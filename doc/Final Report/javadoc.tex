\documentclass[11pt]{report}
\pagenumbering{Roman}
\def\bl{\mbox{}\newline\mbox{}\newline{}}
\usepackage{ifthen}
\renewcommand{\contentsname}
{
\centering
ANNE API JavaDoc
\\ \LARGE APPENDIX B of ANNE Report
\\ \LARGE Table of Contents
}
\newcommand{\hide}[2]{
\ifthenelse{\equal{#1}{inherited}}%
{}%
{}%
}
\newcommand{\entityintro}[3]{%
  \hbox to \hsize{%
    \vbox{%
      \hbox to .2in{}%
    }%
    {\bf #1}%
    \dotfill\pageref{#2}%
  }
  \makebox[\hsize]{%
    \parbox{.4in}{}%
    \parbox[l]{5in}{%
      \vspace{1mm}\it%
      #3%
      \vspace{1mm}%
    }%
  }%
}
\newcommand{\isep}[0]{%
\setlength{\itemsep}{-.4ex}
}
\newcommand{\sld}[0]{%
\setlength{\topsep}{0em}
\setlength{\partopsep}{0em}
\setlength{\parskip}{0em}
\setlength{\parsep}{-1em}
}
\newcommand{\headref}[3]{%
\ifthenelse{#1 = 1}{%
\addcontentsline{toc}{section}{\hspace{\qquad}\protect\numberline{}{#3}}%
}{}%
\ifthenelse{#1 = 2}{%
\addcontentsline{toc}{subsection}{\hspace{\qquad}\protect\numerline{}{#3}}%
}{}%
\ifthenelse{#1 = 3}{%
\addcontentsline{toc}{subsubsection}{\hspace{\qquad}\protect\numerline{}{#3}}%
}{}%
\label{#3}%
\makebox[\textwidth][l]{#2 #3}%
}%
\newcommand{\membername}[1]{{\it #1}\linebreak}
\newcommand{\divideents}[1]{\vskip -1em\indent\rule{2in}{.5mm}}
\newcommand{\refdefined}[1]{
\expandafter\ifx\csname r@#1\endcsname\relax
\relax\else
{$($ in \ref{#1}, page \pageref{#1}$)$}
\fi}
\newcommand{\startsection}[4]{
\gdef\classname{#2}
\subsection{\label{#3}{\bf {\sc #1} #2}}{
\rule[1em]{\hsize}{4pt}\vskip -1em
\vskip .1in 
#4
}%
}
\newcommand{\startsubsubsection}[2]{
\subsubsection{\sc #1}{%
\rule[1em]{\hsize}{2pt}%
#2}
}
\usepackage{color}
\date{\today}
\pagestyle{myheadings}
\addtocontents{toc}{\protect\def\protect\packagename{}}
\addtocontents{toc}{\protect\def\protect\classname{}}
\markboth{\protect\packagename -- \protect\classname}{\protect\packagename -- \protect\classname}
\oddsidemargin 0in
\evensidemargin 0in
% \topmargin -.8in
\chardef\bslash=`\\
\textheight 9.4in
\textwidth 6.5in
\begin{document}
\sloppy
\raggedright
\tableofcontents
\gdef\packagename{}
\gdef\classname{}
\newpage
\def\packagename{uk.ac.ic.doc.neuralnets.gui}
\chapter{\bf Package uk.ac.ic.doc.neuralnets.gui}{
\vskip -.25in
\hbox to \hsize{\it Package Contents\hfil Page}
\rule{\hsize}{.7mm}
\vskip .13in
\hbox{\bf Classes}
\entityintro{CommandMenu}{l0}{...no description...}
\entityintro{CommandToolbar}{l1}{...no description...}
\entityintro{GUILayout}{l2}{This class lays out the GUI skeleton in a given a shell giving access to the
 main pane, side pane and bottom pane.}
\entityintro{GUILog}{l3}{Creates the log box in the bottom bar}
\entityintro{GUIMain}{l4}{Bootstrap.}
\entityintro{GUIManager}{l5}{Manages the GUI representation of a layered neural network.}
\entityintro{GUIMenu}{l6}{Constructs the application menu.}
\entityintro{GUISideBar}{l7}{Controls the Sidebar of the UI.}
\entityintro{GUIToolbar}{l8}{Constructs the application toolbar from {\tt ToolbarPlugins}.}
\entityintro{ImageHandler}{l9}{The ImageHandleris responsible for retrieving 
 {\tt Image} instances for named image files.}
\entityintro{MenuPlugin}{l10}{Menu plugins create the application menu structure.}
\entityintro{NetworkModifier}{l11}{Network Modifiers are pluggable units in the Modify tab.}
\entityintro{NeuroneCombo}{l12}{...no description...}
\entityintro{RunPanel}{l13}{Creates the user interface for the Run tab.}
\entityintro{ScrollingTextAppender}{l14}{...no description...}
\entityintro{ToolbarPlugin}{l15}{ToolbarPlugins add buttons to the application toolbar.}
\entityintro{TrainingPanel}{l16}{Create the Training Panel}
\vskip .1in
\rule{\hsize}{.7mm}
\vskip .1in
\newpage
\section{Classes}{
\startsection{Class}{CommandMenu}{l0}{%
\startsubsubsection{Declaration}{
\fbox{\vbox{
\hbox{\vbox{\small public 
class 
CommandMenu}}
\noindent\hbox{\vbox{{\bf extends} uk.ac.ic.doc.neuralnets.gui.MenuPlugin}}
\noindent\hbox{\vbox{{\bf implements} 
uk.ac.ic.doc.neuralnets.events.EventHandler, java.lang.Runnable}}
}}}
\startsubsubsection{Constructors}{
\vskip -2em
\begin{itemize}
\item{\vskip -1.9ex 
\membername{CommandMenu}
{\tt public {\bf CommandMenu}(  )
\label{l17}\label{l18}}%end signature
}%end item
\end{itemize}
}
\startsubsubsection{Methods}{
\vskip -2em
\begin{itemize}
\item{\vskip -1.9ex 
\membername{flush}
{\tt public void {\bf flush}(  )
\label{l19}\label{l20}}%end signature
}%end item
\divideents{getName}
\item{\vskip -1.9ex 
\membername{getName}
{\tt public String {\bf getName}(  )
\label{l21}\label{l22}}%end signature
}%end item
\divideents{getPriority}
\item{\vskip -1.9ex 
\membername{getPriority}
{\tt public int {\bf getPriority}(  )
\label{l23}\label{l24}}%end signature
}%end item
\divideents{handle}
\item{\vskip -1.9ex 
\membername{handle}
{\tt public void {\bf handle}( {\tt uk.ac.ic.doc.neuralnets.events.Event } {\bf e} )
\label{l25}\label{l26}}%end signature
}%end item
\divideents{isValid}
\item{\vskip -1.9ex 
\membername{isValid}
{\tt public boolean {\bf isValid}(  )
\label{l27}\label{l28}}%end signature
}%end item
\divideents{load}
\item{\vskip -1.9ex 
\membername{load}
{\tt public void {\bf load}( {\tt uk.ac.ic.doc.neuralnets.gui.GUIMenu } {\bf menu} )
\label{l29}\label{l30}}%end signature
}%end item
\divideents{run}
\item{\vskip -1.9ex 
\membername{run}
{\tt public void {\bf run}(  )
\label{l31}\label{l32}}%end signature
}%end item
\end{itemize}
}
\startsubsubsection{Methods inherited from class {\tt uk.ac.ic.doc.neuralnets.gui.MenuPlugin}}{
\par{\small 
\refdefined{l10}\vskip -2em
\begin{itemize}
\item{\vskip -1.9ex 
\membername{load}
{\tt public abstract void {\bf load}( {\tt uk.ac.ic.doc.neuralnets.gui.GUIMenu } {\bf menu} )
}%end signature
\begin{itemize}
\sld
\item{
\sld
{\bf Usage}
  \begin{itemize}\isep
   \item{
Creates the menu for the plugin.s
}%end item
  \end{itemize}
}
\item{
\sld
{\bf Parameters}
\sld\isep
  \begin{itemize}
\sld\isep
   \item{
\sld
{\tt menu} - }
  \end{itemize}
}%end item
\end{itemize}
}%end item
\end{itemize}
}}
\startsubsubsection{Methods inherited from class {\tt uk.ac.ic.doc.neuralnets.util.plugins.PriorityPlugin}}{
\par{\small 
\refdefined{l33}\vskip -2em
\begin{itemize}
\item{\vskip -1.9ex 
\membername{compareTo}
{\tt public int {\bf compareTo}( {\tt uk.ac.ic.doc.neuralnets.util.plugins.PriorityPlugin } {\bf o} )
}%end signature
}%end item
\divideents{getPriority}
\item{\vskip -1.9ex 
\membername{getPriority}
{\tt public abstract int {\bf getPriority}(  )
}%end signature
\begin{itemize}
\sld
\item{
\sld
{\bf Usage}
  \begin{itemize}\isep
   \item{
The plugin's priority.
}%end item
  \end{itemize}
}
\item{{\bf Returns} - 
the priority 
}%end item
\end{itemize}
}%end item
\end{itemize}
}}
}
\startsection{Class}{CommandToolbar}{l1}{%
\startsubsubsection{Declaration}{
\fbox{\vbox{
\hbox{\vbox{\small public 
class 
CommandToolbar}}
\noindent\hbox{\vbox{{\bf extends} uk.ac.ic.doc.neuralnets.gui.ToolbarPlugin}}
\noindent\hbox{\vbox{{\bf implements} 
uk.ac.ic.doc.neuralnets.events.EventHandler, java.lang.Runnable}}
}}}
\startsubsubsection{Constructors}{
\vskip -2em
\begin{itemize}
\item{\vskip -1.9ex 
\membername{CommandToolbar}
{\tt public {\bf CommandToolbar}(  )
\label{l34}\label{l35}}%end signature
}%end item
\end{itemize}
}
\startsubsubsection{Methods}{
\vskip -2em
\begin{itemize}
\item{\vskip -1.9ex 
\membername{create}
{\tt public void {\bf create}( {\tt uk.ac.ic.doc.neuralnets.gui.GUIToolbar } {\bf toolbar} )
\label{l36}\label{l37}}%end signature
}%end item
\divideents{flush}
\item{\vskip -1.9ex 
\membername{flush}
{\tt public void {\bf flush}(  )
\label{l38}\label{l39}}%end signature
}%end item
\divideents{getName}
\item{\vskip -1.9ex 
\membername{getName}
{\tt public String {\bf getName}(  )
\label{l40}\label{l41}}%end signature
}%end item
\divideents{getPriority}
\item{\vskip -1.9ex 
\membername{getPriority}
{\tt public int {\bf getPriority}(  )
\label{l42}\label{l43}}%end signature
}%end item
\divideents{handle}
\item{\vskip -1.9ex 
\membername{handle}
{\tt public void {\bf handle}( {\tt uk.ac.ic.doc.neuralnets.events.Event } {\bf e} )
\label{l44}\label{l45}}%end signature
}%end item
\divideents{isValid}
\item{\vskip -1.9ex 
\membername{isValid}
{\tt public boolean {\bf isValid}(  )
\label{l46}\label{l47}}%end signature
}%end item
\divideents{run}
\item{\vskip -1.9ex 
\membername{run}
{\tt public void {\bf run}(  )
\label{l48}\label{l49}}%end signature
}%end item
\end{itemize}
}
\startsubsubsection{Methods inherited from class {\tt uk.ac.ic.doc.neuralnets.gui.ToolbarPlugin}}{
\par{\small 
\refdefined{l15}\vskip -2em
\begin{itemize}
\item{\vskip -1.9ex 
\membername{create}
{\tt public abstract void {\bf create}( {\tt uk.ac.ic.doc.neuralnets.gui.GUIToolbar } {\bf toolbar} )
}%end signature
\begin{itemize}
\sld
\item{
\sld
{\bf Usage}
  \begin{itemize}\isep
   \item{
Create buttons to add to the toolbar.
 
 For example: {\tt 
 		toolbar.addItem("MyItem");
 		toolbar.addButton("MyItem", "MyButton");
 }
}%end item
  \end{itemize}
}
\item{
\sld
{\bf Parameters}
\sld\isep
  \begin{itemize}
\sld\isep
   \item{
\sld
{\tt toolbar} - - the application toolbar to which to add buttons}
  \end{itemize}
}%end item
\end{itemize}
}%end item
\end{itemize}
}}
\startsubsubsection{Methods inherited from class {\tt uk.ac.ic.doc.neuralnets.util.plugins.PriorityPlugin}}{
\par{\small 
\refdefined{l33}\vskip -2em
\begin{itemize}
\item{\vskip -1.9ex 
\membername{compareTo}
{\tt public int {\bf compareTo}( {\tt uk.ac.ic.doc.neuralnets.util.plugins.PriorityPlugin } {\bf o} )
}%end signature
}%end item
\divideents{getPriority}
\item{\vskip -1.9ex 
\membername{getPriority}
{\tt public abstract int {\bf getPriority}(  )
}%end signature
\begin{itemize}
\sld
\item{
\sld
{\bf Usage}
  \begin{itemize}\isep
   \item{
The plugin's priority.
}%end item
  \end{itemize}
}
\item{{\bf Returns} - 
the priority 
}%end item
\end{itemize}
}%end item
\end{itemize}
}}
}
\startsection{Class}{GUILayout}{l2}{%
{\small This class lays out the GUI skeleton in a given a shell giving access to the
 main pane, side pane and bottom pane.}
\vskip .1in 
\startsubsubsection{Declaration}{
\fbox{\vbox{
\hbox{\vbox{\small public 
class 
GUILayout}}
\noindent\hbox{\vbox{{\bf extends} java.lang.Object}}
}}}
\startsubsubsection{Constructors}{
\vskip -2em
\begin{itemize}
\item{\vskip -1.9ex 
\membername{GUILayout}
{\tt public {\bf GUILayout}( {\tt org.eclipse.swt.widgets.Shell } {\bf shell} )
\label{l50}\label{l51}}%end signature
\begin{itemize}
\sld
\item{
\sld
{\bf Usage}
  \begin{itemize}\isep
   \item{
Adds layout containers to the shell.
}%end item
  \end{itemize}
}
\item{
\sld
{\bf Parameters}
\sld\isep
  \begin{itemize}
\sld\isep
   \item{
\sld
{\tt shell} - }
  \end{itemize}
}%end item
\end{itemize}
}%end item
\end{itemize}
}
\startsubsubsection{Methods}{
\vskip -2em
\begin{itemize}
\item{\vskip -1.9ex 
\membername{getBottomContainer}
{\tt public Composite {\bf getBottomContainer}(  )
\label{l52}\label{l53}}%end signature
\begin{itemize}
\sld
\item{
\sld
{\bf Usage}
  \begin{itemize}\isep
   \item{
Get the bottom pane
}%end item
  \end{itemize}
}
\item{{\bf Returns} - 
the Composite for the bottom container 
}%end item
\end{itemize}
}%end item
\divideents{getGraphContainer}
\item{\vskip -1.9ex 
\membername{getGraphContainer}
{\tt public Composite {\bf getGraphContainer}(  )
\label{l54}\label{l55}}%end signature
\begin{itemize}
\sld
\item{
\sld
{\bf Usage}
  \begin{itemize}\isep
   \item{
Gets the main window pane
}%end item
  \end{itemize}
}
\item{{\bf Returns} - 
the Composite for the graph container 
}%end item
\end{itemize}
}%end item
\divideents{getSidebarContainer}
\item{\vskip -1.9ex 
\membername{getSidebarContainer}
{\tt public Composite {\bf getSidebarContainer}(  )
\label{l56}\label{l57}}%end signature
\begin{itemize}
\sld
\item{
\sld
{\bf Usage}
  \begin{itemize}\isep
   \item{
Gets the side pane
}%end item
  \end{itemize}
}
\item{{\bf Returns} - 
the Composite for the side container 
}%end item
\end{itemize}
}%end item
\divideents{getToolbar}
\item{\vskip -1.9ex 
\membername{getToolbar}
{\tt public CoolBar {\bf getToolbar}(  )
\label{l58}\label{l59}}%end signature
\begin{itemize}
\sld
\item{
\sld
{\bf Usage}
  \begin{itemize}\isep
   \item{
Get the toolbar
}%end item
  \end{itemize}
}
\item{{\bf Returns} - 
the application toolbar as a CoolBar 
}%end item
\end{itemize}
}%end item
\end{itemize}
}
}
\startsection{Class}{GUILog}{l3}{%
{\small Creates the log box in the bottom bar}
\vskip .1in 
\startsubsubsection{Declaration}{
\fbox{\vbox{
\hbox{\vbox{\small public 
class 
GUILog}}
\noindent\hbox{\vbox{{\bf extends} java.lang.Object}}
}}}
\startsubsubsection{Constructors}{
\vskip -2em
\begin{itemize}
\item{\vskip -1.9ex 
\membername{GUILog}
{\tt public {\bf GUILog}( {\tt org.eclipse.swt.widgets.Composite } {\bf container} )
\label{l60}\label{l61}}%end signature
}%end item
\end{itemize}
}
}
\startsection{Class}{GUIMain}{l4}{%
{\small Bootstrap.}
\vskip .1in 
\startsubsubsection{Declaration}{
\fbox{\vbox{
\hbox{\vbox{\small public 
class 
GUIMain}}
\noindent\hbox{\vbox{{\bf extends} java.lang.Object}}
}}}
\startsubsubsection{Constructors}{
\vskip -2em
\begin{itemize}
\item{\vskip -1.9ex 
\membername{GUIMain}
{\tt public {\bf GUIMain}(  )
\label{l62}\label{l63}}%end signature
}%end item
\end{itemize}
}
\startsubsubsection{Methods}{
\vskip -2em
\begin{itemize}
\item{\vskip -1.9ex 
\membername{main}
{\tt public static void {\bf main}( {\tt java.lang.String []} {\bf args} )
\label{l64}\label{l65}}%end signature
\begin{itemize}
\sld
\item{
\sld
{\bf Parameters}
\sld\isep
  \begin{itemize}
\sld\isep
   \item{
\sld
{\tt args} - }
  \end{itemize}
}%end item
\end{itemize}
}%end item
\end{itemize}
}
}
\startsection{Class}{GUIManager}{l5}{%
{\small Manages the GUI representation of a layered neural network. Controls
 importing and exporting networks to and from their standard model
 representation, zooming into and out of layers of the network, and tooltips.
 
 Listens synchronously for GraphUpdateEvents, NewNeuroneTypeEvents,
 NeuralNetworkTickEvents and NeuralNetworkSimulationEvents}
\vskip .1in 
\startsubsubsection{Declaration}{
\fbox{\vbox{
\hbox{\vbox{\small public 
class 
GUIManager}}
\noindent\hbox{\vbox{{\bf extends} uk.ac.ic.doc.neuralnets.coreui.ZoomingInterfaceManager}}
}}}
\startsubsubsection{Constructors}{
\vskip -2em
\begin{itemize}
\item{\vskip -1.9ex 
\membername{GUIManager}
{\tt public {\bf GUIManager}( {\tt org.eclipse.zest.core.widgets.IContainer } {\bf graph},
{\tt uk.ac.ic.doc.neuralnets.graph.neural.NeuralNetwork } {\bf network} )
\label{l66}\label{l67}}%end signature
\begin{itemize}
\sld
\item{
\sld
{\bf Usage}
  \begin{itemize}\isep
   \item{
Creates a GUIManager to display a given Neural Network on a given SWT
 IContainer canvas.
}%end item
  \end{itemize}
}
\item{
\sld
{\bf Parameters}
\sld\isep
  \begin{itemize}
\sld\isep
   \item{
\sld
{\tt graph} - the canvas on which to display the network}
   \item{
\sld
{\tt network} - the network to be displayed in the GUI}
  \end{itemize}
}%end item
\end{itemize}
}%end item
\divideents{GUIManager}
\item{\vskip -1.9ex 
\membername{GUIManager}
{\tt public {\bf GUIManager}( {\tt org.eclipse.zest.core.widgets.IContainer } {\bf graph},
{\tt uk.ac.ic.doc.neuralnets.graph.neural.NeuralNetwork } {\bf network},
{\tt uk.ac.ic.doc.neuralnets.persistence.FileSpecification } {\bf location} )
\label{l68}\label{l69}}%end signature
\begin{itemize}
\sld
\item{
\sld
{\bf Usage}
  \begin{itemize}\isep
   \item{
Creates a GUIManager to display a given Neural Network, from a given
 location, on a given SWT IContainer canvas.
}%end item
  \end{itemize}
}
\item{
\sld
{\bf Parameters}
\sld\isep
  \begin{itemize}
\sld\isep
   \item{
\sld
{\tt graph} - the canvas on which to display the network}
   \item{
\sld
{\tt network} - the network to be displayed in the GUI}
   \item{
\sld
{\tt location} - the location of the network}
  \end{itemize}
}%end item
\end{itemize}
}%end item
\end{itemize}
}
\startsubsubsection{Methods}{
\vskip -2em
\begin{itemize}
\item{\vskip -1.9ex 
\membername{addConnection}
{\tt public void {\bf addConnection}( {\tt uk.ac.ic.doc.neuralnets.graph.Edge } {\bf e} )
\label{l70}\label{l71}}%end signature
}%end item
\divideents{canZoomIn}
\item{\vskip -1.9ex 
\membername{canZoomIn}
{\tt public boolean {\bf canZoomIn}(  )
\label{l72}\label{l73}}%end signature
}%end item
\divideents{canZoomOut}
\item{\vskip -1.9ex 
\membername{canZoomOut}
{\tt public boolean {\bf canZoomOut}(  )
\label{l74}\label{l75}}%end signature
}%end item
\divideents{disableGraph}
\item{\vskip -1.9ex 
\membername{disableGraph}
{\tt public void {\bf disableGraph}(  )
\label{l76}\label{l77}}%end signature
\begin{itemize}
\sld
\item{
\sld
{\bf Usage}
  \begin{itemize}\isep
   \item{
Disable clicks to the graph area.
}%end item
  \end{itemize}
}
\end{itemize}
}%end item
\divideents{enableGraph}
\item{\vskip -1.9ex 
\membername{enableGraph}
{\tt public void {\bf enableGraph}(  )
\label{l78}\label{l79}}%end signature
\begin{itemize}
\sld
\item{
\sld
{\bf Usage}
  \begin{itemize}\isep
   \item{
Enable clicks to the graph area
}%end item
  \end{itemize}
}
\end{itemize}
}%end item
\divideents{getCurrentNetwork}
\item{\vskip -1.9ex 
\membername{getCurrentNetwork}
{\tt public NeuralNetwork {\bf getCurrentNetwork}(  )
\label{l80}\label{l81}}%end signature
}%end item
\divideents{getGraph}
\item{\vskip -1.9ex 
\membername{getGraph}
{\tt public Graph {\bf getGraph}(  )
\label{l82}\label{l83}}%end signature
}%end item
\divideents{getNode}
\item{\vskip -1.9ex 
\membername{getNode}
{\tt public GraphItem {\bf getNode}( {\tt uk.ac.ic.doc.neuralnets.graph.neural.Neurone } {\bf n} )
\label{l84}\label{l85}}%end signature
}%end item
\divideents{getZoomIDs}
\item{\vskip -1.9ex 
\membername{getZoomIDs}
{\tt public Stack {\bf getZoomIDs}(  )
\label{l86}\label{l87}}%end signature
}%end item
\divideents{getZoomLevels}
\item{\vskip -1.9ex 
\membername{getZoomLevels}
{\tt public Stack {\bf getZoomLevels}(  )
\label{l88}\label{l89}}%end signature
}%end item
\divideents{persistLocations}
\item{\vskip -1.9ex 
\membername{persistLocations}
{\tt public void {\bf persistLocations}(  )
\label{l90}\label{l91}}%end signature
}%end item
\divideents{redrawCurrentView}
\item{\vskip -1.9ex 
\membername{redrawCurrentView}
{\tt public void {\bf redrawCurrentView}(  )
\label{l92}\label{l93}}%end signature
}%end item
\divideents{remove}
\item{\vskip -1.9ex 
\membername{remove}
{\tt public void {\bf remove}( {\tt org.eclipse.zest.core.widgets.GraphItem } {\bf i} )
\label{l94}\label{l95}}%end signature
}%end item
\divideents{removeNetwork}
\item{\vskip -1.9ex 
\membername{removeNetwork}
{\tt public void {\bf removeNetwork}( {\tt uk.ac.ic.doc.neuralnets.graph.neural.NeuralNetwork } {\bf n} )
\label{l96}\label{l97}}%end signature
\begin{itemize}
\sld
\item{
\sld
{\bf Usage}
  \begin{itemize}\isep
   \item{
Removes the given neural network from the current view, and redraws the
 screen as necessary.
}%end item
  \end{itemize}
}
\item{
\sld
{\bf Parameters}
\sld\isep
  \begin{itemize}
\sld\isep
   \item{
\sld
{\tt n} - the neural network to add to the current section of the neural
            network}
  \end{itemize}
}%end item
\end{itemize}
}%end item
\divideents{reset}
\item{\vskip -1.9ex 
\membername{reset}
{\tt protected void {\bf reset}(  )
\label{l98}\label{l99}}%end signature
}%end item
\divideents{updateInterfaceHints}
\item{\vskip -1.9ex 
\membername{updateInterfaceHints}
{\tt public void {\bf updateInterfaceHints}(  )
\label{l100}\label{l101}}%end signature
}%end item
\divideents{zoomIn}
\item{\vskip -1.9ex 
\membername{zoomIn}
{\tt public void {\bf zoomIn}( {\tt uk.ac.ic.doc.neuralnets.graph.neural.NeuralNetwork } {\bf n} )
\label{l102}\label{l103}}%end signature
}%end item
\divideents{zoomOut}
\item{\vskip -1.9ex 
\membername{zoomOut}
{\tt public void {\bf zoomOut}(  )
\label{l104}\label{l105}}%end signature
}%end item
\end{itemize}
}
\startsubsubsection{Methods inherited from class {\tt uk.ac.ic.doc.neuralnets.coreui.ZoomingInterfaceManager}}{
\par{\small 
\refdefined{l106}\vskip -2em
\begin{itemize}
\item{\vskip -1.9ex 
\membername{canZoomIn}
{\tt public abstract boolean {\bf canZoomIn}(  )
}%end signature
\begin{itemize}
\sld
\item{
\sld
{\bf Usage}
  \begin{itemize}\isep
   \item{
Checks whether or not it is possible to zoom in. It is only possible to
 zoom in if exactly one internal network layer is selected.
}%end item
  \end{itemize}
}
\item{{\bf Returns} - 
whether or not it is possible to zoom in 
}%end item
\end{itemize}
}%end item
\divideents{canZoomOut}
\item{\vskip -1.9ex 
\membername{canZoomOut}
{\tt public abstract boolean {\bf canZoomOut}(  )
}%end signature
\begin{itemize}
\sld
\item{
\sld
{\bf Usage}
  \begin{itemize}\isep
   \item{
Checks whether or not it is possible to zoom out. It is always possible
 to zoom out unless the current view is the root network.
}%end item
  \end{itemize}
}
\item{{\bf Returns} - 
whether or not it is possible to zoom out 
}%end item
\end{itemize}
}%end item
\divideents{getZoomIDs}
\item{\vskip -1.9ex 
\membername{getZoomIDs}
{\tt public abstract Stack {\bf getZoomIDs}(  )
}%end signature
\begin{itemize}
\sld
\item{
\sld
{\bf Usage}
  \begin{itemize}\isep
   \item{
Returns a stack containing the IDs of each network layer that has
 currently been zoomed into. This can be used to trace the current zoom
 path from the root of the neural network.
}%end item
  \end{itemize}
}
\item{{\bf Returns} - 
a stack of IDs of each network layer that is currently zoomed
         into 
}%end item
\end{itemize}
}%end item
\divideents{getZoomLevels}
\item{\vskip -1.9ex 
\membername{getZoomLevels}
{\tt public abstract Stack {\bf getZoomLevels}(  )
}%end signature
\begin{itemize}
\sld
\item{
\sld
{\bf Usage}
  \begin{itemize}\isep
   \item{
Returns a stack containing each network layer that has currently been
 zoomed into, starting with the root network.
}%end item
  \end{itemize}
}
\item{{\bf Returns} - 
a stack containing each network layer that has currently been
         zoomed into. 
}%end item
\end{itemize}
}%end item
\divideents{zoomIn}
\item{\vskip -1.9ex 
\membername{zoomIn}
{\tt public abstract void {\bf zoomIn}( {\tt uk.ac.ic.doc.neuralnets.graph.neural.NeuralNetwork } {\bf n} )
}%end signature
\begin{itemize}
\sld
\item{
\sld
{\bf Usage}
  \begin{itemize}\isep
   \item{
Zooms into the selected network layer. Clears the current view, and
 instead shows the contents of the selected network layer.
}%end item
  \end{itemize}
}
\item{
\sld
{\bf Parameters}
\sld\isep
  \begin{itemize}
\sld\isep
   \item{
\sld
{\tt n} - the network to zoom into.}
  \end{itemize}
}%end item
\end{itemize}
}%end item
\divideents{zoomOut}
\item{\vskip -1.9ex 
\membername{zoomOut}
{\tt public abstract void {\bf zoomOut}(  )
}%end signature
\begin{itemize}
\sld
\item{
\sld
{\bf Usage}
  \begin{itemize}\isep
   \item{
Zooms out one layer. Clears the current view, and instead shows the
 contents of the current layer's parent. If the current view is the root
 network, then nothing happens as it is not possible to zoom out further.
}%end item
  \end{itemize}
}
\end{itemize}
}%end item
\end{itemize}
}}
\startsubsubsection{Methods inherited from class {\tt uk.ac.ic.doc.neuralnets.coreui.InterfaceManager}}{
\par{\small 
\refdefined{l107}\vskip -2em
\begin{itemize}
\item{\vskip -1.9ex 
\membername{addConnection}
{\tt public void {\bf addConnection}( {\tt uk.ac.ic.doc.neuralnets.graph.Edge } {\bf e} )
}%end signature
\begin{itemize}
\sld
\item{
\sld
{\bf Usage}
  \begin{itemize}\isep
   \item{
Adds the given edge to the current view, and redraws the screen as
 necessary.
}%end item
  \end{itemize}
}
\item{
\sld
{\bf Parameters}
\sld\isep
  \begin{itemize}
\sld\isep
   \item{
\sld
{\tt e} - }
  \end{itemize}
}%end item
\end{itemize}
}%end item
\divideents{addNetwork}
\item{\vskip -1.9ex 
\membername{addNetwork}
{\tt public void {\bf addNetwork}( {\tt uk.ac.ic.doc.neuralnets.graph.neural.NeuralNetwork } {\bf n} )
}%end signature
\begin{itemize}
\sld
\item{
\sld
{\bf Usage}
  \begin{itemize}\isep
   \item{
Adds the given neural network to the current view, and redraws the screen
 as necessary.
}%end item
  \end{itemize}
}
\item{
\sld
{\bf Parameters}
\sld\isep
  \begin{itemize}
\sld\isep
   \item{
\sld
{\tt n} - the neural network to add to the current section of the neural
            network}
  \end{itemize}
}%end item
\end{itemize}
}%end item
\divideents{addNeurone}
\item{\vskip -1.9ex 
\membername{addNeurone}
{\tt public void {\bf addNeurone}( {\tt uk.ac.ic.doc.neuralnets.graph.neural.Neurone } {\bf n} )
}%end signature
\begin{itemize}
\sld
\item{
\sld
{\bf Usage}
  \begin{itemize}\isep
   \item{
Adds the given neurone to the current view, and redraws the screen
 as necessary.
}%end item
  \end{itemize}
}
\item{
\sld
{\bf Parameters}
\sld\isep
  \begin{itemize}
\sld\isep
   \item{
\sld
{\tt n} - the neurone to add to the current section of the neural
            network}
  \end{itemize}
}%end item
\end{itemize}
}%end item
\divideents{addNode}
\item{\vskip -1.9ex 
\membername{addNode}
{\tt public void {\bf addNode}( {\tt uk.ac.ic.doc.neuralnets.graph.Node } {\bf n} )
}%end signature
\begin{itemize}
\sld
\item{
\sld
{\bf Usage}
  \begin{itemize}\isep
   \item{
Adds the given node to the current view, and redraws the screen
 as necessary.
}%end item
  \end{itemize}
}
\item{
\sld
{\bf Parameters}
\sld\isep
  \begin{itemize}
\sld\isep
   \item{
\sld
{\tt n} - the node to add to the current section of the neural
            network}
  \end{itemize}
}%end item
\end{itemize}
}%end item
\divideents{addNode}
\item{\vskip -1.9ex 
\membername{addNode}
{\tt public void {\bf addNode}( {\tt uk.ac.ic.doc.neuralnets.graph.neural.NodeSpecification } {\bf spec} )
}%end signature
\begin{itemize}
\sld
\item{
\sld
{\bf Usage}
  \begin{itemize}\isep
   \item{
Creates a node from the give specification, adds to the current view, and
 redraws the screen as necessary.
}%end item
  \end{itemize}
}
\item{
\sld
{\bf Parameters}
\sld\isep
  \begin{itemize}
\sld\isep
   \item{
\sld
{\tt spec} - the specification of the node to add to the current section of
            the neural network}
  \end{itemize}
}%end item
\end{itemize}
}%end item
\divideents{getCommandControl}
\item{\vskip -1.9ex 
\membername{getCommandControl}
{\tt public CommandControl {\bf getCommandControl}(  )
}%end signature
\begin{itemize}
\sld
\item{
\sld
{\bf Usage}
  \begin{itemize}\isep
   \item{
Gets the command control used by the GUIManager. This object handles the
 undo and redo stacks as commands are executed and undone.
}%end item
  \end{itemize}
}
\item{{\bf Returns} - 
the CommandControl object used by the GUIManager 
}%end item
\end{itemize}
}%end item
\divideents{getCurrentNetwork}
\item{\vskip -1.9ex 
\membername{getCurrentNetwork}
{\tt public abstract NeuralNetwork {\bf getCurrentNetwork}(  )
}%end signature
\begin{itemize}
\sld
\item{
\sld
{\bf Usage}
  \begin{itemize}\isep
   \item{
Returns the neural network layer currently being viewed in the
 GUIManager.
}%end item
  \end{itemize}
}
\item{{\bf Returns} - 
the current neural network layer 
}%end item
\end{itemize}
}%end item
\divideents{getGraph}
\item{\vskip -1.9ex 
\membername{getGraph}
{\tt public abstract Object {\bf getGraph}(  )
}%end signature
\begin{itemize}
\sld
\item{
\sld
{\bf Usage}
  \begin{itemize}\isep
   \item{
Returns the Graph representation used by this UI Manager.
}%end item
  \end{itemize}
}
\item{{\bf Returns} - 
the Graph that the Manager draws onto 
}%end item
\end{itemize}
}%end item
\divideents{getNode}
\item{\vskip -1.9ex 
\membername{getNode}
{\tt public abstract Object {\bf getNode}( {\tt uk.ac.ic.doc.neuralnets.graph.neural.Neurone } {\bf n} )
}%end signature
\begin{itemize}
\sld
\item{
\sld
{\bf Usage}
  \begin{itemize}\isep
   \item{
Finds the GUINode in the GUI corresponding to the given Neurone and
 returns it. Returns null if the given Neurone is not loaded in the GUI.
}%end item
  \end{itemize}
}
\item{
\sld
{\bf Parameters}
\sld\isep
  \begin{itemize}
\sld\isep
   \item{
\sld
{\tt n} - the Neurone to look up in the GUI}
  \end{itemize}
}%end item
\item{{\bf Returns} - 
the GUINode in the GUI corresponding to the given Neurone 
}%end item
\end{itemize}
}%end item
\divideents{getRootNetwork}
\item{\vskip -1.9ex 
\membername{getRootNetwork}
{\tt public NeuralNetwork {\bf getRootNetwork}(  )
}%end signature
\begin{itemize}
\sld
\item{
\sld
{\bf Usage}
  \begin{itemize}\isep
   \item{
Gets the root of the layered neural network stored in the GUIManager.
}%end item
  \end{itemize}
}
\item{{\bf Returns} - 
the root of the main neural network 
}%end item
\end{itemize}
}%end item
\divideents{getSaveLocation}
\item{\vskip -1.9ex 
\membername{getSaveLocation}
{\tt public FileSpecification {\bf getSaveLocation}(  )
}%end signature
\begin{itemize}
\sld
\item{
\sld
{\bf Usage}
  \begin{itemize}\isep
   \item{
Gets the location to save the network to, or null if no such location
 exists.
}%end item
  \end{itemize}
}
\item{{\bf Returns} - 
the network's save location, or null if none exists 
}%end item
\end{itemize}
}%end item
\divideents{getUtils}
\item{\vskip -1.9ex 
\membername{getUtils}
{\tt public InteractionUtils {\bf getUtils}(  )
}%end signature
\begin{itemize}
\sld
\item{
\sld
{\bf Usage}
  \begin{itemize}\isep
   \item{
Returns the GUIManager's interaction utilities.
}%end item
  \end{itemize}
}
\item{{\bf Returns} - 
the InteractionUtils object used by the GUIManager 
}%end item
\end{itemize}
}%end item
\divideents{persistLocations}
\item{\vskip -1.9ex 
\membername{persistLocations}
{\tt public abstract void {\bf persistLocations}(  )
}%end signature
\begin{itemize}
\sld
\item{
\sld
{\bf Usage}
  \begin{itemize}\isep
   \item{
Pushes down the locations of all Nodes to the model. Allows positions
 to be persisted to storage and reloaded.
}%end item
  \end{itemize}
}
\end{itemize}
}%end item
\divideents{redrawCurrentView}
\item{\vskip -1.9ex 
\membername{redrawCurrentView}
{\tt public abstract void {\bf redrawCurrentView}(  )
}%end signature
\begin{itemize}
\sld
\item{
\sld
{\bf Usage}
  \begin{itemize}\isep
   \item{
Draws the current view of the graph. Imports the current network layer
 from the internal model and applies the current layout.
}%end item
  \end{itemize}
}
\end{itemize}
}%end item
\divideents{remove}
\item{\vskip -1.9ex 
\membername{remove}
{\tt public abstract void {\bf remove}( {\tt java.lang.Object } {\bf i} )
}%end signature
\begin{itemize}
\sld
\item{
\sld
{\bf Usage}
  \begin{itemize}\isep
   \item{
Removes the given GraphItem from the view.
}%end item
  \end{itemize}
}
\item{
\sld
{\bf Parameters}
\sld\isep
  \begin{itemize}
\sld\isep
   \item{
\sld
{\tt i} - the graphitem to be removed from the view}
  \end{itemize}
}%end item
\end{itemize}
}%end item
\divideents{removeNetwork}
\item{\vskip -1.9ex 
\membername{removeNetwork}
{\tt public void {\bf removeNetwork}( {\tt uk.ac.ic.doc.neuralnets.graph.neural.NeuralNetwork } {\bf n} )
}%end signature
\begin{itemize}
\sld
\item{
\sld
{\bf Usage}
  \begin{itemize}\isep
   \item{
Removes the given neural network from the current view, and redraws the
 screen as necessary.
}%end item
  \end{itemize}
}
\item{
\sld
{\bf Parameters}
\sld\isep
  \begin{itemize}
\sld\isep
   \item{
\sld
{\tt n} - the neural network to remove from the current section of the neural
            network}
  \end{itemize}
}%end item
\end{itemize}
}%end item
\divideents{reset}
\item{\vskip -1.9ex 
\membername{reset}
{\tt protected abstract void {\bf reset}(  )
}%end signature
\begin{itemize}
\sld
\item{
\sld
{\bf Usage}
  \begin{itemize}\isep
   \item{
Reset the current manager, e.g. when a new network is loaded
}%end item
  \end{itemize}
}
\end{itemize}
}%end item
\divideents{setNetwork}
\item{\vskip -1.9ex 
\membername{setNetwork}
{\tt public void {\bf setNetwork}( {\tt uk.ac.ic.doc.neuralnets.graph.neural.NeuralNetwork } {\bf network},
{\tt uk.ac.ic.doc.neuralnets.persistence.FileSpecification } {\bf location} )
}%end signature
\begin{itemize}
\sld
\item{
\sld
{\bf Usage}
  \begin{itemize}\isep
   \item{
Loads the given neural network into the GUIManager, from the given
 location.
}%end item
  \end{itemize}
}
\item{
\sld
{\bf Parameters}
\sld\isep
  \begin{itemize}
\sld\isep
   \item{
\sld
{\tt network} - the network to be loaded into the GUIManager}
   \item{
\sld
{\tt location} - the location to load the network from}
  \end{itemize}
}%end item
\end{itemize}
}%end item
\divideents{setSaveLocation}
\item{\vskip -1.9ex 
\membername{setSaveLocation}
{\tt public void {\bf setSaveLocation}( {\tt uk.ac.ic.doc.neuralnets.persistence.FileSpecification } {\bf saveLoc} )
}%end signature
\begin{itemize}
\sld
\item{
\sld
{\bf Usage}
  \begin{itemize}\isep
   \item{
Sets the network's save location.
}%end item
  \end{itemize}
}
\item{
\sld
{\bf Parameters}
\sld\isep
  \begin{itemize}
\sld\isep
   \item{
\sld
{\tt saveLoc} - }
  \end{itemize}
}%end item
\end{itemize}
}%end item
\divideents{updateInterfaceHints}
\item{\vskip -1.9ex 
\membername{updateInterfaceHints}
{\tt public abstract void {\bf updateInterfaceHints}(  )
}%end signature
\begin{itemize}
\sld
\item{
\sld
{\bf Usage}
  \begin{itemize}\isep
   \item{
Updates the tooltips or other UI hints of all graph elements 
 in the current view.
}%end item
  \end{itemize}
}
\end{itemize}
}%end item
\end{itemize}
}}
}
\startsection{Class}{GUIMenu}{l6}{%
{\small Constructs the application menu. Looks for {\tt MenuPlugins}, sorts
 them according to priority, then loads them into the menu.}
\vskip .1in 
\startsubsubsection{Declaration}{
\fbox{\vbox{
\hbox{\vbox{\small public 
class 
GUIMenu}}
\noindent\hbox{\vbox{{\bf extends} java.lang.Object}}
}}}
\startsubsubsection{Constructors}{
\vskip -2em
\begin{itemize}
\item{\vskip -1.9ex 
\membername{GUIMenu}
{\tt public {\bf GUIMenu}( {\tt org.eclipse.swt.widgets.Shell } {\bf rootShell},
{\tt uk.ac.ic.doc.neuralnets.coreui.ZoomingInterfaceManager } {\bf gm} )
\label{l108}\label{l109}}%end signature
\begin{itemize}
\sld
\item{
\sld
{\bf Usage}
  \begin{itemize}\isep
   \item{
Creates the application menu by requesting {\tt MenuPlugins} from
 the PluginManager.
}%end item
  \end{itemize}
}
\item{
\sld
{\bf Parameters}
\sld\isep
  \begin{itemize}
\sld\isep
   \item{
\sld
{\tt rootShell} - - the shell the menu is for}
   \item{
\sld
{\tt gm} - - the graph manager.}
  \end{itemize}
}%end item
\item{{\bf See Also}
  \begin{itemize}
   \item{{\tt uk.ac.ic.doc.neuralnets.util.plugins.PluginManager} {\small 
\refdefined{l110}}%end \small
}%end item
  \end{itemize}
}%end item
\end{itemize}
}%end item
\end{itemize}
}
\startsubsubsection{Methods}{
\vskip -2em
\begin{itemize}
\item{\vskip -1.9ex 
\membername{addMenuItem}
{\tt public MenuItem {\bf addMenuItem}( {\tt java.lang.String } {\bf parent},
{\tt java.lang.String } {\bf name} )
\label{l111}\label{l112}}%end signature
\begin{itemize}
\sld
\item{
\sld
{\bf Usage}
  \begin{itemize}\isep
   \item{
Adds a named menu item to a parent menu
}%end item
  \end{itemize}
}
\item{
\sld
{\bf Parameters}
\sld\isep
  \begin{itemize}
\sld\isep
   \item{
\sld
{\tt parent} - - the menu to add the item to. If the parent menu isn't found
            then the root menu is used.}
   \item{
\sld
{\tt name} - - the name for the new menu item.}
  \end{itemize}
}%end item
\item{{\bf Returns} - 
the newly created {\tt MenuItem} 
}%end item
\end{itemize}
}%end item
\divideents{addMenuSeparator}
\item{\vskip -1.9ex 
\membername{addMenuSeparator}
{\tt public void {\bf addMenuSeparator}( {\tt java.lang.String } {\bf parent} )
\label{l113}\label{l114}}%end signature
\begin{itemize}
\sld
\item{
\sld
{\bf Usage}
  \begin{itemize}\isep
   \item{
Add a separator to parent menu
}%end item
  \end{itemize}
}
\item{
\sld
{\bf Parameters}
\sld\isep
  \begin{itemize}
\sld\isep
   \item{
\sld
{\tt parent} - - menu to separate}
  \end{itemize}
}%end item
\end{itemize}
}%end item
\divideents{addSubMenu}
\item{\vskip -1.9ex 
\membername{addSubMenu}
{\tt public MenuItem {\bf addSubMenu}( {\tt java.lang.String } {\bf parent},
{\tt java.lang.String } {\bf name} )
\label{l115}\label{l116}}%end signature
\begin{itemize}
\sld
\item{
\sld
{\bf Usage}
  \begin{itemize}\isep
   \item{
Adds a menu item to the parent menu and connects an empty menu to it. The
 highest level menu is {\bf "root"} which is automatically created.
}%end item
  \end{itemize}
}
\item{
\sld
{\bf Parameters}
\sld\isep
  \begin{itemize}
\sld\isep
   \item{
\sld
{\tt parent} - - name of the parent menu, e.g. "root", if the parent menu is
            not found then the root menu will be used.}
   \item{
\sld
{\tt name} - - name of the new submenu}
  \end{itemize}
}%end item
\item{{\bf Returns} - 
{\tt MenuItem} for the new submenu, if the submenu already
         exists then that {\tt MenuItem} is returned. 
}%end item
\end{itemize}
}%end item
\divideents{getManager}
\item{\vskip -1.9ex 
\membername{getManager}
{\tt public ZoomingInterfaceManager {\bf getManager}(  )
\label{l117}\label{l118}}%end signature
\begin{itemize}
\sld
\item{
\sld
{\bf Usage}
  \begin{itemize}\isep
   \item{
Get the graph manager.
}%end item
  \end{itemize}
}
\item{{\bf Returns} - 
the ZoomingInterfaceManager for the graph. 
}%end item
\end{itemize}
}%end item
\divideents{getShell}
\item{\vskip -1.9ex 
\membername{getShell}
{\tt public Shell {\bf getShell}(  )
\label{l119}\label{l120}}%end signature
\begin{itemize}
\sld
\item{
\sld
{\bf Usage}
  \begin{itemize}\isep
   \item{
Get the parent shell of the menu.
}%end item
  \end{itemize}
}
\item{{\bf Returns} - 
the main program shell 
}%end item
\end{itemize}
}%end item
\end{itemize}
}
}
\startsection{Class}{GUISideBar}{l7}{%
{\small Controls the Sidebar of the UI.}
\vskip .1in 
\startsubsubsection{Declaration}{
\fbox{\vbox{
\hbox{\vbox{\small public 
class 
GUISideBar}}
\noindent\hbox{\vbox{{\bf extends} java.lang.Object}}
}}}
\startsubsubsection{Constructors}{
\vskip -2em
\begin{itemize}
\item{\vskip -1.9ex 
\membername{GUISideBar}
{\tt public {\bf GUISideBar}( {\tt org.eclipse.swt.widgets.Composite } {\bf container},
{\tt uk.ac.ic.doc.neuralnets.coreui.ZoomingInterfaceManager } {\bf gm} )
\label{l121}\label{l122}}%end signature
\begin{itemize}
\sld
\item{
\sld
{\bf Usage}
  \begin{itemize}\isep
   \item{
Create the Sidebar.
}%end item
  \end{itemize}
}
\item{
\sld
{\bf Parameters}
\sld\isep
  \begin{itemize}
\sld\isep
   \item{
\sld
{\tt container} - - sidebar container}
   \item{
\sld
{\tt gm} - - graph manager.}
  \end{itemize}
}%end item
\end{itemize}
}%end item
\end{itemize}
}
}
\startsection{Class}{GUIToolbar}{l8}{%
{\small Constructs the application toolbar from {\tt ToolbarPlugins}. The
 toolbar is a collection of groups which can each contain a number of
 buttons$/$controls.}
\vskip .1in 
\startsubsubsection{Declaration}{
\fbox{\vbox{
\hbox{\vbox{\small public 
class 
GUIToolbar}}
\noindent\hbox{\vbox{{\bf extends} java.lang.Object}}
}}}
\startsubsubsection{Constructors}{
\vskip -2em
\begin{itemize}
\item{\vskip -1.9ex 
\membername{GUIToolbar}
{\tt public {\bf GUIToolbar}( {\tt org.eclipse.swt.widgets.CoolBar } {\bf coolbar},
{\tt uk.ac.ic.doc.neuralnets.coreui.ZoomingInterfaceManager } {\bf gm} )
\label{l123}\label{l124}}%end signature
\begin{itemize}
\sld
\item{
\sld
{\bf Usage}
  \begin{itemize}\isep
   \item{
Creates the application toolbar by requesting {\tt ToolbarPlugins}
 from the plugin manager.
}%end item
  \end{itemize}
}
\item{
\sld
{\bf Parameters}
\sld\isep
  \begin{itemize}
\sld\isep
   \item{
\sld
{\tt coolbar} - }
   \item{
\sld
{\tt gm} - }
  \end{itemize}
}%end item
\end{itemize}
}%end item
\end{itemize}
}
\startsubsubsection{Methods}{
\vskip -2em
\begin{itemize}
\item{\vskip -1.9ex 
\membername{addButton}
{\tt public ToolItem {\bf addButton}( {\tt java.lang.String } {\bf parent},
{\tt org.eclipse.swt.graphics.Image } {\bf icon} )
\label{l125}\label{l126}}%end signature
\begin{itemize}
\sld
\item{
\sld
{\bf Usage}
  \begin{itemize}\isep
   \item{
Add a button to a parent group with an icon.
}%end item
  \end{itemize}
}
\item{
\sld
{\bf Parameters}
\sld\isep
  \begin{itemize}
\sld\isep
   \item{
\sld
{\tt parent} - - the parent group.}
   \item{
\sld
{\tt icon} - - the icon {\tt Image}.}
  \end{itemize}
}%end item
\item{{\bf Returns} - 
- the new button 
}%end item
\end{itemize}
}%end item
\divideents{addButton}
\item{\vskip -1.9ex 
\membername{addButton}
{\tt public ToolItem {\bf addButton}( {\tt java.lang.String } {\bf parent},
{\tt java.lang.String } {\bf name} )
\label{l127}\label{l128}}%end signature
\begin{itemize}
\sld
\item{
\sld
{\bf Usage}
  \begin{itemize}\isep
   \item{
Add a button to a parent group with text
}%end item
  \end{itemize}
}
\item{
\sld
{\bf Parameters}
\sld\isep
  \begin{itemize}
\sld\isep
   \item{
\sld
{\tt parent} - - the name parent group}
   \item{
\sld
{\tt name} - - text to appear on the button}
  \end{itemize}
}%end item
\item{{\bf Returns} - 
- the new button 
}%end item
\end{itemize}
}%end item
\divideents{addButton}
\item{\vskip -1.9ex 
\membername{addButton}
{\tt public ToolItem {\bf addButton}( {\tt java.lang.String } {\bf parent},
{\tt java.lang.String } {\bf name},
{\tt int } {\bf type} )
\label{l129}\label{l130}}%end signature
\begin{itemize}
\sld
\item{
\sld
{\bf Usage}
  \begin{itemize}\isep
   \item{
Add a radio$/$toggle button to a parent group.
}%end item
  \end{itemize}
}
\item{
\sld
{\bf Parameters}
\sld\isep
  \begin{itemize}
\sld\isep
   \item{
\sld
{\tt parent} - - the parent group}
   \item{
\sld
{\tt name} - - the button name}
   \item{
\sld
{\tt type} - - the button type SWT.CHECK$/$SWT.RADIO$/$SWT.SEPARATOR}
  \end{itemize}
}%end item
\item{{\bf Returns} - 
- the new button 
}%end item
\end{itemize}
}%end item
\divideents{addGroup}
\item{\vskip -1.9ex 
\membername{addGroup}
{\tt public CoolItem {\bf addGroup}( {\tt java.lang.String } {\bf name} )
\label{l131}\label{l132}}%end signature
\begin{itemize}
\sld
\item{
\sld
{\bf Usage}
  \begin{itemize}\isep
   \item{
Add a new group to the toolbar.
}%end item
  \end{itemize}
}
\item{
\sld
{\bf Parameters}
\sld\isep
  \begin{itemize}
\sld\isep
   \item{
\sld
{\tt name} - - name of the new toolbar.}
  \end{itemize}
}%end item
\end{itemize}
}%end item
\divideents{getManager}
\item{\vskip -1.9ex 
\membername{getManager}
{\tt public ZoomingInterfaceManager {\bf getManager}(  )
\label{l133}\label{l134}}%end signature
\begin{itemize}
\sld
\item{
\sld
{\bf Usage}
  \begin{itemize}\isep
   \item{
Get the graph manager. Allows toolbar buttons to have listeners which
 modify the graph.
}%end item
  \end{itemize}
}
\item{{\bf Returns} - 
- the manager for the graph. 
}%end item
\end{itemize}
}%end item
\divideents{getShell}
\item{\vskip -1.9ex 
\membername{getShell}
{\tt public Shell {\bf getShell}(  )
\label{l135}\label{l136}}%end signature
\begin{itemize}
\sld
\item{
\sld
{\bf Usage}
  \begin{itemize}\isep
   \item{
Get the parent shell. Allows toolbar buttons to have listeners which
 create new shells.
}%end item
  \end{itemize}
}
\item{{\bf Returns} - 
- the toolbars parent shell 
}%end item
\end{itemize}
}%end item
\divideents{repackGroup}
\item{\vskip -1.9ex 
\membername{repackGroup}
{\tt public void {\bf repackGroup}( {\tt java.lang.String } {\bf itemGroup} )
\label{l137}\label{l138}}%end signature
\begin{itemize}
\sld
\item{
\sld
{\bf Usage}
  \begin{itemize}\isep
   \item{
Recalculate the size of the toolbar group
}%end item
  \end{itemize}
}
\item{
\sld
{\bf Parameters}
\sld\isep
  \begin{itemize}
\sld\isep
   \item{
\sld
{\tt itemGroup} - }
  \end{itemize}
}%end item
\end{itemize}
}%end item
\end{itemize}
}
}
\startsection{Class}{ImageHandler}{l9}{%
{\small The ImageHandleris responsible for retrieving 
 {\tt Image} instances for named image files.}
\vskip .1in 
\startsubsubsection{Declaration}{
\fbox{\vbox{
\hbox{\vbox{\small public 
class 
ImageHandler}}
\noindent\hbox{\vbox{{\bf extends} java.lang.Object}}
}}}
\startsubsubsection{Methods}{
\vskip -2em
\begin{itemize}
\item{\vskip -1.9ex 
\membername{get}
{\tt public static ImageHandler {\bf get}(  )
\label{l139}\label{l140}}%end signature
\begin{itemize}
\sld
\item{
\sld
{\bf Usage}
  \begin{itemize}\isep
   \item{
Get the ImageHandler.
}%end item
  \end{itemize}
}
\item{{\bf Returns} - 
the ImageHandler 
}%end item
\end{itemize}
}%end item
\divideents{getIcon}
\item{\vskip -1.9ex 
\membername{getIcon}
{\tt public Image {\bf getIcon}( {\tt java.lang.String } {\bf name} )
\label{l141}\label{l142}}%end signature
\begin{itemize}
\sld
\item{
\sld
{\bf Usage}
  \begin{itemize}\isep
   \item{
Create an SWT Image for the named icon file from the {\it res$/$icons}
  folder
}%end item
  \end{itemize}
}
\item{
\sld
{\bf Parameters}
\sld\isep
  \begin{itemize}
\sld\isep
   \item{
\sld
{\tt name} - - Icon file name with or without .png extension}
  \end{itemize}
}%end item
\item{{\bf Returns} - 
Image object for file or null if the file is not found. 
}%end item
\end{itemize}
}%end item
\end{itemize}
}
}
\startsection{Class}{MenuPlugin}{l10}{%
{\small Menu plugins create the application menu structure. See GUIMenu for the
 interface used to create menus.}
\vskip .1in 
\startsubsubsection{Declaration}{
\fbox{\vbox{
\hbox{\vbox{\small public abstract 
class 
MenuPlugin}}
\noindent\hbox{\vbox{{\bf extends} uk.ac.ic.doc.neuralnets.util.plugins.PriorityPlugin}}
}}}
\startsubsubsection{Constructors}{
\vskip -2em
\begin{itemize}
\item{\vskip -1.9ex 
\membername{MenuPlugin}
{\tt public {\bf MenuPlugin}(  )
\label{l143}\label{l144}}%end signature
}%end item
\end{itemize}
}
\startsubsubsection{Methods}{
\vskip -2em
\begin{itemize}
\item{\vskip -1.9ex 
\membername{load}
{\tt public abstract void {\bf load}( {\tt uk.ac.ic.doc.neuralnets.gui.GUIMenu } {\bf menu} )
\label{l145}\label{l146}}%end signature
\begin{itemize}
\sld
\item{
\sld
{\bf Usage}
  \begin{itemize}\isep
   \item{
Creates the menu for the plugin.s
}%end item
  \end{itemize}
}
\item{
\sld
{\bf Parameters}
\sld\isep
  \begin{itemize}
\sld\isep
   \item{
\sld
{\tt menu} - }
  \end{itemize}
}%end item
\end{itemize}
}%end item
\end{itemize}
}
\startsubsubsection{Methods inherited from class {\tt uk.ac.ic.doc.neuralnets.util.plugins.PriorityPlugin}}{
\par{\small 
\refdefined{l33}\vskip -2em
\begin{itemize}
\item{\vskip -1.9ex 
\membername{compareTo}
{\tt public int {\bf compareTo}( {\tt uk.ac.ic.doc.neuralnets.util.plugins.PriorityPlugin } {\bf o} )
}%end signature
}%end item
\divideents{getPriority}
\item{\vskip -1.9ex 
\membername{getPriority}
{\tt public abstract int {\bf getPriority}(  )
}%end signature
\begin{itemize}
\sld
\item{
\sld
{\bf Usage}
  \begin{itemize}\isep
   \item{
The plugin's priority.
}%end item
  \end{itemize}
}
\item{{\bf Returns} - 
the priority 
}%end item
\end{itemize}
}%end item
\end{itemize}
}}
}
\startsection{Class}{NetworkModifier}{l11}{%
{\small Network Modifiers are pluggable units in the Modify tab.}
\vskip .1in 
\startsubsubsection{Declaration}{
\fbox{\vbox{
\hbox{\vbox{\small public abstract 
class 
NetworkModifier}}
\noindent\hbox{\vbox{{\bf extends} uk.ac.ic.doc.neuralnets.util.plugins.PriorityPlugin}}
}}}
\startsubsubsection{Constructors}{
\vskip -2em
\begin{itemize}
\item{\vskip -1.9ex 
\membername{NetworkModifier}
{\tt public {\bf NetworkModifier}(  )
\label{l147}\label{l148}}%end signature
}%end item
\end{itemize}
}
\startsubsubsection{Methods}{
\vskip -2em
\begin{itemize}
\item{\vskip -1.9ex 
\membername{getConfigurationGUI}
{\tt public abstract Composite {\bf getConfigurationGUI}( {\tt org.eclipse.swt.widgets.Composite } {\bf parent},
{\tt uk.ac.ic.doc.neuralnets.coreui.ZoomingInterfaceManager } {\bf gm},
{\tt org.eclipse.swt.widgets.ExpandItem } {\bf ei} )
\label{l149}\label{l150}}%end signature
\begin{itemize}
\sld
\item{
\sld
{\bf Usage}
  \begin{itemize}\isep
   \item{
Create the UI for the unit, called during the initialization of the
 modify tab.
}%end item
  \end{itemize}
}
\item{
\sld
{\bf Parameters}
\sld\isep
  \begin{itemize}
\sld\isep
   \item{
\sld
{\tt parent} - - the expand bar for modifiers}
   \item{
\sld
{\tt gm} - - the graph manager}
   \item{
\sld
{\tt ei} - - the expand item for the modifier.}
  \end{itemize}
}%end item
\item{{\bf Returns} - 
composite containing the UI components for the modifier 
}%end item
\end{itemize}
}%end item
\divideents{toString}
\item{\vskip -1.9ex 
\membername{toString}
{\tt public abstract String {\bf toString}(  )
\label{l151}\label{l152}}%end signature
}%end item
\end{itemize}
}
\startsubsubsection{Methods inherited from class {\tt uk.ac.ic.doc.neuralnets.util.plugins.PriorityPlugin}}{
\par{\small 
\refdefined{l33}\vskip -2em
\begin{itemize}
\item{\vskip -1.9ex 
\membername{compareTo}
{\tt public int {\bf compareTo}( {\tt uk.ac.ic.doc.neuralnets.util.plugins.PriorityPlugin } {\bf o} )
}%end signature
}%end item
\divideents{getPriority}
\item{\vskip -1.9ex 
\membername{getPriority}
{\tt public abstract int {\bf getPriority}(  )
}%end signature
\begin{itemize}
\sld
\item{
\sld
{\bf Usage}
  \begin{itemize}\isep
   \item{
The plugin's priority.
}%end item
  \end{itemize}
}
\item{{\bf Returns} - 
the priority 
}%end item
\end{itemize}
}%end item
\end{itemize}
}}
}
\startsection{Class}{NeuroneCombo}{l12}{%
\startsubsubsection{Declaration}{
\fbox{\vbox{
\hbox{\vbox{\small public 
class 
NeuroneCombo}}
\noindent\hbox{\vbox{{\bf extends} java.lang.Object}}
\noindent\hbox{\vbox{{\bf implements} 
uk.ac.ic.doc.neuralnets.events.EventHandler}}
}}}
\startsubsubsection{Constructors}{
\vskip -2em
\begin{itemize}
\item{\vskip -1.9ex 
\membername{NeuroneCombo}
{\tt public {\bf NeuroneCombo}( {\tt org.eclipse.swt.widgets.Composite } {\bf parent},
{\tt java.lang.Class } {\bf filter} )
\label{l153}\label{l154}}%end signature
}%end item
\end{itemize}
}
\startsubsubsection{Methods}{
\vskip -2em
\begin{itemize}
\item{\vskip -1.9ex 
\membername{flush}
{\tt public void {\bf flush}(  )
\label{l155}\label{l156}}%end signature
}%end item
\divideents{getCombo}
\item{\vskip -1.9ex 
\membername{getCombo}
{\tt public Combo {\bf getCombo}(  )
\label{l157}\label{l158}}%end signature
}%end item
\divideents{getName}
\item{\vskip -1.9ex 
\membername{getName}
{\tt public String {\bf getName}(  )
\label{l159}\label{l160}}%end signature
}%end item
\divideents{getSpecification}
\item{\vskip -1.9ex 
\membername{getSpecification}
{\tt public NodeSpecification {\bf getSpecification}(  )
\label{l161}\label{l162}}%end signature
}%end item
\divideents{handle}
\item{\vskip -1.9ex 
\membername{handle}
{\tt public void {\bf handle}( {\tt uk.ac.ic.doc.neuralnets.events.Event } {\bf e} )
\label{l163}\label{l164}}%end signature
}%end item
\divideents{isValid}
\item{\vskip -1.9ex 
\membername{isValid}
{\tt public boolean {\bf isValid}(  )
\label{l165}\label{l166}}%end signature
}%end item
\divideents{setLayoutData}
\item{\vskip -1.9ex 
\membername{setLayoutData}
{\tt public void {\bf setLayoutData}( {\tt java.lang.Object } {\bf layout} )
\label{l167}\label{l168}}%end signature
}%end item
\divideents{setSpecification}
\item{\vskip -1.9ex 
\membername{setSpecification}
{\tt public void {\bf setSpecification}( {\tt uk.ac.ic.doc.neuralnets.graph.neural.NodeSpecification } {\bf spec} )
\label{l169}\label{l170}}%end signature
}%end item
\divideents{updateSpecification}
\item{\vskip -1.9ex 
\membername{updateSpecification}
{\tt public void {\bf updateSpecification}(  )
\label{l171}\label{l172}}%end signature
}%end item
\end{itemize}
}
}
\startsection{Class}{RunPanel}{l13}{%
{\small Creates the user interface for the Run tab. The Run tab listens syncronously
 for NeuralNetworkSimulationEvents and NeuralNetworkTickEvents.}
\vskip .1in 
\startsubsubsection{Declaration}{
\fbox{\vbox{
\hbox{\vbox{\small public 
class 
RunPanel}}
\noindent\hbox{\vbox{{\bf extends} java.lang.Object}}
\noindent\hbox{\vbox{{\bf implements} 
uk.ac.ic.doc.neuralnets.events.EventHandler}}
}}}
\startsubsubsection{Constructors}{
\vskip -2em
\begin{itemize}
\item{\vskip -1.9ex 
\membername{RunPanel}
{\tt public {\bf RunPanel}( {\tt org.eclipse.swt.widgets.Composite } {\bf parent},
{\tt uk.ac.ic.doc.neuralnets.coreui.ZoomingInterfaceManager } {\bf gm} )
\label{l173}\label{l174}}%end signature
\begin{itemize}
\sld
\item{
\sld
{\bf Usage}
  \begin{itemize}\isep
   \item{
Create the Run tab.
}%end item
  \end{itemize}
}
\item{
\sld
{\bf Parameters}
\sld\isep
  \begin{itemize}
\sld\isep
   \item{
\sld
{\tt parent} - - the tab container}
   \item{
\sld
{\tt gm} - - the graph manager}
  \end{itemize}
}%end item
\end{itemize}
}%end item
\end{itemize}
}
\startsubsubsection{Methods}{
\vskip -2em
\begin{itemize}
\item{\vskip -1.9ex 
\membername{flush}
{\tt public void {\bf flush}(  )
\label{l175}\label{l176}}%end signature
}%end item
\divideents{getName}
\item{\vskip -1.9ex 
\membername{getName}
{\tt public String {\bf getName}(  )
\label{l177}\label{l178}}%end signature
}%end item
\divideents{handle}
\item{\vskip -1.9ex 
\membername{handle}
{\tt public void {\bf handle}( {\tt uk.ac.ic.doc.neuralnets.events.Event } {\bf e} )
\label{l179}\label{l180}}%end signature
}%end item
\divideents{isValid}
\item{\vskip -1.9ex 
\membername{isValid}
{\tt public boolean {\bf isValid}(  )
\label{l181}\label{l182}}%end signature
}%end item
\end{itemize}
}
}
\startsection{Class}{ScrollingTextAppender}{l14}{%
\startsubsubsection{Declaration}{
\fbox{\vbox{
\hbox{\vbox{\small public 
class 
ScrollingTextAppender}}
\noindent\hbox{\vbox{{\bf extends} org.apache.log4j.AppenderSkeleton}}
}}}
\startsubsubsection{Constructors}{
\vskip -2em
\begin{itemize}
\item{\vskip -1.9ex 
\membername{ScrollingTextAppender}
{\tt public {\bf ScrollingTextAppender}(  )
\label{l183}\label{l184}}%end signature
}%end item
\end{itemize}
}
\startsubsubsection{Methods}{
\vskip -2em
\begin{itemize}
\item{\vskip -1.9ex 
\membername{append}
{\tt protected void {\bf append}( {\tt org.apache.log4j.spi.LoggingEvent } {\bf e} )
\label{l185}\label{l186}}%end signature
}%end item
\divideents{close}
\item{\vskip -1.9ex 
\membername{close}
{\tt public void {\bf close}(  )
\label{l187}\label{l188}}%end signature
}%end item
\divideents{requiresLayout}
\item{\vskip -1.9ex 
\membername{requiresLayout}
{\tt public boolean {\bf requiresLayout}(  )
\label{l189}\label{l190}}%end signature
}%end item
\divideents{setText}
\item{\vskip -1.9ex 
\membername{setText}
{\tt public static void {\bf setText}( {\tt org.eclipse.swt.custom.StyledText } {\bf t} )
\label{l191}\label{l192}}%end signature
}%end item
\end{itemize}
}
\startsubsubsection{Methods inherited from class {\tt org.apache.log4j.AppenderSkeleton}}{
\par{\small 
\refdefined{l193}\vskip -2em
\begin{itemize}
\item{\vskip -1.9ex 
\membername{activateOptions}
{\tt public void {\bf activateOptions}(  )
}%end signature
}%end item
\divideents{addFilter}
\item{\vskip -1.9ex 
\membername{addFilter}
{\tt public void {\bf addFilter}( {\tt org.apache.log4j.spi.Filter } {\bf arg0} )
}%end signature
}%end item
\divideents{append}
\item{\vskip -1.9ex 
\membername{append}
{\tt protected abstract void {\bf append}( {\tt org.apache.log4j.spi.LoggingEvent } {\bf arg0} )
}%end signature
}%end item
\divideents{clearFilters}
\item{\vskip -1.9ex 
\membername{clearFilters}
{\tt public void {\bf clearFilters}(  )
}%end signature
}%end item
\divideents{doAppend}
\item{\vskip -1.9ex 
\membername{doAppend}
{\tt public synchronized void {\bf doAppend}( {\tt org.apache.log4j.spi.LoggingEvent } {\bf arg0} )
}%end signature
}%end item
\divideents{finalize}
\item{\vskip -1.9ex 
\membername{finalize}
{\tt public void {\bf finalize}(  )
}%end signature
}%end item
\divideents{getErrorHandler}
\item{\vskip -1.9ex 
\membername{getErrorHandler}
{\tt public ErrorHandler {\bf getErrorHandler}(  )
}%end signature
}%end item
\divideents{getFilter}
\item{\vskip -1.9ex 
\membername{getFilter}
{\tt public Filter {\bf getFilter}(  )
}%end signature
}%end item
\divideents{getFirstFilter}
\item{\vskip -1.9ex 
\membername{getFirstFilter}
{\tt public final Filter {\bf getFirstFilter}(  )
}%end signature
}%end item
\divideents{getLayout}
\item{\vskip -1.9ex 
\membername{getLayout}
{\tt public Layout {\bf getLayout}(  )
}%end signature
}%end item
\divideents{getName}
\item{\vskip -1.9ex 
\membername{getName}
{\tt public final String {\bf getName}(  )
}%end signature
}%end item
\divideents{getThreshold}
\item{\vskip -1.9ex 
\membername{getThreshold}
{\tt public Priority {\bf getThreshold}(  )
}%end signature
}%end item
\divideents{isAsSevereAsThreshold}
\item{\vskip -1.9ex 
\membername{isAsSevereAsThreshold}
{\tt public boolean {\bf isAsSevereAsThreshold}( {\tt org.apache.log4j.Priority } {\bf arg0} )
}%end signature
}%end item
\divideents{setErrorHandler}
\item{\vskip -1.9ex 
\membername{setErrorHandler}
{\tt public synchronized void {\bf setErrorHandler}( {\tt org.apache.log4j.spi.ErrorHandler } {\bf arg0} )
}%end signature
}%end item
\divideents{setLayout}
\item{\vskip -1.9ex 
\membername{setLayout}
{\tt public void {\bf setLayout}( {\tt org.apache.log4j.Layout } {\bf arg0} )
}%end signature
}%end item
\divideents{setName}
\item{\vskip -1.9ex 
\membername{setName}
{\tt public void {\bf setName}( {\tt java.lang.String } {\bf arg0} )
}%end signature
}%end item
\divideents{setThreshold}
\item{\vskip -1.9ex 
\membername{setThreshold}
{\tt public void {\bf setThreshold}( {\tt org.apache.log4j.Priority } {\bf arg0} )
}%end signature
}%end item
\end{itemize}
}}
}
\startsection{Class}{ToolbarPlugin}{l15}{%
{\small ToolbarPlugins add buttons to the application toolbar.}
\vskip .1in 
\startsubsubsection{Declaration}{
\fbox{\vbox{
\hbox{\vbox{\small public abstract 
class 
ToolbarPlugin}}
\noindent\hbox{\vbox{{\bf extends} uk.ac.ic.doc.neuralnets.util.plugins.PriorityPlugin}}
}}}
\startsubsubsection{Constructors}{
\vskip -2em
\begin{itemize}
\item{\vskip -1.9ex 
\membername{ToolbarPlugin}
{\tt public {\bf ToolbarPlugin}(  )
\label{l194}\label{l195}}%end signature
}%end item
\end{itemize}
}
\startsubsubsection{Methods}{
\vskip -2em
\begin{itemize}
\item{\vskip -1.9ex 
\membername{create}
{\tt public abstract void {\bf create}( {\tt uk.ac.ic.doc.neuralnets.gui.GUIToolbar } {\bf toolbar} )
\label{l196}\label{l197}}%end signature
\begin{itemize}
\sld
\item{
\sld
{\bf Usage}
  \begin{itemize}\isep
   \item{
Create buttons to add to the toolbar.
 
 For example: {\tt 
 		toolbar.addItem("MyItem");
 		toolbar.addButton("MyItem", "MyButton");
 }
}%end item
  \end{itemize}
}
\item{
\sld
{\bf Parameters}
\sld\isep
  \begin{itemize}
\sld\isep
   \item{
\sld
{\tt toolbar} - - the application toolbar to which to add buttons}
  \end{itemize}
}%end item
\end{itemize}
}%end item
\end{itemize}
}
\startsubsubsection{Methods inherited from class {\tt uk.ac.ic.doc.neuralnets.util.plugins.PriorityPlugin}}{
\par{\small 
\refdefined{l33}\vskip -2em
\begin{itemize}
\item{\vskip -1.9ex 
\membername{compareTo}
{\tt public int {\bf compareTo}( {\tt uk.ac.ic.doc.neuralnets.util.plugins.PriorityPlugin } {\bf o} )
}%end signature
}%end item
\divideents{getPriority}
\item{\vskip -1.9ex 
\membername{getPriority}
{\tt public abstract int {\bf getPriority}(  )
}%end signature
\begin{itemize}
\sld
\item{
\sld
{\bf Usage}
  \begin{itemize}\isep
   \item{
The plugin's priority.
}%end item
  \end{itemize}
}
\item{{\bf Returns} - 
the priority 
}%end item
\end{itemize}
}%end item
\end{itemize}
}}
}
\startsection{Class}{TrainingPanel}{l16}{%
{\small Create the Training Panel}
\vskip .1in 
\startsubsubsection{Declaration}{
\fbox{\vbox{
\hbox{\vbox{\small public 
class 
TrainingPanel}}
\noindent\hbox{\vbox{{\bf extends} java.lang.Object}}
}}}
\startsubsubsection{Constructors}{
\vskip -2em
\begin{itemize}
\item{\vskip -1.9ex 
\membername{TrainingPanel}
{\tt public {\bf TrainingPanel}( {\tt org.eclipse.swt.widgets.Composite } {\bf c},
{\tt uk.ac.ic.doc.neuralnets.coreui.ZoomingInterfaceManager } {\bf gm} )
\label{l198}\label{l199}}%end signature
}%end item
\end{itemize}
}
}
}
}
\newpage
\def\packagename{uk.ac.ic.doc.neuralnets.graph.neural.manipulation}
\chapter{\bf Package uk.ac.ic.doc.neuralnets.graph.neural.manipulation}{
\vskip -.25in
\hbox to \hsize{\it Package Contents\hfil Page}
\rule{\hsize}{.7mm}
\vskip .13in
\hbox{\bf Classes}
\entityintro{EdgeCreatedEvent}{l200}{Event to indicate an edge has been created}
\entityintro{EdgeFactory}{l201}{EdgeFactory creates Edges from EdgeSpecifications}
\entityintro{GraphFactory}{l202}{GraphFactory makes Graphs from GraphSpecifications}
\entityintro{GraphSpecification}{l203}{Encodes the details of the Graph to be created}
\entityintro{HomogenousNetworkSpecification}{l204}{...no description...}
\entityintro{InhibitoryNodeSpecification}{l205}{Default NodeSpecification for Inhibitory Spiking neurones.}
\entityintro{InteractionUtils}{l206}{...no description...}
\entityintro{InteractionUtils.NetworkRunner}{l207}{The thread used to run the network asynchronously with the UI}
\entityintro{NodeCreatedEvent}{l208}{Indicates a node has been created by the factory}
\entityintro{NodeFactory}{l209}{NodeFactory creates Node objects from NodeSpecifications.}
\entityintro{PerceptronSpecification}{l210}{Default NodeSpecification for Perceptrons.}
\entityintro{SpikingNodeSpecification}{l211}{Default NodeSpecification for SpikingNeurones}
\vskip .1in
\rule{\hsize}{.7mm}
\vskip .1in
\newpage
\section{Classes}{
\startsection{Class}{EdgeCreatedEvent}{l200}{%
{\small Event to indicate an edge has been created}
\vskip .1in 
\startsubsubsection{Declaration}{
\fbox{\vbox{
\hbox{\vbox{\small public 
class 
EdgeCreatedEvent}}
\noindent\hbox{\vbox{{\bf extends} uk.ac.ic.doc.neuralnets.events.Event}}
}}}
\startsubsubsection{Constructors}{
\vskip -2em
\begin{itemize}
\item{\vskip -1.9ex 
\membername{EdgeCreatedEvent}
{\tt public {\bf EdgeCreatedEvent}( {\tt int } {\bf num},
{\tt int } {\bf count} )
\label{l212}\label{l213}}%end signature
}%end item
\end{itemize}
}
\startsubsubsection{Methods}{
\vskip -2em
\begin{itemize}
\item{\vskip -1.9ex 
\membername{getEdgeCount}
{\tt public int {\bf getEdgeCount}(  )
\label{l214}\label{l215}}%end signature
\begin{itemize}
\sld
\item{
\sld
{\bf Usage}
  \begin{itemize}\isep
   \item{
Answer the approximate number of edges to be created; this may be 
 probabilistic and thus differ to the actual number created
}%end item
  \end{itemize}
}
\item{{\bf Returns} - 
A guess at the number of edges that will be created 
}%end item
\end{itemize}
}%end item
\divideents{getEdgeNumber}
\item{\vskip -1.9ex 
\membername{getEdgeNumber}
{\tt public int {\bf getEdgeNumber}(  )
\label{l216}\label{l217}}%end signature
\begin{itemize}
\sld
\item{
\sld
{\bf Usage}
  \begin{itemize}\isep
   \item{
Answer the number of edges thus far created
}%end item
  \end{itemize}
}
\item{{\bf Returns} - 
How many edges were created at the point of this event 
}%end item
\end{itemize}
}%end item
\divideents{toString}
\item{\vskip -1.9ex 
\membername{toString}
{\tt public String {\bf toString}(  )
\label{l218}\label{l219}}%end signature
}%end item
\end{itemize}
}
\startsubsubsection{Methods inherited from class {\tt uk.ac.ic.doc.neuralnets.events.Event}}{
\par{\small 
\refdefined{l220}\vskip -2em
\begin{itemize}
\item{\vskip -1.9ex 
\membername{toString}
{\tt public abstract String {\bf toString}(  )
}%end signature
}%end item
\end{itemize}
}}
}
\startsection{Class}{EdgeFactory}{l201}{%
{\small EdgeFactory creates Edges from EdgeSpecifications}
\vskip .1in 
\startsubsubsection{Declaration}{
\fbox{\vbox{
\hbox{\vbox{\small public 
class 
EdgeFactory}}
\noindent\hbox{\vbox{{\bf extends} java.lang.Object}}
\noindent\hbox{\vbox{{\bf implements} 
java.io.Serializable}}
}}}
\startsubsubsection{Constructors}{
\vskip -2em
\begin{itemize}
\item{\vskip -1.9ex 
\membername{EdgeFactory}
{\tt public {\bf EdgeFactory}(  )
\label{l221}\label{l222}}%end signature
}%end item
\end{itemize}
}
\startsubsubsection{Methods}{
\vskip -2em
\begin{itemize}
\item{\vskip -1.9ex 
\membername{create}
{\tt public Edge {\bf create}( {\tt uk.ac.ic.doc.neuralnets.graph.neural.EdgeSpecification } {\bf s} )
\label{l223}\label{l224}}%end signature
\begin{itemize}
\sld
\item{
\sld
{\bf Usage}
  \begin{itemize}\isep
   \item{
Create an edge conforming to the given EdgeSpecification. Currently it
 is required that \textless From\textgreater  and \textless To\textgreater  are the same type. If they are both
 Neurones, a Synapse is created. If they are both NeuralNetworks, a
 NetworkBridge is constructed.
}%end item
  \end{itemize}
}
\item{
\sld
{\bf Parameters}
\sld\isep
  \begin{itemize}
\sld\isep
   \item{
\sld
{\tt s} - The EdgeSpecification to use}
  \end{itemize}
}%end item
\item{{\bf Returns} - 
The created edge 
}%end item
\item{{\bf Exceptions}
  \begin{itemize}
\sld
   \item{\vskip -.6ex{\tt java.lang.UnsupportedOperationException} - When the types of the nodes are
 unsupported in this version of the factory.}
  \end{itemize}
}%end item
\item{{\bf See Also}
  \begin{itemize}
   \item{{\tt uk.ac.ic.doc.neuralnets.graph.neural.Neurone} {\small 
\refdefined{l225}}%end \small
}%end item
   \item{{\tt uk.ac.ic.doc.neuralnets.graph.neural.NeuralNetwork} {\small 
\refdefined{l226}}%end \small
}%end item
   \item{{\tt uk.ac.ic.doc.neuralnets.graph.Edge} {\small 
\refdefined{l227}}%end \small
}%end item
   \item{{\tt uk.ac.ic.doc.neuralnets.graph.Node} {\small 
\refdefined{l228}}%end \small
}%end item
   \item{{\tt uk.ac.ic.doc.neuralnets.graph.EdgeSpecification} {\small 
\refdefined{l229}}%end \small
}%end item
  \end{itemize}
}%end item
\end{itemize}
}%end item
\divideents{create}
\item{\vskip -1.9ex 
\membername{create}
{\tt public Edge {\bf create}( {\tt uk.ac.ic.doc.neuralnets.graph.Node } {\bf f},
{\tt uk.ac.ic.doc.neuralnets.graph.Node } {\bf t} )
\label{l230}\label{l231}}%end signature
\begin{itemize}
\sld
\item{
\sld
{\bf Usage}
  \begin{itemize}\isep
   \item{
Create an edge between the supplied nodes
}%end item
  \end{itemize}
}
\item{
\sld
{\bf Parameters}
\sld\isep
  \begin{itemize}
\sld\isep
   \item{
\sld
{\tt f} - The start node}
   \item{
\sld
{\tt t} - The end node}
  \end{itemize}
}%end item
\item{{\bf Returns} - 
The created edge 
}%end item
\item{{\bf See Also}
  \begin{itemize}
   \item{{\tt uk.ac.ic.doc.neuralnets.graph.Edge} {\small 
\refdefined{l227}}%end \small
}%end item
   \item{{\tt uk.ac.ic.doc.neuralnets.graph.Node} {\small 
\refdefined{l228}}%end \small
}%end item
  \end{itemize}
}%end item
\end{itemize}
}%end item
\divideents{get}
\item{\vskip -1.9ex 
\membername{get}
{\tt public static EdgeFactory {\bf get}(  )
\label{l232}\label{l233}}%end signature
\begin{itemize}
\sld
\item{
\sld
{\bf Usage}
  \begin{itemize}\isep
   \item{
Get the factory instance.
}%end item
  \end{itemize}
}
\item{{\bf Returns} - 
the EdgeFactory 
}%end item
\end{itemize}
}%end item
\end{itemize}
}
}
\startsection{Class}{GraphFactory}{l202}{%
{\small GraphFactory makes Graphs from GraphSpecifications}
\vskip .1in 
\startsubsubsection{Declaration}{
\fbox{\vbox{
\hbox{\vbox{\small public 
class 
GraphFactory}}
\noindent\hbox{\vbox{{\bf extends} java.lang.Object}}
}}}
\startsubsubsection{Fields}{
\begin{itemize}
\item{
public static final int EVENT\_RESOLUTION\begin{itemize}\item{\vskip -.9ex }\end{itemize}
}
\end{itemize}
}
\startsubsubsection{Constructors}{
\vskip -2em
\begin{itemize}
\item{\vskip -1.9ex 
\membername{GraphFactory}
{\tt public {\bf GraphFactory}(  )
\label{l234}\label{l235}}%end signature
}%end item
\end{itemize}
}
\startsubsubsection{Methods}{
\vskip -2em
\begin{itemize}
\item{\vskip -1.9ex 
\membername{create}
{\tt public Graph {\bf create}( {\tt java.lang.Class } {\bf type},
{\tt uk.ac.ic.doc.neuralnets.graph.neural.NodeSpecification } {\bf ntype},
{\tt int } {\bf quantity} )
\label{l236}\label{l237}}%end signature
\begin{itemize}
\sld
\item{
\sld
{\bf Usage}
  \begin{itemize}\isep
   \item{
Create a Graph of the given type, with the supplied quantity and type of nodes
}%end item
  \end{itemize}
}
\item{
\sld
{\bf Parameters}
\sld\isep
  \begin{itemize}
\sld\isep
   \item{
\sld
{\tt type} - the Class of graph to create}
   \item{
\sld
{\tt ntype} - The NodeSpecification encoding the type of node to include}
   \item{
\sld
{\tt quantity} - The quantity of nodes to produce}
  \end{itemize}
}%end item
\item{{\bf Returns} - 
The given neural network 
}%end item
\end{itemize}
}%end item
\divideents{create}
\item{\vskip -1.9ex 
\membername{create}
{\tt public Graph {\bf create}( {\tt uk.ac.ic.doc.neuralnets.graph.neural.manipulation.GraphSpecification } {\bf spec} )
\label{l238}\label{l239}}%end signature
\begin{itemize}
\sld
\item{
\sld
{\bf Usage}
  \begin{itemize}\isep
   \item{
Create a Graph conforming to the given GraphSpecification.
}%end item
  \end{itemize}
}
\item{
\sld
{\bf Parameters}
\sld\isep
  \begin{itemize}
\sld\isep
   \item{
\sld
{\tt spec} - The specification of the graph. Supports some specialisation
 for homogeneous networks}
  \end{itemize}
}%end item
\item{{\bf Returns} - 
The created Graph 
}%end item
\item{{\bf See Also}
  \begin{itemize}
   \item{{\tt uk.ac.ic.doc.neuralnets.graph.neural.manipulation.HomogenousNetworkSpecification} {\small 
\refdefined{l204}}%end \small
}%end item
  \end{itemize}
}%end item
\end{itemize}
}%end item
\divideents{get}
\item{\vskip -1.9ex 
\membername{get}
{\tt public static GraphFactory {\bf get}(  )
\label{l240}\label{l241}}%end signature
\begin{itemize}
\sld
\item{
\sld
{\bf Usage}
  \begin{itemize}\isep
   \item{
Get the instance of this factory
}%end item
  \end{itemize}
}
\item{{\bf Returns} - 
The GraphFactory. 
}%end item
\end{itemize}
}%end item
\divideents{makeNetwork}
\item{\vskip -1.9ex 
\membername{makeNetwork}
{\tt public NeuralNetwork {\bf makeNetwork}( {\tt int } {\bf n},
{\tt double } {\bf edgeProb} )
\label{l242}\label{l243}}%end signature
\begin{itemize}
\sld
\item{
\sld
{\bf Usage}
  \begin{itemize}\isep
   \item{
Make a homogeneous network of n nodes, connected with edgeProb 
 probability. Utilises the default node type.
}%end item
  \end{itemize}
}
\item{
\sld
{\bf Parameters}
\sld\isep
  \begin{itemize}
\sld\isep
   \item{
\sld
{\tt n} - the number of nodes to create}
   \item{
\sld
{\tt edgeProb} - The probability of edge created}
  \end{itemize}
}%end item
\item{{\bf Returns} - 
The NeuralNetwork created 
}%end item
\end{itemize}
}%end item
\end{itemize}
}
}
\startsection{Class}{GraphSpecification}{l203}{%
{\small Encodes the details of the Graph to be created}
\vskip .1in 
\startsubsubsection{Declaration}{
\fbox{\vbox{
\hbox{\vbox{\small public abstract 
class 
GraphSpecification}}
\noindent\hbox{\vbox{{\bf extends} java.lang.Object}}
}}}
\startsubsubsection{Constructors}{
\vskip -2em
\begin{itemize}
\item{\vskip -1.9ex 
\membername{GraphSpecification}
{\tt public {\bf GraphSpecification}(  )
\label{l244}\label{l245}}%end signature
\begin{itemize}
\sld
\item{
\sld
{\bf Usage}
  \begin{itemize}\isep
   \item{
Create a default, empty graph.
}%end item
  \end{itemize}
}
\end{itemize}
}%end item
\divideents{GraphSpecification}
\item{\vskip -1.9ex 
\membername{GraphSpecification}
{\tt public {\bf GraphSpecification}( {\tt java.util.List } {\bf nodes} )
\label{l246}\label{l247}}%end signature
\begin{itemize}
\sld
\item{
\sld
{\bf Usage}
  \begin{itemize}\isep
   \item{
Create a graph of the default node type, in the supplied quantity
}%end item
  \end{itemize}
}
\item{
\sld
{\bf Parameters}
\sld\isep
  \begin{itemize}
\sld\isep
   \item{
\sld
{\tt nodes} - The number of nodes to creaet}
  \end{itemize}
}%end item
\end{itemize}
}%end item
\divideents{GraphSpecification}
\item{\vskip -1.9ex 
\membername{GraphSpecification}
{\tt public {\bf GraphSpecification}( {\tt java.util.List } {\bf s},
{\tt java.util.List } {\bf ns},
{\tt uk.ac.ic.doc.neuralnets.util.Transformer } {\bf builder} )
\label{l248}\label{l249}}%end signature
\begin{itemize}
\sld
\item{
\sld
{\bf Usage}
  \begin{itemize}\isep
   \item{
Create a graph with the given node types and quantities, and use
 the supplied transformer to build edges
}%end item
  \end{itemize}
}
\item{
\sld
{\bf Parameters}
\sld\isep
  \begin{itemize}
\sld\isep
   \item{
\sld
{\tt s} - The list of node types (indices map to ns)}
   \item{
\sld
{\tt ns} - The list of quantities of node (indices map to s)}
   \item{
\sld
{\tt builder} - The edge building transformer}
  \end{itemize}
}%end item
\end{itemize}
}%end item
\divideents{GraphSpecification}
\item{\vskip -1.9ex 
\membername{GraphSpecification}
{\tt public {\bf GraphSpecification}( {\tt uk.ac.ic.doc.neuralnets.util.Transformer } {\bf builder} )
\label{l250}\label{l251}}%end signature
\begin{itemize}
\sld
\item{
\sld
{\bf Usage}
  \begin{itemize}\isep
   \item{
Create a default empty graph, with the supplied edge builder
}%end item
  \end{itemize}
}
\item{
\sld
{\bf Parameters}
\sld\isep
  \begin{itemize}
\sld\isep
   \item{
\sld
{\tt builder} - The edge builder to use to transform the graph}
  \end{itemize}
}%end item
\end{itemize}
}%end item
\end{itemize}
}
\startsubsubsection{Methods}{
\vskip -2em
\begin{itemize}
\item{\vskip -1.9ex 
\membername{getEdgeBuilder}
{\tt public Transformer {\bf getEdgeBuilder}(  )
\label{l252}\label{l253}}%end signature
\begin{itemize}
\sld
\item{
\sld
{\bf Usage}
  \begin{itemize}\isep
   \item{
Get the edge building transformer for this specification
}%end item
  \end{itemize}
}
\item{{\bf Returns} - 
A transformer used to build edges 
}%end item
\end{itemize}
}%end item
\divideents{getNodes}
\item{\vskip -1.9ex 
\membername{getNodes}
{\tt public List {\bf getNodes}(  )
\label{l254}\label{l255}}%end signature
\begin{itemize}
\sld
\item{
\sld
{\bf Usage}
  \begin{itemize}\isep
   \item{
Answer the quantities of nodes in this specification
}%end item
  \end{itemize}
}
\item{{\bf Returns} - 
The list of integer values. Modifications to this list are
 retained in the specification 
}%end item
\end{itemize}
}%end item
\divideents{getSpecifications}
\item{\vskip -1.9ex 
\membername{getSpecifications}
{\tt public List {\bf getSpecifications}(  )
\label{l256}\label{l257}}%end signature
\begin{itemize}
\sld
\item{
\sld
{\bf Usage}
  \begin{itemize}\isep
   \item{
Return the list of node types in this specification
}%end item
  \end{itemize}
}
\item{{\bf Returns} - 
The list of node types. Modifications to this list are
 retained in the specification 
}%end item
\end{itemize}
}%end item
\divideents{getTarget}
\item{\vskip -1.9ex 
\membername{getTarget}
{\tt public abstract Class {\bf getTarget}(  )
\label{l258}\label{l259}}%end signature
\begin{itemize}
\sld
\item{
\sld
{\bf Usage}
  \begin{itemize}\isep
   \item{
Stores the type of graph to create
}%end item
  \end{itemize}
}
\item{{\bf Returns} - 
The Class of the Graph encoded by this specification 
}%end item
\end{itemize}
}%end item
\divideents{separateNetworks}
\item{\vskip -1.9ex 
\membername{separateNetworks}
{\tt public abstract boolean {\bf separateNetworks}(  )
\label{l260}\label{l261}}%end signature
\begin{itemize}
\sld
\item{
\sld
{\bf Usage}
  \begin{itemize}\isep
   \item{
Answers whether or not the node types in this specification should be
 separated into their own sub-networks
}%end item
  \end{itemize}
}
\item{{\bf Returns} - 
True iff nodes are to be separated 
}%end item
\end{itemize}
}%end item
\end{itemize}
}
}
\startsection{Class}{HomogenousNetworkSpecification}{l204}{%
\startsubsubsection{Declaration}{
\fbox{\vbox{
\hbox{\vbox{\small public 
class 
HomogenousNetworkSpecification}}
\noindent\hbox{\vbox{{\bf extends} uk.ac.ic.doc.neuralnets.graph.neural.manipulation.GraphSpecification}}
}}}
\startsubsubsection{Constructors}{
\vskip -2em
\begin{itemize}
\item{\vskip -1.9ex 
\membername{HomogenousNetworkSpecification}
{\tt public {\bf HomogenousNetworkSpecification}( {\tt java.lang.Integer } {\bf nodes},
{\tt double } {\bf edgeProb} )
\label{l262}\label{l263}}%end signature
}%end item
\divideents{HomogenousNetworkSpecification}
\item{\vskip -1.9ex 
\membername{HomogenousNetworkSpecification}
{\tt public {\bf HomogenousNetworkSpecification}( {\tt java.util.List } {\bf nodes},
{\tt double } {\bf edgeProb} )
\label{l264}\label{l265}}%end signature
}%end item
\divideents{HomogenousNetworkSpecification}
\item{\vskip -1.9ex 
\membername{HomogenousNetworkSpecification}
{\tt public {\bf HomogenousNetworkSpecification}( {\tt java.util.List } {\bf specs},
{\tt java.util.List } {\bf nodes} )
\label{l266}\label{l267}}%end signature
}%end item
\divideents{HomogenousNetworkSpecification}
\item{\vskip -1.9ex 
\membername{HomogenousNetworkSpecification}
{\tt public {\bf HomogenousNetworkSpecification}( {\tt java.util.List } {\bf specs},
{\tt java.util.List } {\bf nodes},
{\tt double } {\bf edgeProb} )
\label{l268}\label{l269}}%end signature
}%end item
\divideents{HomogenousNetworkSpecification}
\item{\vskip -1.9ex 
\membername{HomogenousNetworkSpecification}
{\tt public {\bf HomogenousNetworkSpecification}( {\tt uk.ac.ic.doc.neuralnets.graph.neural.NodeSpecification } {\bf spec},
{\tt double } {\bf edgeProb} )
\label{l270}\label{l271}}%end signature
}%end item
\divideents{HomogenousNetworkSpecification}
\item{\vskip -1.9ex 
\membername{HomogenousNetworkSpecification}
{\tt public {\bf HomogenousNetworkSpecification}( {\tt uk.ac.ic.doc.neuralnets.graph.neural.NodeSpecification } {\bf spec},
{\tt java.lang.Integer } {\bf nodes} )
\label{l272}\label{l273}}%end signature
}%end item
\divideents{HomogenousNetworkSpecification}
\item{\vskip -1.9ex 
\membername{HomogenousNetworkSpecification}
{\tt public {\bf HomogenousNetworkSpecification}( {\tt uk.ac.ic.doc.neuralnets.graph.neural.NodeSpecification } {\bf spec},
{\tt java.lang.Integer } {\bf nodes},
{\tt double } {\bf edgeProb} )
\label{l274}\label{l275}}%end signature
}%end item
\end{itemize}
}
\startsubsubsection{Methods}{
\vskip -2em
\begin{itemize}
\item{\vskip -1.9ex 
\membername{getTarget}
{\tt public Class {\bf getTarget}(  )
\label{l276}\label{l277}}%end signature
}%end item
\divideents{separateNetworks}
\item{\vskip -1.9ex 
\membername{separateNetworks}
{\tt public boolean {\bf separateNetworks}(  )
\label{l278}\label{l279}}%end signature
}%end item
\end{itemize}
}
\startsubsubsection{Methods inherited from class {\tt uk.ac.ic.doc.neuralnets.graph.neural.manipulation.GraphSpecification}}{
\par{\small 
\refdefined{l203}\vskip -2em
\begin{itemize}
\item{\vskip -1.9ex 
\membername{getEdgeBuilder}
{\tt public Transformer {\bf getEdgeBuilder}(  )
}%end signature
\begin{itemize}
\sld
\item{
\sld
{\bf Usage}
  \begin{itemize}\isep
   \item{
Get the edge building transformer for this specification
}%end item
  \end{itemize}
}
\item{{\bf Returns} - 
A transformer used to build edges 
}%end item
\end{itemize}
}%end item
\divideents{getNodes}
\item{\vskip -1.9ex 
\membername{getNodes}
{\tt public List {\bf getNodes}(  )
}%end signature
\begin{itemize}
\sld
\item{
\sld
{\bf Usage}
  \begin{itemize}\isep
   \item{
Answer the quantities of nodes in this specification
}%end item
  \end{itemize}
}
\item{{\bf Returns} - 
The list of integer values. Modifications to this list are
 retained in the specification 
}%end item
\end{itemize}
}%end item
\divideents{getSpecifications}
\item{\vskip -1.9ex 
\membername{getSpecifications}
{\tt public List {\bf getSpecifications}(  )
}%end signature
\begin{itemize}
\sld
\item{
\sld
{\bf Usage}
  \begin{itemize}\isep
   \item{
Return the list of node types in this specification
}%end item
  \end{itemize}
}
\item{{\bf Returns} - 
The list of node types. Modifications to this list are
 retained in the specification 
}%end item
\end{itemize}
}%end item
\divideents{getTarget}
\item{\vskip -1.9ex 
\membername{getTarget}
{\tt public abstract Class {\bf getTarget}(  )
}%end signature
\begin{itemize}
\sld
\item{
\sld
{\bf Usage}
  \begin{itemize}\isep
   \item{
Stores the type of graph to create
}%end item
  \end{itemize}
}
\item{{\bf Returns} - 
The Class of the Graph encoded by this specification 
}%end item
\end{itemize}
}%end item
\divideents{separateNetworks}
\item{\vskip -1.9ex 
\membername{separateNetworks}
{\tt public abstract boolean {\bf separateNetworks}(  )
}%end signature
\begin{itemize}
\sld
\item{
\sld
{\bf Usage}
  \begin{itemize}\isep
   \item{
Answers whether or not the node types in this specification should be
 separated into their own sub-networks
}%end item
  \end{itemize}
}
\item{{\bf Returns} - 
True iff nodes are to be separated 
}%end item
\end{itemize}
}%end item
\end{itemize}
}}
}
\startsection{Class}{InhibitoryNodeSpecification}{l205}{%
{\small Default NodeSpecification for Inhibitory Spiking neurones.}
\vskip .1in 
\startsubsubsection{Declaration}{
\fbox{\vbox{
\hbox{\vbox{\small public 
class 
InhibitoryNodeSpecification}}
\noindent\hbox{\vbox{{\bf extends} uk.ac.ic.doc.neuralnets.graph.neural.manipulation.SpikingNodeSpecification}}
}}}
\startsubsubsection{Constructors}{
\vskip -2em
\begin{itemize}
\item{\vskip -1.9ex 
\membername{InhibitoryNodeSpecification}
{\tt public {\bf InhibitoryNodeSpecification}(  )
\label{l280}\label{l281}}%end signature
\begin{itemize}
\sld
\item{
\sld
{\bf Usage}
  \begin{itemize}\isep
   \item{
Creates a inhibitory spiking neurone specification with default 
 parameters according to Izhikevich's model.
 
 \begin{itemize}

 	\item[Squash Function\textless $/$dt\textgreater ]{-1\textless $/$dd\textgreater 
  }
\item[Trigger\textless $/$dt\textgreater ]{30\textless $/$dd\textgreater 
  }
\item[Initial Charge\textless $/$dt\textgreater ]{-65\textless $/$dd\textgreater 
  }
\item[Recovery Scale\textless $/$dt\textgreater ]{0.02 + 0.08 * RAND()\textless $/$dd\textgreater 
  }
\item[Recovery Sensitivity\textless $/$dt\textgreater ]{0.25 - 0.05 * RAND()\textless $/$dd\textgreater 
  }
\item[Post Spike Reset\textless $/$dt\textgreater ]{-65\textless $/$dd\textgreater 
  }
\item[PSRRecovery\textless $/$dt\textgreater ]{2\textless $/$dd\textgreater 
  }
\item[Thalamic Input\textless $/$dt\textgreater ]{2 * GRAND()\textless $/$dd\textgreater 
  }
\item[Synaptic Delay\textless $/$dt\textgreater ]{20 * RAND()\textless $/$dd\textgreater 
 }
\end{itemize}
  where RAND() is a  uniformly distributed random number between 0 and 1 
  and GRAND() is a Gaussian distributed random number.
}%end item
  \end{itemize}
}
\item{{\bf See Also}
  \begin{itemize}
   \item{{\tt java.util.Random} {\small 
\refdefined{l282}}%end \small
}%end item
  \end{itemize}
}%end item
\end{itemize}
}%end item
\end{itemize}
}
\startsubsubsection{Methods inherited from class {\tt uk.ac.ic.doc.neuralnets.graph.neural.manipulation.SpikingNodeSpecification}}{
\par{\small 
\refdefined{l211}}}
\startsubsubsection{Methods inherited from class {\tt uk.ac.ic.doc.neuralnets.graph.neural.NodeSpecification}}{
\par{\small 
\refdefined{l283}\vskip -2em
\begin{itemize}
\item{\vskip -1.9ex 
\membername{get}
{\tt public ASTExpression {\bf get}( {\tt java.lang.String } {\bf param} )
}%end signature
\begin{itemize}
\sld
\item{
\sld
{\bf Usage}
  \begin{itemize}\isep
   \item{
Get the AST expression for input parameter.
}%end item
  \end{itemize}
}
\item{
\sld
{\bf Parameters}
\sld\isep
  \begin{itemize}
\sld\isep
   \item{
\sld
{\tt param} - String}
  \end{itemize}
}%end item
\item{{\bf Returns} - 
AST expression 
}%end item
\end{itemize}
}%end item
\divideents{getEdgeDecoration}
\item{\vskip -1.9ex 
\membername{getEdgeDecoration}
{\tt public EdgeDecoration {\bf getEdgeDecoration}(  )
}%end signature
\begin{itemize}
\sld
\item{
\sld
{\bf Usage}
  \begin{itemize}\isep
   \item{
Get the edge decoration for the node specification.
}%end item
  \end{itemize}
}
\item{{\bf Returns} - 
The edge decoration. 
}%end item
\end{itemize}
}%end item
\divideents{getName}
\item{\vskip -1.9ex 
\membername{getName}
{\tt public String {\bf getName}(  )
}%end signature
\begin{itemize}
\sld
\item{
\sld
{\bf Usage}
  \begin{itemize}\isep
   \item{
Get the name of the node specification.
}%end item
  \end{itemize}
}
\item{{\bf Returns} - 
The name. 
}%end item
\end{itemize}
}%end item
\divideents{getParameters}
\item{\vskip -1.9ex 
\membername{getParameters}
{\tt public Set {\bf getParameters}(  )
}%end signature
\begin{itemize}
\sld
\item{
\sld
{\bf Usage}
  \begin{itemize}\isep
   \item{
Get the parameter key set.
}%end item
  \end{itemize}
}
\item{{\bf Returns} - 
Parameter key set. 
}%end item
\end{itemize}
}%end item
\divideents{getTarget}
\item{\vskip -1.9ex 
\membername{getTarget}
{\tt public Class {\bf getTarget}(  )
}%end signature
\begin{itemize}
\sld
\item{
\sld
{\bf Usage}
  \begin{itemize}\isep
   \item{
Get target of node specification.
}%end item
  \end{itemize}
}
\item{{\bf Returns} - 
Target 
}%end item
\end{itemize}
}%end item
\divideents{set}
\item{\vskip -1.9ex 
\membername{set}
{\tt public NodeSpecification {\bf set}( {\tt java.lang.String } {\bf param},
{\tt uk.ac.ic.doc.neuralnets.expressions.ast.ASTExpression } {\bf target} )
}%end signature
\begin{itemize}
\sld
\item{
\sld
{\bf Usage}
  \begin{itemize}\isep
   \item{
Set a parameter to an AST expresion.
}%end item
  \end{itemize}
}
\item{
\sld
{\bf Parameters}
\sld\isep
  \begin{itemize}
\sld\isep
   \item{
\sld
{\tt param} - Parameter name}
   \item{
\sld
{\tt target} - AST expression value.}
  \end{itemize}
}%end item
\item{{\bf Returns} - 
Itself. 
}%end item
\end{itemize}
}%end item
\divideents{setEdgeDecoration}
\item{\vskip -1.9ex 
\membername{setEdgeDecoration}
{\tt public void {\bf setEdgeDecoration}( {\tt uk.ac.ic.doc.neuralnets.graph.neural.EdgeDecoration } {\bf ed} )
}%end signature
\begin{itemize}
\sld
\item{
\sld
{\bf Usage}
  \begin{itemize}\isep
   \item{
Set the edge decorator for the node specification.
}%end item
  \end{itemize}
}
\item{
\sld
{\bf Parameters}
\sld\isep
  \begin{itemize}
\sld\isep
   \item{
\sld
{\tt ed} - The edge decoration.}
  \end{itemize}
}%end item
\end{itemize}
}%end item
\divideents{setName}
\item{\vskip -1.9ex 
\membername{setName}
{\tt public void {\bf setName}( {\tt java.lang.String } {\bf n} )
}%end signature
\begin{itemize}
\sld
\item{
\sld
{\bf Usage}
  \begin{itemize}\isep
   \item{
Set name of node specification.
}%end item
  \end{itemize}
}
\item{
\sld
{\bf Parameters}
\sld\isep
  \begin{itemize}
\sld\isep
   \item{
\sld
{\tt n} - Name}
  \end{itemize}
}%end item
\end{itemize}
}%end item
\end{itemize}
}}
}
\startsection{Class}{InteractionUtils}{l206}{%
\startsubsubsection{Declaration}{
\fbox{\vbox{
\hbox{\vbox{\small public 
class 
InteractionUtils}}
\noindent\hbox{\vbox{{\bf extends} java.lang.Object}}
}}}
\startsubsubsection{Constructors}{
\vskip -2em
\begin{itemize}
\item{\vskip -1.9ex 
\membername{InteractionUtils}
{\tt public {\bf InteractionUtils}( {\tt uk.ac.ic.doc.neuralnets.graph.neural.NeuralNetwork } {\bf n} )
\label{l284}\label{l285}}%end signature
\begin{itemize}
\sld
\item{
\sld
{\bf Parameters}
\sld\isep
  \begin{itemize}
\sld\isep
   \item{
\sld
{\tt n} - The NeuralNetwork to operate over}
  \end{itemize}
}%end item
\end{itemize}
}%end item
\end{itemize}
}
\startsubsubsection{Methods}{
\vskip -2em
\begin{itemize}
\item{\vskip -1.9ex 
\membername{bifurcate}
{\tt public NeuralNetwork {\bf bifurcate}( {\tt uk.ac.ic.doc.neuralnets.graph.neural.NeuralNetwork } {\bf n},
{\tt uk.ac.ic.doc.neuralnets.util.Transformer } {\bf knife} )
\label{l286}\label{l287}}%end signature
\begin{itemize}
\sld
\item{
\sld
{\bf Usage}
  \begin{itemize}\isep
   \item{
Extract the nodes from n that are selected by the knife, removing them
 from the network and instead creating a new network.
 
 Any edges in n that are into or out of knife are instead routed via a
 NetworkBridge.
 
 The resultant network is added to the parent network of n automatically.
}%end item
  \end{itemize}
}
\item{
\sld
{\bf Parameters}
\sld\isep
  \begin{itemize}
\sld\isep
   \item{
\sld
{\tt n} - The network to bifurcate}
   \item{
\sld
{\tt knife} - A transformer to select the nodes to remove}
  \end{itemize}
}%end item
\item{{\bf Returns} - 
The resultant (new) bifurcated network 
}%end item
\end{itemize}
}%end item
\divideents{connect}
\item{\vskip -1.9ex 
\membername{connect}
{\tt public Collection {\bf connect}( {\tt java.util.Collection } {\bf f},
{\tt java.util.Collection } {\bf t} )
\label{l288}\label{l289}}%end signature
\begin{itemize}
\sld
\item{
\sld
{\bf Usage}
  \begin{itemize}\isep
   \item{
Fully connect the given sets of nodes in the network
}%end item
  \end{itemize}
}
\item{
\sld
{\bf Parameters}
\sld\isep
  \begin{itemize}
\sld\isep
   \item{
\sld
{\tt f} - The source node}
   \item{
\sld
{\tt t} - The target node}
  \end{itemize}
}%end item
\item{{\bf Returns} - 
The collection of created edges 
}%end item
\end{itemize}
}%end item
\divideents{connect}
\item{\vskip -1.9ex 
\membername{connect}
{\tt public Collection {\bf connect}( {\tt java.util.Collection } {\bf f},
{\tt java.util.Collection } {\bf t},
{\tt double } {\bf edgeProb} )
\label{l290}\label{l291}}%end signature
\begin{itemize}
\sld
\item{
\sld
{\bf Usage}
  \begin{itemize}\isep
   \item{
Connect the given sets of nodes in the network with the chosen 
 probability of edge creation
}%end item
  \end{itemize}
}
\item{
\sld
{\bf Parameters}
\sld\isep
  \begin{itemize}
\sld\isep
   \item{
\sld
{\tt f} - The source node}
   \item{
\sld
{\tt t} - The target node}
   \item{
\sld
{\tt edgeProb} - The probability a given edge is created}
  \end{itemize}
}%end item
\item{{\bf Returns} - 
The collection of created edges 
}%end item
\end{itemize}
}%end item
\divideents{connect}
\item{\vskip -1.9ex 
\membername{connect}
{\tt public Edge {\bf connect}( {\tt uk.ac.ic.doc.neuralnets.graph.Node } {\bf f},
{\tt uk.ac.ic.doc.neuralnets.graph.Node } {\bf t} )
\label{l292}\label{l293}}%end signature
\begin{itemize}
\sld
\item{
\sld
{\bf Usage}
  \begin{itemize}\isep
   \item{
Connect the given nodes in any networks. If the network of f is the
 same as the network of t, return a synpase in that network. Otherwise,
 create a bridge from network of f to network of t, and route a synapse
 through its bundle. If network of f is a super-node of the network of 
 t, then bridges are still created. Bridges and synapses are always
 re-used where possible.
 
 Given a network with two sub-networks, n1 and n2, and n2 containing n3,
 a synapse from a neurone in n1 to a neurone in n3 most route over a
 network bridge to n2, then a network bridge from n2 to n3, and finally
 act as a synapse from n3's input to the synapse.
 
 Connecting a network to its parent results in a null connection, as it
 is not necessary.
}%end item
  \end{itemize}
}
\item{
\sld
{\bf Parameters}
\sld\isep
  \begin{itemize}
\sld\isep
   \item{
\sld
{\tt f} - The node to connect from}
   \item{
\sld
{\tt t} - The node to connect to}
  \end{itemize}
}%end item
\item{{\bf Returns} - 
The edge that connects these nodes, or null if no such
 connection is possible 
}%end item
\end{itemize}
}%end item
\divideents{connect1to1}
\item{\vskip -1.9ex 
\membername{connect1to1}
{\tt public Collection {\bf connect1to1}( {\tt java.util.Collection } {\bf f},
{\tt java.util.Collection } {\bf t} )
\label{l294}\label{l295}}%end signature
\begin{itemize}
\sld
\item{
\sld
{\bf Usage}
  \begin{itemize}\isep
   \item{
Connect the given sets of nodes in the network with a 1-1 connection
 mapping (i.e. each node in f connects to one node in t) to as great an
 extent as possible. If there are insufficient nodes in t, some may be
 re-used
}%end item
  \end{itemize}
}
\item{
\sld
{\bf Parameters}
\sld\isep
  \begin{itemize}
\sld\isep
   \item{
\sld
{\tt f} - The source node}
   \item{
\sld
{\tt t} - The target node}
  \end{itemize}
}%end item
\item{{\bf Returns} - 
The collection of created edges 
}%end item
\end{itemize}
}%end item
\divideents{createNodes}
\item{\vskip -1.9ex 
\membername{createNodes}
{\tt public NeuralNetwork {\bf createNodes}( {\tt uk.ac.ic.doc.neuralnets.graph.neural.manipulation.GraphSpecification } {\bf spec} )
\label{l296}\label{l297}}%end signature
\begin{itemize}
\sld
\item{
\sld
{\bf Usage}
  \begin{itemize}\isep
   \item{
Create some nodes in the network
}%end item
  \end{itemize}
}
\item{
\sld
{\bf Parameters}
\sld\isep
  \begin{itemize}
\sld\isep
   \item{
\sld
{\tt spec} - The specification of how to add nodes and edges}
  \end{itemize}
}%end item
\item{{\bf Returns} - 
The nodes added, as a new network 
}%end item
\end{itemize}
}%end item
\divideents{createNodes}
\item{\vskip -1.9ex 
\membername{createNodes}
{\tt public NeuralNetwork {\bf createNodes}( {\tt int } {\bf nodes},
{\tt double } {\bf edgeProb} )
\label{l298}\label{l299}}%end signature
\begin{itemize}
\sld
\item{
\sld
{\bf Usage}
  \begin{itemize}\isep
   \item{
Create some nodes in the network
}%end item
  \end{itemize}
}
\item{
\sld
{\bf Parameters}
\sld\isep
  \begin{itemize}
\sld\isep
   \item{
\sld
{\tt nodes} - The number of nodes to create}
   \item{
\sld
{\tt edgeProb} - The probability a given edge should be made}
  \end{itemize}
}%end item
\item{{\bf Returns} - 
The nodes added, as a new network 
}%end item
\end{itemize}
}%end item
\divideents{findNetwork}
\item{\vskip -1.9ex 
\membername{findNetwork}
{\tt public NeuralNetwork {\bf findNetwork}( {\tt uk.ac.ic.doc.neuralnets.graph.Node } {\bf n} )
\label{l300}\label{l301}}%end signature
\begin{itemize}
\sld
\item{
\sld
{\bf Usage}
  \begin{itemize}\isep
   \item{
Find the network which contains the given node. NB: Our semantics of
 containment dictate that the root network is contained by itself.
}%end item
  \end{itemize}
}
\item{
\sld
{\bf Parameters}
\sld\isep
  \begin{itemize}
\sld\isep
   \item{
\sld
{\tt n} - The node to seek}
  \end{itemize}
}%end item
\item{{\bf Returns} - 
The NeuralNetwork that contains it, or null if such could not
 be found 
}%end item
\end{itemize}
}%end item
\divideents{getNetwork}
\item{\vskip -1.9ex 
\membername{getNetwork}
{\tt public NeuralNetwork {\bf getNetwork}(  )
\label{l302}\label{l303}}%end signature
\begin{itemize}
\sld
\item{{\bf Returns} - 
The NeuralNetwork that backs these utils 
}%end item
\end{itemize}
}%end item
\divideents{isSuper}
\item{\vskip -1.9ex 
\membername{isSuper}
{\tt public boolean {\bf isSuper}( {\tt uk.ac.ic.doc.neuralnets.graph.neural.NeuralNetwork } {\bf a},
{\tt uk.ac.ic.doc.neuralnets.graph.neural.NeuralNetwork } {\bf b} )
\label{l304}\label{l305}}%end signature
\begin{itemize}
\sld
\item{
\sld
{\bf Usage}
  \begin{itemize}\isep
   \item{
Answers whether network a is a parent of network b
}%end item
  \end{itemize}
}
\item{
\sld
{\bf Parameters}
\sld\isep
  \begin{itemize}
\sld\isep
   \item{
\sld
{\tt a} - The parent node to test}
   \item{
\sld
{\tt b} - The child node to seek}
  \end{itemize}
}%end item
\item{{\bf Returns} - 
true iff a is a parent of b 
}%end item
\end{itemize}
}%end item
\divideents{isSuper}
\item{\vskip -1.9ex 
\membername{isSuper}
{\tt public boolean {\bf isSuper}( {\tt uk.ac.ic.doc.neuralnets.graph.Node } {\bf a},
{\tt uk.ac.ic.doc.neuralnets.graph.Node } {\bf b} )
\label{l306}\label{l307}}%end signature
\begin{itemize}
\sld
\item{
\sld
{\bf Usage}
  \begin{itemize}\isep
   \item{
Answers whether Node a is a super-node of node b (i.e. a parent)
}%end item
  \end{itemize}
}
\item{
\sld
{\bf Parameters}
\sld\isep
  \begin{itemize}
\sld\isep
   \item{
\sld
{\tt a} - The parent node to test}
   \item{
\sld
{\tt b} - The child node to seek}
  \end{itemize}
}%end item
\item{{\bf Returns} - 
true iff a is a parent of b 
}%end item
\end{itemize}
}%end item
\divideents{lowestCommonAncestor}
\item{\vskip -1.9ex 
\membername{lowestCommonAncestor}
{\tt public NeuralNetwork {\bf lowestCommonAncestor}( {\tt uk.ac.ic.doc.neuralnets.graph.Node } {\bf a},
{\tt uk.ac.ic.doc.neuralnets.graph.Node } {\bf b} )
\label{l308}\label{l309}}%end signature
\begin{itemize}
\sld
\item{
\sld
{\bf Usage}
  \begin{itemize}\isep
   \item{
Find the lowest common ancestor of Nodes a and b; i.e. the deepest
 NeuralNetwork in the tree of networks that contains both a and b.
 
 Algorithm:
 Iterate up the parents of a and b until an intersection in the sets
 of their ancestors is found; at that point, we have th lowest common
 ancestor and can return
}%end item
  \end{itemize}
}
\item{
\sld
{\bf Parameters}
\sld\isep
  \begin{itemize}
\sld\isep
   \item{
\sld
{\tt a} - The first node to seek}
   \item{
\sld
{\tt b} - The second node to seek}
  \end{itemize}
}%end item
\item{{\bf Returns} - 
The lowest common ancestor of a and b, or null if it could not
 be found (in a correct network, this shouldn't be possible) 
}%end item
\end{itemize}
}%end item
\divideents{pauseNetwork}
\item{\vskip -1.9ex 
\membername{pauseNetwork}
{\tt public void {\bf pauseNetwork}(  )
\label{l310}\label{l311}}%end signature
\begin{itemize}
\sld
\item{
\sld
{\bf Usage}
  \begin{itemize}\isep
   \item{
Pause the network from running
}%end item
  \end{itemize}
}
\end{itemize}
}%end item
\divideents{prettyPrintNetwork}
\item{\vskip -1.9ex 
\membername{prettyPrintNetwork}
{\tt public void {\bf prettyPrintNetwork}( {\tt java.io.PrintStream } {\bf out} )
\label{l312}\label{l313}}%end signature
\begin{itemize}
\sld
\item{
\sld
{\bf Usage}
  \begin{itemize}\isep
   \item{
Print out the network to the given PrintStream
}%end item
  \end{itemize}
}
\item{
\sld
{\bf Parameters}
\sld\isep
  \begin{itemize}
\sld\isep
   \item{
\sld
{\tt out} - The PrintStream to which to print}
  \end{itemize}
}%end item
\end{itemize}
}%end item
\divideents{resetNetwork}
\item{\vskip -1.9ex 
\membername{resetNetwork}
{\tt public void {\bf resetNetwork}(  )
\label{l314}\label{l315}}%end signature
}%end item
\divideents{runNetwork}
\item{\vskip -1.9ex 
\membername{runNetwork}
{\tt public void {\bf runNetwork}(  )
\label{l316}\label{l317}}%end signature
\begin{itemize}
\sld
\item{
\sld
{\bf Usage}
  \begin{itemize}\isep
   \item{
Run the network from the last tick state (i.e. resume)
}%end item
  \end{itemize}
}
\end{itemize}
}%end item
\divideents{runNetwork}
\item{\vskip -1.9ex 
\membername{runNetwork}
{\tt public void {\bf runNetwork}( {\tt int } {\bf ticks} )
\label{l318}\label{l319}}%end signature
\begin{itemize}
\sld
\item{
\sld
{\bf Usage}
  \begin{itemize}\isep
   \item{
Run the network for the given number of ticks
}%end item
  \end{itemize}
}
\item{
\sld
{\bf Parameters}
\sld\isep
  \begin{itemize}
\sld\isep
   \item{
\sld
{\tt ticks} - How long to run for, or \textless  0 for "forever"}
  \end{itemize}
}%end item
\end{itemize}
}%end item
\divideents{setNetwork}
\item{\vskip -1.9ex 
\membername{setNetwork}
{\tt public void {\bf setNetwork}( {\tt uk.ac.ic.doc.neuralnets.graph.neural.NeuralNetwork } {\bf n} )
\label{l320}\label{l321}}%end signature
\begin{itemize}
\sld
\item{
\sld
{\bf Parameters}
\sld\isep
  \begin{itemize}
\sld\isep
   \item{
\sld
{\tt n} - The NeuralNetwork to operate over}
  \end{itemize}
}%end item
\end{itemize}
}%end item
\divideents{teardown}
\item{\vskip -1.9ex 
\membername{teardown}
{\tt public void {\bf teardown}(  )
\label{l322}\label{l323}}%end signature
\begin{itemize}
\sld
\item{
\sld
{\bf Usage}
  \begin{itemize}\isep
   \item{
Cause this instance to stop any threads it may have spawned, and release
 its resources. Any further operations have undefined behaviour.
}%end item
  \end{itemize}
}
\end{itemize}
}%end item
\end{itemize}
}
}
\startsection{Class}{InteractionUtils.NetworkRunner}{l207}{%
{\small The thread used to run the network asynchronously with the UI}
\vskip .1in 
\startsubsubsection{Declaration}{
\fbox{\vbox{
\hbox{\vbox{\small protected 
class 
InteractionUtils.NetworkRunner}}
\noindent\hbox{\vbox{{\bf extends} java.lang.Thread}}
}}}
\startsubsubsection{Constructors}{
\vskip -2em
\begin{itemize}
\item{\vskip -1.9ex 
\membername{InteractionUtils.NetworkRunner}
{\tt protected {\bf InteractionUtils.NetworkRunner}(  )
\label{l324}\label{l325}}%end signature
}%end item
\end{itemize}
}
\startsubsubsection{Methods}{
\vskip -2em
\begin{itemize}
\item{\vskip -1.9ex 
\membername{getRemainingTicks}
{\tt public int {\bf getRemainingTicks}(  )
\label{l326}\label{l327}}%end signature
}%end item
\divideents{kill}
\item{\vskip -1.9ex 
\membername{kill}
{\tt public void {\bf kill}(  )
\label{l328}\label{l329}}%end signature
}%end item
\divideents{pauseNetwork}
\item{\vskip -1.9ex 
\membername{pauseNetwork}
{\tt public void {\bf pauseNetwork}(  )
\label{l330}\label{l331}}%end signature
}%end item
\divideents{run}
\item{\vskip -1.9ex 
\membername{run}
{\tt public void {\bf run}(  )
\label{l332}\label{l333}}%end signature
}%end item
\divideents{runNetwork}
\item{\vskip -1.9ex 
\membername{runNetwork}
{\tt public void {\bf runNetwork}(  )
\label{l334}\label{l335}}%end signature
}%end item
\divideents{runNetwork}
\item{\vskip -1.9ex 
\membername{runNetwork}
{\tt public void {\bf runNetwork}( {\tt int } {\bf ticks} )
\label{l336}\label{l337}}%end signature
}%end item
\divideents{setTicks}
\item{\vskip -1.9ex 
\membername{setTicks}
{\tt public void {\bf setTicks}( {\tt int } {\bf ticks} )
\label{l338}\label{l339}}%end signature
}%end item
\end{itemize}
}
\startsubsubsection{Methods inherited from class {\tt java.lang.Thread}}{
\par{\small 
\refdefined{l340}\vskip -2em
\begin{itemize}
\item{\vskip -1.9ex 
\membername{activeCount}
{\tt public static int {\bf activeCount}(  )
}%end signature
}%end item
\divideents{checkAccess}
\item{\vskip -1.9ex 
\membername{checkAccess}
{\tt public final void {\bf checkAccess}(  )
}%end signature
}%end item
\divideents{countStackFrames}
\item{\vskip -1.9ex 
\membername{countStackFrames}
{\tt public native int {\bf countStackFrames}(  )
}%end signature
}%end item
\divideents{currentThread}
\item{\vskip -1.9ex 
\membername{currentThread}
{\tt public static native Thread {\bf currentThread}(  )
}%end signature
}%end item
\divideents{destroy}
\item{\vskip -1.9ex 
\membername{destroy}
{\tt public void {\bf destroy}(  )
}%end signature
}%end item
\divideents{dumpStack}
\item{\vskip -1.9ex 
\membername{dumpStack}
{\tt public static void {\bf dumpStack}(  )
}%end signature
}%end item
\divideents{enumerate}
\item{\vskip -1.9ex 
\membername{enumerate}
{\tt public static int {\bf enumerate}( {\tt java.lang.Thread []} {\bf arg0} )
}%end signature
}%end item
\divideents{getAllStackTraces}
\item{\vskip -1.9ex 
\membername{getAllStackTraces}
{\tt public static Map {\bf getAllStackTraces}(  )
}%end signature
}%end item
\divideents{getContextClassLoader}
\item{\vskip -1.9ex 
\membername{getContextClassLoader}
{\tt public ClassLoader {\bf getContextClassLoader}(  )
}%end signature
}%end item
\divideents{getDefaultUncaughtExceptionHandler}
\item{\vskip -1.9ex 
\membername{getDefaultUncaughtExceptionHandler}
{\tt public static Thread.UncaughtExceptionHandler {\bf getDefaultUncaughtExceptionHandler}(  )
}%end signature
}%end item
\divideents{getId}
\item{\vskip -1.9ex 
\membername{getId}
{\tt public long {\bf getId}(  )
}%end signature
}%end item
\divideents{getName}
\item{\vskip -1.9ex 
\membername{getName}
{\tt public final String {\bf getName}(  )
}%end signature
}%end item
\divideents{getPriority}
\item{\vskip -1.9ex 
\membername{getPriority}
{\tt public final int {\bf getPriority}(  )
}%end signature
}%end item
\divideents{getStackTrace}
\item{\vskip -1.9ex 
\membername{getStackTrace}
{\tt public StackTraceElement {\bf getStackTrace}(  )
}%end signature
}%end item
\divideents{getState}
\item{\vskip -1.9ex 
\membername{getState}
{\tt public Thread.State {\bf getState}(  )
}%end signature
}%end item
\divideents{getThreadGroup}
\item{\vskip -1.9ex 
\membername{getThreadGroup}
{\tt public final ThreadGroup {\bf getThreadGroup}(  )
}%end signature
}%end item
\divideents{getUncaughtExceptionHandler}
\item{\vskip -1.9ex 
\membername{getUncaughtExceptionHandler}
{\tt public Thread.UncaughtExceptionHandler {\bf getUncaughtExceptionHandler}(  )
}%end signature
}%end item
\divideents{holdsLock}
\item{\vskip -1.9ex 
\membername{holdsLock}
{\tt public static native boolean {\bf holdsLock}( {\tt java.lang.Object } {\bf arg0} )
}%end signature
}%end item
\divideents{interrupt}
\item{\vskip -1.9ex 
\membername{interrupt}
{\tt public void {\bf interrupt}(  )
}%end signature
}%end item
\divideents{interrupted}
\item{\vskip -1.9ex 
\membername{interrupted}
{\tt public static boolean {\bf interrupted}(  )
}%end signature
}%end item
\divideents{isAlive}
\item{\vskip -1.9ex 
\membername{isAlive}
{\tt public final native boolean {\bf isAlive}(  )
}%end signature
}%end item
\divideents{isDaemon}
\item{\vskip -1.9ex 
\membername{isDaemon}
{\tt public final boolean {\bf isDaemon}(  )
}%end signature
}%end item
\divideents{isInterrupted}
\item{\vskip -1.9ex 
\membername{isInterrupted}
{\tt public boolean {\bf isInterrupted}(  )
}%end signature
}%end item
\divideents{join}
\item{\vskip -1.9ex 
\membername{join}
{\tt public final void {\bf join}(  )
}%end signature
}%end item
\divideents{join}
\item{\vskip -1.9ex 
\membername{join}
{\tt public final synchronized void {\bf join}( {\tt long } {\bf arg0} )
}%end signature
}%end item
\divideents{join}
\item{\vskip -1.9ex 
\membername{join}
{\tt public final synchronized void {\bf join}( {\tt long } {\bf arg0},
{\tt int } {\bf arg1} )
}%end signature
}%end item
\divideents{resume}
\item{\vskip -1.9ex 
\membername{resume}
{\tt public final void {\bf resume}(  )
}%end signature
}%end item
\divideents{run}
\item{\vskip -1.9ex 
\membername{run}
{\tt public void {\bf run}(  )
}%end signature
}%end item
\divideents{setContextClassLoader}
\item{\vskip -1.9ex 
\membername{setContextClassLoader}
{\tt public void {\bf setContextClassLoader}( {\tt java.lang.ClassLoader } {\bf arg0} )
}%end signature
}%end item
\divideents{setDaemon}
\item{\vskip -1.9ex 
\membername{setDaemon}
{\tt public final void {\bf setDaemon}( {\tt boolean } {\bf arg0} )
}%end signature
}%end item
\divideents{setDefaultUncaughtExceptionHandler}
\item{\vskip -1.9ex 
\membername{setDefaultUncaughtExceptionHandler}
{\tt public static void {\bf setDefaultUncaughtExceptionHandler}( {\tt java.lang.Thread.UncaughtExceptionHandler } {\bf arg0} )
}%end signature
}%end item
\divideents{setName}
\item{\vskip -1.9ex 
\membername{setName}
{\tt public final void {\bf setName}( {\tt java.lang.String } {\bf arg0} )
}%end signature
}%end item
\divideents{setPriority}
\item{\vskip -1.9ex 
\membername{setPriority}
{\tt public final void {\bf setPriority}( {\tt int } {\bf arg0} )
}%end signature
}%end item
\divideents{setUncaughtExceptionHandler}
\item{\vskip -1.9ex 
\membername{setUncaughtExceptionHandler}
{\tt public void {\bf setUncaughtExceptionHandler}( {\tt java.lang.Thread.UncaughtExceptionHandler } {\bf arg0} )
}%end signature
}%end item
\divideents{sleep}
\item{\vskip -1.9ex 
\membername{sleep}
{\tt public static native void {\bf sleep}( {\tt long } {\bf arg0} )
}%end signature
}%end item
\divideents{sleep}
\item{\vskip -1.9ex 
\membername{sleep}
{\tt public static void {\bf sleep}( {\tt long } {\bf arg0},
{\tt int } {\bf arg1} )
}%end signature
}%end item
\divideents{start}
\item{\vskip -1.9ex 
\membername{start}
{\tt public synchronized void {\bf start}(  )
}%end signature
}%end item
\divideents{stop}
\item{\vskip -1.9ex 
\membername{stop}
{\tt public final void {\bf stop}(  )
}%end signature
}%end item
\divideents{stop}
\item{\vskip -1.9ex 
\membername{stop}
{\tt public final synchronized void {\bf stop}( {\tt java.lang.Throwable } {\bf arg0} )
}%end signature
}%end item
\divideents{suspend}
\item{\vskip -1.9ex 
\membername{suspend}
{\tt public final void {\bf suspend}(  )
}%end signature
}%end item
\divideents{toString}
\item{\vskip -1.9ex 
\membername{toString}
{\tt public String {\bf toString}(  )
}%end signature
}%end item
\divideents{yield}
\item{\vskip -1.9ex 
\membername{yield}
{\tt public static native void {\bf yield}(  )
}%end signature
}%end item
\end{itemize}
}}
}
\startsection{Class}{NodeCreatedEvent}{l208}{%
{\small Indicates a node has been created by the factory}
\vskip .1in 
\startsubsubsection{Declaration}{
\fbox{\vbox{
\hbox{\vbox{\small public 
class 
NodeCreatedEvent}}
\noindent\hbox{\vbox{{\bf extends} uk.ac.ic.doc.neuralnets.events.Event}}
}}}
\startsubsubsection{Constructors}{
\vskip -2em
\begin{itemize}
\item{\vskip -1.9ex 
\membername{NodeCreatedEvent}
{\tt public {\bf NodeCreatedEvent}( {\tt int } {\bf num},
{\tt int } {\bf count} )
\label{l341}\label{l342}}%end signature
}%end item
\end{itemize}
}
\startsubsubsection{Methods}{
\vskip -2em
\begin{itemize}
\item{\vskip -1.9ex 
\membername{getNodeCount}
{\tt public int {\bf getNodeCount}(  )
\label{l343}\label{l344}}%end signature
\begin{itemize}
\sld
\item{
\sld
{\bf Usage}
  \begin{itemize}\isep
   \item{
Get the number of nodes that need to be created
}%end item
  \end{itemize}
}
\item{{\bf Returns} - 
The maximum number of nodes to be created 
}%end item
\end{itemize}
}%end item
\divideents{getNodeNumber}
\item{\vskip -1.9ex 
\membername{getNodeNumber}
{\tt public int {\bf getNodeNumber}(  )
\label{l345}\label{l346}}%end signature
\begin{itemize}
\sld
\item{
\sld
{\bf Usage}
  \begin{itemize}\isep
   \item{
Get the number of nodes created so far
}%end item
  \end{itemize}
}
\item{{\bf Returns} - 
The quantity of nodes thus far created 
}%end item
\end{itemize}
}%end item
\divideents{toString}
\item{\vskip -1.9ex 
\membername{toString}
{\tt public String {\bf toString}(  )
\label{l347}\label{l348}}%end signature
}%end item
\end{itemize}
}
\startsubsubsection{Methods inherited from class {\tt uk.ac.ic.doc.neuralnets.events.Event}}{
\par{\small 
\refdefined{l220}\vskip -2em
\begin{itemize}
\item{\vskip -1.9ex 
\membername{toString}
{\tt public abstract String {\bf toString}(  )
}%end signature
}%end item
\end{itemize}
}}
}
\startsection{Class}{NodeFactory}{l209}{%
{\small NodeFactory creates Node objects from NodeSpecifications.}
\vskip .1in 
\startsubsubsection{Declaration}{
\fbox{\vbox{
\hbox{\vbox{\small public 
class 
NodeFactory}}
\noindent\hbox{\vbox{{\bf extends} java.lang.Object}}
\noindent\hbox{\vbox{{\bf implements} 
java.io.Serializable}}
}}}
\startsubsubsection{Constructors}{
\vskip -2em
\begin{itemize}
\item{\vskip -1.9ex 
\membername{NodeFactory}
{\tt public {\bf NodeFactory}(  )
\label{l349}\label{l350}}%end signature
}%end item
\end{itemize}
}
\startsubsubsection{Methods}{
\vskip -2em
\begin{itemize}
\item{\vskip -1.9ex 
\membername{create}
{\tt public Neurone {\bf create}(  )
\label{l351}\label{l352}}%end signature
\begin{itemize}
\sld
\item{
\sld
{\bf Usage}
  \begin{itemize}\isep
   \item{
Create a default neurone
}%end item
  \end{itemize}
}
\item{{\bf Returns} - 
a neurone with default spiking neurone parameters. 
}%end item
\end{itemize}
}%end item
\divideents{create}
\item{\vskip -1.9ex 
\membername{create}
{\tt public Node {\bf create}( {\tt uk.ac.ic.doc.neuralnets.graph.neural.NodeSpecification } {\bf s} )
\label{l353}\label{l354}}%end signature
\begin{itemize}
\sld
\item{
\sld
{\bf Parameters}
\sld\isep
  \begin{itemize}
\sld\isep
   \item{
\sld
{\tt s} - the specification of the node}
  \end{itemize}
}%end item
\item{{\bf Returns} - 
node with parameters conforming to the specification. 
}%end item
\end{itemize}
}%end item
\divideents{get}
\item{\vskip -1.9ex 
\membername{get}
{\tt public static NodeFactory {\bf get}(  )
\label{l355}\label{l356}}%end signature
\begin{itemize}
\sld
\item{
\sld
{\bf Usage}
  \begin{itemize}\isep
   \item{
Get the factory instance.
}%end item
  \end{itemize}
}
\item{{\bf Returns} - 
the NodeFactory 
}%end item
\end{itemize}
}%end item
\end{itemize}
}
}
\startsection{Class}{PerceptronSpecification}{l210}{%
{\small Default NodeSpecification for Perceptrons.}
\vskip .1in 
\startsubsubsection{Declaration}{
\fbox{\vbox{
\hbox{\vbox{\small public 
class 
PerceptronSpecification}}
\noindent\hbox{\vbox{{\bf extends} uk.ac.ic.doc.neuralnets.graph.neural.NodeSpecification}}
}}}
\startsubsubsection{Constructors}{
\vskip -2em
\begin{itemize}
\item{\vskip -1.9ex 
\membername{PerceptronSpecification}
{\tt public {\bf PerceptronSpecification}(  )
\label{l357}\label{l358}}%end signature
\begin{itemize}
\sld
\item{
\sld
{\bf Usage}
  \begin{itemize}\isep
   \item{
Creates a perceptron specifcation with default sigmoid parameters.
 
 \begin{itemize}

  \item[Squash Function\textless $/$dt\textgreater ]{ 1 $/$ (1 + e \textless sup\textgreater -charge\textless $/$sup\textgreater ) \textless $/$dd\textgreater 
  }
\item[Trigger\textless $/$dt\textgreater ]{1\textless $/$dd\textgreater 
 }
\end{itemize}
}%end item
  \end{itemize}
}
\end{itemize}
}%end item
\end{itemize}
}
\startsubsubsection{Methods inherited from class {\tt uk.ac.ic.doc.neuralnets.graph.neural.NodeSpecification}}{
\par{\small 
\refdefined{l283}\vskip -2em
\begin{itemize}
\item{\vskip -1.9ex 
\membername{get}
{\tt public ASTExpression {\bf get}( {\tt java.lang.String } {\bf param} )
}%end signature
\begin{itemize}
\sld
\item{
\sld
{\bf Usage}
  \begin{itemize}\isep
   \item{
Get the AST expression for input parameter.
}%end item
  \end{itemize}
}
\item{
\sld
{\bf Parameters}
\sld\isep
  \begin{itemize}
\sld\isep
   \item{
\sld
{\tt param} - String}
  \end{itemize}
}%end item
\item{{\bf Returns} - 
AST expression 
}%end item
\end{itemize}
}%end item
\divideents{getEdgeDecoration}
\item{\vskip -1.9ex 
\membername{getEdgeDecoration}
{\tt public EdgeDecoration {\bf getEdgeDecoration}(  )
}%end signature
\begin{itemize}
\sld
\item{
\sld
{\bf Usage}
  \begin{itemize}\isep
   \item{
Get the edge decoration for the node specification.
}%end item
  \end{itemize}
}
\item{{\bf Returns} - 
The edge decoration. 
}%end item
\end{itemize}
}%end item
\divideents{getName}
\item{\vskip -1.9ex 
\membername{getName}
{\tt public String {\bf getName}(  )
}%end signature
\begin{itemize}
\sld
\item{
\sld
{\bf Usage}
  \begin{itemize}\isep
   \item{
Get the name of the node specification.
}%end item
  \end{itemize}
}
\item{{\bf Returns} - 
The name. 
}%end item
\end{itemize}
}%end item
\divideents{getParameters}
\item{\vskip -1.9ex 
\membername{getParameters}
{\tt public Set {\bf getParameters}(  )
}%end signature
\begin{itemize}
\sld
\item{
\sld
{\bf Usage}
  \begin{itemize}\isep
   \item{
Get the parameter key set.
}%end item
  \end{itemize}
}
\item{{\bf Returns} - 
Parameter key set. 
}%end item
\end{itemize}
}%end item
\divideents{getTarget}
\item{\vskip -1.9ex 
\membername{getTarget}
{\tt public Class {\bf getTarget}(  )
}%end signature
\begin{itemize}
\sld
\item{
\sld
{\bf Usage}
  \begin{itemize}\isep
   \item{
Get target of node specification.
}%end item
  \end{itemize}
}
\item{{\bf Returns} - 
Target 
}%end item
\end{itemize}
}%end item
\divideents{set}
\item{\vskip -1.9ex 
\membername{set}
{\tt public NodeSpecification {\bf set}( {\tt java.lang.String } {\bf param},
{\tt uk.ac.ic.doc.neuralnets.expressions.ast.ASTExpression } {\bf target} )
}%end signature
\begin{itemize}
\sld
\item{
\sld
{\bf Usage}
  \begin{itemize}\isep
   \item{
Set a parameter to an AST expresion.
}%end item
  \end{itemize}
}
\item{
\sld
{\bf Parameters}
\sld\isep
  \begin{itemize}
\sld\isep
   \item{
\sld
{\tt param} - Parameter name}
   \item{
\sld
{\tt target} - AST expression value.}
  \end{itemize}
}%end item
\item{{\bf Returns} - 
Itself. 
}%end item
\end{itemize}
}%end item
\divideents{setEdgeDecoration}
\item{\vskip -1.9ex 
\membername{setEdgeDecoration}
{\tt public void {\bf setEdgeDecoration}( {\tt uk.ac.ic.doc.neuralnets.graph.neural.EdgeDecoration } {\bf ed} )
}%end signature
\begin{itemize}
\sld
\item{
\sld
{\bf Usage}
  \begin{itemize}\isep
   \item{
Set the edge decorator for the node specification.
}%end item
  \end{itemize}
}
\item{
\sld
{\bf Parameters}
\sld\isep
  \begin{itemize}
\sld\isep
   \item{
\sld
{\tt ed} - The edge decoration.}
  \end{itemize}
}%end item
\end{itemize}
}%end item
\divideents{setName}
\item{\vskip -1.9ex 
\membername{setName}
{\tt public void {\bf setName}( {\tt java.lang.String } {\bf n} )
}%end signature
\begin{itemize}
\sld
\item{
\sld
{\bf Usage}
  \begin{itemize}\isep
   \item{
Set name of node specification.
}%end item
  \end{itemize}
}
\item{
\sld
{\bf Parameters}
\sld\isep
  \begin{itemize}
\sld\isep
   \item{
\sld
{\tt n} - Name}
  \end{itemize}
}%end item
\end{itemize}
}%end item
\end{itemize}
}}
}
\startsection{Class}{SpikingNodeSpecification}{l211}{%
{\small Default NodeSpecification for SpikingNeurones}
\vskip .1in 
\startsubsubsection{Declaration}{
\fbox{\vbox{
\hbox{\vbox{\small public 
class 
SpikingNodeSpecification}}
\noindent\hbox{\vbox{{\bf extends} uk.ac.ic.doc.neuralnets.graph.neural.NodeSpecification}}
}}}
\startsubsubsection{Constructors}{
\vskip -2em
\begin{itemize}
\item{\vskip -1.9ex 
\membername{SpikingNodeSpecification}
{\tt public {\bf SpikingNodeSpecification}(  )
\label{l359}\label{l360}}%end signature
\begin{itemize}
\sld
\item{
\sld
{\bf Usage}
  \begin{itemize}\isep
   \item{
Creates a spiking neurone specification with default parameters according
 to Izhikevich's model.
 
 \begin{itemize}

 	\item[Squash Function\textless $/$dt\textgreater ]{0.5\textless $/$dd\textgreater 
  }
\item[Trigger\textless $/$dt\textgreater ]{30\textless $/$dd\textgreater 
  }
\item[Initial Charge\textless $/$dt\textgreater ]{-65\textless $/$dd\textgreater 
  }
\item[Recovery Scale\textless $/$dt\textgreater ]{0.02\textless $/$dd\textgreater 
  }
\item[Recovery Sensitivity\textless $/$dt\textgreater ]{0.2\textless $/$dd\textgreater 
  }
\item[Post Spike Reset\textless $/$dt\textgreater ]{-65 + 15 * RAND()\textless sup\textgreater 2\textless $/$sup\textgreater \textless $/$dd\textgreater 
  }
\item[PSRRecovery\textless $/$dt\textgreater ]{8 - 6 * RAND()\textless sup\textgreater 2\textless $/$sup\textgreater \textless $/$dd\textgreater 
  }
\item[Thalamic Input\textless $/$dt\textgreater ]{5 * GRAND()\textless $/$dd\textgreater 
  }
\item[Synaptic Delay\textless $/$dt\textgreater ]{20 * RAND()\textless $/$dd\textgreater 
 }
\end{itemize}
  where RAND() is a  uniformly distributed random number between 0 and 1 
  and GRAND() is a Gaussian distributed random number.
}%end item
  \end{itemize}
}
\item{{\bf See Also}
  \begin{itemize}
   \item{{\tt java.util.Random} {\small 
\refdefined{l282}}%end \small
}%end item
  \end{itemize}
}%end item
\end{itemize}
}%end item
\end{itemize}
}
\startsubsubsection{Methods inherited from class {\tt uk.ac.ic.doc.neuralnets.graph.neural.NodeSpecification}}{
\par{\small 
\refdefined{l283}\vskip -2em
\begin{itemize}
\item{\vskip -1.9ex 
\membername{get}
{\tt public ASTExpression {\bf get}( {\tt java.lang.String } {\bf param} )
}%end signature
\begin{itemize}
\sld
\item{
\sld
{\bf Usage}
  \begin{itemize}\isep
   \item{
Get the AST expression for input parameter.
}%end item
  \end{itemize}
}
\item{
\sld
{\bf Parameters}
\sld\isep
  \begin{itemize}
\sld\isep
   \item{
\sld
{\tt param} - String}
  \end{itemize}
}%end item
\item{{\bf Returns} - 
AST expression 
}%end item
\end{itemize}
}%end item
\divideents{getEdgeDecoration}
\item{\vskip -1.9ex 
\membername{getEdgeDecoration}
{\tt public EdgeDecoration {\bf getEdgeDecoration}(  )
}%end signature
\begin{itemize}
\sld
\item{
\sld
{\bf Usage}
  \begin{itemize}\isep
   \item{
Get the edge decoration for the node specification.
}%end item
  \end{itemize}
}
\item{{\bf Returns} - 
The edge decoration. 
}%end item
\end{itemize}
}%end item
\divideents{getName}
\item{\vskip -1.9ex 
\membername{getName}
{\tt public String {\bf getName}(  )
}%end signature
\begin{itemize}
\sld
\item{
\sld
{\bf Usage}
  \begin{itemize}\isep
   \item{
Get the name of the node specification.
}%end item
  \end{itemize}
}
\item{{\bf Returns} - 
The name. 
}%end item
\end{itemize}
}%end item
\divideents{getParameters}
\item{\vskip -1.9ex 
\membername{getParameters}
{\tt public Set {\bf getParameters}(  )
}%end signature
\begin{itemize}
\sld
\item{
\sld
{\bf Usage}
  \begin{itemize}\isep
   \item{
Get the parameter key set.
}%end item
  \end{itemize}
}
\item{{\bf Returns} - 
Parameter key set. 
}%end item
\end{itemize}
}%end item
\divideents{getTarget}
\item{\vskip -1.9ex 
\membername{getTarget}
{\tt public Class {\bf getTarget}(  )
}%end signature
\begin{itemize}
\sld
\item{
\sld
{\bf Usage}
  \begin{itemize}\isep
   \item{
Get target of node specification.
}%end item
  \end{itemize}
}
\item{{\bf Returns} - 
Target 
}%end item
\end{itemize}
}%end item
\divideents{set}
\item{\vskip -1.9ex 
\membername{set}
{\tt public NodeSpecification {\bf set}( {\tt java.lang.String } {\bf param},
{\tt uk.ac.ic.doc.neuralnets.expressions.ast.ASTExpression } {\bf target} )
}%end signature
\begin{itemize}
\sld
\item{
\sld
{\bf Usage}
  \begin{itemize}\isep
   \item{
Set a parameter to an AST expresion.
}%end item
  \end{itemize}
}
\item{
\sld
{\bf Parameters}
\sld\isep
  \begin{itemize}
\sld\isep
   \item{
\sld
{\tt param} - Parameter name}
   \item{
\sld
{\tt target} - AST expression value.}
  \end{itemize}
}%end item
\item{{\bf Returns} - 
Itself. 
}%end item
\end{itemize}
}%end item
\divideents{setEdgeDecoration}
\item{\vskip -1.9ex 
\membername{setEdgeDecoration}
{\tt public void {\bf setEdgeDecoration}( {\tt uk.ac.ic.doc.neuralnets.graph.neural.EdgeDecoration } {\bf ed} )
}%end signature
\begin{itemize}
\sld
\item{
\sld
{\bf Usage}
  \begin{itemize}\isep
   \item{
Set the edge decorator for the node specification.
}%end item
  \end{itemize}
}
\item{
\sld
{\bf Parameters}
\sld\isep
  \begin{itemize}
\sld\isep
   \item{
\sld
{\tt ed} - The edge decoration.}
  \end{itemize}
}%end item
\end{itemize}
}%end item
\divideents{setName}
\item{\vskip -1.9ex 
\membername{setName}
{\tt public void {\bf setName}( {\tt java.lang.String } {\bf n} )
}%end signature
\begin{itemize}
\sld
\item{
\sld
{\bf Usage}
  \begin{itemize}\isep
   \item{
Set name of node specification.
}%end item
  \end{itemize}
}
\item{
\sld
{\bf Parameters}
\sld\isep
  \begin{itemize}
\sld\isep
   \item{
\sld
{\tt n} - Name}
  \end{itemize}
}%end item
\end{itemize}
}%end item
\end{itemize}
}}
}
}
}
\newpage
\def\packagename{uk.ac.ic.doc.neuralnets.gui.graph.events}
\chapter{\bf Package uk.ac.ic.doc.neuralnets.gui.graph.events}{
\vskip -.25in
\hbox to \hsize{\it Package Contents\hfil Page}
\rule{\hsize}{.7mm}
\vskip .13in
\hbox{\bf Classes}
\entityintro{ChargeUpdateHandler}{l361}{...no description...}
\entityintro{NeuroneTypesPersister}{l362}{...no description...}
\entityintro{NodeLocationUpdater}{l363}{...no description...}
\entityintro{ToolTipUpdater}{l364}{...no description...}
\vskip .1in
\rule{\hsize}{.7mm}
\vskip .1in
\newpage
\section{Classes}{
\startsection{Class}{ChargeUpdateHandler}{l361}{%
\startsubsubsection{Declaration}{
\fbox{\vbox{
\hbox{\vbox{\small public 
class 
ChargeUpdateHandler}}
\noindent\hbox{\vbox{{\bf extends} java.lang.Object}}
\noindent\hbox{\vbox{{\bf implements} 
uk.ac.ic.doc.neuralnets.events.EventHandler}}
}}}
\startsubsubsection{Constructors}{
\vskip -2em
\begin{itemize}
\item{\vskip -1.9ex 
\membername{ChargeUpdateHandler}
{\tt public {\bf ChargeUpdateHandler}(  )
\label{l365}\label{l366}}%end signature
}%end item
\divideents{ChargeUpdateHandler}
\item{\vskip -1.9ex 
\membername{ChargeUpdateHandler}
{\tt public {\bf ChargeUpdateHandler}( {\tt uk.ac.ic.doc.neuralnets.coreui.ZoomingInterfaceManager } {\bf m} )
\label{l367}\label{l368}}%end signature
}%end item
\end{itemize}
}
\startsubsubsection{Methods}{
\vskip -2em
\begin{itemize}
\item{\vskip -1.9ex 
\membername{flush}
{\tt public void {\bf flush}(  )
\label{l369}\label{l370}}%end signature
}%end item
\divideents{getName}
\item{\vskip -1.9ex 
\membername{getName}
{\tt public String {\bf getName}(  )
\label{l371}\label{l372}}%end signature
}%end item
\divideents{handle}
\item{\vskip -1.9ex 
\membername{handle}
{\tt public void {\bf handle}( {\tt uk.ac.ic.doc.neuralnets.events.Event } {\bf e} )
\label{l373}\label{l374}}%end signature
}%end item
\divideents{isValid}
\item{\vskip -1.9ex 
\membername{isValid}
{\tt public boolean {\bf isValid}(  )
\label{l375}\label{l376}}%end signature
}%end item
\divideents{setGUIManager}
\item{\vskip -1.9ex 
\membername{setGUIManager}
{\tt public void {\bf setGUIManager}( {\tt uk.ac.ic.doc.neuralnets.coreui.ZoomingInterfaceManager } {\bf m} )
\label{l377}\label{l378}}%end signature
}%end item
\end{itemize}
}
}
\startsection{Class}{NeuroneTypesPersister}{l362}{%
\startsubsubsection{Declaration}{
\fbox{\vbox{
\hbox{\vbox{\small public 
class 
NeuroneTypesPersister}}
\noindent\hbox{\vbox{{\bf extends} java.lang.Object}}
\noindent\hbox{\vbox{{\bf implements} 
uk.ac.ic.doc.neuralnets.events.EventHandler}}
}}}
\startsubsubsection{Constructors}{
\vskip -2em
\begin{itemize}
\item{\vskip -1.9ex 
\membername{NeuroneTypesPersister}
{\tt public {\bf NeuroneTypesPersister}(  )
\label{l379}\label{l380}}%end signature
}%end item
\end{itemize}
}
\startsubsubsection{Methods}{
\vskip -2em
\begin{itemize}
\item{\vskip -1.9ex 
\membername{flush}
{\tt public void {\bf flush}(  )
\label{l381}\label{l382}}%end signature
}%end item
\divideents{getName}
\item{\vskip -1.9ex 
\membername{getName}
{\tt public String {\bf getName}(  )
\label{l383}\label{l384}}%end signature
}%end item
\divideents{handle}
\item{\vskip -1.9ex 
\membername{handle}
{\tt public void {\bf handle}( {\tt uk.ac.ic.doc.neuralnets.events.Event } {\bf e} )
\label{l385}\label{l386}}%end signature
}%end item
\divideents{isValid}
\item{\vskip -1.9ex 
\membername{isValid}
{\tt public boolean {\bf isValid}(  )
\label{l387}\label{l388}}%end signature
}%end item
\end{itemize}
}
}
\startsection{Class}{NodeLocationUpdater}{l363}{%
\startsubsubsection{Declaration}{
\fbox{\vbox{
\hbox{\vbox{\small public 
class 
NodeLocationUpdater}}
\noindent\hbox{\vbox{{\bf extends} java.lang.Object}}
\noindent\hbox{\vbox{{\bf implements} 
uk.ac.ic.doc.neuralnets.events.EventHandler}}
}}}
\startsubsubsection{Constructors}{
\vskip -2em
\begin{itemize}
\item{\vskip -1.9ex 
\membername{NodeLocationUpdater}
{\tt public {\bf NodeLocationUpdater}( {\tt uk.ac.ic.doc.neuralnets.coreui.ZoomingInterfaceManager } {\bf gm} )
\label{l389}\label{l390}}%end signature
}%end item
\end{itemize}
}
\startsubsubsection{Methods}{
\vskip -2em
\begin{itemize}
\item{\vskip -1.9ex 
\membername{flush}
{\tt public void {\bf flush}(  )
\label{l391}\label{l392}}%end signature
}%end item
\divideents{getName}
\item{\vskip -1.9ex 
\membername{getName}
{\tt public String {\bf getName}(  )
\label{l393}\label{l394}}%end signature
}%end item
\divideents{handle}
\item{\vskip -1.9ex 
\membername{handle}
{\tt public void {\bf handle}( {\tt uk.ac.ic.doc.neuralnets.events.Event } {\bf e} )
\label{l395}\label{l396}}%end signature
}%end item
\divideents{isValid}
\item{\vskip -1.9ex 
\membername{isValid}
{\tt public boolean {\bf isValid}(  )
\label{l397}\label{l398}}%end signature
}%end item
\end{itemize}
}
}
\startsection{Class}{ToolTipUpdater}{l364}{%
\startsubsubsection{Declaration}{
\fbox{\vbox{
\hbox{\vbox{\small public 
class 
ToolTipUpdater}}
\noindent\hbox{\vbox{{\bf extends} java.lang.Object}}
\noindent\hbox{\vbox{{\bf implements} 
uk.ac.ic.doc.neuralnets.events.EventHandler}}
}}}
\startsubsubsection{Constructors}{
\vskip -2em
\begin{itemize}
\item{\vskip -1.9ex 
\membername{ToolTipUpdater}
{\tt public {\bf ToolTipUpdater}( {\tt uk.ac.ic.doc.neuralnets.coreui.ZoomingInterfaceManager } {\bf gm} )
\label{l399}\label{l400}}%end signature
}%end item
\end{itemize}
}
\startsubsubsection{Methods}{
\vskip -2em
\begin{itemize}
\item{\vskip -1.9ex 
\membername{flush}
{\tt public void {\bf flush}(  )
\label{l401}\label{l402}}%end signature
}%end item
\divideents{getName}
\item{\vskip -1.9ex 
\membername{getName}
{\tt public String {\bf getName}(  )
\label{l403}\label{l404}}%end signature
}%end item
\divideents{handle}
\item{\vskip -1.9ex 
\membername{handle}
{\tt public void {\bf handle}( {\tt uk.ac.ic.doc.neuralnets.events.Event } {\bf e} )
\label{l405}\label{l406}}%end signature
}%end item
\divideents{isValid}
\item{\vskip -1.9ex 
\membername{isValid}
{\tt public boolean {\bf isValid}(  )
\label{l407}\label{l408}}%end signature
}%end item
\end{itemize}
}
}
}
}
\newpage
\def\packagename{uk.ac.ic.doc.neuralnets.gui.statistics}
\chapter{\bf Package uk.ac.ic.doc.neuralnets.gui.statistics}{
\vskip -.25in
\hbox to \hsize{\it Package Contents\hfil Page}
\rule{\hsize}{.7mm}
\vskip .13in
\hbox{\bf Classes}
\entityintro{StatisticianConfig}{l409}{Basic Statistician Configuration interface.}
\vskip .1in
\rule{\hsize}{.7mm}
\vskip .1in
\newpage
\section{Classes}{
\startsection{Class}{StatisticianConfig}{l409}{%
{\small Basic Statistician Configuration interface. Statisticians are EventHandlers 
 designed to harvest data from events during the running of a neural network.
 StatisticianConfigs can be used to configure$/$disable Statisticians.}
\vskip .1in 
\startsubsubsection{Declaration}{
\fbox{\vbox{
\hbox{\vbox{\small public abstract 
class 
StatisticianConfig}}
\noindent\hbox{\vbox{{\bf extends} uk.ac.ic.doc.neuralnets.util.plugins.PriorityPlugin}}
}}}
\startsubsubsection{Constructors}{
\vskip -2em
\begin{itemize}
\item{\vskip -1.9ex 
\membername{StatisticianConfig}
{\tt public {\bf StatisticianConfig}(  )
\label{l410}\label{l411}}%end signature
}%end item
\end{itemize}
}
\startsubsubsection{Methods}{
\vskip -2em
\begin{itemize}
\item{\vskip -1.9ex 
\membername{configure}
{\tt public abstract EventHandler {\bf configure}( {\tt org.eclipse.swt.widgets.Shell } {\bf parent} )
\label{l412}\label{l413}}%end signature
\begin{itemize}
\sld
\item{
\sld
{\bf Usage}
  \begin{itemize}\isep
   \item{
Perform an operations required to configure a new statistician.
}%end item
  \end{itemize}
}
\item{
\sld
{\bf Parameters}
\sld\isep
  \begin{itemize}
\sld\isep
   \item{
\sld
{\tt parent} - - shell access, for user interaction}
  \end{itemize}
}%end item
\item{{\bf Returns} - 
the configured event handler 
}%end item
\end{itemize}
}%end item
\divideents{disable}
\item{\vskip -1.9ex 
\membername{disable}
{\tt public abstract void {\bf disable}( {\tt uk.ac.ic.doc.neuralnets.events.EventHandler } {\bf h} )
\label{l414}\label{l415}}%end signature
\begin{itemize}
\sld
\item{
\sld
{\bf Usage}
  \begin{itemize}\isep
   \item{
Disable a statistician
}%end item
  \end{itemize}
}
\item{
\sld
{\bf Parameters}
\sld\isep
  \begin{itemize}
\sld\isep
   \item{
\sld
{\tt h} - the event handler to disable}
  \end{itemize}
}%end item
\end{itemize}
}%end item
\divideents{getTargetEvents}
\item{\vskip -1.9ex 
\membername{getTargetEvents}
{\tt public Class {\bf getTargetEvents}(  )
\label{l416}\label{l417}}%end signature
\begin{itemize}
\sld
\item{
\sld
{\bf Usage}
  \begin{itemize}\isep
   \item{
Defines which events this statistician listens for.
}%end item
  \end{itemize}
}
\item{{\bf Returns} - 
An array of Event classes to be registered to handle 
}%end item
\end{itemize}
}%end item
\end{itemize}
}
\startsubsubsection{Methods inherited from class {\tt uk.ac.ic.doc.neuralnets.util.plugins.PriorityPlugin}}{
\par{\small 
\refdefined{l33}\vskip -2em
\begin{itemize}
\item{\vskip -1.9ex 
\membername{compareTo}
{\tt public int {\bf compareTo}( {\tt uk.ac.ic.doc.neuralnets.util.plugins.PriorityPlugin } {\bf o} )
}%end signature
}%end item
\divideents{getPriority}
\item{\vskip -1.9ex 
\membername{getPriority}
{\tt public abstract int {\bf getPriority}(  )
}%end signature
\begin{itemize}
\sld
\item{
\sld
{\bf Usage}
  \begin{itemize}\isep
   \item{
The plugin's priority.
}%end item
  \end{itemize}
}
\item{{\bf Returns} - 
the priority 
}%end item
\end{itemize}
}%end item
\end{itemize}
}}
}
}
}
\newpage
\def\packagename{uk.ac.ic.doc.neuralnets.util}
\chapter{\bf Package uk.ac.ic.doc.neuralnets.util}{
\vskip -.25in
\hbox to \hsize{\it Package Contents\hfil Page}
\rule{\hsize}{.7mm}
\vskip .13in
\hbox{\bf Interfaces}
\entityintro{Transformer}{l418}{General purpose Transformer from one data-type to another}
\vskip .13in
\hbox{\bf Classes}
\entityintro{Container}{l419}{Simple container for another object, for use when a final object is required
 but cannot be furnished yet}
\vskip .1in
\rule{\hsize}{.7mm}
\vskip .1in
\newpage
\section{Interfaces}{
\startsection{Interface}{Transformer}{l418}{%
{\small General purpose Transformer from one data-type to another}
\vskip .1in 
\startsubsubsection{Declaration}{
\fbox{\vbox{
\hbox{\vbox{\small public interface 
Transformer}}
\noindent\hbox{\vbox{{\bf implements} 
java.io.Serializable}}
}}}
\startsubsubsection{Methods}{
\vskip -2em
\begin{itemize}
\item{\vskip -1.9ex 
\membername{transform}
{\tt public Object {\bf transform}( {\tt java.lang.Object } {\bf input} )
\label{l420}\label{l421}}%end signature
\begin{itemize}
\sld
\item{
\sld
{\bf Usage}
  \begin{itemize}\isep
   \item{
Transform input object
}%end item
  \end{itemize}
}
\item{
\sld
{\bf Parameters}
\sld\isep
  \begin{itemize}
\sld\isep
   \item{
\sld
{\tt input} - - the object to transform}
  \end{itemize}
}%end item
\item{{\bf Returns} - 
the transformed object 
}%end item
\end{itemize}
}%end item
\end{itemize}
}
}
}
\section{Classes}{
\startsection{Class}{Container}{l419}{%
{\small Simple container for another object, for use when a final object is required
 but cannot be furnished yet}
\vskip .1in 
\startsubsubsection{Declaration}{
\fbox{\vbox{
\hbox{\vbox{\small public 
class 
Container}}
\noindent\hbox{\vbox{{\bf extends} java.lang.Object}}
}}}
\startsubsubsection{Constructors}{
\vskip -2em
\begin{itemize}
\item{\vskip -1.9ex 
\membername{Container}
{\tt public {\bf Container}(  )
\label{l422}\label{l423}}%end signature
\begin{itemize}
\sld
\item{
\sld
{\bf Usage}
  \begin{itemize}\isep
   \item{
Create an empty container
}%end item
  \end{itemize}
}
\end{itemize}
}%end item
\divideents{Container}
\item{\vskip -1.9ex 
\membername{Container}
{\tt public {\bf Container}( {\tt java.lang.Object } {\bf contents} )
\label{l424}\label{l425}}%end signature
\begin{itemize}
\sld
\item{
\sld
{\bf Usage}
  \begin{itemize}\isep
   \item{
Create a container with contents of type T.
}%end item
  \end{itemize}
}
\item{
\sld
{\bf Parameters}
\sld\isep
  \begin{itemize}
\sld\isep
   \item{
\sld
{\tt contents} - }
  \end{itemize}
}%end item
\end{itemize}
}%end item
\end{itemize}
}
\startsubsubsection{Methods}{
\vskip -2em
\begin{itemize}
\item{\vskip -1.9ex 
\membername{get}
{\tt public Object {\bf get}(  )
\label{l426}\label{l427}}%end signature
\begin{itemize}
\sld
\item{
\sld
{\bf Usage}
  \begin{itemize}\isep
   \item{
Get the content of the container.
}%end item
  \end{itemize}
}
\item{{\bf Returns} - 
the container contents 
}%end item
\end{itemize}
}%end item
\divideents{set}
\item{\vskip -1.9ex 
\membername{set}
{\tt public void {\bf set}( {\tt java.lang.Object } {\bf t} )
\label{l428}\label{l429}}%end signature
\begin{itemize}
\sld
\item{
\sld
{\bf Usage}
  \begin{itemize}\isep
   \item{
Set the content of the container.
}%end item
  \end{itemize}
}
\item{
\sld
{\bf Parameters}
\sld\isep
  \begin{itemize}
\sld\isep
   \item{
\sld
{\tt t} - - the object to store in the container}
  \end{itemize}
}%end item
\end{itemize}
}%end item
\end{itemize}
}
}
}
}
\newpage
\def\packagename{uk.ac.ic.doc.neuralnets.util.configuration}
\chapter{\bf Package uk.ac.ic.doc.neuralnets.util.configuration}{
\vskip -.25in
\hbox to \hsize{\it Package Contents\hfil Page}
\rule{\hsize}{.7mm}
\vskip .13in
\hbox{\bf Interfaces}
\entityintro{Configurator}{l430}{Configurators are Plugins that are run once at application load-time.}
\vskip .13in
\hbox{\bf Classes}
\entityintro{ConfigurationManager}{l431}{The ConfigurationManager controls Configurator objects, calling their 
 {\tt configure} methods at application load time.}
\vskip .1in
\rule{\hsize}{.7mm}
\vskip .1in
\newpage
\section{Interfaces}{
\startsection{Interface}{Configurator}{l430}{%
{\small Configurators are Plugins that are run once at application load-time. They 
 are intended for configuring external libraries such as Log4J.}
\vskip .1in 
\startsubsubsection{Declaration}{
\fbox{\vbox{
\hbox{\vbox{\small public interface 
Configurator}}
\noindent\hbox{\vbox{{\bf implements} 
uk.ac.ic.doc.neuralnets.util.plugins.Plugin}}
}}}
\startsubsubsection{Methods}{
\vskip -2em
\begin{itemize}
\item{\vskip -1.9ex 
\membername{configure}
{\tt public void {\bf configure}(  )
\label{l432}\label{l433}}%end signature
\begin{itemize}
\sld
\item{
\sld
{\bf Usage}
  \begin{itemize}\isep
   \item{
Perform any required actions for configuration
}%end item
  \end{itemize}
}
\end{itemize}
}%end item
\end{itemize}
}
}
}
\section{Classes}{
\startsection{Class}{ConfigurationManager}{l431}{%
{\small The ConfigurationManager controls Configurator objects, calling their 
 {\tt configure} methods at application load time.}
\vskip .1in 
\startsubsubsection{Declaration}{
\fbox{\vbox{
\hbox{\vbox{\small public 
class 
ConfigurationManager}}
\noindent\hbox{\vbox{{\bf extends} java.lang.Object}}
}}}
\startsubsubsection{Fields}{
\begin{itemize}
\item{
public static final File config\begin{itemize}\item{\vskip -.9ex Master configuration file.}\end{itemize}
}
\end{itemize}
}
\startsubsubsection{Constructors}{
\vskip -2em
\begin{itemize}
\item{\vskip -1.9ex 
\membername{ConfigurationManager}
{\tt public {\bf ConfigurationManager}(  )
\label{l434}\label{l435}}%end signature
}%end item
\end{itemize}
}
\startsubsubsection{Methods}{
\vskip -2em
\begin{itemize}
\item{\vskip -1.9ex 
\membername{configure}
{\tt public static void {\bf configure}(  )
\label{l436}\label{l437}}%end signature
\begin{itemize}
\sld
\item{
\sld
{\bf Usage}
  \begin{itemize}\isep
   \item{
Configure all configurators found in conf$/$configurator.cfg.
}%end item
  \end{itemize}
}
\end{itemize}
}%end item
\end{itemize}
}
}
}
}
\newpage
\def\packagename{uk.ac.ic.doc.neuralnets.util.plugins}
\chapter{\bf Package uk.ac.ic.doc.neuralnets.util.plugins}{
\vskip -.25in
\hbox to \hsize{\it Package Contents\hfil Page}
\rule{\hsize}{.7mm}
\vskip .13in
\hbox{\bf Interfaces}
\entityintro{Plugin}{l438}{Generic Plugin interface.}
\vskip .13in
\hbox{\bf Classes}
\entityintro{PluginLoader}{l439}{The PluginLoader is responsible for loading plugin class files from the 
 $/$plugin  directory into the virtual machine.}
\entityintro{PluginLoadException}{l440}{Throw when there are unrecoverable errors whilst attempting to instantiate a 
 plugin.}
\entityintro{PluginManager}{l110}{The PluginManager is responsible for managing the class loading and 
 instantiation of plugins from the plugins directory.}
\entityintro{PriorityPlugin}{l33}{PriorityPlugin extends the plugin interface allowing an ordering to be 
 applied.}
\vskip .1in
\rule{\hsize}{.7mm}
\vskip .1in
\newpage
\section{Interfaces}{
\startsection{Interface}{Plugin}{l438}{%
{\small Generic Plugin interface. All plugin types must extend or implement this
 interface. The class name of an extending plugin type must be unique. Plugins
 can not directly implement the Plugin interface, i.e. a plugin must be a 
 descendant of a sub-type of Plugin.}
\vskip .1in 
\startsubsubsection{Declaration}{
\fbox{\vbox{
\hbox{\vbox{\small public interface 
Plugin}}
}}}
\startsubsubsection{Methods}{
\vskip -2em
\begin{itemize}
\item{\vskip -1.9ex 
\membername{getName}
{\tt public String {\bf getName}(  )
\label{l441}\label{l442}}%end signature
\begin{itemize}
\sld
\item{
\sld
{\bf Usage}
  \begin{itemize}\isep
   \item{
Get the canonical name of this Plugin, used to identify it
}%end item
  \end{itemize}
}
\item{{\bf Returns} - 
The canonical name of the loaded plugin 
}%end item
\end{itemize}
}%end item
\end{itemize}
}
}
}
\section{Classes}{
\startsection{Class}{PluginLoader}{l439}{%
{\small The PluginLoader is responsible for loading plugin class files from the 
 $/$plugin  directory into the virtual machine.}
\vskip .1in 
\startsubsubsection{Declaration}{
\fbox{\vbox{
\hbox{\vbox{\small public 
class 
PluginLoader}}
\noindent\hbox{\vbox{{\bf extends} java.lang.ClassLoader}}
}}}
\startsubsubsection{Constructors}{
\vskip -2em
\begin{itemize}
\item{\vskip -1.9ex 
\membername{PluginLoader}
{\tt public {\bf PluginLoader}( {\tt java.lang.String } {\bf searchPath} )
\label{l443}\label{l444}}%end signature
}%end item
\end{itemize}
}
\startsubsubsection{Methods}{
\vskip -2em
\begin{itemize}
\item{\vskip -1.9ex 
\membername{findClass}
{\tt public Class {\bf findClass}( {\tt java.lang.String } {\bf name} )
\label{l445}\label{l446}}%end signature
}%end item
\end{itemize}
}
\startsubsubsection{Methods inherited from class {\tt java.lang.ClassLoader}}{
\par{\small 
\refdefined{l447}\vskip -2em
\begin{itemize}
\item{\vskip -1.9ex 
\membername{clearAssertionStatus}
{\tt public synchronized void {\bf clearAssertionStatus}(  )
}%end signature
}%end item
\divideents{defineClass}
\item{\vskip -1.9ex 
\membername{defineClass}
{\tt protected final Class {\bf defineClass}( {\tt byte []} {\bf arg0},
{\tt int } {\bf arg1},
{\tt int } {\bf arg2} )
}%end signature
}%end item
\divideents{defineClass}
\item{\vskip -1.9ex 
\membername{defineClass}
{\tt protected final Class {\bf defineClass}( {\tt java.lang.String } {\bf arg0},
{\tt byte []} {\bf arg1},
{\tt int } {\bf arg2},
{\tt int } {\bf arg3} )
}%end signature
}%end item
\divideents{defineClass}
\item{\vskip -1.9ex 
\membername{defineClass}
{\tt protected final Class {\bf defineClass}( {\tt java.lang.String } {\bf arg0},
{\tt byte []} {\bf arg1},
{\tt int } {\bf arg2},
{\tt int } {\bf arg3},
{\tt java.security.ProtectionDomain } {\bf arg4} )
}%end signature
}%end item
\divideents{defineClass}
\item{\vskip -1.9ex 
\membername{defineClass}
{\tt protected final Class {\bf defineClass}( {\tt java.lang.String } {\bf arg0},
{\tt java.nio.ByteBuffer } {\bf arg1},
{\tt java.security.ProtectionDomain } {\bf arg2} )
}%end signature
}%end item
\divideents{definePackage}
\item{\vskip -1.9ex 
\membername{definePackage}
{\tt protected Package {\bf definePackage}( {\tt java.lang.String } {\bf arg0},
{\tt java.lang.String } {\bf arg1},
{\tt java.lang.String } {\bf arg2},
{\tt java.lang.String } {\bf arg3},
{\tt java.lang.String } {\bf arg4},
{\tt java.lang.String } {\bf arg5},
{\tt java.lang.String } {\bf arg6},
{\tt java.net.URL } {\bf arg7} )
}%end signature
}%end item
\divideents{findClass}
\item{\vskip -1.9ex 
\membername{findClass}
{\tt protected Class {\bf findClass}( {\tt java.lang.String } {\bf arg0} )
}%end signature
}%end item
\divideents{findLibrary}
\item{\vskip -1.9ex 
\membername{findLibrary}
{\tt protected String {\bf findLibrary}( {\tt java.lang.String } {\bf arg0} )
}%end signature
}%end item
\divideents{findLoadedClass}
\item{\vskip -1.9ex 
\membername{findLoadedClass}
{\tt protected final Class {\bf findLoadedClass}( {\tt java.lang.String } {\bf arg0} )
}%end signature
}%end item
\divideents{findResource}
\item{\vskip -1.9ex 
\membername{findResource}
{\tt protected URL {\bf findResource}( {\tt java.lang.String } {\bf arg0} )
}%end signature
}%end item
\divideents{findResources}
\item{\vskip -1.9ex 
\membername{findResources}
{\tt protected Enumeration {\bf findResources}( {\tt java.lang.String } {\bf arg0} )
}%end signature
}%end item
\divideents{findSystemClass}
\item{\vskip -1.9ex 
\membername{findSystemClass}
{\tt protected final Class {\bf findSystemClass}( {\tt java.lang.String } {\bf arg0} )
}%end signature
}%end item
\divideents{getPackage}
\item{\vskip -1.9ex 
\membername{getPackage}
{\tt protected Package {\bf getPackage}( {\tt java.lang.String } {\bf arg0} )
}%end signature
}%end item
\divideents{getPackages}
\item{\vskip -1.9ex 
\membername{getPackages}
{\tt protected Package {\bf getPackages}(  )
}%end signature
}%end item
\divideents{getParent}
\item{\vskip -1.9ex 
\membername{getParent}
{\tt public final ClassLoader {\bf getParent}(  )
}%end signature
}%end item
\divideents{getResource}
\item{\vskip -1.9ex 
\membername{getResource}
{\tt public URL {\bf getResource}( {\tt java.lang.String } {\bf arg0} )
}%end signature
}%end item
\divideents{getResourceAsStream}
\item{\vskip -1.9ex 
\membername{getResourceAsStream}
{\tt public InputStream {\bf getResourceAsStream}( {\tt java.lang.String } {\bf arg0} )
}%end signature
}%end item
\divideents{getResources}
\item{\vskip -1.9ex 
\membername{getResources}
{\tt public Enumeration {\bf getResources}( {\tt java.lang.String } {\bf arg0} )
}%end signature
}%end item
\divideents{getSystemClassLoader}
\item{\vskip -1.9ex 
\membername{getSystemClassLoader}
{\tt public static ClassLoader {\bf getSystemClassLoader}(  )
}%end signature
}%end item
\divideents{getSystemResource}
\item{\vskip -1.9ex 
\membername{getSystemResource}
{\tt public static URL {\bf getSystemResource}( {\tt java.lang.String } {\bf arg0} )
}%end signature
}%end item
\divideents{getSystemResourceAsStream}
\item{\vskip -1.9ex 
\membername{getSystemResourceAsStream}
{\tt public static InputStream {\bf getSystemResourceAsStream}( {\tt java.lang.String } {\bf arg0} )
}%end signature
}%end item
\divideents{getSystemResources}
\item{\vskip -1.9ex 
\membername{getSystemResources}
{\tt public static Enumeration {\bf getSystemResources}( {\tt java.lang.String } {\bf arg0} )
}%end signature
}%end item
\divideents{loadClass}
\item{\vskip -1.9ex 
\membername{loadClass}
{\tt public Class {\bf loadClass}( {\tt java.lang.String } {\bf arg0} )
}%end signature
}%end item
\divideents{loadClass}
\item{\vskip -1.9ex 
\membername{loadClass}
{\tt protected synchronized Class {\bf loadClass}( {\tt java.lang.String } {\bf arg0},
{\tt boolean } {\bf arg1} )
}%end signature
}%end item
\divideents{resolveClass}
\item{\vskip -1.9ex 
\membername{resolveClass}
{\tt protected final void {\bf resolveClass}( {\tt java.lang.Class } {\bf arg0} )
}%end signature
}%end item
\divideents{setClassAssertionStatus}
\item{\vskip -1.9ex 
\membername{setClassAssertionStatus}
{\tt public synchronized void {\bf setClassAssertionStatus}( {\tt java.lang.String } {\bf arg0},
{\tt boolean } {\bf arg1} )
}%end signature
}%end item
\divideents{setDefaultAssertionStatus}
\item{\vskip -1.9ex 
\membername{setDefaultAssertionStatus}
{\tt public synchronized void {\bf setDefaultAssertionStatus}( {\tt boolean } {\bf arg0} )
}%end signature
}%end item
\divideents{setPackageAssertionStatus}
\item{\vskip -1.9ex 
\membername{setPackageAssertionStatus}
{\tt public synchronized void {\bf setPackageAssertionStatus}( {\tt java.lang.String } {\bf arg0},
{\tt boolean } {\bf arg1} )
}%end signature
}%end item
\divideents{setSigners}
\item{\vskip -1.9ex 
\membername{setSigners}
{\tt protected final void {\bf setSigners}( {\tt java.lang.Class } {\bf arg0},
{\tt java.lang.Object []} {\bf arg1} )
}%end signature
}%end item
\end{itemize}
}}
}
\startsection{Class}{PluginLoadException}{l440}{%
{\small Throw when there are unrecoverable errors whilst attempting to instantiate a 
 plugin.}
\vskip .1in 
\startsubsubsection{Declaration}{
\fbox{\vbox{
\hbox{\vbox{\small public 
class 
PluginLoadException}}
\noindent\hbox{\vbox{{\bf extends} java.lang.Exception}}
}}}
\startsubsubsection{Fields}{
\begin{itemize}
\item{
public static final long serialVersionUID\begin{itemize}\item{\vskip -.9ex }\end{itemize}
}
\end{itemize}
}
\startsubsubsection{Constructors}{
\vskip -2em
\begin{itemize}
\item{\vskip -1.9ex 
\membername{PluginLoadException}
{\tt public {\bf PluginLoadException}( {\tt java.lang.String } {\bf m} )
\label{l448}\label{l449}}%end signature
}%end item
\divideents{PluginLoadException}
\item{\vskip -1.9ex 
\membername{PluginLoadException}
{\tt public {\bf PluginLoadException}( {\tt java.lang.String } {\bf m},
{\tt java.lang.Throwable } {\bf e} )
\label{l450}\label{l451}}%end signature
}%end item
\divideents{PluginLoadException}
\item{\vskip -1.9ex 
\membername{PluginLoadException}
{\tt public {\bf PluginLoadException}( {\tt java.lang.Throwable } {\bf e} )
\label{l452}\label{l453}}%end signature
}%end item
\end{itemize}
}
\startsubsubsection{Methods inherited from class {\tt java.lang.Exception}}{
\par{\small 
\refdefined{l454}}}
\startsubsubsection{Methods inherited from class {\tt java.lang.Throwable}}{
\par{\small 
\refdefined{l455}\vskip -2em
\begin{itemize}
\item{\vskip -1.9ex 
\membername{fillInStackTrace}
{\tt public synchronized native Throwable {\bf fillInStackTrace}(  )
}%end signature
}%end item
\divideents{getCause}
\item{\vskip -1.9ex 
\membername{getCause}
{\tt public Throwable {\bf getCause}(  )
}%end signature
}%end item
\divideents{getLocalizedMessage}
\item{\vskip -1.9ex 
\membername{getLocalizedMessage}
{\tt public String {\bf getLocalizedMessage}(  )
}%end signature
}%end item
\divideents{getMessage}
\item{\vskip -1.9ex 
\membername{getMessage}
{\tt public String {\bf getMessage}(  )
}%end signature
}%end item
\divideents{getStackTrace}
\item{\vskip -1.9ex 
\membername{getStackTrace}
{\tt public StackTraceElement {\bf getStackTrace}(  )
}%end signature
}%end item
\divideents{initCause}
\item{\vskip -1.9ex 
\membername{initCause}
{\tt public synchronized Throwable {\bf initCause}( {\tt java.lang.Throwable } {\bf arg0} )
}%end signature
}%end item
\divideents{printStackTrace}
\item{\vskip -1.9ex 
\membername{printStackTrace}
{\tt public void {\bf printStackTrace}(  )
}%end signature
}%end item
\divideents{printStackTrace}
\item{\vskip -1.9ex 
\membername{printStackTrace}
{\tt public void {\bf printStackTrace}( {\tt java.io.PrintStream } {\bf arg0} )
}%end signature
}%end item
\divideents{printStackTrace}
\item{\vskip -1.9ex 
\membername{printStackTrace}
{\tt public void {\bf printStackTrace}( {\tt java.io.PrintWriter } {\bf arg0} )
}%end signature
}%end item
\divideents{setStackTrace}
\item{\vskip -1.9ex 
\membername{setStackTrace}
{\tt public void {\bf setStackTrace}( {\tt java.lang.StackTraceElement []} {\bf arg0} )
}%end signature
}%end item
\divideents{toString}
\item{\vskip -1.9ex 
\membername{toString}
{\tt public String {\bf toString}(  )
}%end signature
}%end item
\end{itemize}
}}
}
\startsection{Class}{PluginManager}{l110}{%
{\small The PluginManager is responsible for managing the class loading and 
 instantiation of plugins from the plugins directory. Plugins are loaded and 
 cached by the PluginLoader.}
\vskip .1in 
\startsubsubsection{Declaration}{
\fbox{\vbox{
\hbox{\vbox{\small public 
class 
PluginManager}}
\noindent\hbox{\vbox{{\bf extends} java.lang.Object}}
}}}
\startsubsubsection{Fields}{
\begin{itemize}
\item{
public static final File searchPath\begin{itemize}\item{\vskip -.9ex Path to plugin directory}\end{itemize}
}
\end{itemize}
}
\startsubsubsection{Methods}{
\vskip -2em
\begin{itemize}
\item{\vskip -1.9ex 
\membername{checkValidity}
{\tt public void {\bf checkValidity}(  )
\label{l456}\label{l457}}%end signature
\begin{itemize}
\sld
\item{
\sld
{\bf Usage}
  \begin{itemize}\isep
   \item{
Check the validity of all the plugins in this PluginManager. If any have
 been loaded that are invalid, remove them from this PluginManager
}%end item
  \end{itemize}
}
\end{itemize}
}%end item
\divideents{checkValidity}
\item{\vskip -1.9ex 
\membername{checkValidity}
{\tt public void {\bf checkValidity}( {\tt java.lang.Class } {\bf clazz} )
\label{l458}\label{l459}}%end signature
\begin{itemize}
\sld
\item{
\sld
{\bf Usage}
  \begin{itemize}\isep
   \item{
Check the validity of all the plugins of the given type. If any have 
 been loaded that are invalid, remove them from this PluginManager
}%end item
  \end{itemize}
}
\item{
\sld
{\bf Parameters}
\sld\isep
  \begin{itemize}
\sld\isep
   \item{
\sld
{\tt clazz} - The class of the plugin type}
  \end{itemize}
}%end item
\end{itemize}
}%end item
\divideents{checkValidity}
\item{\vskip -1.9ex 
\membername{checkValidity}
{\tt public void {\bf checkValidity}( {\tt java.lang.String } {\bf type} )
\label{l460}\label{l461}}%end signature
\begin{itemize}
\sld
\item{
\sld
{\bf Usage}
  \begin{itemize}\isep
   \item{
Check the validity of all the plugins of the given type. If any have
 been loaded that are invalid, remove them from this PluginManager
}%end item
  \end{itemize}
}
\item{
\sld
{\bf Parameters}
\sld\isep
  \begin{itemize}
\sld\isep
   \item{
\sld
{\tt type} - The type name of the plugin}
  \end{itemize}
}%end item
\end{itemize}
}%end item
\divideents{get}
\item{\vskip -1.9ex 
\membername{get}
{\tt public static PluginManager {\bf get}(  )
\label{l462}\label{l463}}%end signature
\begin{itemize}
\sld
\item{
\sld
{\bf Usage}
  \begin{itemize}\isep
   \item{
Retrieve the instance of the PluginManager.
}%end item
  \end{itemize}
}
\item{{\bf Returns} - 
the PluginManager instance 
}%end item
\item{{\bf Exceptions}
  \begin{itemize}
\sld
   \item{\vskip -.6ex{\tt uk.ac.ic.doc.neuralnets.util.plugins.PluginLoadException} - }
  \end{itemize}
}%end item
\end{itemize}
}%end item
\divideents{getPlugin}
\item{\vskip -1.9ex 
\membername{getPlugin}
{\tt public Plugin {\bf getPlugin}( {\tt java.lang.String } {\bf name},
{\tt java.lang.Class } {\bf clazz} )
\label{l464}\label{l465}}%end signature
\begin{itemize}
\sld
\item{
\sld
{\bf Usage}
  \begin{itemize}\isep
   \item{
Load the requested plugin and cast it to the given class
}%end item
  \end{itemize}
}
\item{
\sld
{\bf Parameters}
\sld\isep
  \begin{itemize}
\sld\isep
   \item{
\sld
{\tt name} - The name of the plugin}
   \item{
\sld
{\tt clazz} - The class to which it must be cast}
  \end{itemize}
}%end item
\item{{\bf Returns} - 
A Plugin object of type T 
}%end item
\end{itemize}
}%end item
\divideents{getPlugin}
\item{\vskip -1.9ex 
\membername{getPlugin}
{\tt public Plugin {\bf getPlugin}( {\tt java.lang.String } {\bf name},
{\tt java.lang.String } {\bf type} )
\label{l466}\label{l467}}%end signature
\begin{itemize}
\sld
\item{
\sld
{\bf Usage}
  \begin{itemize}\isep
   \item{
Load the requested plugin and cast it to the given class
}%end item
  \end{itemize}
}
\item{
\sld
{\bf Parameters}
\sld\isep
  \begin{itemize}
\sld\isep
   \item{
\sld
{\tt name} - The name of the plugin}
   \item{
\sld
{\tt type} - The type of the plugin to fetch}
  \end{itemize}
}%end item
\item{{\bf Returns} - 
A Plugin object of the given name and type 
}%end item
\end{itemize}
}%end item
\divideents{getPluginsOftype}
\item{\vskip -1.9ex 
\membername{getPluginsOftype}
{\tt public Set {\bf getPluginsOftype}( {\tt java.lang.Class } {\bf clazz} )
\label{l468}\label{l469}}%end signature
\begin{itemize}
\sld
\item{
\sld
{\bf Usage}
  \begin{itemize}\isep
   \item{
Answer all the plugins of the given type
}%end item
  \end{itemize}
}
\item{
\sld
{\bf Parameters}
\sld\isep
  \begin{itemize}
\sld\isep
   \item{
\sld
{\tt clazz} - The class of the type of plugin to find}
  \end{itemize}
}%end item
\item{{\bf Returns} - 
A set of plugin names 
}%end item
\end{itemize}
}%end item
\divideents{getPluginsOfType}
\item{\vskip -1.9ex 
\membername{getPluginsOfType}
{\tt public Set {\bf getPluginsOfType}( {\tt java.lang.String } {\bf type} )
\label{l470}\label{l471}}%end signature
\begin{itemize}
\sld
\item{
\sld
{\bf Usage}
  \begin{itemize}\isep
   \item{
Answer all the plugins of the given type
}%end item
  \end{itemize}
}
\item{
\sld
{\bf Parameters}
\sld\isep
  \begin{itemize}
\sld\isep
   \item{
\sld
{\tt type} - The type of the plugin to find}
  \end{itemize}
}%end item
\item{{\bf Returns} - 
A set of plugin names 
}%end item
\end{itemize}
}%end item
\divideents{refreshPlugins}
\item{\vskip -1.9ex 
\membername{refreshPlugins}
{\tt public void {\bf refreshPlugins}(  )
\label{l472}\label{l473}}%end signature
}%end item
\end{itemize}
}
}
\startsection{Class}{PriorityPlugin}{l33}{%
{\small PriorityPlugin extends the plugin interface allowing an ordering to be 
 applied. The ordering can be achieved in two ways: by implementing the
 {\tt getPriority} to return the plugin's priority, or by overriding
 the {\tt compareTo} method if more detailed comparison is required.}
\vskip .1in 
\startsubsubsection{Declaration}{
\fbox{\vbox{
\hbox{\vbox{\small public abstract 
class 
PriorityPlugin}}
\noindent\hbox{\vbox{{\bf extends} java.lang.Object}}
\noindent\hbox{\vbox{{\bf implements} 
java.lang.Comparable, Plugin}}
}}}
\startsubsubsection{Constructors}{
\vskip -2em
\begin{itemize}
\item{\vskip -1.9ex 
\membername{PriorityPlugin}
{\tt public {\bf PriorityPlugin}(  )
\label{l474}\label{l475}}%end signature
}%end item
\end{itemize}
}
\startsubsubsection{Methods}{
\vskip -2em
\begin{itemize}
\item{\vskip -1.9ex 
\membername{compareTo}
{\tt public int {\bf compareTo}( {\tt uk.ac.ic.doc.neuralnets.util.plugins.PriorityPlugin } {\bf o} )
\label{l476}\label{l477}}%end signature
}%end item
\divideents{getPriority}
\item{\vskip -1.9ex 
\membername{getPriority}
{\tt public abstract int {\bf getPriority}(  )
\label{l478}\label{l479}}%end signature
\begin{itemize}
\sld
\item{
\sld
{\bf Usage}
  \begin{itemize}\isep
   \item{
The plugin's priority.
}%end item
  \end{itemize}
}
\item{{\bf Returns} - 
the priority 
}%end item
\end{itemize}
}%end item
\end{itemize}
}
}
}
}
\newpage
\def\packagename{uk.ac.ic.doc.neuralnets.graph.neural.io}
\chapter{\bf Package uk.ac.ic.doc.neuralnets.graph.neural.io}{
\vskip -.25in
\hbox to \hsize{\it Package Contents\hfil Page}
\rule{\hsize}{.7mm}
\vskip .13in
\hbox{\bf Interfaces}
\entityintro{Foldable}{l480}{Denotes that an InputNode can be used for N-Fold training.}
\vskip .13in
\hbox{\bf Classes}
\entityintro{InputNode}{l481}{InputNodes are the default method for passing data, from the user or 
 from external sources, through the network.}
\entityintro{IONeurone}{l482}{Purely a class to "mark" a neurone as being for I$/$O purposes.}
\entityintro{OutputNode}{l483}{OutputNodes are the default method for harvesting data from a neural network 
 for use in external cases.}
\entityintro{ValueReportingOutputNode}{l484}{...no description...}
\vskip .1in
\rule{\hsize}{.7mm}
\vskip .1in
\newpage
\section{Interfaces}{
\startsection{Interface}{Foldable}{l480}{%
{\small Denotes that an InputNode can be used for N-Fold training.}
\vskip .1in 
\startsubsubsection{Declaration}{
\fbox{\vbox{
\hbox{\vbox{\small public interface 
Foldable}}
}}}
\startsubsubsection{Methods}{
\vskip -2em
\begin{itemize}
\item{\vskip -1.9ex 
\membername{fold}
{\tt public void {\bf fold}( {\tt int } {\bf foldNumber},
{\tt int } {\bf folds} )
\label{l485}\label{l486}}%end signature
\begin{itemize}
\sld
\item{
\sld
{\bf Usage}
  \begin{itemize}\isep
   \item{
Instruct this foldable to prepare for the next fold
}%end item
  \end{itemize}
}
\item{
\sld
{\bf Parameters}
\sld\isep
  \begin{itemize}
\sld\isep
   \item{
\sld
{\tt foldNumber} - The number of the current fold to prepare}
   \item{
\sld
{\tt folds} - The number of folds total}
  \end{itemize}
}%end item
\end{itemize}
}%end item
\end{itemize}
}
}
}
\section{Classes}{
\startsection{Class}{InputNode}{l481}{%
{\small InputNodes are the default method for passing data, from the user or 
 from external sources, through the network.
 
 InputNodes contain a matrix of Doubles which are fired into the network row 
 by row whenever a network ticks.
 
 They can optionally contain a corresponding matrix of target values which can
  be used for training.}
\vskip .1in 
\startsubsubsection{Declaration}{
\fbox{\vbox{
\hbox{\vbox{\small public abstract 
class 
InputNode}}
\noindent\hbox{\vbox{{\bf extends} uk.ac.ic.doc.neuralnets.graph.neural.NeuralNetwork}}
\noindent\hbox{\vbox{{\bf implements} 
uk.ac.ic.doc.neuralnets.util.plugins.Plugin, Foldable}}
}}}
\startsubsubsection{Constructors}{
\vskip -2em
\begin{itemize}
\item{\vskip -1.9ex 
\membername{InputNode}
{\tt public {\bf InputNode}(  )
\label{l487}\label{l488}}%end signature
\begin{itemize}
\sld
\item{
\sld
{\bf Usage}
  \begin{itemize}\isep
   \item{
Configures and adds the input node to the network.
}%end item
  \end{itemize}
}
\end{itemize}
}%end item
\end{itemize}
}
\startsubsubsection{Methods}{
\vskip -2em
\begin{itemize}
\item{\vskip -1.9ex 
\membername{configure}
{\tt public abstract void {\bf configure}(  )
\label{l489}\label{l490}}%end signature
\begin{itemize}
\sld
\item{
\sld
{\bf Usage}
  \begin{itemize}\isep
   \item{
Called before nodes are added to the network. Can be used to prompt for 
 the location of input data for instance.
}%end item
  \end{itemize}
}
\end{itemize}
}%end item
\divideents{destroy}
\item{\vskip -1.9ex 
\membername{destroy}
{\tt public abstract void {\bf destroy}(  )
\label{l491}\label{l492}}%end signature
\begin{itemize}
\sld
\item{
\sld
{\bf Usage}
  \begin{itemize}\isep
   \item{
Tear-down housekeeping for when the node is removed from the graph.
}%end item
  \end{itemize}
}
\end{itemize}
}%end item
\divideents{fold}
\item{\vskip -1.9ex 
\membername{fold}
{\tt public void {\bf fold}( {\tt int } {\bf foldNumber},
{\tt int } {\bf folds} )
\label{l493}\label{l494}}%end signature
}%end item
\divideents{getData}
\item{\vskip -1.9ex 
\membername{getData}
{\tt public PartitionableMatrix {\bf getData}(  )
\label{l495}\label{l496}}%end signature
\begin{itemize}
\sld
\item{
\sld
{\bf Usage}
  \begin{itemize}\isep
   \item{
Matrix of data to be passed through the network.
}%end item
  \end{itemize}
}
\item{{\bf Returns} - 
matrix of data values 
}%end item
\end{itemize}
}%end item
\divideents{getTargets}
\item{\vskip -1.9ex 
\membername{getTargets}
{\tt public PartitionableMatrix {\bf getTargets}(  )
\label{l497}\label{l498}}%end signature
\begin{itemize}
\sld
\item{
\sld
{\bf Usage}
  \begin{itemize}\isep
   \item{
Matrix of target test data
}%end item
  \end{itemize}
}
\item{{\bf Returns} - 
matrix of target values 
}%end item
\end{itemize}
}%end item
\divideents{recreate}
\item{\vskip -1.9ex 
\membername{recreate}
{\tt public abstract void {\bf recreate}(  )
\label{l499}\label{l500}}%end signature
\begin{itemize}
\sld
\item{
\sld
{\bf Usage}
  \begin{itemize}\isep
   \item{
Called when configuration data is already in memory and the user need not
 be promted for it again.
}%end item
  \end{itemize}
}
\end{itemize}
}%end item
\divideents{setRow}
\item{\vskip -1.9ex 
\membername{setRow}
{\tt public void {\bf setRow}( {\tt int } {\bf row} )
\label{l501}\label{l502}}%end signature
\begin{itemize}
\sld
\item{
\sld
{\bf Usage}
  \begin{itemize}\isep
   \item{
Set the current row of data to use for input. Is fold-sensitive (row N is
 different per fold).
}%end item
  \end{itemize}
}
\item{
\sld
{\bf Parameters}
\sld\isep
  \begin{itemize}
\sld\isep
   \item{
\sld
{\tt row} - The number of the row to seek to}
  \end{itemize}
}%end item
\end{itemize}
}%end item
\divideents{toNetwork}
\item{\vskip -1.9ex 
\membername{toNetwork}
{\tt public NeuralNetwork {\bf toNetwork}(  )
\label{l503}\label{l504}}%end signature
\begin{itemize}
\sld
\item{
\sld
{\bf Usage}
  \begin{itemize}\isep
   \item{
Sends data to the network.
}%end item
  \end{itemize}
}
\item{{\bf Returns} - 
Itself. 
}%end item
\end{itemize}
}%end item
\divideents{toString}
\item{\vskip -1.9ex 
\membername{toString}
{\tt public String {\bf toString}(  )
\label{l505}\label{l506}}%end signature
}%end item
\end{itemize}
}
\startsubsubsection{Methods inherited from class {\tt uk.ac.ic.doc.neuralnets.graph.neural.NeuralNetwork}}{
\par{\small 
\refdefined{l226}\vskip -2em
\begin{itemize}
\item{\vskip -1.9ex 
\membername{connect}
{\tt public Node {\bf connect}( {\tt uk.ac.ic.doc.neuralnets.graph.neural.NetworkBridge } {\bf e} )
}%end signature
}%end item
\divideents{getIncoming}
\item{\vskip -1.9ex 
\membername{getIncoming}
{\tt public Collection {\bf getIncoming}(  )
}%end signature
}%end item
\divideents{getMetadata}
\item{\vskip -1.9ex 
\membername{getMetadata}
{\tt public String {\bf getMetadata}( {\tt java.lang.String } {\bf key} )
}%end signature
}%end item
\divideents{getOutgoing}
\item{\vskip -1.9ex 
\membername{getOutgoing}
{\tt public Collection {\bf getOutgoing}(  )
}%end signature
}%end item
\divideents{getTicks}
\item{\vskip -1.9ex 
\membername{getTicks}
{\tt public int {\bf getTicks}(  )
}%end signature
}%end item
\divideents{getX}
\item{\vskip -1.9ex 
\membername{getX}
{\tt public int {\bf getX}(  )
}%end signature
}%end item
\divideents{getY}
\item{\vskip -1.9ex 
\membername{getY}
{\tt public int {\bf getY}(  )
}%end signature
}%end item
\divideents{getZ}
\item{\vskip -1.9ex 
\membername{getZ}
{\tt public int {\bf getZ}(  )
}%end signature
}%end item
\divideents{resetTicks}
\item{\vskip -1.9ex 
\membername{resetTicks}
{\tt public void {\bf resetTicks}(  )
}%end signature
}%end item
\divideents{setMetadata}
\item{\vskip -1.9ex 
\membername{setMetadata}
{\tt public Node {\bf setMetadata}( {\tt java.lang.String } {\bf key},
{\tt java.lang.String } {\bf item} )
}%end signature
}%end item
\divideents{setPos}
\item{\vskip -1.9ex 
\membername{setPos}
{\tt public void {\bf setPos}( {\tt int } {\bf x},
{\tt int } {\bf y},
{\tt int } {\bf z} )
}%end signature
}%end item
\divideents{tick}
\item{\vskip -1.9ex 
\membername{tick}
{\tt public Node {\bf tick}(  )
}%end signature
}%end item
\divideents{type}
\item{\vskip -1.9ex 
\membername{type}
{\tt protected String {\bf type}(  )
}%end signature
}%end item
\end{itemize}
}}
\startsubsubsection{Methods inherited from class {\tt uk.ac.ic.doc.neuralnets.graph.Graph}}{
\par{\small 
\refdefined{l507}\vskip -2em
\begin{itemize}
\item{\vskip -1.9ex 
\membername{addAllNodes}
{\tt public Graph {\bf addAllNodes}( {\tt java.util.Collection } {\bf ns} )
}%end signature
\begin{itemize}
\sld
\item{
\sld
{\bf Usage}
  \begin{itemize}\isep
   \item{
Adds a collection of nodes to the graph, only if that collection doesn't
 contain itself.
}%end item
  \end{itemize}
}
\item{
\sld
{\bf Parameters}
\sld\isep
  \begin{itemize}
\sld\isep
   \item{
\sld
{\tt ns} - Collection of nodes to add.}
  \end{itemize}
}%end item
\item{{\bf Returns} - 
Itself with the nodes added or not added. 
}%end item
\end{itemize}
}%end item
\divideents{addEdge}
\item{\vskip -1.9ex 
\membername{addEdge}
{\tt public Graph {\bf addEdge}( {\tt uk.ac.ic.doc.neuralnets.graph.Edge } {\bf e} )
}%end signature
\begin{itemize}
\sld
\item{
\sld
{\bf Usage}
  \begin{itemize}\isep
   \item{
Adds an edge to the graph and adds its start and end nodes to the graph.
}%end item
  \end{itemize}
}
\item{
\sld
{\bf Parameters}
\sld\isep
  \begin{itemize}
\sld\isep
   \item{
\sld
{\tt e} - Edge to add.}
  \end{itemize}
}%end item
\item{{\bf Returns} - 
Itself 
}%end item
\end{itemize}
}%end item
\divideents{addNode}
\item{\vskip -1.9ex 
\membername{addNode}
{\tt public Graph {\bf addNode}( {\tt uk.ac.ic.doc.neuralnets.graph.Node } {\bf n} )
}%end signature
\begin{itemize}
\sld
\item{
\sld
{\bf Usage}
  \begin{itemize}\isep
   \item{
Adds input node to the graph as long as input node is not itself, returns
 itself.
}%end item
  \end{itemize}
}
\item{
\sld
{\bf Parameters}
\sld\isep
  \begin{itemize}
\sld\isep
   \item{
\sld
{\tt n} - Node to add.}
  \end{itemize}
}%end item
\item{{\bf Returns} - 
Itself with the node added or not added. 
}%end item
\end{itemize}
}%end item
\divideents{forEachEdge}
\item{\vskip -1.9ex 
\membername{forEachEdge}
{\tt public Graph {\bf forEachEdge}( {\tt uk.ac.ic.doc.neuralnets.graph.Graph.Command } {\bf c} )
}%end signature
\begin{itemize}
\sld
\item{
\sld
{\bf Usage}
  \begin{itemize}\isep
   \item{
Conducts a command on each edge within the graph.
}%end item
  \end{itemize}
}
\item{
\sld
{\bf Parameters}
\sld\isep
  \begin{itemize}
\sld\isep
   \item{
\sld
{\tt c} - Command to execute.}
  \end{itemize}
}%end item
\item{{\bf Returns} - 
Itself. 
}%end item
\end{itemize}
}%end item
\divideents{forEachNode}
\item{\vskip -1.9ex 
\membername{forEachNode}
{\tt public Graph {\bf forEachNode}( {\tt uk.ac.ic.doc.neuralnets.graph.Graph.Command } {\bf c} )
}%end signature
\begin{itemize}
\sld
\item{
\sld
{\bf Usage}
  \begin{itemize}\isep
   \item{
Conducts a command on each node within the graph.
}%end item
  \end{itemize}
}
\item{
\sld
{\bf Parameters}
\sld\isep
  \begin{itemize}
\sld\isep
   \item{
\sld
{\tt c} - Command to execute.}
  \end{itemize}
}%end item
\item{{\bf Returns} - 
Itself. 
}%end item
\end{itemize}
}%end item
\divideents{getEdges}
\item{\vskip -1.9ex 
\membername{getEdges}
{\tt public Collection {\bf getEdges}(  )
}%end signature
\begin{itemize}
\sld
\item{
\sld
{\bf Usage}
  \begin{itemize}\isep
   \item{
Gets the edges from within.
}%end item
  \end{itemize}
}
\item{{\bf Returns} - 
The edges. 
}%end item
\end{itemize}
}%end item
\divideents{getFreshID}
\item{\vskip -1.9ex 
\membername{getFreshID}
{\tt public void {\bf getFreshID}(  )
}%end signature
\begin{itemize}
\sld
\item{
\sld
{\bf Usage}
  \begin{itemize}\isep
   \item{
Sets the id of the object to a new fresh id.
}%end item
  \end{itemize}
}
\end{itemize}
}%end item
\divideents{getID}
\item{\vskip -1.9ex 
\membername{getID}
{\tt public int {\bf getID}(  )
}%end signature
\begin{itemize}
\sld
\item{
\sld
{\bf Usage}
  \begin{itemize}\isep
   \item{
Gets the id of the object.
}%end item
  \end{itemize}
}
\item{{\bf Returns} - 
The id. 
}%end item
\end{itemize}
}%end item
\divideents{getNodes}
\item{\vskip -1.9ex 
\membername{getNodes}
{\tt public Collection {\bf getNodes}(  )
}%end signature
\begin{itemize}
\sld
\item{
\sld
{\bf Usage}
  \begin{itemize}\isep
   \item{
Gets the nodes from within.
}%end item
  \end{itemize}
}
\item{{\bf Returns} - 
The nodes. 
}%end item
\end{itemize}
}%end item
\divideents{merge}
\item{\vskip -1.9ex 
\membername{merge}
{\tt public Graph {\bf merge}( {\tt uk.ac.ic.doc.neuralnets.graph.Graph } {\bf o} )
}%end signature
\begin{itemize}
\sld
\item{
\sld
{\bf Usage}
  \begin{itemize}\isep
   \item{
Merges one graph with its self, as all the edges and nodes.
}%end item
  \end{itemize}
}
\item{
\sld
{\bf Parameters}
\sld\isep
  \begin{itemize}
\sld\isep
   \item{
\sld
{\tt o} - Graph to merge with.}
  \end{itemize}
}%end item
\item{{\bf Returns} - 
Itself 
}%end item
\end{itemize}
}%end item
\divideents{setID}
\item{\vskip -1.9ex 
\membername{setID}
{\tt public void {\bf setID}( {\tt int } {\bf id} )
}%end signature
\begin{itemize}
\sld
\item{
\sld
{\bf Usage}
  \begin{itemize}\isep
   \item{
Sets the id of the object to parameter.
}%end item
  \end{itemize}
}
\item{
\sld
{\bf Parameters}
\sld\isep
  \begin{itemize}
\sld\isep
   \item{
\sld
{\tt int} - New id.}
  \end{itemize}
}%end item
\end{itemize}
}%end item
\divideents{toString}
\item{\vskip -1.9ex 
\membername{toString}
{\tt public String {\bf toString}(  )
}%end signature
}%end item
\divideents{type}
\item{\vskip -1.9ex 
\membername{type}
{\tt protected String {\bf type}(  )
}%end signature
\begin{itemize}
\sld
\item{
\sld
{\bf Usage}
  \begin{itemize}\isep
   \item{
Returns the object type.
}%end item
  \end{itemize}
}
\item{{\bf Returns} - 
Object type. 
}%end item
\end{itemize}
}%end item
\end{itemize}
}}
}
\startsection{Class}{IONeurone}{l482}{%
{\small Purely a class to "mark" a neurone as being for I$/$O purposes.}
\vskip .1in 
\startsubsubsection{Declaration}{
\fbox{\vbox{
\hbox{\vbox{\small public 
class 
IONeurone}}
\noindent\hbox{\vbox{{\bf extends} uk.ac.ic.doc.neuralnets.graph.neural.Neurone}}
}}}
\startsubsubsection{Serializable Fields}{
\begin{itemize}
\item{
private boolean concrete\begin{itemize}\item{\vskip -.9ex }\end{itemize}
}
\end{itemize}
}
\startsubsubsection{Constructors}{
\vskip -2em
\begin{itemize}
\item{\vskip -1.9ex 
\membername{IONeurone}
{\tt public {\bf IONeurone}(  )
\label{l508}\label{l509}}%end signature
}%end item
\end{itemize}
}
\startsubsubsection{Methods}{
\vskip -2em
\begin{itemize}
\item{\vskip -1.9ex 
\membername{getCharge}
{\tt public double {\bf getCharge}(  )
\label{l510}\label{l511}}%end signature
}%end item
\divideents{toString}
\item{\vskip -1.9ex 
\membername{toString}
{\tt public String {\bf toString}(  )
\label{l512}\label{l513}}%end signature
}%end item
\end{itemize}
}
\startsubsubsection{Methods inherited from class {\tt uk.ac.ic.doc.neuralnets.graph.neural.Neurone}}{
\par{\small 
\refdefined{l225}\vskip -2em
\begin{itemize}
\item{\vskip -1.9ex 
\membername{charge}
{\tt public Neurone {\bf charge}( {\tt double } {\bf amt} )
}%end signature
}%end item
\divideents{getCharge}
\item{\vskip -1.9ex 
\membername{getCharge}
{\tt public double {\bf getCharge}(  )
}%end signature
}%end item
\divideents{getCurrentCharge}
\item{\vskip -1.9ex 
\membername{getCurrentCharge}
{\tt public Double {\bf getCurrentCharge}(  )
}%end signature
}%end item
\divideents{getEdgeDecoration}
\item{\vskip -1.9ex 
\membername{getEdgeDecoration}
{\tt public EdgeDecoration {\bf getEdgeDecoration}(  )
}%end signature
}%end item
\divideents{getFreshID}
\item{\vskip -1.9ex 
\membername{getFreshID}
{\tt public void {\bf getFreshID}(  )
}%end signature
}%end item
\divideents{getID}
\item{\vskip -1.9ex 
\membername{getID}
{\tt public int {\bf getID}(  )
}%end signature
}%end item
\divideents{getSquashFunction}
\item{\vskip -1.9ex 
\membername{getSquashFunction}
{\tt public ASTExpression {\bf getSquashFunction}(  )
}%end signature
}%end item
\divideents{getTrigger}
\item{\vskip -1.9ex 
\membername{getTrigger}
{\tt public double {\bf getTrigger}(  )
}%end signature
}%end item
\divideents{reset}
\item{\vskip -1.9ex 
\membername{reset}
{\tt public void {\bf reset}(  )
}%end signature
}%end item
\divideents{setCharge}
\item{\vskip -1.9ex 
\membername{setCharge}
{\tt public void {\bf setCharge}( {\tt double } {\bf charge} )
}%end signature
}%end item
\divideents{setEdgeDecoration}
\item{\vskip -1.9ex 
\membername{setEdgeDecoration}
{\tt public void {\bf setEdgeDecoration}( {\tt uk.ac.ic.doc.neuralnets.graph.neural.EdgeDecoration } {\bf ed} )
}%end signature
}%end item
\divideents{setID}
\item{\vskip -1.9ex 
\membername{setID}
{\tt public void {\bf setID}( {\tt int } {\bf id} )
}%end signature
}%end item
\divideents{setInitialCharge}
\item{\vskip -1.9ex 
\membername{setInitialCharge}
{\tt public void {\bf setInitialCharge}( {\tt uk.ac.ic.doc.neuralnets.expressions.ast.ASTExpression } {\bf c} )
}%end signature
}%end item
\divideents{setSquashFunction}
\item{\vskip -1.9ex 
\membername{setSquashFunction}
{\tt public void {\bf setSquashFunction}( {\tt uk.ac.ic.doc.neuralnets.expressions.ast.ASTExpression } {\bf e} )
}%end signature
}%end item
\divideents{setTrigger}
\item{\vskip -1.9ex 
\membername{setTrigger}
{\tt public void {\bf setTrigger}( {\tt uk.ac.ic.doc.neuralnets.expressions.ast.ASTExpression } {\bf t} )
}%end signature
}%end item
\divideents{setTrigger}
\item{\vskip -1.9ex 
\membername{setTrigger}
{\tt public void {\bf setTrigger}( {\tt double } {\bf d} )
}%end signature
}%end item
\divideents{tick}
\item{\vskip -1.9ex 
\membername{tick}
{\tt public Node {\bf tick}(  )
}%end signature
\begin{itemize}
\sld
\item{
\sld
{\bf Usage}
  \begin{itemize}\isep
   \item{
Ticks the neurone one step forward. Fires the neurone is appropriate.
}%end item
  \end{itemize}
}
\item{{\bf Returns} - 
Itself. 
}%end item
\end{itemize}
}%end item
\divideents{toString}
\item{\vskip -1.9ex 
\membername{toString}
{\tt public String {\bf toString}(  )
}%end signature
}%end item
\end{itemize}
}}
\startsubsubsection{Methods inherited from class {\tt uk.ac.ic.doc.neuralnets.graph.neural.NodeBase}}{
\par{\small 
\refdefined{l514}\vskip -2em
\begin{itemize}
\item{\vskip -1.9ex 
\membername{connect}
{\tt public Node {\bf connect}( {\tt uk.ac.ic.doc.neuralnets.graph.Edge } {\bf e} )
}%end signature
\begin{itemize}
\sld
\item{
\sld
{\bf Usage}
  \begin{itemize}\isep
   \item{
Connect this node up with the input edge.
}%end item
  \end{itemize}
}
\end{itemize}
}%end item
\divideents{getIncoming}
\item{\vskip -1.9ex 
\membername{getIncoming}
{\tt public Collection {\bf getIncoming}(  )
}%end signature
\begin{itemize}
\sld
\item{
\sld
{\bf Usage}
  \begin{itemize}\isep
   \item{
Get incoming edges.
}%end item
  \end{itemize}
}
\end{itemize}
}%end item
\divideents{getMetadata}
\item{\vskip -1.9ex 
\membername{getMetadata}
{\tt public String {\bf getMetadata}( {\tt java.lang.String } {\bf key} )
}%end signature
\begin{itemize}
\sld
\item{
\sld
{\bf Usage}
  \begin{itemize}\isep
   \item{
Returns the meta data for the key input.
}%end item
  \end{itemize}
}
\item{
\sld
{\bf Parameters}
\sld\isep
  \begin{itemize}
\sld\isep
   \item{
\sld
{\tt key} - To look for.}
  \end{itemize}
}%end item
\item{{\bf Returns} - 
item Found. 
}%end item
\end{itemize}
}%end item
\divideents{getOutgoing}
\item{\vskip -1.9ex 
\membername{getOutgoing}
{\tt public Collection {\bf getOutgoing}(  )
}%end signature
\begin{itemize}
\sld
\item{
\sld
{\bf Usage}
  \begin{itemize}\isep
   \item{
Get outgoing edges.
}%end item
  \end{itemize}
}
\end{itemize}
}%end item
\divideents{getX}
\item{\vskip -1.9ex 
\membername{getX}
{\tt public int {\bf getX}(  )
}%end signature
\begin{itemize}
\sld
\item{
\sld
{\bf Usage}
  \begin{itemize}\isep
   \item{
Returns the position of the node on the x axis.
}%end item
  \end{itemize}
}
\item{{\bf Returns} - 
x axis position. 
}%end item
\end{itemize}
}%end item
\divideents{getY}
\item{\vskip -1.9ex 
\membername{getY}
{\tt public int {\bf getY}(  )
}%end signature
\begin{itemize}
\sld
\item{
\sld
{\bf Usage}
  \begin{itemize}\isep
   \item{
Returns the position of the node on the y axis.
}%end item
  \end{itemize}
}
\item{{\bf Returns} - 
y axis position. 
}%end item
\end{itemize}
}%end item
\divideents{getZ}
\item{\vskip -1.9ex 
\membername{getZ}
{\tt public int {\bf getZ}(  )
}%end signature
\begin{itemize}
\sld
\item{
\sld
{\bf Usage}
  \begin{itemize}\isep
   \item{
Returns the position of the node on the z axis.
}%end item
  \end{itemize}
}
\item{{\bf Returns} - 
z axis position. 
}%end item
\end{itemize}
}%end item
\divideents{setMetadata}
\item{\vskip -1.9ex 
\membername{setMetadata}
{\tt public Node {\bf setMetadata}( {\tt java.lang.String } {\bf key},
{\tt java.lang.String } {\bf item} )
}%end signature
\begin{itemize}
\sld
\item{
\sld
{\bf Usage}
  \begin{itemize}\isep
   \item{
Set meta data for the object.
}%end item
  \end{itemize}
}
\item{
\sld
{\bf Parameters}
\sld\isep
  \begin{itemize}
\sld\isep
   \item{
\sld
{\tt key} - String key}
   \item{
\sld
{\tt item} - String item}
  \end{itemize}
}%end item
\end{itemize}
}%end item
\divideents{setPos}
\item{\vskip -1.9ex 
\membername{setPos}
{\tt public void {\bf setPos}( {\tt int } {\bf x},
{\tt int } {\bf y},
{\tt int } {\bf z} )
}%end signature
\begin{itemize}
\sld
\item{
\sld
{\bf Usage}
  \begin{itemize}\isep
   \item{
Sets the position of the node.
}%end item
  \end{itemize}
}
\item{
\sld
{\bf Parameters}
\sld\isep
  \begin{itemize}
\sld\isep
   \item{
\sld
{\tt x} - Position on x axis.}
   \item{
\sld
{\tt y} - Position on y axis.}
   \item{
\sld
{\tt z} - Position on z axis.}
  \end{itemize}
}%end item
\end{itemize}
}%end item
\divideents{setX}
\item{\vskip -1.9ex 
\membername{setX}
{\tt public void {\bf setX}( {\tt int } {\bf x} )
}%end signature
\begin{itemize}
\sld
\item{
\sld
{\bf Usage}
  \begin{itemize}\isep
   \item{
Sets the position of the node on the x axis.
}%end item
  \end{itemize}
}
\item{
\sld
{\bf Parameters}
\sld\isep
  \begin{itemize}
\sld\isep
   \item{
\sld
{\tt x} - Position on x axis.}
  \end{itemize}
}%end item
\end{itemize}
}%end item
\divideents{setY}
\item{\vskip -1.9ex 
\membername{setY}
{\tt public void {\bf setY}( {\tt int } {\bf y} )
}%end signature
\begin{itemize}
\sld
\item{
\sld
{\bf Usage}
  \begin{itemize}\isep
   \item{
Sets the position of the node on the y axis.
}%end item
  \end{itemize}
}
\item{
\sld
{\bf Parameters}
\sld\isep
  \begin{itemize}
\sld\isep
   \item{
\sld
{\tt y} - Position on y axis.}
  \end{itemize}
}%end item
\end{itemize}
}%end item
\divideents{setZ}
\item{\vskip -1.9ex 
\membername{setZ}
{\tt public void {\bf setZ}( {\tt int } {\bf z} )
}%end signature
\begin{itemize}
\sld
\item{
\sld
{\bf Usage}
  \begin{itemize}\isep
   \item{
Sets the position of the node on the z axis.
}%end item
  \end{itemize}
}
\item{
\sld
{\bf Parameters}
\sld\isep
  \begin{itemize}
\sld\isep
   \item{
\sld
{\tt z} - Position on z axis.}
  \end{itemize}
}%end item
\end{itemize}
}%end item
\divideents{tick}
\item{\vskip -1.9ex 
\membername{tick}
{\tt public abstract Node {\bf tick}(  )
}%end signature
}%end item
\divideents{toString}
\item{\vskip -1.9ex 
\membername{toString}
{\tt public abstract String {\bf toString}(  )
}%end signature
}%end item
\end{itemize}
}}
}
\startsection{Class}{OutputNode}{l483}{%
{\small OutputNodes are the default method for harvesting data from a neural network 
 for use in external cases.\bl  Each time an output node fires the abstract fire method is called.}
\vskip .1in 
\startsubsubsection{Declaration}{
\fbox{\vbox{
\hbox{\vbox{\small public abstract 
class 
OutputNode}}
\noindent\hbox{\vbox{{\bf extends} uk.ac.ic.doc.neuralnets.graph.neural.NeuralNetwork}}
\noindent\hbox{\vbox{{\bf implements} 
uk.ac.ic.doc.neuralnets.util.plugins.Plugin}}
}}}
\startsubsubsection{Constructors}{
\vskip -2em
\begin{itemize}
\item{\vskip -1.9ex 
\membername{OutputNode}
{\tt public {\bf OutputNode}(  )
\label{l515}\label{l516}}%end signature
\begin{itemize}
\sld
\item{
\sld
{\bf Usage}
  \begin{itemize}\isep
   \item{
Create the empty output node. A call to toNetwork should be made soon.
}%end item
  \end{itemize}
}
\end{itemize}
}%end item
\divideents{OutputNode}
\item{\vskip -1.9ex 
\membername{OutputNode}
{\tt public {\bf OutputNode}( {\tt int } {\bf nodes} )
\label{l517}\label{l518}}%end signature
\begin{itemize}
\sld
\item{
\sld
{\bf Usage}
  \begin{itemize}\isep
   \item{
Create the output nodes
}%end item
  \end{itemize}
}
\item{
\sld
{\bf Parameters}
\sld\isep
  \begin{itemize}
\sld\isep
   \item{
\sld
{\tt nodes} - - the number of nodes to create}
  \end{itemize}
}%end item
\end{itemize}
}%end item
\end{itemize}
}
\startsubsubsection{Methods}{
\vskip -2em
\begin{itemize}
\item{\vskip -1.9ex 
\membername{destroy}
{\tt public abstract void {\bf destroy}(  )
\label{l519}\label{l520}}%end signature
\begin{itemize}
\sld
\item{
\sld
{\bf Usage}
  \begin{itemize}\isep
   \item{
Tear-down housekeeping for when the node is removed from the graph.
}%end item
  \end{itemize}
}
\end{itemize}
}%end item
\divideents{fire}
\item{\vskip -1.9ex 
\membername{fire}
{\tt protected abstract void {\bf fire}( {\tt int } {\bf n},
{\tt java.lang.Double } {\bf amt} )
\label{l521}\label{l522}}%end signature
\begin{itemize}
\sld
\item{
\sld
{\bf Usage}
  \begin{itemize}\isep
   \item{
Called when an output node fires.
}%end item
  \end{itemize}
}
\item{
\sld
{\bf Parameters}
\sld\isep
  \begin{itemize}
\sld\isep
   \item{
\sld
{\tt n} - the index of the node.}
   \item{
\sld
{\tt amt} - the charge passed through.}
  \end{itemize}
}%end item
\end{itemize}
}%end item
\divideents{recreate}
\item{\vskip -1.9ex 
\membername{recreate}
{\tt public abstract void {\bf recreate}(  )
\label{l523}\label{l524}}%end signature
\begin{itemize}
\sld
\item{
\sld
{\bf Usage}
  \begin{itemize}\isep
   \item{
Called when configuration data is already in memory and the user need not
 be promted for it again.
}%end item
  \end{itemize}
}
\end{itemize}
}%end item
\divideents{setNodes}
\item{\vskip -1.9ex 
\membername{setNodes}
{\tt protected abstract void {\bf setNodes}( {\tt int } {\bf n} )
\label{l525}\label{l526}}%end signature
\begin{itemize}
\sld
\item{
\sld
{\bf Usage}
  \begin{itemize}\isep
   \item{
Configures the nodes in the OutputNode after they've been added to the 
 network.
}%end item
  \end{itemize}
}
\item{
\sld
{\bf Parameters}
\sld\isep
  \begin{itemize}
\sld\isep
   \item{
\sld
{\tt n} - - the}
  \end{itemize}
}%end item
\end{itemize}
}%end item
\divideents{toNetwork}
\item{\vskip -1.9ex 
\membername{toNetwork}
{\tt public NeuralNetwork {\bf toNetwork}( {\tt int } {\bf nodes} )
\label{l527}\label{l528}}%end signature
\begin{itemize}
\sld
\item{
\sld
{\bf Usage}
  \begin{itemize}\isep
   \item{
Sends data to the network.
}%end item
  \end{itemize}
}
\item{{\bf Returns} - 
Itself. 
}%end item
\end{itemize}
}%end item
\divideents{toString}
\item{\vskip -1.9ex 
\membername{toString}
{\tt public String {\bf toString}(  )
\label{l529}\label{l530}}%end signature
}%end item
\end{itemize}
}
\startsubsubsection{Methods inherited from class {\tt uk.ac.ic.doc.neuralnets.graph.neural.NeuralNetwork}}{
\par{\small 
\refdefined{l226}\vskip -2em
\begin{itemize}
\item{\vskip -1.9ex 
\membername{connect}
{\tt public Node {\bf connect}( {\tt uk.ac.ic.doc.neuralnets.graph.neural.NetworkBridge } {\bf e} )
}%end signature
}%end item
\divideents{getIncoming}
\item{\vskip -1.9ex 
\membername{getIncoming}
{\tt public Collection {\bf getIncoming}(  )
}%end signature
}%end item
\divideents{getMetadata}
\item{\vskip -1.9ex 
\membername{getMetadata}
{\tt public String {\bf getMetadata}( {\tt java.lang.String } {\bf key} )
}%end signature
}%end item
\divideents{getOutgoing}
\item{\vskip -1.9ex 
\membername{getOutgoing}
{\tt public Collection {\bf getOutgoing}(  )
}%end signature
}%end item
\divideents{getTicks}
\item{\vskip -1.9ex 
\membername{getTicks}
{\tt public int {\bf getTicks}(  )
}%end signature
}%end item
\divideents{getX}
\item{\vskip -1.9ex 
\membername{getX}
{\tt public int {\bf getX}(  )
}%end signature
}%end item
\divideents{getY}
\item{\vskip -1.9ex 
\membername{getY}
{\tt public int {\bf getY}(  )
}%end signature
}%end item
\divideents{getZ}
\item{\vskip -1.9ex 
\membername{getZ}
{\tt public int {\bf getZ}(  )
}%end signature
}%end item
\divideents{resetTicks}
\item{\vskip -1.9ex 
\membername{resetTicks}
{\tt public void {\bf resetTicks}(  )
}%end signature
}%end item
\divideents{setMetadata}
\item{\vskip -1.9ex 
\membername{setMetadata}
{\tt public Node {\bf setMetadata}( {\tt java.lang.String } {\bf key},
{\tt java.lang.String } {\bf item} )
}%end signature
}%end item
\divideents{setPos}
\item{\vskip -1.9ex 
\membername{setPos}
{\tt public void {\bf setPos}( {\tt int } {\bf x},
{\tt int } {\bf y},
{\tt int } {\bf z} )
}%end signature
}%end item
\divideents{tick}
\item{\vskip -1.9ex 
\membername{tick}
{\tt public Node {\bf tick}(  )
}%end signature
}%end item
\divideents{type}
\item{\vskip -1.9ex 
\membername{type}
{\tt protected String {\bf type}(  )
}%end signature
}%end item
\end{itemize}
}}
\startsubsubsection{Methods inherited from class {\tt uk.ac.ic.doc.neuralnets.graph.Graph}}{
\par{\small 
\refdefined{l507}\vskip -2em
\begin{itemize}
\item{\vskip -1.9ex 
\membername{addAllNodes}
{\tt public Graph {\bf addAllNodes}( {\tt java.util.Collection } {\bf ns} )
}%end signature
\begin{itemize}
\sld
\item{
\sld
{\bf Usage}
  \begin{itemize}\isep
   \item{
Adds a collection of nodes to the graph, only if that collection doesn't
 contain itself.
}%end item
  \end{itemize}
}
\item{
\sld
{\bf Parameters}
\sld\isep
  \begin{itemize}
\sld\isep
   \item{
\sld
{\tt ns} - Collection of nodes to add.}
  \end{itemize}
}%end item
\item{{\bf Returns} - 
Itself with the nodes added or not added. 
}%end item
\end{itemize}
}%end item
\divideents{addEdge}
\item{\vskip -1.9ex 
\membername{addEdge}
{\tt public Graph {\bf addEdge}( {\tt uk.ac.ic.doc.neuralnets.graph.Edge } {\bf e} )
}%end signature
\begin{itemize}
\sld
\item{
\sld
{\bf Usage}
  \begin{itemize}\isep
   \item{
Adds an edge to the graph and adds its start and end nodes to the graph.
}%end item
  \end{itemize}
}
\item{
\sld
{\bf Parameters}
\sld\isep
  \begin{itemize}
\sld\isep
   \item{
\sld
{\tt e} - Edge to add.}
  \end{itemize}
}%end item
\item{{\bf Returns} - 
Itself 
}%end item
\end{itemize}
}%end item
\divideents{addNode}
\item{\vskip -1.9ex 
\membername{addNode}
{\tt public Graph {\bf addNode}( {\tt uk.ac.ic.doc.neuralnets.graph.Node } {\bf n} )
}%end signature
\begin{itemize}
\sld
\item{
\sld
{\bf Usage}
  \begin{itemize}\isep
   \item{
Adds input node to the graph as long as input node is not itself, returns
 itself.
}%end item
  \end{itemize}
}
\item{
\sld
{\bf Parameters}
\sld\isep
  \begin{itemize}
\sld\isep
   \item{
\sld
{\tt n} - Node to add.}
  \end{itemize}
}%end item
\item{{\bf Returns} - 
Itself with the node added or not added. 
}%end item
\end{itemize}
}%end item
\divideents{forEachEdge}
\item{\vskip -1.9ex 
\membername{forEachEdge}
{\tt public Graph {\bf forEachEdge}( {\tt uk.ac.ic.doc.neuralnets.graph.Graph.Command } {\bf c} )
}%end signature
\begin{itemize}
\sld
\item{
\sld
{\bf Usage}
  \begin{itemize}\isep
   \item{
Conducts a command on each edge within the graph.
}%end item
  \end{itemize}
}
\item{
\sld
{\bf Parameters}
\sld\isep
  \begin{itemize}
\sld\isep
   \item{
\sld
{\tt c} - Command to execute.}
  \end{itemize}
}%end item
\item{{\bf Returns} - 
Itself. 
}%end item
\end{itemize}
}%end item
\divideents{forEachNode}
\item{\vskip -1.9ex 
\membername{forEachNode}
{\tt public Graph {\bf forEachNode}( {\tt uk.ac.ic.doc.neuralnets.graph.Graph.Command } {\bf c} )
}%end signature
\begin{itemize}
\sld
\item{
\sld
{\bf Usage}
  \begin{itemize}\isep
   \item{
Conducts a command on each node within the graph.
}%end item
  \end{itemize}
}
\item{
\sld
{\bf Parameters}
\sld\isep
  \begin{itemize}
\sld\isep
   \item{
\sld
{\tt c} - Command to execute.}
  \end{itemize}
}%end item
\item{{\bf Returns} - 
Itself. 
}%end item
\end{itemize}
}%end item
\divideents{getEdges}
\item{\vskip -1.9ex 
\membername{getEdges}
{\tt public Collection {\bf getEdges}(  )
}%end signature
\begin{itemize}
\sld
\item{
\sld
{\bf Usage}
  \begin{itemize}\isep
   \item{
Gets the edges from within.
}%end item
  \end{itemize}
}
\item{{\bf Returns} - 
The edges. 
}%end item
\end{itemize}
}%end item
\divideents{getFreshID}
\item{\vskip -1.9ex 
\membername{getFreshID}
{\tt public void {\bf getFreshID}(  )
}%end signature
\begin{itemize}
\sld
\item{
\sld
{\bf Usage}
  \begin{itemize}\isep
   \item{
Sets the id of the object to a new fresh id.
}%end item
  \end{itemize}
}
\end{itemize}
}%end item
\divideents{getID}
\item{\vskip -1.9ex 
\membername{getID}
{\tt public int {\bf getID}(  )
}%end signature
\begin{itemize}
\sld
\item{
\sld
{\bf Usage}
  \begin{itemize}\isep
   \item{
Gets the id of the object.
}%end item
  \end{itemize}
}
\item{{\bf Returns} - 
The id. 
}%end item
\end{itemize}
}%end item
\divideents{getNodes}
\item{\vskip -1.9ex 
\membername{getNodes}
{\tt public Collection {\bf getNodes}(  )
}%end signature
\begin{itemize}
\sld
\item{
\sld
{\bf Usage}
  \begin{itemize}\isep
   \item{
Gets the nodes from within.
}%end item
  \end{itemize}
}
\item{{\bf Returns} - 
The nodes. 
}%end item
\end{itemize}
}%end item
\divideents{merge}
\item{\vskip -1.9ex 
\membername{merge}
{\tt public Graph {\bf merge}( {\tt uk.ac.ic.doc.neuralnets.graph.Graph } {\bf o} )
}%end signature
\begin{itemize}
\sld
\item{
\sld
{\bf Usage}
  \begin{itemize}\isep
   \item{
Merges one graph with its self, as all the edges and nodes.
}%end item
  \end{itemize}
}
\item{
\sld
{\bf Parameters}
\sld\isep
  \begin{itemize}
\sld\isep
   \item{
\sld
{\tt o} - Graph to merge with.}
  \end{itemize}
}%end item
\item{{\bf Returns} - 
Itself 
}%end item
\end{itemize}
}%end item
\divideents{setID}
\item{\vskip -1.9ex 
\membername{setID}
{\tt public void {\bf setID}( {\tt int } {\bf id} )
}%end signature
\begin{itemize}
\sld
\item{
\sld
{\bf Usage}
  \begin{itemize}\isep
   \item{
Sets the id of the object to parameter.
}%end item
  \end{itemize}
}
\item{
\sld
{\bf Parameters}
\sld\isep
  \begin{itemize}
\sld\isep
   \item{
\sld
{\tt int} - New id.}
  \end{itemize}
}%end item
\end{itemize}
}%end item
\divideents{toString}
\item{\vskip -1.9ex 
\membername{toString}
{\tt public String {\bf toString}(  )
}%end signature
}%end item
\divideents{type}
\item{\vskip -1.9ex 
\membername{type}
{\tt protected String {\bf type}(  )
}%end signature
\begin{itemize}
\sld
\item{
\sld
{\bf Usage}
  \begin{itemize}\isep
   \item{
Returns the object type.
}%end item
  \end{itemize}
}
\item{{\bf Returns} - 
Object type. 
}%end item
\end{itemize}
}%end item
\end{itemize}
}}
}
\startsection{Class}{ValueReportingOutputNode}{l484}{%
\startsubsubsection{Declaration}{
\fbox{\vbox{
\hbox{\vbox{\small public 
class 
ValueReportingOutputNode}}
\noindent\hbox{\vbox{{\bf extends} uk.ac.ic.doc.neuralnets.graph.neural.io.OutputNode}}
}}}
\startsubsubsection{Serializable Fields}{
\begin{itemize}
\item{
private List values\begin{itemize}\item{\vskip -.9ex }\end{itemize}
}
\end{itemize}
}
\startsubsubsection{Constructors}{
\vskip -2em
\begin{itemize}
\item{\vskip -1.9ex 
\membername{ValueReportingOutputNode}
{\tt public {\bf ValueReportingOutputNode}(  )
\label{l531}\label{l532}}%end signature
}%end item
\end{itemize}
}
\startsubsubsection{Methods}{
\vskip -2em
\begin{itemize}
\item{\vskip -1.9ex 
\membername{destroy}
{\tt public void {\bf destroy}(  )
\label{l533}\label{l534}}%end signature
}%end item
\divideents{fire}
\item{\vskip -1.9ex 
\membername{fire}
{\tt protected void {\bf fire}( {\tt int } {\bf n},
{\tt java.lang.Double } {\bf amt} )
\label{l535}\label{l536}}%end signature
}%end item
\divideents{getName}
\item{\vskip -1.9ex 
\membername{getName}
{\tt public String {\bf getName}(  )
\label{l537}\label{l538}}%end signature
}%end item
\divideents{getValues}
\item{\vskip -1.9ex 
\membername{getValues}
{\tt public List {\bf getValues}(  )
\label{l539}\label{l540}}%end signature
}%end item
\divideents{recreate}
\item{\vskip -1.9ex 
\membername{recreate}
{\tt public void {\bf recreate}(  )
\label{l541}\label{l542}}%end signature
}%end item
\divideents{setNodes}
\item{\vskip -1.9ex 
\membername{setNodes}
{\tt protected void {\bf setNodes}( {\tt int } {\bf n} )
\label{l543}\label{l544}}%end signature
}%end item
\end{itemize}
}
\startsubsubsection{Methods inherited from class {\tt uk.ac.ic.doc.neuralnets.graph.neural.io.OutputNode}}{
\par{\small 
\refdefined{l483}\vskip -2em
\begin{itemize}
\item{\vskip -1.9ex 
\membername{destroy}
{\tt public abstract void {\bf destroy}(  )
}%end signature
\begin{itemize}
\sld
\item{
\sld
{\bf Usage}
  \begin{itemize}\isep
   \item{
Tear-down housekeeping for when the node is removed from the graph.
}%end item
  \end{itemize}
}
\end{itemize}
}%end item
\divideents{fire}
\item{\vskip -1.9ex 
\membername{fire}
{\tt protected abstract void {\bf fire}( {\tt int } {\bf n},
{\tt java.lang.Double } {\bf amt} )
}%end signature
\begin{itemize}
\sld
\item{
\sld
{\bf Usage}
  \begin{itemize}\isep
   \item{
Called when an output node fires.
}%end item
  \end{itemize}
}
\item{
\sld
{\bf Parameters}
\sld\isep
  \begin{itemize}
\sld\isep
   \item{
\sld
{\tt n} - the index of the node.}
   \item{
\sld
{\tt amt} - the charge passed through.}
  \end{itemize}
}%end item
\end{itemize}
}%end item
\divideents{recreate}
\item{\vskip -1.9ex 
\membername{recreate}
{\tt public abstract void {\bf recreate}(  )
}%end signature
\begin{itemize}
\sld
\item{
\sld
{\bf Usage}
  \begin{itemize}\isep
   \item{
Called when configuration data is already in memory and the user need not
 be promted for it again.
}%end item
  \end{itemize}
}
\end{itemize}
}%end item
\divideents{setNodes}
\item{\vskip -1.9ex 
\membername{setNodes}
{\tt protected abstract void {\bf setNodes}( {\tt int } {\bf n} )
}%end signature
\begin{itemize}
\sld
\item{
\sld
{\bf Usage}
  \begin{itemize}\isep
   \item{
Configures the nodes in the OutputNode after they've been added to the 
 network.
}%end item
  \end{itemize}
}
\item{
\sld
{\bf Parameters}
\sld\isep
  \begin{itemize}
\sld\isep
   \item{
\sld
{\tt n} - - the}
  \end{itemize}
}%end item
\end{itemize}
}%end item
\divideents{toNetwork}
\item{\vskip -1.9ex 
\membername{toNetwork}
{\tt public NeuralNetwork {\bf toNetwork}( {\tt int } {\bf nodes} )
}%end signature
\begin{itemize}
\sld
\item{
\sld
{\bf Usage}
  \begin{itemize}\isep
   \item{
Sends data to the network.
}%end item
  \end{itemize}
}
\item{{\bf Returns} - 
Itself. 
}%end item
\end{itemize}
}%end item
\divideents{toString}
\item{\vskip -1.9ex 
\membername{toString}
{\tt public String {\bf toString}(  )
}%end signature
}%end item
\end{itemize}
}}
\startsubsubsection{Methods inherited from class {\tt uk.ac.ic.doc.neuralnets.graph.neural.NeuralNetwork}}{
\par{\small 
\refdefined{l226}\vskip -2em
\begin{itemize}
\item{\vskip -1.9ex 
\membername{connect}
{\tt public Node {\bf connect}( {\tt uk.ac.ic.doc.neuralnets.graph.neural.NetworkBridge } {\bf e} )
}%end signature
}%end item
\divideents{getIncoming}
\item{\vskip -1.9ex 
\membername{getIncoming}
{\tt public Collection {\bf getIncoming}(  )
}%end signature
}%end item
\divideents{getMetadata}
\item{\vskip -1.9ex 
\membername{getMetadata}
{\tt public String {\bf getMetadata}( {\tt java.lang.String } {\bf key} )
}%end signature
}%end item
\divideents{getOutgoing}
\item{\vskip -1.9ex 
\membername{getOutgoing}
{\tt public Collection {\bf getOutgoing}(  )
}%end signature
}%end item
\divideents{getTicks}
\item{\vskip -1.9ex 
\membername{getTicks}
{\tt public int {\bf getTicks}(  )
}%end signature
}%end item
\divideents{getX}
\item{\vskip -1.9ex 
\membername{getX}
{\tt public int {\bf getX}(  )
}%end signature
}%end item
\divideents{getY}
\item{\vskip -1.9ex 
\membername{getY}
{\tt public int {\bf getY}(  )
}%end signature
}%end item
\divideents{getZ}
\item{\vskip -1.9ex 
\membername{getZ}
{\tt public int {\bf getZ}(  )
}%end signature
}%end item
\divideents{resetTicks}
\item{\vskip -1.9ex 
\membername{resetTicks}
{\tt public void {\bf resetTicks}(  )
}%end signature
}%end item
\divideents{setMetadata}
\item{\vskip -1.9ex 
\membername{setMetadata}
{\tt public Node {\bf setMetadata}( {\tt java.lang.String } {\bf key},
{\tt java.lang.String } {\bf item} )
}%end signature
}%end item
\divideents{setPos}
\item{\vskip -1.9ex 
\membername{setPos}
{\tt public void {\bf setPos}( {\tt int } {\bf x},
{\tt int } {\bf y},
{\tt int } {\bf z} )
}%end signature
}%end item
\divideents{tick}
\item{\vskip -1.9ex 
\membername{tick}
{\tt public Node {\bf tick}(  )
}%end signature
}%end item
\divideents{type}
\item{\vskip -1.9ex 
\membername{type}
{\tt protected String {\bf type}(  )
}%end signature
}%end item
\end{itemize}
}}
\startsubsubsection{Methods inherited from class {\tt uk.ac.ic.doc.neuralnets.graph.Graph}}{
\par{\small 
\refdefined{l507}\vskip -2em
\begin{itemize}
\item{\vskip -1.9ex 
\membername{addAllNodes}
{\tt public Graph {\bf addAllNodes}( {\tt java.util.Collection } {\bf ns} )
}%end signature
\begin{itemize}
\sld
\item{
\sld
{\bf Usage}
  \begin{itemize}\isep
   \item{
Adds a collection of nodes to the graph, only if that collection doesn't
 contain itself.
}%end item
  \end{itemize}
}
\item{
\sld
{\bf Parameters}
\sld\isep
  \begin{itemize}
\sld\isep
   \item{
\sld
{\tt ns} - Collection of nodes to add.}
  \end{itemize}
}%end item
\item{{\bf Returns} - 
Itself with the nodes added or not added. 
}%end item
\end{itemize}
}%end item
\divideents{addEdge}
\item{\vskip -1.9ex 
\membername{addEdge}
{\tt public Graph {\bf addEdge}( {\tt uk.ac.ic.doc.neuralnets.graph.Edge } {\bf e} )
}%end signature
\begin{itemize}
\sld
\item{
\sld
{\bf Usage}
  \begin{itemize}\isep
   \item{
Adds an edge to the graph and adds its start and end nodes to the graph.
}%end item
  \end{itemize}
}
\item{
\sld
{\bf Parameters}
\sld\isep
  \begin{itemize}
\sld\isep
   \item{
\sld
{\tt e} - Edge to add.}
  \end{itemize}
}%end item
\item{{\bf Returns} - 
Itself 
}%end item
\end{itemize}
}%end item
\divideents{addNode}
\item{\vskip -1.9ex 
\membername{addNode}
{\tt public Graph {\bf addNode}( {\tt uk.ac.ic.doc.neuralnets.graph.Node } {\bf n} )
}%end signature
\begin{itemize}
\sld
\item{
\sld
{\bf Usage}
  \begin{itemize}\isep
   \item{
Adds input node to the graph as long as input node is not itself, returns
 itself.
}%end item
  \end{itemize}
}
\item{
\sld
{\bf Parameters}
\sld\isep
  \begin{itemize}
\sld\isep
   \item{
\sld
{\tt n} - Node to add.}
  \end{itemize}
}%end item
\item{{\bf Returns} - 
Itself with the node added or not added. 
}%end item
\end{itemize}
}%end item
\divideents{forEachEdge}
\item{\vskip -1.9ex 
\membername{forEachEdge}
{\tt public Graph {\bf forEachEdge}( {\tt uk.ac.ic.doc.neuralnets.graph.Graph.Command } {\bf c} )
}%end signature
\begin{itemize}
\sld
\item{
\sld
{\bf Usage}
  \begin{itemize}\isep
   \item{
Conducts a command on each edge within the graph.
}%end item
  \end{itemize}
}
\item{
\sld
{\bf Parameters}
\sld\isep
  \begin{itemize}
\sld\isep
   \item{
\sld
{\tt c} - Command to execute.}
  \end{itemize}
}%end item
\item{{\bf Returns} - 
Itself. 
}%end item
\end{itemize}
}%end item
\divideents{forEachNode}
\item{\vskip -1.9ex 
\membername{forEachNode}
{\tt public Graph {\bf forEachNode}( {\tt uk.ac.ic.doc.neuralnets.graph.Graph.Command } {\bf c} )
}%end signature
\begin{itemize}
\sld
\item{
\sld
{\bf Usage}
  \begin{itemize}\isep
   \item{
Conducts a command on each node within the graph.
}%end item
  \end{itemize}
}
\item{
\sld
{\bf Parameters}
\sld\isep
  \begin{itemize}
\sld\isep
   \item{
\sld
{\tt c} - Command to execute.}
  \end{itemize}
}%end item
\item{{\bf Returns} - 
Itself. 
}%end item
\end{itemize}
}%end item
\divideents{getEdges}
\item{\vskip -1.9ex 
\membername{getEdges}
{\tt public Collection {\bf getEdges}(  )
}%end signature
\begin{itemize}
\sld
\item{
\sld
{\bf Usage}
  \begin{itemize}\isep
   \item{
Gets the edges from within.
}%end item
  \end{itemize}
}
\item{{\bf Returns} - 
The edges. 
}%end item
\end{itemize}
}%end item
\divideents{getFreshID}
\item{\vskip -1.9ex 
\membername{getFreshID}
{\tt public void {\bf getFreshID}(  )
}%end signature
\begin{itemize}
\sld
\item{
\sld
{\bf Usage}
  \begin{itemize}\isep
   \item{
Sets the id of the object to a new fresh id.
}%end item
  \end{itemize}
}
\end{itemize}
}%end item
\divideents{getID}
\item{\vskip -1.9ex 
\membername{getID}
{\tt public int {\bf getID}(  )
}%end signature
\begin{itemize}
\sld
\item{
\sld
{\bf Usage}
  \begin{itemize}\isep
   \item{
Gets the id of the object.
}%end item
  \end{itemize}
}
\item{{\bf Returns} - 
The id. 
}%end item
\end{itemize}
}%end item
\divideents{getNodes}
\item{\vskip -1.9ex 
\membername{getNodes}
{\tt public Collection {\bf getNodes}(  )
}%end signature
\begin{itemize}
\sld
\item{
\sld
{\bf Usage}
  \begin{itemize}\isep
   \item{
Gets the nodes from within.
}%end item
  \end{itemize}
}
\item{{\bf Returns} - 
The nodes. 
}%end item
\end{itemize}
}%end item
\divideents{merge}
\item{\vskip -1.9ex 
\membername{merge}
{\tt public Graph {\bf merge}( {\tt uk.ac.ic.doc.neuralnets.graph.Graph } {\bf o} )
}%end signature
\begin{itemize}
\sld
\item{
\sld
{\bf Usage}
  \begin{itemize}\isep
   \item{
Merges one graph with its self, as all the edges and nodes.
}%end item
  \end{itemize}
}
\item{
\sld
{\bf Parameters}
\sld\isep
  \begin{itemize}
\sld\isep
   \item{
\sld
{\tt o} - Graph to merge with.}
  \end{itemize}
}%end item
\item{{\bf Returns} - 
Itself 
}%end item
\end{itemize}
}%end item
\divideents{setID}
\item{\vskip -1.9ex 
\membername{setID}
{\tt public void {\bf setID}( {\tt int } {\bf id} )
}%end signature
\begin{itemize}
\sld
\item{
\sld
{\bf Usage}
  \begin{itemize}\isep
   \item{
Sets the id of the object to parameter.
}%end item
  \end{itemize}
}
\item{
\sld
{\bf Parameters}
\sld\isep
  \begin{itemize}
\sld\isep
   \item{
\sld
{\tt int} - New id.}
  \end{itemize}
}%end item
\end{itemize}
}%end item
\divideents{toString}
\item{\vskip -1.9ex 
\membername{toString}
{\tt public String {\bf toString}(  )
}%end signature
}%end item
\divideents{type}
\item{\vskip -1.9ex 
\membername{type}
{\tt protected String {\bf type}(  )
}%end signature
\begin{itemize}
\sld
\item{
\sld
{\bf Usage}
  \begin{itemize}\isep
   \item{
Returns the object type.
}%end item
  \end{itemize}
}
\item{{\bf Returns} - 
Object type. 
}%end item
\end{itemize}
}%end item
\end{itemize}
}}
}
}
}
\newpage
\def\packagename{uk.ac.ic.doc.neuralnets.graph.neural.train}
\chapter{\bf Package uk.ac.ic.doc.neuralnets.graph.neural.train}{
\vskip -.25in
\hbox to \hsize{\it Package Contents\hfil Page}
\rule{\hsize}{.7mm}
\vskip .13in
\hbox{\bf Interfaces}
\entityintro{Trainer}{l545}{...no description...}
\vskip .1in
\rule{\hsize}{.7mm}
\vskip .1in
\newpage
\section{Interfaces}{
\startsection{Interface}{Trainer}{l545}{%
\startsubsubsection{Declaration}{
\fbox{\vbox{
\hbox{\vbox{\small public interface 
Trainer}}
\noindent\hbox{\vbox{{\bf implements} 
uk.ac.ic.doc.neuralnets.util.plugins.Plugin}}
}}}
\startsubsubsection{Methods}{
\vskip -2em
\begin{itemize}
\item{\vskip -1.9ex 
\membername{setInputs}
{\tt public void {\bf setInputs}( {\tt java.util.Collection } {\bf in} )
\label{l546}\label{l547}}%end signature
}%end item
\divideents{setInputs}
\item{\vskip -1.9ex 
\membername{setInputs}
{\tt public void {\bf setInputs}( {\tt uk.ac.ic.doc.neuralnets.graph.neural.io.InputNode } {\bf in} )
\label{l548}\label{l549}}%end signature
}%end item
\divideents{setTestLength}
\item{\vskip -1.9ex 
\membername{setTestLength}
{\tt public void {\bf setTestLength}( {\tt int } {\bf it} )
\label{l550}\label{l551}}%end signature
}%end item
\divideents{trainFully}
\item{\vskip -1.9ex 
\membername{trainFully}
{\tt public double {\bf trainFully}( {\tt uk.ac.ic.doc.neuralnets.graph.neural.NeuralNetwork } {\bf n},
{\tt double } {\bf errorTarget},
{\tt int } {\bf maxIt} )
\label{l552}\label{l553}}%end signature
\begin{itemize}
\sld
\item{
\sld
{\bf Usage}
  \begin{itemize}\isep
   \item{
Train this network until the accuracy \textgreater = target
}%end item
  \end{itemize}
}
\item{
\sld
{\bf Parameters}
\sld\isep
  \begin{itemize}
\sld\isep
   \item{
\sld
{\tt n} - The network to train}
   \item{
\sld
{\tt errorTarget} - The target accuracy}
   \item{
\sld
{\tt maxIt} - The maximum number of iterations}
  \end{itemize}
}%end item
\item{{\bf Returns} - 
The accuracy of the network after training 
}%end item
\end{itemize}
}%end item
\divideents{trainOnce}
\item{\vskip -1.9ex 
\membername{trainOnce}
{\tt public double {\bf trainOnce}( {\tt uk.ac.ic.doc.neuralnets.graph.neural.NeuralNetwork } {\bf n} )
\label{l554}\label{l555}}%end signature
\begin{itemize}
\sld
\item{
\sld
{\bf Usage}
  \begin{itemize}\isep
   \item{
Train this network with one iteration
}%end item
  \end{itemize}
}
\item{
\sld
{\bf Parameters}
\sld\isep
  \begin{itemize}
\sld\isep
   \item{
\sld
{\tt n} - The network to train}
  \end{itemize}
}%end item
\item{{\bf Returns} - 
The accuracy of the network after training 
}%end item
\end{itemize}
}%end item
\end{itemize}
}
}
}
}
\newpage
\def\packagename{uk.ac.ic.doc.neuralnets.gui.connector}
\chapter{\bf Package uk.ac.ic.doc.neuralnets.gui.connector}{
\vskip -.25in
\hbox to \hsize{\it Package Contents\hfil Page}
\rule{\hsize}{.7mm}
\vskip .13in
\hbox{\bf Classes}
\entityintro{NetworkConnector}{l556}{...no description...}
\vskip .1in
\rule{\hsize}{.7mm}
\vskip .1in
\newpage
\section{Classes}{
\startsection{Class}{NetworkConnector}{l556}{%
\startsubsubsection{Declaration}{
\fbox{\vbox{
\hbox{\vbox{\small public abstract 
class 
NetworkConnector}}
\noindent\hbox{\vbox{{\bf extends} java.lang.Object}}
\noindent\hbox{\vbox{{\bf implements} 
uk.ac.ic.doc.neuralnets.util.plugins.Plugin}}
}}}
\startsubsubsection{Constructors}{
\vskip -2em
\begin{itemize}
\item{\vskip -1.9ex 
\membername{NetworkConnector}
{\tt public {\bf NetworkConnector}(  )
\label{l557}\label{l558}}%end signature
}%end item
\divideents{NetworkConnector}
\item{\vskip -1.9ex 
\membername{NetworkConnector}
{\tt public {\bf NetworkConnector}( {\tt uk.ac.ic.doc.neuralnets.coreui.ZoomingInterfaceManager } {\bf gm} )
\label{l559}\label{l560}}%end signature
}%end item
\end{itemize}
}
\startsubsubsection{Methods}{
\vskip -2em
\begin{itemize}
\item{\vskip -1.9ex 
\membername{connect}
{\tt public abstract Collection {\bf connect}( {\tt java.util.List } {\bf nodes} )
\label{l561}\label{l562}}%end signature
}%end item
\divideents{getConfigurationPanel}
\item{\vskip -1.9ex 
\membername{getConfigurationPanel}
{\tt public abstract Composite {\bf getConfigurationPanel}( {\tt org.eclipse.swt.widgets.Composite } {\bf parent} )
\label{l563}\label{l564}}%end signature
}%end item
\divideents{setGUIManager}
\item{\vskip -1.9ex 
\membername{setGUIManager}
{\tt public void {\bf setGUIManager}( {\tt uk.ac.ic.doc.neuralnets.coreui.ZoomingInterfaceManager } {\bf gm} )
\label{l565}\label{l566}}%end signature
}%end item
\end{itemize}
}
}
}
}
\newpage
\def\packagename{uk.ac.ic.doc.neuralnets.persistence}
\chapter{\bf Package uk.ac.ic.doc.neuralnets.persistence}{
\vskip -.25in
\hbox to \hsize{\it Package Contents\hfil Page}
\rule{\hsize}{.7mm}
\vskip .13in
\hbox{\bf Interfaces}
\entityintro{LoadSpecification}{l567}{LoadSpecifications provide an abstract method for parameterising a
 LoadService in order to load a neural network in to the program.}
\entityintro{SaveSpecification}{l568}{SaveSpecification provide an abstract way of parameterising a SaveService in
 order to save a network.}
\vskip .13in
\hbox{\bf Classes}
\entityintro{FileSpecification}{l569}{The FileSpecification provides parameters for persistence of networks to$/$from
 the file system, i.e.}
\entityintro{LoadException}{l570}{Denotes an error whilst attempting to load a network.}
\entityintro{LoadManager}{l571}{The LoadManager is responsible for creating networks for use in the 
 application from data in persistable storage using pluggable LoadServices,
 which are parameterised by LoadSpecifications.}
\entityintro{LoadService}{l572}{Classes that implement this interface should be able to create 
 neural networks for use in the application from data in persistable storage.}
\entityintro{MethodSelector}{l573}{...no description...}
\entityintro{SaveException}{l574}{Denotes there was an error whilst attempting to save a network.}
\entityintro{SaveManager}{l575}{The SaveManager is responsible for persisting a given network via parameters
 specified in a SaveSpecification using pluggable SaveServices.}
\entityintro{SaveService}{l576}{Classes that implement this interface should be able to create a persistent 
 representation of a given neural network in some format.}
\vskip .1in
\rule{\hsize}{.7mm}
\vskip .1in
\newpage
\section{Interfaces}{
\startsection{Interface}{LoadSpecification}{l567}{%
{\small LoadSpecifications provide an abstract method for parameterising a
 LoadService in order to load a neural network in to the program. To load a
 network a LoadSpecification is created which names the LoadService to use as
 the load process. The specification is passed to the LoadManager which
 retrieves the requested LoadService and passes the specification on to it.}
\vskip .1in 
\startsubsubsection{Declaration}{
\fbox{\vbox{
\hbox{\vbox{\small public interface 
LoadSpecification}}
}}}
\startsubsubsection{Methods}{
\vskip -2em
\begin{itemize}
\item{\vskip -1.9ex 
\membername{getServiceName}
{\tt public String {\bf getServiceName}(  )
\label{l577}\label{l578}}%end signature
\begin{itemize}
\sld
\item{
\sld
{\bf Usage}
  \begin{itemize}\isep
   \item{
The LoadService used by this specification.
}%end item
  \end{itemize}
}
\item{{\bf Returns} - 
the load service plugin name. 
}%end item
\end{itemize}
}%end item
\end{itemize}
}
}
\startsection{Interface}{SaveSpecification}{l568}{%
{\small SaveSpecification provide an abstract way of parameterising a SaveService in
 order to save a network. To save a network a SaveSpecification is created
 which names the SaveService to use as the save process. The specification is
 passed to the SaveManager which retrieves the requested SaveService and
 passes the specification on to it.}
\vskip .1in 
\startsubsubsection{Declaration}{
\fbox{\vbox{
\hbox{\vbox{\small public interface 
SaveSpecification}}
}}}
\startsubsubsection{Methods}{
\vskip -2em
\begin{itemize}
\item{\vskip -1.9ex 
\membername{getServiceName}
{\tt public String {\bf getServiceName}(  )
\label{l579}\label{l580}}%end signature
\begin{itemize}
\sld
\item{
\sld
{\bf Usage}
  \begin{itemize}\isep
   \item{
The SaveService used by this specification.
}%end item
  \end{itemize}
}
\item{{\bf Returns} - 
the save service plugin name. 
}%end item
\end{itemize}
}%end item
\end{itemize}
}
}
}
\section{Classes}{
\startsection{Class}{FileSpecification}{l569}{%
{\small The FileSpecification provides parameters for persistence of networks to$/$from
 the file system, i.e. a file path.}
\vskip .1in 
\startsubsubsection{Declaration}{
\fbox{\vbox{
\hbox{\vbox{\small public 
class 
FileSpecification}}
\noindent\hbox{\vbox{{\bf extends} java.lang.Object}}
\noindent\hbox{\vbox{{\bf implements} 
SaveSpecification, LoadSpecification}}
}}}
\startsubsubsection{Constructors}{
\vskip -2em
\begin{itemize}
\item{\vskip -1.9ex 
\membername{FileSpecification}
{\tt public {\bf FileSpecification}( {\tt java.lang.String } {\bf pathname},
{\tt java.lang.String } {\bf serviceName} )
\label{l581}\label{l582}}%end signature
\begin{itemize}
\sld
\item{
\sld
{\bf Usage}
  \begin{itemize}\isep
   \item{
Create a new specification.
}%end item
  \end{itemize}
}
\item{
\sld
{\bf Parameters}
\sld\isep
  \begin{itemize}
\sld\isep
   \item{
\sld
{\tt pathname} - - path to save$/$load to from}
   \item{
\sld
{\tt serviceName} - - the service to use.}
  \end{itemize}
}%end item
\end{itemize}
}%end item
\end{itemize}
}
\startsubsubsection{Methods}{
\vskip -2em
\begin{itemize}
\item{\vskip -1.9ex 
\membername{getSavePath}
{\tt public String {\bf getSavePath}(  )
\label{l583}\label{l584}}%end signature
\begin{itemize}
\sld
\item{
\sld
{\bf Usage}
  \begin{itemize}\isep
   \item{
Get the file system location.
}%end item
  \end{itemize}
}
\item{{\bf Returns} - 
the file path 
}%end item
\end{itemize}
}%end item
\divideents{getServiceName}
\item{\vskip -1.9ex 
\membername{getServiceName}
{\tt public String {\bf getServiceName}(  )
\label{l585}\label{l586}}%end signature
}%end item
\divideents{setPath}
\item{\vskip -1.9ex 
\membername{setPath}
{\tt public void {\bf setPath}( {\tt java.lang.String } {\bf savePath} )
\label{l587}\label{l588}}%end signature
\begin{itemize}
\sld
\item{
\sld
{\bf Usage}
  \begin{itemize}\isep
   \item{
Set the file system location
}%end item
  \end{itemize}
}
\item{
\sld
{\bf Parameters}
\sld\isep
  \begin{itemize}
\sld\isep
   \item{
\sld
{\tt savePath} - the new file path}
  \end{itemize}
}%end item
\end{itemize}
}%end item
\end{itemize}
}
}
\startsection{Class}{LoadException}{l570}{%
{\small Denotes an error whilst attempting to load a network.}
\vskip .1in 
\startsubsubsection{Declaration}{
\fbox{\vbox{
\hbox{\vbox{\small public 
class 
LoadException}}
\noindent\hbox{\vbox{{\bf extends} java.lang.Exception}}
}}}
\startsubsubsection{Constructors}{
\vskip -2em
\begin{itemize}
\item{\vskip -1.9ex 
\membername{LoadException}
{\tt public {\bf LoadException}(  )
\label{l589}\label{l590}}%end signature
}%end item
\divideents{LoadException}
\item{\vskip -1.9ex 
\membername{LoadException}
{\tt public {\bf LoadException}( {\tt java.lang.String } {\bf message} )
\label{l591}\label{l592}}%end signature
}%end item
\divideents{LoadException}
\item{\vskip -1.9ex 
\membername{LoadException}
{\tt public {\bf LoadException}( {\tt java.lang.String } {\bf message},
{\tt java.lang.Throwable } {\bf cause} )
\label{l593}\label{l594}}%end signature
}%end item
\divideents{LoadException}
\item{\vskip -1.9ex 
\membername{LoadException}
{\tt public {\bf LoadException}( {\tt java.lang.Throwable } {\bf cause} )
\label{l595}\label{l596}}%end signature
}%end item
\end{itemize}
}
\startsubsubsection{Methods inherited from class {\tt java.lang.Exception}}{
\par{\small 
\refdefined{l454}}}
\startsubsubsection{Methods inherited from class {\tt java.lang.Throwable}}{
\par{\small 
\refdefined{l455}\vskip -2em
\begin{itemize}
\item{\vskip -1.9ex 
\membername{fillInStackTrace}
{\tt public synchronized native Throwable {\bf fillInStackTrace}(  )
}%end signature
}%end item
\divideents{getCause}
\item{\vskip -1.9ex 
\membername{getCause}
{\tt public Throwable {\bf getCause}(  )
}%end signature
}%end item
\divideents{getLocalizedMessage}
\item{\vskip -1.9ex 
\membername{getLocalizedMessage}
{\tt public String {\bf getLocalizedMessage}(  )
}%end signature
}%end item
\divideents{getMessage}
\item{\vskip -1.9ex 
\membername{getMessage}
{\tt public String {\bf getMessage}(  )
}%end signature
}%end item
\divideents{getStackTrace}
\item{\vskip -1.9ex 
\membername{getStackTrace}
{\tt public StackTraceElement {\bf getStackTrace}(  )
}%end signature
}%end item
\divideents{initCause}
\item{\vskip -1.9ex 
\membername{initCause}
{\tt public synchronized Throwable {\bf initCause}( {\tt java.lang.Throwable } {\bf arg0} )
}%end signature
}%end item
\divideents{printStackTrace}
\item{\vskip -1.9ex 
\membername{printStackTrace}
{\tt public void {\bf printStackTrace}(  )
}%end signature
}%end item
\divideents{printStackTrace}
\item{\vskip -1.9ex 
\membername{printStackTrace}
{\tt public void {\bf printStackTrace}( {\tt java.io.PrintStream } {\bf arg0} )
}%end signature
}%end item
\divideents{printStackTrace}
\item{\vskip -1.9ex 
\membername{printStackTrace}
{\tt public void {\bf printStackTrace}( {\tt java.io.PrintWriter } {\bf arg0} )
}%end signature
}%end item
\divideents{setStackTrace}
\item{\vskip -1.9ex 
\membername{setStackTrace}
{\tt public void {\bf setStackTrace}( {\tt java.lang.StackTraceElement []} {\bf arg0} )
}%end signature
}%end item
\divideents{toString}
\item{\vskip -1.9ex 
\membername{toString}
{\tt public String {\bf toString}(  )
}%end signature
}%end item
\end{itemize}
}}
}
\startsection{Class}{LoadManager}{l571}{%
{\small The LoadManager is responsible for creating networks for use in the 
 application from data in persistable storage using pluggable LoadServices,
 which are parameterised by LoadSpecifications.}
\vskip .1in 
\startsubsubsection{Declaration}{
\fbox{\vbox{
\hbox{\vbox{\small public 
class 
LoadManager}}
\noindent\hbox{\vbox{{\bf extends} java.lang.Object}}
}}}
\startsubsubsection{Methods}{
\vskip -2em
\begin{itemize}
\item{\vskip -1.9ex 
\membername{get}
{\tt public static LoadManager {\bf get}(  )
\label{l597}\label{l598}}%end signature
\begin{itemize}
\sld
\item{
\sld
{\bf Usage}
  \begin{itemize}\isep
   \item{
Retrieve the instance of the LoadManager.
}%end item
  \end{itemize}
}
\item{{\bf Returns} - 
the LoadManager instance. 
}%end item
\end{itemize}
}%end item
\divideents{load}
\item{\vskip -1.9ex 
\membername{load}
{\tt public Saveable {\bf load}( {\tt uk.ac.ic.doc.neuralnets.persistence.LoadSpecification } {\bf spec} )
\label{l599}\label{l600}}%end signature
\begin{itemize}
\sld
\item{
\sld
{\bf Usage}
  \begin{itemize}\isep
   \item{
Reads in a external object using a load service parameterised by a 
 load specification.
}%end item
  \end{itemize}
}
\item{
\sld
{\bf Parameters}
\sld\isep
  \begin{itemize}
\sld\isep
   \item{
\sld
{\tt spec} - paramaters for loading}
  \end{itemize}
}%end item
\item{{\bf Returns} - 
the loaded Saveable object. 
}%end item
\item{{\bf Exceptions}
  \begin{itemize}
\sld
   \item{\vskip -.6ex{\tt uk.ac.ic.doc.neuralnets.persistence.LoadException} - }
  \end{itemize}
}%end item
\end{itemize}
}%end item
\end{itemize}
}
}
\startsection{Class}{LoadService}{l572}{%
{\small Classes that implement this interface should be able to create 
 neural networks for use in the application from data in persistable storage.
 They can be fully parameterised through the use of a LoadSpecification.}
\vskip .1in 
\startsubsubsection{Declaration}{
\fbox{\vbox{
\hbox{\vbox{\small public abstract 
class 
LoadService}}
\noindent\hbox{\vbox{{\bf extends} uk.ac.ic.doc.neuralnets.util.plugins.PriorityPlugin}}
}}}
\startsubsubsection{Constructors}{
\vskip -2em
\begin{itemize}
\item{\vskip -1.9ex 
\membername{LoadService}
{\tt public {\bf LoadService}(  )
\label{l601}\label{l602}}%end signature
}%end item
\end{itemize}
}
\startsubsubsection{Methods}{
\vskip -2em
\begin{itemize}
\item{\vskip -1.9ex 
\membername{getFileType}
{\tt public abstract String {\bf getFileType}(  )
\label{l603}\label{l604}}%end signature
\begin{itemize}
\sld
\item{
\sld
{\bf Usage}
  \begin{itemize}\isep
   \item{
Get the string form of the file type that this load service should seek
 e.g. "*.xml"
}%end item
  \end{itemize}
}
\item{{\bf Returns} - 
The lexical form of the file extension 
}%end item
\end{itemize}
}%end item
\divideents{load}
\item{\vskip -1.9ex 
\membername{load}
{\tt public abstract Saveable {\bf load}( {\tt uk.ac.ic.doc.neuralnets.persistence.LoadSpecification } {\bf spec} )
\label{l605}\label{l606}}%end signature
\begin{itemize}
\sld
\item{
\sld
{\bf Usage}
  \begin{itemize}\isep
   \item{
Imports a neural network from persistent storage.
}%end item
  \end{itemize}
}
\item{
\sld
{\bf Parameters}
\sld\isep
  \begin{itemize}
\sld\isep
   \item{
\sld
{\tt spec} - - the load service parameters}
  \end{itemize}
}%end item
\item{{\bf Returns} - 
the loaded network 
}%end item
\item{{\bf Exceptions}
  \begin{itemize}
\sld
   \item{\vskip -.6ex{\tt uk.ac.ic.doc.neuralnets.persistence.LoadException} - in event of error during loading.}
  \end{itemize}
}%end item
\end{itemize}
}%end item
\end{itemize}
}
\startsubsubsection{Methods inherited from class {\tt uk.ac.ic.doc.neuralnets.util.plugins.PriorityPlugin}}{
\par{\small 
\refdefined{l33}\vskip -2em
\begin{itemize}
\item{\vskip -1.9ex 
\membername{compareTo}
{\tt public int {\bf compareTo}( {\tt uk.ac.ic.doc.neuralnets.util.plugins.PriorityPlugin } {\bf o} )
}%end signature
}%end item
\divideents{getPriority}
\item{\vskip -1.9ex 
\membername{getPriority}
{\tt public abstract int {\bf getPriority}(  )
}%end signature
\begin{itemize}
\sld
\item{
\sld
{\bf Usage}
  \begin{itemize}\isep
   \item{
The plugin's priority.
}%end item
  \end{itemize}
}
\item{{\bf Returns} - 
the priority 
}%end item
\end{itemize}
}%end item
\end{itemize}
}}
}
\startsection{Class}{MethodSelector}{l573}{%
\startsubsubsection{Declaration}{
\fbox{\vbox{
\hbox{\vbox{\small public 
class 
MethodSelector}}
\noindent\hbox{\vbox{{\bf extends} java.lang.Object}}
}}}
\startsubsubsection{Constructors}{
\vskip -2em
\begin{itemize}
\item{\vskip -1.9ex 
\membername{MethodSelector}
{\tt public {\bf MethodSelector}(  )
\label{l607}\label{l608}}%end signature
}%end item
\end{itemize}
}
\startsubsubsection{Methods}{
\vskip -2em
\begin{itemize}
\item{\vskip -1.9ex 
\membername{getPersistableFields}
{\tt public Set {\bf getPersistableFields}( {\tt java.lang.Class } {\bf c} )
\label{l609}\label{l610}}%end signature
}%end item
\divideents{getPersistableMethods}
\item{\vskip -1.9ex 
\membername{getPersistableMethods}
{\tt public Set {\bf getPersistableMethods}( {\tt java.lang.Class } {\bf c} )
\label{l611}\label{l612}}%end signature
}%end item
\divideents{getPersistableMethodsAndFields}
\item{\vskip -1.9ex 
\membername{getPersistableMethodsAndFields}
{\tt public Set {\bf getPersistableMethodsAndFields}( {\tt java.lang.Class } {\bf c} )
\label{l613}\label{l614}}%end signature
}%end item
\end{itemize}
}
}
\startsection{Class}{SaveException}{l574}{%
{\small Denotes there was an error whilst attempting to save a network.}
\vskip .1in 
\startsubsubsection{Declaration}{
\fbox{\vbox{
\hbox{\vbox{\small public 
class 
SaveException}}
\noindent\hbox{\vbox{{\bf extends} java.lang.Exception}}
}}}
\startsubsubsection{Constructors}{
\vskip -2em
\begin{itemize}
\item{\vskip -1.9ex 
\membername{SaveException}
{\tt public {\bf SaveException}(  )
\label{l615}\label{l616}}%end signature
}%end item
\divideents{SaveException}
\item{\vskip -1.9ex 
\membername{SaveException}
{\tt public {\bf SaveException}( {\tt java.lang.String } {\bf message} )
\label{l617}\label{l618}}%end signature
}%end item
\divideents{SaveException}
\item{\vskip -1.9ex 
\membername{SaveException}
{\tt public {\bf SaveException}( {\tt java.lang.String } {\bf message},
{\tt java.lang.Throwable } {\bf cause} )
\label{l619}\label{l620}}%end signature
}%end item
\divideents{SaveException}
\item{\vskip -1.9ex 
\membername{SaveException}
{\tt public {\bf SaveException}( {\tt java.lang.Throwable } {\bf cause} )
\label{l621}\label{l622}}%end signature
}%end item
\end{itemize}
}
\startsubsubsection{Methods inherited from class {\tt java.lang.Exception}}{
\par{\small 
\refdefined{l454}}}
\startsubsubsection{Methods inherited from class {\tt java.lang.Throwable}}{
\par{\small 
\refdefined{l455}\vskip -2em
\begin{itemize}
\item{\vskip -1.9ex 
\membername{fillInStackTrace}
{\tt public synchronized native Throwable {\bf fillInStackTrace}(  )
}%end signature
}%end item
\divideents{getCause}
\item{\vskip -1.9ex 
\membername{getCause}
{\tt public Throwable {\bf getCause}(  )
}%end signature
}%end item
\divideents{getLocalizedMessage}
\item{\vskip -1.9ex 
\membername{getLocalizedMessage}
{\tt public String {\bf getLocalizedMessage}(  )
}%end signature
}%end item
\divideents{getMessage}
\item{\vskip -1.9ex 
\membername{getMessage}
{\tt public String {\bf getMessage}(  )
}%end signature
}%end item
\divideents{getStackTrace}
\item{\vskip -1.9ex 
\membername{getStackTrace}
{\tt public StackTraceElement {\bf getStackTrace}(  )
}%end signature
}%end item
\divideents{initCause}
\item{\vskip -1.9ex 
\membername{initCause}
{\tt public synchronized Throwable {\bf initCause}( {\tt java.lang.Throwable } {\bf arg0} )
}%end signature
}%end item
\divideents{printStackTrace}
\item{\vskip -1.9ex 
\membername{printStackTrace}
{\tt public void {\bf printStackTrace}(  )
}%end signature
}%end item
\divideents{printStackTrace}
\item{\vskip -1.9ex 
\membername{printStackTrace}
{\tt public void {\bf printStackTrace}( {\tt java.io.PrintStream } {\bf arg0} )
}%end signature
}%end item
\divideents{printStackTrace}
\item{\vskip -1.9ex 
\membername{printStackTrace}
{\tt public void {\bf printStackTrace}( {\tt java.io.PrintWriter } {\bf arg0} )
}%end signature
}%end item
\divideents{setStackTrace}
\item{\vskip -1.9ex 
\membername{setStackTrace}
{\tt public void {\bf setStackTrace}( {\tt java.lang.StackTraceElement []} {\bf arg0} )
}%end signature
}%end item
\divideents{toString}
\item{\vskip -1.9ex 
\membername{toString}
{\tt public String {\bf toString}(  )
}%end signature
}%end item
\end{itemize}
}}
}
\startsection{Class}{SaveManager}{l575}{%
{\small The SaveManager is responsible for persisting a given network via parameters
 specified in a SaveSpecification using pluggable SaveServices.}
\vskip .1in 
\startsubsubsection{Declaration}{
\fbox{\vbox{
\hbox{\vbox{\small public 
class 
SaveManager}}
\noindent\hbox{\vbox{{\bf extends} java.lang.Object}}
}}}
\startsubsubsection{Methods}{
\vskip -2em
\begin{itemize}
\item{\vskip -1.9ex 
\membername{get}
{\tt public static SaveManager {\bf get}(  )
\label{l623}\label{l624}}%end signature
\begin{itemize}
\sld
\item{
\sld
{\bf Usage}
  \begin{itemize}\isep
   \item{
Retrieves the instance of the SaveManager.
}%end item
  \end{itemize}
}
\item{{\bf Returns} - 
the SaveManager instance. 
}%end item
\end{itemize}
}%end item
\divideents{save}
\item{\vskip -1.9ex 
\membername{save}
{\tt public void {\bf save}( {\tt uk.ac.ic.doc.neuralnets.graph.Saveable } {\bf net},
{\tt uk.ac.ic.doc.neuralnets.persistence.SaveSpecification } {\bf spec} )
\label{l625}\label{l626}}%end signature
\begin{itemize}
\sld
\item{
\sld
{\bf Usage}
  \begin{itemize}\isep
   \item{
Saves a network through the SaveService named in the SaveSpecification.
}%end item
  \end{itemize}
}
\item{
\sld
{\bf Parameters}
\sld\isep
  \begin{itemize}
\sld\isep
   \item{
\sld
{\tt net} - the Neural Network to save.}
   \item{
\sld
{\tt spec} - SaveSpecification, which contains parameters for the save
            service.}
  \end{itemize}
}%end item
\item{{\bf Exceptions}
  \begin{itemize}
\sld
   \item{\vskip -.6ex{\tt uk.ac.ic.doc.neuralnets.persistence.SaveException} - in the event something goes wrong during saving.}
  \end{itemize}
}%end item
\end{itemize}
}%end item
\end{itemize}
}
}
\startsection{Class}{SaveService}{l576}{%
{\small Classes that implement this interface should be able to create a persistent 
 representation of a given neural network in some format. They can be fully
 parameterised through the use of a SaveSpecification.}
\vskip .1in 
\startsubsubsection{Declaration}{
\fbox{\vbox{
\hbox{\vbox{\small public abstract 
class 
SaveService}}
\noindent\hbox{\vbox{{\bf extends} uk.ac.ic.doc.neuralnets.util.plugins.PriorityPlugin}}
}}}
\startsubsubsection{Constructors}{
\vskip -2em
\begin{itemize}
\item{\vskip -1.9ex 
\membername{SaveService}
{\tt public {\bf SaveService}(  )
\label{l627}\label{l628}}%end signature
}%end item
\end{itemize}
}
\startsubsubsection{Methods}{
\vskip -2em
\begin{itemize}
\item{\vskip -1.9ex 
\membername{getFileType}
{\tt public abstract String {\bf getFileType}(  )
\label{l629}\label{l630}}%end signature
\begin{itemize}
\sld
\item{
\sld
{\bf Usage}
  \begin{itemize}\isep
   \item{
Get the string form of the file type that this save service should seek
 e.g. "*.xml"
}%end item
  \end{itemize}
}
\item{{\bf Returns} - 
The lexical form of the file extension 
}%end item
\end{itemize}
}%end item
\divideents{save}
\item{\vskip -1.9ex 
\membername{save}
{\tt public abstract void {\bf save}( {\tt uk.ac.ic.doc.neuralnets.graph.Saveable } {\bf network},
{\tt uk.ac.ic.doc.neuralnets.persistence.SaveSpecification } {\bf spec} )
\label{l631}\label{l632}}%end signature
\begin{itemize}
\sld
\item{
\sld
{\bf Usage}
  \begin{itemize}\isep
   \item{
Exports the given neural network to persistent storage in a given format
}%end item
  \end{itemize}
}
\item{
\sld
{\bf Parameters}
\sld\isep
  \begin{itemize}
\sld\isep
   \item{
\sld
{\tt network} - - the network to save}
   \item{
\sld
{\tt spec} - - the save service parameters}
  \end{itemize}
}%end item
\item{{\bf Exceptions}
  \begin{itemize}
\sld
   \item{\vskip -.6ex{\tt uk.ac.ic.doc.neuralnets.persistence.SaveException} - in the event of error during saving}
  \end{itemize}
}%end item
\end{itemize}
}%end item
\end{itemize}
}
\startsubsubsection{Methods inherited from class {\tt uk.ac.ic.doc.neuralnets.util.plugins.PriorityPlugin}}{
\par{\small 
\refdefined{l33}\vskip -2em
\begin{itemize}
\item{\vskip -1.9ex 
\membername{compareTo}
{\tt public int {\bf compareTo}( {\tt uk.ac.ic.doc.neuralnets.util.plugins.PriorityPlugin } {\bf o} )
}%end signature
}%end item
\divideents{getPriority}
\item{\vskip -1.9ex 
\membername{getPriority}
{\tt public abstract int {\bf getPriority}(  )
}%end signature
\begin{itemize}
\sld
\item{
\sld
{\bf Usage}
  \begin{itemize}\isep
   \item{
The plugin's priority.
}%end item
  \end{itemize}
}
\item{{\bf Returns} - 
the priority 
}%end item
\end{itemize}
}%end item
\end{itemize}
}}
}
}
}
\newpage
\def\packagename{uk.ac.ic.doc.neuralnets.matrix}
\chapter{\bf Package uk.ac.ic.doc.neuralnets.matrix}{
\vskip -.25in
\hbox to \hsize{\it Package Contents\hfil Page}
\rule{\hsize}{.7mm}
\vskip .13in
\hbox{\bf Interfaces}
\entityintro{Matrix.Command}{l633}{...no description...}
\vskip .13in
\hbox{\bf Classes}
\entityintro{Matrix}{l634}{Matrix class that almost supports dynamic resizing
 May not be needed for our use cases, so didn't invest any more effort
 Resizing half-works (specify no-bound with width or height == 0), can
 put effort in if it's needed
 Wherever possible, instead of returning void from a public method,
 returns itself instead to permit chaining of calls}
\entityintro{PartitionableMatrix}{l635}{...no description...}
\entityintro{RollUpMatrix}{l636}{...no description...}
\vskip .1in
\rule{\hsize}{.7mm}
\vskip .1in
\newpage
\section{Interfaces}{
\startsection{Interface}{Matrix.Command}{l633}{%
\startsubsubsection{Declaration}{
\fbox{\vbox{
\hbox{\vbox{\small public static interface 
Matrix.Command}}
}}}
\startsubsubsection{Methods}{
\vskip -2em
\begin{itemize}
\item{\vskip -1.9ex 
\membername{exec}
{\tt public void {\bf exec}( {\tt int } {\bf x},
{\tt int } {\bf y},
{\tt java.lang.Object } {\bf item} )
\label{l637}\label{l638}}%end signature
}%end item
\end{itemize}
}
}
}
\section{Classes}{
\startsection{Class}{Matrix}{l634}{%
{\small Matrix class that almost supports dynamic resizing
 May not be needed for our use cases, so didn't invest any more effort
 Resizing half-works (specify no-bound with width or height == 0), can
 put effort in if it's needed
 Wherever possible, instead of returning void from a public method,
 returns itself instead to permit chaining of calls}
\vskip .1in 
\startsubsubsection{Declaration}{
\fbox{\vbox{
\hbox{\vbox{\small public 
class 
Matrix}}
\noindent\hbox{\vbox{{\bf extends} java.lang.Object}}
\noindent\hbox{\vbox{{\bf implements} 
java.io.Serializable}}
}}}
\startsubsubsection{Constructors}{
\vskip -2em
\begin{itemize}
\item{\vskip -1.9ex 
\membername{Matrix}
{\tt public {\bf Matrix}( {\tt int } {\bf width},
{\tt int } {\bf height} )
\label{l639}\label{l640}}%end signature
}%end item
\end{itemize}
}
\startsubsubsection{Methods}{
\vskip -2em
\begin{itemize}
\item{\vskip -1.9ex 
\membername{add}
{\tt public synchronized Matrix {\bf add}( {\tt java.lang.Object } {\bf item} )
\label{l641}\label{l642}}%end signature
}%end item
\divideents{add}
\item{\vskip -1.9ex 
\membername{add}
{\tt public synchronized Matrix {\bf add}( {\tt java.lang.Object } {\bf item},
{\tt int } {\bf x} )
\label{l643}\label{l644}}%end signature
}%end item
\divideents{bounds}
\item{\vskip -1.9ex 
\membername{bounds}
{\tt protected final void {\bf bounds}( {\tt int } {\bf x},
{\tt int } {\bf y} )
\label{l645}\label{l646}}%end signature
}%end item
\divideents{boundsX}
\item{\vskip -1.9ex 
\membername{boundsX}
{\tt protected final void {\bf boundsX}( {\tt int } {\bf x} )
\label{l647}\label{l648}}%end signature
}%end item
\divideents{boundsY}
\item{\vskip -1.9ex 
\membername{boundsY}
{\tt protected final void {\bf boundsY}( {\tt int } {\bf y} )
\label{l649}\label{l650}}%end signature
}%end item
\divideents{forEach}
\item{\vskip -1.9ex 
\membername{forEach}
{\tt public synchronized Matrix {\bf forEach}( {\tt uk.ac.ic.doc.neuralnets.matrix.Matrix.Command } {\bf c} )
\label{l651}\label{l652}}%end signature
}%end item
\divideents{get}
\item{\vskip -1.9ex 
\membername{get}
{\tt public synchronized Object {\bf get}( {\tt int } {\bf x},
{\tt int } {\bf y} )
\label{l653}\label{l654}}%end signature
}%end item
\divideents{getHeight}
\item{\vskip -1.9ex 
\membername{getHeight}
{\tt public int {\bf getHeight}(  )
\label{l655}\label{l656}}%end signature
}%end item
\divideents{getWidth}
\item{\vskip -1.9ex 
\membername{getWidth}
{\tt public int {\bf getWidth}(  )
\label{l657}\label{l658}}%end signature
}%end item
\divideents{set}
\item{\vskip -1.9ex 
\membername{set}
{\tt public synchronized Matrix {\bf set}( {\tt java.lang.Object } {\bf item},
{\tt int } {\bf x},
{\tt int } {\bf y} )
\label{l659}\label{l660}}%end signature
}%end item
\divideents{toString}
\item{\vskip -1.9ex 
\membername{toString}
{\tt public synchronized String {\bf toString}(  )
\label{l661}\label{l662}}%end signature
}%end item
\end{itemize}
}
}
\startsection{Class}{PartitionableMatrix}{l635}{%
\startsubsubsection{Declaration}{
\fbox{\vbox{
\hbox{\vbox{\small public 
class 
PartitionableMatrix}}
\noindent\hbox{\vbox{{\bf extends} uk.ac.ic.doc.neuralnets.matrix.Matrix}}
}}}
\startsubsubsection{Serializable Fields}{
\begin{itemize}
\item{
private int pX1\begin{itemize}\item{\vskip -.9ex }\end{itemize}
}
\item{
private int pY1\begin{itemize}\item{\vskip -.9ex }\end{itemize}
}
\item{
private int pX2\begin{itemize}\item{\vskip -.9ex }\end{itemize}
}
\item{
private int pY2\begin{itemize}\item{\vskip -.9ex }\end{itemize}
}
\end{itemize}
}
\startsubsubsection{Constructors}{
\vskip -2em
\begin{itemize}
\item{\vskip -1.9ex 
\membername{PartitionableMatrix}
{\tt public {\bf PartitionableMatrix}( {\tt int } {\bf width},
{\tt int } {\bf height} )
\label{l663}\label{l664}}%end signature
}%end item
\end{itemize}
}
\startsubsubsection{Methods}{
\vskip -2em
\begin{itemize}
\item{\vskip -1.9ex 
\membername{clearPartition}
{\tt public synchronized PartitionableMatrix {\bf clearPartition}(  )
\label{l665}\label{l666}}%end signature
}%end item
\divideents{forEachPartitioned}
\item{\vskip -1.9ex 
\membername{forEachPartitioned}
{\tt public synchronized PartitionableMatrix {\bf forEachPartitioned}( {\tt uk.ac.ic.doc.neuralnets.matrix.Matrix.Command } {\bf c} )
\label{l667}\label{l668}}%end signature
}%end item
\divideents{getPartitioned}
\item{\vskip -1.9ex 
\membername{getPartitioned}
{\tt public synchronized Object {\bf getPartitioned}( {\tt int } {\bf x},
{\tt int } {\bf y} )
\label{l669}\label{l670}}%end signature
}%end item
\divideents{getPartitionedMatrix}
\item{\vskip -1.9ex 
\membername{getPartitionedMatrix}
{\tt public synchronized PartitionableMatrix {\bf getPartitionedMatrix}(  )
\label{l671}\label{l672}}%end signature
}%end item
\divideents{newMatrix}
\item{\vskip -1.9ex 
\membername{newMatrix}
{\tt protected PartitionableMatrix {\bf newMatrix}( {\tt int } {\bf w},
{\tt int } {\bf h} )
\label{l673}\label{l674}}%end signature
}%end item
\divideents{partition}
\item{\vskip -1.9ex 
\membername{partition}
{\tt public synchronized PartitionableMatrix {\bf partition}( {\tt int } {\bf x1},
{\tt int } {\bf y1},
{\tt int } {\bf x2},
{\tt int } {\bf y2} )
\label{l675}\label{l676}}%end signature
}%end item
\end{itemize}
}
\startsubsubsection{Methods inherited from class {\tt uk.ac.ic.doc.neuralnets.matrix.Matrix}}{
\par{\small 
\refdefined{l634}\vskip -2em
\begin{itemize}
\item{\vskip -1.9ex 
\membername{add}
{\tt public synchronized Matrix {\bf add}( {\tt java.lang.Object } {\bf item} )
}%end signature
}%end item
\divideents{add}
\item{\vskip -1.9ex 
\membername{add}
{\tt public synchronized Matrix {\bf add}( {\tt java.lang.Object } {\bf item},
{\tt int } {\bf x} )
}%end signature
}%end item
\divideents{bounds}
\item{\vskip -1.9ex 
\membername{bounds}
{\tt protected final void {\bf bounds}( {\tt int } {\bf x},
{\tt int } {\bf y} )
}%end signature
}%end item
\divideents{boundsX}
\item{\vskip -1.9ex 
\membername{boundsX}
{\tt protected final void {\bf boundsX}( {\tt int } {\bf x} )
}%end signature
}%end item
\divideents{boundsY}
\item{\vskip -1.9ex 
\membername{boundsY}
{\tt protected final void {\bf boundsY}( {\tt int } {\bf y} )
}%end signature
}%end item
\divideents{forEach}
\item{\vskip -1.9ex 
\membername{forEach}
{\tt public synchronized Matrix {\bf forEach}( {\tt uk.ac.ic.doc.neuralnets.matrix.Matrix.Command } {\bf c} )
}%end signature
}%end item
\divideents{get}
\item{\vskip -1.9ex 
\membername{get}
{\tt public synchronized Object {\bf get}( {\tt int } {\bf x},
{\tt int } {\bf y} )
}%end signature
}%end item
\divideents{getHeight}
\item{\vskip -1.9ex 
\membername{getHeight}
{\tt public int {\bf getHeight}(  )
}%end signature
}%end item
\divideents{getWidth}
\item{\vskip -1.9ex 
\membername{getWidth}
{\tt public int {\bf getWidth}(  )
}%end signature
}%end item
\divideents{set}
\item{\vskip -1.9ex 
\membername{set}
{\tt public synchronized Matrix {\bf set}( {\tt java.lang.Object } {\bf item},
{\tt int } {\bf x},
{\tt int } {\bf y} )
}%end signature
}%end item
\divideents{toString}
\item{\vskip -1.9ex 
\membername{toString}
{\tt public synchronized String {\bf toString}(  )
}%end signature
}%end item
\end{itemize}
}}
}
\startsection{Class}{RollUpMatrix}{l636}{%
\startsubsubsection{Declaration}{
\fbox{\vbox{
\hbox{\vbox{\small public 
class 
RollUpMatrix}}
\noindent\hbox{\vbox{{\bf extends} uk.ac.ic.doc.neuralnets.matrix.PartitionableMatrix}}
}}}
\startsubsubsection{Constructors}{
\vskip -2em
\begin{itemize}
\item{\vskip -1.9ex 
\membername{RollUpMatrix}
{\tt public {\bf RollUpMatrix}( {\tt int } {\bf width},
{\tt int } {\bf height} )
\label{l677}\label{l678}}%end signature
}%end item
\end{itemize}
}
\startsubsubsection{Methods}{
\vskip -2em
\begin{itemize}
\item{\vskip -1.9ex 
\membername{newMatrix}
{\tt protected PartitionableMatrix {\bf newMatrix}( {\tt int } {\bf w},
{\tt int } {\bf h} )
\label{l679}\label{l680}}%end signature
}%end item
\divideents{rollUp}
\item{\vskip -1.9ex 
\membername{rollUp}
{\tt public synchronized RollUpMatrix {\bf rollUp}( {\tt int } {\bf width},
{\tt int } {\bf height} )
\label{l681}\label{l682}}%end signature
}%end item
\end{itemize}
}
\startsubsubsection{Methods inherited from class {\tt uk.ac.ic.doc.neuralnets.matrix.PartitionableMatrix}}{
\par{\small 
\refdefined{l635}\vskip -2em
\begin{itemize}
\item{\vskip -1.9ex 
\membername{clearPartition}
{\tt public synchronized PartitionableMatrix {\bf clearPartition}(  )
}%end signature
}%end item
\divideents{forEachPartitioned}
\item{\vskip -1.9ex 
\membername{forEachPartitioned}
{\tt public synchronized PartitionableMatrix {\bf forEachPartitioned}( {\tt uk.ac.ic.doc.neuralnets.matrix.Matrix.Command } {\bf c} )
}%end signature
}%end item
\divideents{getPartitioned}
\item{\vskip -1.9ex 
\membername{getPartitioned}
{\tt public synchronized Object {\bf getPartitioned}( {\tt int } {\bf x},
{\tt int } {\bf y} )
}%end signature
}%end item
\divideents{getPartitionedMatrix}
\item{\vskip -1.9ex 
\membername{getPartitionedMatrix}
{\tt public synchronized PartitionableMatrix {\bf getPartitionedMatrix}(  )
}%end signature
}%end item
\divideents{newMatrix}
\item{\vskip -1.9ex 
\membername{newMatrix}
{\tt protected PartitionableMatrix {\bf newMatrix}( {\tt int } {\bf w},
{\tt int } {\bf h} )
}%end signature
}%end item
\divideents{partition}
\item{\vskip -1.9ex 
\membername{partition}
{\tt public synchronized PartitionableMatrix {\bf partition}( {\tt int } {\bf x1},
{\tt int } {\bf y1},
{\tt int } {\bf x2},
{\tt int } {\bf y2} )
}%end signature
}%end item
\end{itemize}
}}
\startsubsubsection{Methods inherited from class {\tt uk.ac.ic.doc.neuralnets.matrix.Matrix}}{
\par{\small 
\refdefined{l634}\vskip -2em
\begin{itemize}
\item{\vskip -1.9ex 
\membername{add}
{\tt public synchronized Matrix {\bf add}( {\tt java.lang.Object } {\bf item} )
}%end signature
}%end item
\divideents{add}
\item{\vskip -1.9ex 
\membername{add}
{\tt public synchronized Matrix {\bf add}( {\tt java.lang.Object } {\bf item},
{\tt int } {\bf x} )
}%end signature
}%end item
\divideents{bounds}
\item{\vskip -1.9ex 
\membername{bounds}
{\tt protected final void {\bf bounds}( {\tt int } {\bf x},
{\tt int } {\bf y} )
}%end signature
}%end item
\divideents{boundsX}
\item{\vskip -1.9ex 
\membername{boundsX}
{\tt protected final void {\bf boundsX}( {\tt int } {\bf x} )
}%end signature
}%end item
\divideents{boundsY}
\item{\vskip -1.9ex 
\membername{boundsY}
{\tt protected final void {\bf boundsY}( {\tt int } {\bf y} )
}%end signature
}%end item
\divideents{forEach}
\item{\vskip -1.9ex 
\membername{forEach}
{\tt public synchronized Matrix {\bf forEach}( {\tt uk.ac.ic.doc.neuralnets.matrix.Matrix.Command } {\bf c} )
}%end signature
}%end item
\divideents{get}
\item{\vskip -1.9ex 
\membername{get}
{\tt public synchronized Object {\bf get}( {\tt int } {\bf x},
{\tt int } {\bf y} )
}%end signature
}%end item
\divideents{getHeight}
\item{\vskip -1.9ex 
\membername{getHeight}
{\tt public int {\bf getHeight}(  )
}%end signature
}%end item
\divideents{getWidth}
\item{\vskip -1.9ex 
\membername{getWidth}
{\tt public int {\bf getWidth}(  )
}%end signature
}%end item
\divideents{set}
\item{\vskip -1.9ex 
\membername{set}
{\tt public synchronized Matrix {\bf set}( {\tt java.lang.Object } {\bf item},
{\tt int } {\bf x},
{\tt int } {\bf y} )
}%end signature
}%end item
\divideents{toString}
\item{\vskip -1.9ex 
\membername{toString}
{\tt public synchronized String {\bf toString}(  )
}%end signature
}%end item
\end{itemize}
}}
}
}
}
\newpage
\def\packagename{uk.ac.ic.doc.neuralnets.expressions}
\chapter{\bf Package uk.ac.ic.doc.neuralnets.expressions}{
\vskip -.25in
\hbox to \hsize{\it Package Contents\hfil Page}
\rule{\hsize}{.7mm}
\vskip .13in
\hbox{\bf Interfaces}
\entityintro{BindVariable}{l683}{...no description...}
\vskip .13in
\hbox{\bf Classes}
\entityintro{CalculationLexer}{l684}{...no description...}
\entityintro{CalculationParser}{l685}{...no description...}
\entityintro{Expression}{l686}{...no description...}
\entityintro{ExpressionException}{l687}{...no description...}
\vskip .1in
\rule{\hsize}{.7mm}
\vskip .1in
\newpage
\section{Interfaces}{
\startsection{Interface}{BindVariable}{l683}{%
\startsubsubsection{Declaration}{
\fbox{\vbox{
\hbox{\vbox{\small public interface 
BindVariable}}
\noindent\hbox{\vbox{{\bf implements} 
java.lang.annotation.Annotation}}
}}}
\startsubsubsection{Methods}{
\vskip -2em
\begin{itemize}
\item{\vskip -1.9ex 
\membername{rebind}
{\tt public boolean {\bf rebind}(  )
\label{l688}\label{l689}}%end signature
\begin{itemize}
\sld
\item{
\sld
{\bf Usage}
  \begin{itemize}\isep
   \item{
Whether or not an Expression should rebind this method each time it is
 evaluated. Defaults to false.
}%end item
  \end{itemize}
}
\end{itemize}
}%end item
\divideents{value}
\item{\vskip -1.9ex 
\membername{value}
{\tt public String {\bf value}(  )
\label{l690}\label{l691}}%end signature
\begin{itemize}
\sld
\item{
\sld
{\bf Usage}
  \begin{itemize}\isep
   \item{
The variable name to bind the annotated method to
}%end item
  \end{itemize}
}
\end{itemize}
}%end item
\end{itemize}
}
}
}
\section{Classes}{
\startsection{Class}{CalculationLexer}{l684}{%
\startsubsubsection{Declaration}{
\fbox{\vbox{
\hbox{\vbox{\small public 
class 
CalculationLexer}}
\noindent\hbox{\vbox{{\bf extends} org.antlr.runtime.Lexer}}
}}}
\startsubsubsection{Fields}{
\begin{itemize}
\item{
public static final int MOD\begin{itemize}\item{\vskip -.9ex }\end{itemize}
}
\item{
public static final int GRAND\begin{itemize}\item{\vskip -.9ex }\end{itemize}
}
\item{
public static final int INT\begin{itemize}\item{\vskip -.9ex }\end{itemize}
}
\item{
public static final int COSH\begin{itemize}\item{\vskip -.9ex }\end{itemize}
}
\item{
public static final int MULT\begin{itemize}\item{\vskip -.9ex }\end{itemize}
}
\item{
public static final int MINUS\begin{itemize}\item{\vskip -.9ex }\end{itemize}
}
\item{
public static final int EOF\begin{itemize}\item{\vskip -.9ex }\end{itemize}
}
\item{
public static final int SINH\begin{itemize}\item{\vskip -.9ex }\end{itemize}
}
\item{
public static final int LPAREN\begin{itemize}\item{\vskip -.9ex }\end{itemize}
}
\item{
public static final int RPAREN\begin{itemize}\item{\vskip -.9ex }\end{itemize}
}
\item{
public static final int TANH\begin{itemize}\item{\vskip -.9ex }\end{itemize}
}
\item{
public static final int WS\begin{itemize}\item{\vskip -.9ex }\end{itemize}
}
\item{
public static final int POW\begin{itemize}\item{\vskip -.9ex }\end{itemize}
}
\item{
public static final int NEWLINE\begin{itemize}\item{\vskip -.9ex }\end{itemize}
}
\item{
public static final int SIN\begin{itemize}\item{\vskip -.9ex }\end{itemize}
}
\item{
public static final int COS\begin{itemize}\item{\vskip -.9ex }\end{itemize}
}
\item{
public static final int TAN\begin{itemize}\item{\vskip -.9ex }\end{itemize}
}
\item{
public static final int RAND\begin{itemize}\item{\vskip -.9ex }\end{itemize}
}
\item{
public static final int DOUBLE\begin{itemize}\item{\vskip -.9ex }\end{itemize}
}
\item{
public static final int PLUS\begin{itemize}\item{\vskip -.9ex }\end{itemize}
}
\item{
public static final int VAR\begin{itemize}\item{\vskip -.9ex }\end{itemize}
}
\item{
public static final int DIV\begin{itemize}\item{\vskip -.9ex }\end{itemize}
}
\end{itemize}
}
\startsubsubsection{Constructors}{
\vskip -2em
\begin{itemize}
\item{\vskip -1.9ex 
\membername{CalculationLexer}
{\tt public {\bf CalculationLexer}(  )
\label{l692}\label{l693}}%end signature
}%end item
\divideents{CalculationLexer}
\item{\vskip -1.9ex 
\membername{CalculationLexer}
{\tt public {\bf CalculationLexer}( {\tt org.antlr.runtime.CharStream } {\bf input} )
\label{l694}\label{l695}}%end signature
}%end item
\divideents{CalculationLexer}
\item{\vskip -1.9ex 
\membername{CalculationLexer}
{\tt public {\bf CalculationLexer}( {\tt org.antlr.runtime.CharStream } {\bf input},
{\tt org.antlr.runtime.RecognizerSharedState } {\bf state} )
\label{l696}\label{l697}}%end signature
}%end item
\end{itemize}
}
\startsubsubsection{Methods}{
\vskip -2em
\begin{itemize}
\item{\vskip -1.9ex 
\membername{getGrammarFileName}
{\tt public String {\bf getGrammarFileName}(  )
\label{l698}\label{l699}}%end signature
}%end item
\divideents{mCOS}
\item{\vskip -1.9ex 
\membername{mCOS}
{\tt public final void {\bf mCOS}(  )
\label{l700}\label{l701}}%end signature
}%end item
\divideents{mCOSH}
\item{\vskip -1.9ex 
\membername{mCOSH}
{\tt public final void {\bf mCOSH}(  )
\label{l702}\label{l703}}%end signature
}%end item
\divideents{mDIV}
\item{\vskip -1.9ex 
\membername{mDIV}
{\tt public final void {\bf mDIV}(  )
\label{l704}\label{l705}}%end signature
}%end item
\divideents{mDOUBLE}
\item{\vskip -1.9ex 
\membername{mDOUBLE}
{\tt public final void {\bf mDOUBLE}(  )
\label{l706}\label{l707}}%end signature
}%end item
\divideents{mGRAND}
\item{\vskip -1.9ex 
\membername{mGRAND}
{\tt public final void {\bf mGRAND}(  )
\label{l708}\label{l709}}%end signature
}%end item
\divideents{mINT}
\item{\vskip -1.9ex 
\membername{mINT}
{\tt public final void {\bf mINT}(  )
\label{l710}\label{l711}}%end signature
}%end item
\divideents{mLPAREN}
\item{\vskip -1.9ex 
\membername{mLPAREN}
{\tt public final void {\bf mLPAREN}(  )
\label{l712}\label{l713}}%end signature
}%end item
\divideents{mMINUS}
\item{\vskip -1.9ex 
\membername{mMINUS}
{\tt public final void {\bf mMINUS}(  )
\label{l714}\label{l715}}%end signature
}%end item
\divideents{mMOD}
\item{\vskip -1.9ex 
\membername{mMOD}
{\tt public final void {\bf mMOD}(  )
\label{l716}\label{l717}}%end signature
}%end item
\divideents{mMULT}
\item{\vskip -1.9ex 
\membername{mMULT}
{\tt public final void {\bf mMULT}(  )
\label{l718}\label{l719}}%end signature
}%end item
\divideents{mNEWLINE}
\item{\vskip -1.9ex 
\membername{mNEWLINE}
{\tt public final void {\bf mNEWLINE}(  )
\label{l720}\label{l721}}%end signature
}%end item
\divideents{mPLUS}
\item{\vskip -1.9ex 
\membername{mPLUS}
{\tt public final void {\bf mPLUS}(  )
\label{l722}\label{l723}}%end signature
}%end item
\divideents{mPOW}
\item{\vskip -1.9ex 
\membername{mPOW}
{\tt public final void {\bf mPOW}(  )
\label{l724}\label{l725}}%end signature
}%end item
\divideents{mRAND}
\item{\vskip -1.9ex 
\membername{mRAND}
{\tt public final void {\bf mRAND}(  )
\label{l726}\label{l727}}%end signature
}%end item
\divideents{mRPAREN}
\item{\vskip -1.9ex 
\membername{mRPAREN}
{\tt public final void {\bf mRPAREN}(  )
\label{l728}\label{l729}}%end signature
}%end item
\divideents{mSIN}
\item{\vskip -1.9ex 
\membername{mSIN}
{\tt public final void {\bf mSIN}(  )
\label{l730}\label{l731}}%end signature
}%end item
\divideents{mSINH}
\item{\vskip -1.9ex 
\membername{mSINH}
{\tt public final void {\bf mSINH}(  )
\label{l732}\label{l733}}%end signature
}%end item
\divideents{mTAN}
\item{\vskip -1.9ex 
\membername{mTAN}
{\tt public final void {\bf mTAN}(  )
\label{l734}\label{l735}}%end signature
}%end item
\divideents{mTANH}
\item{\vskip -1.9ex 
\membername{mTANH}
{\tt public final void {\bf mTANH}(  )
\label{l736}\label{l737}}%end signature
}%end item
\divideents{mTokens}
\item{\vskip -1.9ex 
\membername{mTokens}
{\tt public void {\bf mTokens}(  )
\label{l738}\label{l739}}%end signature
}%end item
\divideents{mVAR}
\item{\vskip -1.9ex 
\membername{mVAR}
{\tt public final void {\bf mVAR}(  )
\label{l740}\label{l741}}%end signature
}%end item
\divideents{mWS}
\item{\vskip -1.9ex 
\membername{mWS}
{\tt public final void {\bf mWS}(  )
\label{l742}\label{l743}}%end signature
}%end item
\end{itemize}
}
\startsubsubsection{Methods inherited from class {\tt org.antlr.runtime.Lexer}}{
\par{\small 
\refdefined{l744}\vskip -2em
\begin{itemize}
\item{\vskip -1.9ex 
\membername{emit}
{\tt public Token {\bf emit}(  )
}%end signature
}%end item
\divideents{emit}
\item{\vskip -1.9ex 
\membername{emit}
{\tt public void {\bf emit}( {\tt org.antlr.runtime.Token } {\bf arg0} )
}%end signature
}%end item
\divideents{getCharErrorDisplay}
\item{\vskip -1.9ex 
\membername{getCharErrorDisplay}
{\tt public String {\bf getCharErrorDisplay}( {\tt int } {\bf arg0} )
}%end signature
}%end item
\divideents{getCharIndex}
\item{\vskip -1.9ex 
\membername{getCharIndex}
{\tt public int {\bf getCharIndex}(  )
}%end signature
}%end item
\divideents{getCharPositionInLine}
\item{\vskip -1.9ex 
\membername{getCharPositionInLine}
{\tt public int {\bf getCharPositionInLine}(  )
}%end signature
}%end item
\divideents{getCharStream}
\item{\vskip -1.9ex 
\membername{getCharStream}
{\tt public CharStream {\bf getCharStream}(  )
}%end signature
}%end item
\divideents{getErrorMessage}
\item{\vskip -1.9ex 
\membername{getErrorMessage}
{\tt public String {\bf getErrorMessage}( {\tt org.antlr.runtime.RecognitionException } {\bf arg0},
{\tt java.lang.String []} {\bf arg1} )
}%end signature
}%end item
\divideents{getLine}
\item{\vskip -1.9ex 
\membername{getLine}
{\tt public int {\bf getLine}(  )
}%end signature
}%end item
\divideents{getSourceName}
\item{\vskip -1.9ex 
\membername{getSourceName}
{\tt public String {\bf getSourceName}(  )
}%end signature
}%end item
\divideents{getText}
\item{\vskip -1.9ex 
\membername{getText}
{\tt public String {\bf getText}(  )
}%end signature
}%end item
\divideents{match}
\item{\vskip -1.9ex 
\membername{match}
{\tt public void {\bf match}( {\tt int } {\bf arg0} )
}%end signature
}%end item
\divideents{match}
\item{\vskip -1.9ex 
\membername{match}
{\tt public void {\bf match}( {\tt java.lang.String } {\bf arg0} )
}%end signature
}%end item
\divideents{matchAny}
\item{\vskip -1.9ex 
\membername{matchAny}
{\tt public void {\bf matchAny}(  )
}%end signature
}%end item
\divideents{matchRange}
\item{\vskip -1.9ex 
\membername{matchRange}
{\tt public void {\bf matchRange}( {\tt int } {\bf arg0},
{\tt int } {\bf arg1} )
}%end signature
}%end item
\divideents{mTokens}
\item{\vskip -1.9ex 
\membername{mTokens}
{\tt public abstract void {\bf mTokens}(  )
}%end signature
}%end item
\divideents{nextToken}
\item{\vskip -1.9ex 
\membername{nextToken}
{\tt public Token {\bf nextToken}(  )
}%end signature
}%end item
\divideents{recover}
\item{\vskip -1.9ex 
\membername{recover}
{\tt public void {\bf recover}( {\tt org.antlr.runtime.RecognitionException } {\bf arg0} )
}%end signature
}%end item
\divideents{reportError}
\item{\vskip -1.9ex 
\membername{reportError}
{\tt public void {\bf reportError}( {\tt org.antlr.runtime.RecognitionException } {\bf arg0} )
}%end signature
}%end item
\divideents{reset}
\item{\vskip -1.9ex 
\membername{reset}
{\tt public void {\bf reset}(  )
}%end signature
}%end item
\divideents{setCharStream}
\item{\vskip -1.9ex 
\membername{setCharStream}
{\tt public void {\bf setCharStream}( {\tt org.antlr.runtime.CharStream } {\bf arg0} )
}%end signature
}%end item
\divideents{setText}
\item{\vskip -1.9ex 
\membername{setText}
{\tt public void {\bf setText}( {\tt java.lang.String } {\bf arg0} )
}%end signature
}%end item
\divideents{skip}
\item{\vskip -1.9ex 
\membername{skip}
{\tt public void {\bf skip}(  )
}%end signature
}%end item
\divideents{traceIn}
\item{\vskip -1.9ex 
\membername{traceIn}
{\tt public void {\bf traceIn}( {\tt java.lang.String } {\bf arg0},
{\tt int } {\bf arg1} )
}%end signature
}%end item
\divideents{traceOut}
\item{\vskip -1.9ex 
\membername{traceOut}
{\tt public void {\bf traceOut}( {\tt java.lang.String } {\bf arg0},
{\tt int } {\bf arg1} )
}%end signature
}%end item
\end{itemize}
}}
\startsubsubsection{Methods inherited from class {\tt org.antlr.runtime.BaseRecognizer}}{
\par{\small 
\refdefined{l745}\vskip -2em
\begin{itemize}
\item{\vskip -1.9ex 
\membername{alreadyParsedRule}
{\tt public boolean {\bf alreadyParsedRule}( {\tt org.antlr.runtime.IntStream } {\bf arg0},
{\tt int } {\bf arg1} )
}%end signature
}%end item
\divideents{beginResync}
\item{\vskip -1.9ex 
\membername{beginResync}
{\tt public void {\bf beginResync}(  )
}%end signature
}%end item
\divideents{combineFollows}
\item{\vskip -1.9ex 
\membername{combineFollows}
{\tt protected BitSet {\bf combineFollows}( {\tt boolean } {\bf arg0} )
}%end signature
}%end item
\divideents{computeContextSensitiveRuleFOLLOW}
\item{\vskip -1.9ex 
\membername{computeContextSensitiveRuleFOLLOW}
{\tt protected BitSet {\bf computeContextSensitiveRuleFOLLOW}(  )
}%end signature
}%end item
\divideents{computeErrorRecoverySet}
\item{\vskip -1.9ex 
\membername{computeErrorRecoverySet}
{\tt protected BitSet {\bf computeErrorRecoverySet}(  )
}%end signature
}%end item
\divideents{consumeUntil}
\item{\vskip -1.9ex 
\membername{consumeUntil}
{\tt public void {\bf consumeUntil}( {\tt org.antlr.runtime.IntStream } {\bf arg0},
{\tt org.antlr.runtime.BitSet } {\bf arg1} )
}%end signature
}%end item
\divideents{consumeUntil}
\item{\vskip -1.9ex 
\membername{consumeUntil}
{\tt public void {\bf consumeUntil}( {\tt org.antlr.runtime.IntStream } {\bf arg0},
{\tt int } {\bf arg1} )
}%end signature
}%end item
\divideents{displayRecognitionError}
\item{\vskip -1.9ex 
\membername{displayRecognitionError}
{\tt public void {\bf displayRecognitionError}( {\tt java.lang.String []} {\bf arg0},
{\tt org.antlr.runtime.RecognitionException } {\bf arg1} )
}%end signature
}%end item
\divideents{emitErrorMessage}
\item{\vskip -1.9ex 
\membername{emitErrorMessage}
{\tt public void {\bf emitErrorMessage}( {\tt java.lang.String } {\bf arg0} )
}%end signature
}%end item
\divideents{endResync}
\item{\vskip -1.9ex 
\membername{endResync}
{\tt public void {\bf endResync}(  )
}%end signature
}%end item
\divideents{getBacktrackingLevel}
\item{\vskip -1.9ex 
\membername{getBacktrackingLevel}
{\tt public int {\bf getBacktrackingLevel}(  )
}%end signature
}%end item
\divideents{getCurrentInputSymbol}
\item{\vskip -1.9ex 
\membername{getCurrentInputSymbol}
{\tt protected Object {\bf getCurrentInputSymbol}( {\tt org.antlr.runtime.IntStream } {\bf arg0} )
}%end signature
}%end item
\divideents{getErrorHeader}
\item{\vskip -1.9ex 
\membername{getErrorHeader}
{\tt public String {\bf getErrorHeader}( {\tt org.antlr.runtime.RecognitionException } {\bf arg0} )
}%end signature
}%end item
\divideents{getErrorMessage}
\item{\vskip -1.9ex 
\membername{getErrorMessage}
{\tt public String {\bf getErrorMessage}( {\tt org.antlr.runtime.RecognitionException } {\bf arg0},
{\tt java.lang.String []} {\bf arg1} )
}%end signature
}%end item
\divideents{getGrammarFileName}
\item{\vskip -1.9ex 
\membername{getGrammarFileName}
{\tt public String {\bf getGrammarFileName}(  )
}%end signature
}%end item
\divideents{getMissingSymbol}
\item{\vskip -1.9ex 
\membername{getMissingSymbol}
{\tt protected Object {\bf getMissingSymbol}( {\tt org.antlr.runtime.IntStream } {\bf arg0},
{\tt org.antlr.runtime.RecognitionException } {\bf arg1},
{\tt int } {\bf arg2},
{\tt org.antlr.runtime.BitSet } {\bf arg3} )
}%end signature
}%end item
\divideents{getNumberOfSyntaxErrors}
\item{\vskip -1.9ex 
\membername{getNumberOfSyntaxErrors}
{\tt public int {\bf getNumberOfSyntaxErrors}(  )
}%end signature
}%end item
\divideents{getRuleInvocationStack}
\item{\vskip -1.9ex 
\membername{getRuleInvocationStack}
{\tt public List {\bf getRuleInvocationStack}(  )
}%end signature
}%end item
\divideents{getRuleInvocationStack}
\item{\vskip -1.9ex 
\membername{getRuleInvocationStack}
{\tt public static List {\bf getRuleInvocationStack}( {\tt java.lang.Throwable } {\bf arg0},
{\tt java.lang.String } {\bf arg1} )
}%end signature
}%end item
\divideents{getRuleMemoization}
\item{\vskip -1.9ex 
\membername{getRuleMemoization}
{\tt public int {\bf getRuleMemoization}( {\tt int } {\bf arg0},
{\tt int } {\bf arg1} )
}%end signature
}%end item
\divideents{getRuleMemoizationCacheSize}
\item{\vskip -1.9ex 
\membername{getRuleMemoizationCacheSize}
{\tt public int {\bf getRuleMemoizationCacheSize}(  )
}%end signature
}%end item
\divideents{getSourceName}
\item{\vskip -1.9ex 
\membername{getSourceName}
{\tt public abstract String {\bf getSourceName}(  )
}%end signature
}%end item
\divideents{getTokenErrorDisplay}
\item{\vskip -1.9ex 
\membername{getTokenErrorDisplay}
{\tt public String {\bf getTokenErrorDisplay}( {\tt org.antlr.runtime.Token } {\bf arg0} )
}%end signature
}%end item
\divideents{getTokenNames}
\item{\vskip -1.9ex 
\membername{getTokenNames}
{\tt public String {\bf getTokenNames}(  )
}%end signature
}%end item
\divideents{match}
\item{\vskip -1.9ex 
\membername{match}
{\tt public Object {\bf match}( {\tt org.antlr.runtime.IntStream } {\bf arg0},
{\tt int } {\bf arg1},
{\tt org.antlr.runtime.BitSet } {\bf arg2} )
}%end signature
}%end item
\divideents{matchAny}
\item{\vskip -1.9ex 
\membername{matchAny}
{\tt public void {\bf matchAny}( {\tt org.antlr.runtime.IntStream } {\bf arg0} )
}%end signature
}%end item
\divideents{memoize}
\item{\vskip -1.9ex 
\membername{memoize}
{\tt public void {\bf memoize}( {\tt org.antlr.runtime.IntStream } {\bf arg0},
{\tt int } {\bf arg1},
{\tt int } {\bf arg2} )
}%end signature
}%end item
\divideents{mismatch}
\item{\vskip -1.9ex 
\membername{mismatch}
{\tt protected void {\bf mismatch}( {\tt org.antlr.runtime.IntStream } {\bf arg0},
{\tt int } {\bf arg1},
{\tt org.antlr.runtime.BitSet } {\bf arg2} )
}%end signature
}%end item
\divideents{mismatchIsMissingToken}
\item{\vskip -1.9ex 
\membername{mismatchIsMissingToken}
{\tt public boolean {\bf mismatchIsMissingToken}( {\tt org.antlr.runtime.IntStream } {\bf arg0},
{\tt org.antlr.runtime.BitSet } {\bf arg1} )
}%end signature
}%end item
\divideents{mismatchIsUnwantedToken}
\item{\vskip -1.9ex 
\membername{mismatchIsUnwantedToken}
{\tt public boolean {\bf mismatchIsUnwantedToken}( {\tt org.antlr.runtime.IntStream } {\bf arg0},
{\tt int } {\bf arg1} )
}%end signature
}%end item
\divideents{pushFollow}
\item{\vskip -1.9ex 
\membername{pushFollow}
{\tt protected void {\bf pushFollow}( {\tt org.antlr.runtime.BitSet } {\bf arg0} )
}%end signature
}%end item
\divideents{recover}
\item{\vskip -1.9ex 
\membername{recover}
{\tt public void {\bf recover}( {\tt org.antlr.runtime.IntStream } {\bf arg0},
{\tt org.antlr.runtime.RecognitionException } {\bf arg1} )
}%end signature
}%end item
\divideents{recoverFromMismatchedSet}
\item{\vskip -1.9ex 
\membername{recoverFromMismatchedSet}
{\tt public Object {\bf recoverFromMismatchedSet}( {\tt org.antlr.runtime.IntStream } {\bf arg0},
{\tt org.antlr.runtime.RecognitionException } {\bf arg1},
{\tt org.antlr.runtime.BitSet } {\bf arg2} )
}%end signature
}%end item
\divideents{recoverFromMismatchedToken}
\item{\vskip -1.9ex 
\membername{recoverFromMismatchedToken}
{\tt protected Object {\bf recoverFromMismatchedToken}( {\tt org.antlr.runtime.IntStream } {\bf arg0},
{\tt int } {\bf arg1},
{\tt org.antlr.runtime.BitSet } {\bf arg2} )
}%end signature
}%end item
\divideents{reportError}
\item{\vskip -1.9ex 
\membername{reportError}
{\tt public void {\bf reportError}( {\tt org.antlr.runtime.RecognitionException } {\bf arg0} )
}%end signature
}%end item
\divideents{reset}
\item{\vskip -1.9ex 
\membername{reset}
{\tt public void {\bf reset}(  )
}%end signature
}%end item
\divideents{toStrings}
\item{\vskip -1.9ex 
\membername{toStrings}
{\tt public List {\bf toStrings}( {\tt java.util.List } {\bf arg0} )
}%end signature
}%end item
\divideents{traceIn}
\item{\vskip -1.9ex 
\membername{traceIn}
{\tt public void {\bf traceIn}( {\tt java.lang.String } {\bf arg0},
{\tt int } {\bf arg1},
{\tt java.lang.Object } {\bf arg2} )
}%end signature
}%end item
\divideents{traceOut}
\item{\vskip -1.9ex 
\membername{traceOut}
{\tt public void {\bf traceOut}( {\tt java.lang.String } {\bf arg0},
{\tt int } {\bf arg1},
{\tt java.lang.Object } {\bf arg2} )
}%end signature
}%end item
\end{itemize}
}}
}
\startsection{Class}{CalculationParser}{l685}{%
\startsubsubsection{Declaration}{
\fbox{\vbox{
\hbox{\vbox{\small public 
class 
CalculationParser}}
\noindent\hbox{\vbox{{\bf extends} org.antlr.runtime.Parser}}
}}}
\startsubsubsection{Fields}{
\begin{itemize}
\item{
public static final String tokenNames\begin{itemize}\item{\vskip -.9ex }\end{itemize}
}
\item{
public static final int MOD\begin{itemize}\item{\vskip -.9ex }\end{itemize}
}
\item{
public static final int INT\begin{itemize}\item{\vskip -.9ex }\end{itemize}
}
\item{
public static final int GRAND\begin{itemize}\item{\vskip -.9ex }\end{itemize}
}
\item{
public static final int COSH\begin{itemize}\item{\vskip -.9ex }\end{itemize}
}
\item{
public static final int MULT\begin{itemize}\item{\vskip -.9ex }\end{itemize}
}
\item{
public static final int MINUS\begin{itemize}\item{\vskip -.9ex }\end{itemize}
}
\item{
public static final int EOF\begin{itemize}\item{\vskip -.9ex }\end{itemize}
}
\item{
public static final int SINH\begin{itemize}\item{\vskip -.9ex }\end{itemize}
}
\item{
public static final int LPAREN\begin{itemize}\item{\vskip -.9ex }\end{itemize}
}
\item{
public static final int RPAREN\begin{itemize}\item{\vskip -.9ex }\end{itemize}
}
\item{
public static final int TANH\begin{itemize}\item{\vskip -.9ex }\end{itemize}
}
\item{
public static final int WS\begin{itemize}\item{\vskip -.9ex }\end{itemize}
}
\item{
public static final int POW\begin{itemize}\item{\vskip -.9ex }\end{itemize}
}
\item{
public static final int NEWLINE\begin{itemize}\item{\vskip -.9ex }\end{itemize}
}
\item{
public static final int SIN\begin{itemize}\item{\vskip -.9ex }\end{itemize}
}
\item{
public static final int COS\begin{itemize}\item{\vskip -.9ex }\end{itemize}
}
\item{
public static final int RAND\begin{itemize}\item{\vskip -.9ex }\end{itemize}
}
\item{
public static final int TAN\begin{itemize}\item{\vskip -.9ex }\end{itemize}
}
\item{
public static final int DOUBLE\begin{itemize}\item{\vskip -.9ex }\end{itemize}
}
\item{
public static final int PLUS\begin{itemize}\item{\vskip -.9ex }\end{itemize}
}
\item{
public static final int VAR\begin{itemize}\item{\vskip -.9ex }\end{itemize}
}
\item{
public static final int DIV\begin{itemize}\item{\vskip -.9ex }\end{itemize}
}
\item{
public static final BitSet FOLLOW\_lowLevelExpr\_in\_stat191\begin{itemize}\item{\vskip -.9ex }\end{itemize}
}
\item{
public static final BitSet FOLLOW\_NEWLINE\_in\_stat193\begin{itemize}\item{\vskip -.9ex }\end{itemize}
}
\item{
public static final BitSet FOLLOW\_multLevelExpr\_in\_lowLevelExpr220\begin{itemize}\item{\vskip -.9ex }\end{itemize}
}
\item{
public static final BitSet FOLLOW\_PLUS\_in\_lowLevelExpr234\begin{itemize}\item{\vskip -.9ex }\end{itemize}
}
\item{
public static final BitSet FOLLOW\_multLevelExpr\_in\_lowLevelExpr238\begin{itemize}\item{\vskip -.9ex }\end{itemize}
}
\item{
public static final BitSet FOLLOW\_MINUS\_in\_lowLevelExpr252\begin{itemize}\item{\vskip -.9ex }\end{itemize}
}
\item{
public static final BitSet FOLLOW\_multLevelExpr\_in\_lowLevelExpr256\begin{itemize}\item{\vskip -.9ex }\end{itemize}
}
\item{
public static final BitSet FOLLOW\_powLevelExpr\_in\_multLevelExpr294\begin{itemize}\item{\vskip -.9ex }\end{itemize}
}
\item{
public static final BitSet FOLLOW\_MULT\_in\_multLevelExpr314\begin{itemize}\item{\vskip -.9ex }\end{itemize}
}
\item{
public static final BitSet FOLLOW\_powLevelExpr\_in\_multLevelExpr318\begin{itemize}\item{\vskip -.9ex }\end{itemize}
}
\item{
public static final BitSet FOLLOW\_DIV\_in\_multLevelExpr329\begin{itemize}\item{\vskip -.9ex }\end{itemize}
}
\item{
public static final BitSet FOLLOW\_powLevelExpr\_in\_multLevelExpr333\begin{itemize}\item{\vskip -.9ex }\end{itemize}
}
\item{
public static final BitSet FOLLOW\_MOD\_in\_multLevelExpr344\begin{itemize}\item{\vskip -.9ex }\end{itemize}
}
\item{
public static final BitSet FOLLOW\_powLevelExpr\_in\_multLevelExpr348\begin{itemize}\item{\vskip -.9ex }\end{itemize}
}
\item{
public static final BitSet FOLLOW\_unary\_in\_powLevelExpr378\begin{itemize}\item{\vskip -.9ex }\end{itemize}
}
\item{
public static final BitSet FOLLOW\_POW\_in\_powLevelExpr386\begin{itemize}\item{\vskip -.9ex }\end{itemize}
}
\item{
public static final BitSet FOLLOW\_unary\_in\_powLevelExpr390\begin{itemize}\item{\vskip -.9ex }\end{itemize}
}
\item{
public static final BitSet FOLLOW\_atom\_in\_unary414\begin{itemize}\item{\vskip -.9ex }\end{itemize}
}
\item{
public static final BitSet FOLLOW\_MINUS\_in\_unary421\begin{itemize}\item{\vskip -.9ex }\end{itemize}
}
\item{
public static final BitSet FOLLOW\_atom\_in\_unary425\begin{itemize}\item{\vskip -.9ex }\end{itemize}
}
\item{
public static final BitSet FOLLOW\_INT\_in\_atom446\begin{itemize}\item{\vskip -.9ex }\end{itemize}
}
\item{
public static final BitSet FOLLOW\_VAR\_in\_atom453\begin{itemize}\item{\vskip -.9ex }\end{itemize}
}
\item{
public static final BitSet FOLLOW\_DOUBLE\_in\_atom460\begin{itemize}\item{\vskip -.9ex }\end{itemize}
}
\item{
public static final BitSet FOLLOW\_RAND\_in\_atom468\begin{itemize}\item{\vskip -.9ex }\end{itemize}
}
\item{
public static final BitSet FOLLOW\_GRAND\_in\_atom476\begin{itemize}\item{\vskip -.9ex }\end{itemize}
}
\item{
public static final BitSet FOLLOW\_LPAREN\_in\_atom486\begin{itemize}\item{\vskip -.9ex }\end{itemize}
}
\item{
public static final BitSet FOLLOW\_lowLevelExpr\_in\_atom488\begin{itemize}\item{\vskip -.9ex }\end{itemize}
}
\item{
public static final BitSet FOLLOW\_RPAREN\_in\_atom490\begin{itemize}\item{\vskip -.9ex }\end{itemize}
}
\item{
public static final BitSet FOLLOW\_SINH\_in\_atom497\begin{itemize}\item{\vskip -.9ex }\end{itemize}
}
\item{
public static final BitSet FOLLOW\_LPAREN\_in\_atom499\begin{itemize}\item{\vskip -.9ex }\end{itemize}
}
\item{
public static final BitSet FOLLOW\_lowLevelExpr\_in\_atom503\begin{itemize}\item{\vskip -.9ex }\end{itemize}
}
\item{
public static final BitSet FOLLOW\_RPAREN\_in\_atom506\begin{itemize}\item{\vskip -.9ex }\end{itemize}
}
\item{
public static final BitSet FOLLOW\_COSH\_in\_atom511\begin{itemize}\item{\vskip -.9ex }\end{itemize}
}
\item{
public static final BitSet FOLLOW\_LPAREN\_in\_atom513\begin{itemize}\item{\vskip -.9ex }\end{itemize}
}
\item{
public static final BitSet FOLLOW\_lowLevelExpr\_in\_atom517\begin{itemize}\item{\vskip -.9ex }\end{itemize}
}
\item{
public static final BitSet FOLLOW\_RPAREN\_in\_atom520\begin{itemize}\item{\vskip -.9ex }\end{itemize}
}
\item{
public static final BitSet FOLLOW\_TANH\_in\_atom525\begin{itemize}\item{\vskip -.9ex }\end{itemize}
}
\item{
public static final BitSet FOLLOW\_LPAREN\_in\_atom527\begin{itemize}\item{\vskip -.9ex }\end{itemize}
}
\item{
public static final BitSet FOLLOW\_lowLevelExpr\_in\_atom531\begin{itemize}\item{\vskip -.9ex }\end{itemize}
}
\item{
public static final BitSet FOLLOW\_RPAREN\_in\_atom534\begin{itemize}\item{\vskip -.9ex }\end{itemize}
}
\item{
public static final BitSet FOLLOW\_SIN\_in\_atom539\begin{itemize}\item{\vskip -.9ex }\end{itemize}
}
\item{
public static final BitSet FOLLOW\_LPAREN\_in\_atom541\begin{itemize}\item{\vskip -.9ex }\end{itemize}
}
\item{
public static final BitSet FOLLOW\_lowLevelExpr\_in\_atom545\begin{itemize}\item{\vskip -.9ex }\end{itemize}
}
\item{
public static final BitSet FOLLOW\_RPAREN\_in\_atom548\begin{itemize}\item{\vskip -.9ex }\end{itemize}
}
\item{
public static final BitSet FOLLOW\_COS\_in\_atom553\begin{itemize}\item{\vskip -.9ex }\end{itemize}
}
\item{
public static final BitSet FOLLOW\_LPAREN\_in\_atom555\begin{itemize}\item{\vskip -.9ex }\end{itemize}
}
\item{
public static final BitSet FOLLOW\_lowLevelExpr\_in\_atom559\begin{itemize}\item{\vskip -.9ex }\end{itemize}
}
\item{
public static final BitSet FOLLOW\_RPAREN\_in\_atom562\begin{itemize}\item{\vskip -.9ex }\end{itemize}
}
\item{
public static final BitSet FOLLOW\_TAN\_in\_atom567\begin{itemize}\item{\vskip -.9ex }\end{itemize}
}
\item{
public static final BitSet FOLLOW\_LPAREN\_in\_atom569\begin{itemize}\item{\vskip -.9ex }\end{itemize}
}
\item{
public static final BitSet FOLLOW\_lowLevelExpr\_in\_atom573\begin{itemize}\item{\vskip -.9ex }\end{itemize}
}
\item{
public static final BitSet FOLLOW\_RPAREN\_in\_atom576\begin{itemize}\item{\vskip -.9ex }\end{itemize}
}
\end{itemize}
}
\startsubsubsection{Constructors}{
\vskip -2em
\begin{itemize}
\item{\vskip -1.9ex 
\membername{CalculationParser}
{\tt public {\bf CalculationParser}( {\tt org.antlr.runtime.TokenStream } {\bf input} )
\label{l746}\label{l747}}%end signature
}%end item
\divideents{CalculationParser}
\item{\vskip -1.9ex 
\membername{CalculationParser}
{\tt public {\bf CalculationParser}( {\tt org.antlr.runtime.TokenStream } {\bf input},
{\tt org.antlr.runtime.RecognizerSharedState } {\bf state} )
\label{l748}\label{l749}}%end signature
}%end item
\end{itemize}
}
\startsubsubsection{Methods}{
\vskip -2em
\begin{itemize}
\item{\vskip -1.9ex 
\membername{atom}
{\tt public final Double {\bf atom}(  )
\label{l750}\label{l751}}%end signature
}%end item
\divideents{bind}
\item{\vskip -1.9ex 
\membername{bind}
{\tt public void {\bf bind}( {\tt java.lang.String } {\bf var},
{\tt java.lang.Double } {\bf val} )
\label{l752}\label{l753}}%end signature
}%end item
\divideents{displayRecognitionError}
\item{\vskip -1.9ex 
\membername{displayRecognitionError}
{\tt public void {\bf displayRecognitionError}( {\tt java.lang.String []} {\bf tokenNames},
{\tt org.antlr.runtime.RecognitionException } {\bf e} )
\label{l754}\label{l755}}%end signature
}%end item
\divideents{evaluate}
\item{\vskip -1.9ex 
\membername{evaluate}
{\tt public Double {\bf evaluate}(  )
\label{l756}\label{l757}}%end signature
}%end item
\divideents{getGrammarFileName}
\item{\vskip -1.9ex 
\membername{getGrammarFileName}
{\tt public String {\bf getGrammarFileName}(  )
\label{l758}\label{l759}}%end signature
}%end item
\divideents{getTokenNames}
\item{\vskip -1.9ex 
\membername{getTokenNames}
{\tt public String {\bf getTokenNames}(  )
\label{l760}\label{l761}}%end signature
}%end item
\divideents{lowLevelExpr}
\item{\vskip -1.9ex 
\membername{lowLevelExpr}
{\tt public final Double {\bf lowLevelExpr}(  )
\label{l762}\label{l763}}%end signature
}%end item
\divideents{multLevelExpr}
\item{\vskip -1.9ex 
\membername{multLevelExpr}
{\tt public final Double {\bf multLevelExpr}(  )
\label{l764}\label{l765}}%end signature
}%end item
\divideents{powLevelExpr}
\item{\vskip -1.9ex 
\membername{powLevelExpr}
{\tt public final Double {\bf powLevelExpr}(  )
\label{l766}\label{l767}}%end signature
}%end item
\divideents{stat}
\item{\vskip -1.9ex 
\membername{stat}
{\tt public final Double {\bf stat}(  )
\label{l768}\label{l769}}%end signature
}%end item
\divideents{unary}
\item{\vskip -1.9ex 
\membername{unary}
{\tt public final Double {\bf unary}(  )
\label{l770}\label{l771}}%end signature
}%end item
\end{itemize}
}
\startsubsubsection{Methods inherited from class {\tt org.antlr.runtime.Parser}}{
\par{\small 
\refdefined{l772}\vskip -2em
\begin{itemize}
\item{\vskip -1.9ex 
\membername{getCurrentInputSymbol}
{\tt protected Object {\bf getCurrentInputSymbol}( {\tt org.antlr.runtime.IntStream } {\bf arg0} )
}%end signature
}%end item
\divideents{getMissingSymbol}
\item{\vskip -1.9ex 
\membername{getMissingSymbol}
{\tt protected Object {\bf getMissingSymbol}( {\tt org.antlr.runtime.IntStream } {\bf arg0},
{\tt org.antlr.runtime.RecognitionException } {\bf arg1},
{\tt int } {\bf arg2},
{\tt org.antlr.runtime.BitSet } {\bf arg3} )
}%end signature
}%end item
\divideents{getSourceName}
\item{\vskip -1.9ex 
\membername{getSourceName}
{\tt public String {\bf getSourceName}(  )
}%end signature
}%end item
\divideents{getTokenStream}
\item{\vskip -1.9ex 
\membername{getTokenStream}
{\tt public TokenStream {\bf getTokenStream}(  )
}%end signature
}%end item
\divideents{reset}
\item{\vskip -1.9ex 
\membername{reset}
{\tt public void {\bf reset}(  )
}%end signature
}%end item
\divideents{setTokenStream}
\item{\vskip -1.9ex 
\membername{setTokenStream}
{\tt public void {\bf setTokenStream}( {\tt org.antlr.runtime.TokenStream } {\bf arg0} )
}%end signature
}%end item
\divideents{traceIn}
\item{\vskip -1.9ex 
\membername{traceIn}
{\tt public void {\bf traceIn}( {\tt java.lang.String } {\bf arg0},
{\tt int } {\bf arg1} )
}%end signature
}%end item
\divideents{traceOut}
\item{\vskip -1.9ex 
\membername{traceOut}
{\tt public void {\bf traceOut}( {\tt java.lang.String } {\bf arg0},
{\tt int } {\bf arg1} )
}%end signature
}%end item
\end{itemize}
}}
\startsubsubsection{Methods inherited from class {\tt org.antlr.runtime.BaseRecognizer}}{
\par{\small 
\refdefined{l745}\vskip -2em
\begin{itemize}
\item{\vskip -1.9ex 
\membername{alreadyParsedRule}
{\tt public boolean {\bf alreadyParsedRule}( {\tt org.antlr.runtime.IntStream } {\bf arg0},
{\tt int } {\bf arg1} )
}%end signature
}%end item
\divideents{beginResync}
\item{\vskip -1.9ex 
\membername{beginResync}
{\tt public void {\bf beginResync}(  )
}%end signature
}%end item
\divideents{combineFollows}
\item{\vskip -1.9ex 
\membername{combineFollows}
{\tt protected BitSet {\bf combineFollows}( {\tt boolean } {\bf arg0} )
}%end signature
}%end item
\divideents{computeContextSensitiveRuleFOLLOW}
\item{\vskip -1.9ex 
\membername{computeContextSensitiveRuleFOLLOW}
{\tt protected BitSet {\bf computeContextSensitiveRuleFOLLOW}(  )
}%end signature
}%end item
\divideents{computeErrorRecoverySet}
\item{\vskip -1.9ex 
\membername{computeErrorRecoverySet}
{\tt protected BitSet {\bf computeErrorRecoverySet}(  )
}%end signature
}%end item
\divideents{consumeUntil}
\item{\vskip -1.9ex 
\membername{consumeUntil}
{\tt public void {\bf consumeUntil}( {\tt org.antlr.runtime.IntStream } {\bf arg0},
{\tt org.antlr.runtime.BitSet } {\bf arg1} )
}%end signature
}%end item
\divideents{consumeUntil}
\item{\vskip -1.9ex 
\membername{consumeUntil}
{\tt public void {\bf consumeUntil}( {\tt org.antlr.runtime.IntStream } {\bf arg0},
{\tt int } {\bf arg1} )
}%end signature
}%end item
\divideents{displayRecognitionError}
\item{\vskip -1.9ex 
\membername{displayRecognitionError}
{\tt public void {\bf displayRecognitionError}( {\tt java.lang.String []} {\bf arg0},
{\tt org.antlr.runtime.RecognitionException } {\bf arg1} )
}%end signature
}%end item
\divideents{emitErrorMessage}
\item{\vskip -1.9ex 
\membername{emitErrorMessage}
{\tt public void {\bf emitErrorMessage}( {\tt java.lang.String } {\bf arg0} )
}%end signature
}%end item
\divideents{endResync}
\item{\vskip -1.9ex 
\membername{endResync}
{\tt public void {\bf endResync}(  )
}%end signature
}%end item
\divideents{getBacktrackingLevel}
\item{\vskip -1.9ex 
\membername{getBacktrackingLevel}
{\tt public int {\bf getBacktrackingLevel}(  )
}%end signature
}%end item
\divideents{getCurrentInputSymbol}
\item{\vskip -1.9ex 
\membername{getCurrentInputSymbol}
{\tt protected Object {\bf getCurrentInputSymbol}( {\tt org.antlr.runtime.IntStream } {\bf arg0} )
}%end signature
}%end item
\divideents{getErrorHeader}
\item{\vskip -1.9ex 
\membername{getErrorHeader}
{\tt public String {\bf getErrorHeader}( {\tt org.antlr.runtime.RecognitionException } {\bf arg0} )
}%end signature
}%end item
\divideents{getErrorMessage}
\item{\vskip -1.9ex 
\membername{getErrorMessage}
{\tt public String {\bf getErrorMessage}( {\tt org.antlr.runtime.RecognitionException } {\bf arg0},
{\tt java.lang.String []} {\bf arg1} )
}%end signature
}%end item
\divideents{getGrammarFileName}
\item{\vskip -1.9ex 
\membername{getGrammarFileName}
{\tt public String {\bf getGrammarFileName}(  )
}%end signature
}%end item
\divideents{getMissingSymbol}
\item{\vskip -1.9ex 
\membername{getMissingSymbol}
{\tt protected Object {\bf getMissingSymbol}( {\tt org.antlr.runtime.IntStream } {\bf arg0},
{\tt org.antlr.runtime.RecognitionException } {\bf arg1},
{\tt int } {\bf arg2},
{\tt org.antlr.runtime.BitSet } {\bf arg3} )
}%end signature
}%end item
\divideents{getNumberOfSyntaxErrors}
\item{\vskip -1.9ex 
\membername{getNumberOfSyntaxErrors}
{\tt public int {\bf getNumberOfSyntaxErrors}(  )
}%end signature
}%end item
\divideents{getRuleInvocationStack}
\item{\vskip -1.9ex 
\membername{getRuleInvocationStack}
{\tt public List {\bf getRuleInvocationStack}(  )
}%end signature
}%end item
\divideents{getRuleInvocationStack}
\item{\vskip -1.9ex 
\membername{getRuleInvocationStack}
{\tt public static List {\bf getRuleInvocationStack}( {\tt java.lang.Throwable } {\bf arg0},
{\tt java.lang.String } {\bf arg1} )
}%end signature
}%end item
\divideents{getRuleMemoization}
\item{\vskip -1.9ex 
\membername{getRuleMemoization}
{\tt public int {\bf getRuleMemoization}( {\tt int } {\bf arg0},
{\tt int } {\bf arg1} )
}%end signature
}%end item
\divideents{getRuleMemoizationCacheSize}
\item{\vskip -1.9ex 
\membername{getRuleMemoizationCacheSize}
{\tt public int {\bf getRuleMemoizationCacheSize}(  )
}%end signature
}%end item
\divideents{getSourceName}
\item{\vskip -1.9ex 
\membername{getSourceName}
{\tt public abstract String {\bf getSourceName}(  )
}%end signature
}%end item
\divideents{getTokenErrorDisplay}
\item{\vskip -1.9ex 
\membername{getTokenErrorDisplay}
{\tt public String {\bf getTokenErrorDisplay}( {\tt org.antlr.runtime.Token } {\bf arg0} )
}%end signature
}%end item
\divideents{getTokenNames}
\item{\vskip -1.9ex 
\membername{getTokenNames}
{\tt public String {\bf getTokenNames}(  )
}%end signature
}%end item
\divideents{match}
\item{\vskip -1.9ex 
\membername{match}
{\tt public Object {\bf match}( {\tt org.antlr.runtime.IntStream } {\bf arg0},
{\tt int } {\bf arg1},
{\tt org.antlr.runtime.BitSet } {\bf arg2} )
}%end signature
}%end item
\divideents{matchAny}
\item{\vskip -1.9ex 
\membername{matchAny}
{\tt public void {\bf matchAny}( {\tt org.antlr.runtime.IntStream } {\bf arg0} )
}%end signature
}%end item
\divideents{memoize}
\item{\vskip -1.9ex 
\membername{memoize}
{\tt public void {\bf memoize}( {\tt org.antlr.runtime.IntStream } {\bf arg0},
{\tt int } {\bf arg1},
{\tt int } {\bf arg2} )
}%end signature
}%end item
\divideents{mismatch}
\item{\vskip -1.9ex 
\membername{mismatch}
{\tt protected void {\bf mismatch}( {\tt org.antlr.runtime.IntStream } {\bf arg0},
{\tt int } {\bf arg1},
{\tt org.antlr.runtime.BitSet } {\bf arg2} )
}%end signature
}%end item
\divideents{mismatchIsMissingToken}
\item{\vskip -1.9ex 
\membername{mismatchIsMissingToken}
{\tt public boolean {\bf mismatchIsMissingToken}( {\tt org.antlr.runtime.IntStream } {\bf arg0},
{\tt org.antlr.runtime.BitSet } {\bf arg1} )
}%end signature
}%end item
\divideents{mismatchIsUnwantedToken}
\item{\vskip -1.9ex 
\membername{mismatchIsUnwantedToken}
{\tt public boolean {\bf mismatchIsUnwantedToken}( {\tt org.antlr.runtime.IntStream } {\bf arg0},
{\tt int } {\bf arg1} )
}%end signature
}%end item
\divideents{pushFollow}
\item{\vskip -1.9ex 
\membername{pushFollow}
{\tt protected void {\bf pushFollow}( {\tt org.antlr.runtime.BitSet } {\bf arg0} )
}%end signature
}%end item
\divideents{recover}
\item{\vskip -1.9ex 
\membername{recover}
{\tt public void {\bf recover}( {\tt org.antlr.runtime.IntStream } {\bf arg0},
{\tt org.antlr.runtime.RecognitionException } {\bf arg1} )
}%end signature
}%end item
\divideents{recoverFromMismatchedSet}
\item{\vskip -1.9ex 
\membername{recoverFromMismatchedSet}
{\tt public Object {\bf recoverFromMismatchedSet}( {\tt org.antlr.runtime.IntStream } {\bf arg0},
{\tt org.antlr.runtime.RecognitionException } {\bf arg1},
{\tt org.antlr.runtime.BitSet } {\bf arg2} )
}%end signature
}%end item
\divideents{recoverFromMismatchedToken}
\item{\vskip -1.9ex 
\membername{recoverFromMismatchedToken}
{\tt protected Object {\bf recoverFromMismatchedToken}( {\tt org.antlr.runtime.IntStream } {\bf arg0},
{\tt int } {\bf arg1},
{\tt org.antlr.runtime.BitSet } {\bf arg2} )
}%end signature
}%end item
\divideents{reportError}
\item{\vskip -1.9ex 
\membername{reportError}
{\tt public void {\bf reportError}( {\tt org.antlr.runtime.RecognitionException } {\bf arg0} )
}%end signature
}%end item
\divideents{reset}
\item{\vskip -1.9ex 
\membername{reset}
{\tt public void {\bf reset}(  )
}%end signature
}%end item
\divideents{toStrings}
\item{\vskip -1.9ex 
\membername{toStrings}
{\tt public List {\bf toStrings}( {\tt java.util.List } {\bf arg0} )
}%end signature
}%end item
\divideents{traceIn}
\item{\vskip -1.9ex 
\membername{traceIn}
{\tt public void {\bf traceIn}( {\tt java.lang.String } {\bf arg0},
{\tt int } {\bf arg1},
{\tt java.lang.Object } {\bf arg2} )
}%end signature
}%end item
\divideents{traceOut}
\item{\vskip -1.9ex 
\membername{traceOut}
{\tt public void {\bf traceOut}( {\tt java.lang.String } {\bf arg0},
{\tt int } {\bf arg1},
{\tt java.lang.Object } {\bf arg2} )
}%end signature
}%end item
\end{itemize}
}}
}
\startsection{Class}{Expression}{l686}{%
\startsubsubsection{Declaration}{
\fbox{\vbox{
\hbox{\vbox{\small public 
class 
Expression}}
\noindent\hbox{\vbox{{\bf extends} java.lang.Object}}
}}}
\startsubsubsection{Constructors}{
\vskip -2em
\begin{itemize}
\item{\vskip -1.9ex 
\membername{Expression}
{\tt public {\bf Expression}( {\tt java.lang.Double } {\bf value} )
\label{l773}\label{l774}}%end signature
\begin{itemize}
\sld
\item{
\sld
{\bf Usage}
  \begin{itemize}\isep
   \item{
Create an Expression to encode the given value
}%end item
  \end{itemize}
}
\item{
\sld
{\bf Parameters}
\sld\isep
  \begin{itemize}
\sld\isep
   \item{
\sld
{\tt value} - The value returned by this Expression}
  \end{itemize}
}%end item
\end{itemize}
}%end item
\divideents{Expression}
\item{\vskip -1.9ex 
\membername{Expression}
{\tt public {\bf Expression}( {\tt java.lang.String } {\bf expr} )
\label{l775}\label{l776}}%end signature
\begin{itemize}
\sld
\item{
\sld
{\bf Usage}
  \begin{itemize}\isep
   \item{
Create an Expression for the given string
}%end item
  \end{itemize}
}
\item{
\sld
{\bf Parameters}
\sld\isep
  \begin{itemize}
\sld\isep
   \item{
\sld
{\tt expr} - The expression to represent}
  \end{itemize}
}%end item
\end{itemize}
}%end item
\end{itemize}
}
\startsubsubsection{Methods}{
\vskip -2em
\begin{itemize}
\item{\vskip -1.9ex 
\membername{bind}
{\tt public void {\bf bind}( {\tt java.lang.Object } {\bf o} )
\label{l777}\label{l778}}%end signature
\begin{itemize}
\sld
\item{
\sld
{\bf Usage}
  \begin{itemize}\isep
   \item{
Bind variables according to BindVariable annotations present in this
 object, and all of its super-classes
}%end item
  \end{itemize}
}
\item{
\sld
{\bf Parameters}
\sld\isep
  \begin{itemize}
\sld\isep
   \item{
\sld
{\tt o} - The object to bind variables from}
  \end{itemize}
}%end item
\end{itemize}
}%end item
\divideents{bind}
\item{\vskip -1.9ex 
\membername{bind}
{\tt public void {\bf bind}( {\tt java.lang.String } {\bf var},
{\tt java.lang.Double } {\bf val} )
\label{l779}\label{l780}}%end signature
\begin{itemize}
\sld
\item{
\sld
{\bf Usage}
  \begin{itemize}\isep
   \item{
Manually bind a variable in the expression
}%end item
  \end{itemize}
}
\item{
\sld
{\bf Parameters}
\sld\isep
  \begin{itemize}
\sld\isep
   \item{
\sld
{\tt var} - The variable to bind}
   \item{
\sld
{\tt val} - The value to bind to}
  \end{itemize}
}%end item
\end{itemize}
}%end item
\divideents{bind}
\item{\vskip -1.9ex 
\membername{bind}
{\tt protected void {\bf bind}( {\tt java.lang.String } {\bf var},
{\tt java.lang.reflect.Method } {\bf m} )
\label{l781}\label{l782}}%end signature
}%end item
\divideents{evaluate}
\item{\vskip -1.9ex 
\membername{evaluate}
{\tt public Double {\bf evaluate}(  )
\label{l783}\label{l784}}%end signature
\begin{itemize}
\sld
\item{
\sld
{\bf Usage}
  \begin{itemize}\isep
   \item{
Evaluate the expression after refreshing its current bindings
}%end item
  \end{itemize}
}
\item{{\bf Returns} - 
The value this expression evaluates to 
}%end item
\item{{\bf Exceptions}
  \begin{itemize}
\sld
   \item{\vskip -.6ex{\tt uk.ac.ic.doc.neuralnets.expressions.ExpressionException} - }
  \end{itemize}
}%end item
\end{itemize}
}%end item
\divideents{evaluate}
\item{\vskip -1.9ex 
\membername{evaluate}
{\tt public Double {\bf evaluate}( {\tt java.lang.Object } {\bf o} )
\label{l785}\label{l786}}%end signature
\begin{itemize}
\sld
\item{
\sld
{\bf Usage}
  \begin{itemize}\isep
   \item{
Re-bind variables, then evaluate the expression
}%end item
  \end{itemize}
}
\item{
\sld
{\bf Parameters}
\sld\isep
  \begin{itemize}
\sld\isep
   \item{
\sld
{\tt o} - The object to bind variables from}
  \end{itemize}
}%end item
\item{{\bf Returns} - 
The value this expression evaluates to 
}%end item
\item{{\bf Exceptions}
  \begin{itemize}
\sld
   \item{\vskip -.6ex{\tt uk.ac.ic.doc.neuralnets.expressions.ExpressionException} - }
  \end{itemize}
}%end item
\end{itemize}
}%end item
\divideents{getExpression}
\item{\vskip -1.9ex 
\membername{getExpression}
{\tt public String {\bf getExpression}(  )
\label{l787}\label{l788}}%end signature
\begin{itemize}
\sld
\item{
\sld
{\bf Usage}
  \begin{itemize}\isep
   \item{
Answer the input expression
}%end item
  \end{itemize}
}
\item{{\bf Returns} - 
The mathematical expression encoded by this object 
}%end item
\end{itemize}
}%end item
\divideents{getParser}
\item{\vskip -1.9ex 
\membername{getParser}
{\tt protected CalculationParser {\bf getParser}( {\tt java.lang.String } {\bf ex} )
\label{l789}\label{l790}}%end signature
}%end item
\divideents{toString}
\item{\vskip -1.9ex 
\membername{toString}
{\tt public String {\bf toString}(  )
\label{l791}\label{l792}}%end signature
}%end item
\end{itemize}
}
}
\startsection{Class}{ExpressionException}{l687}{%
\startsubsubsection{Declaration}{
\fbox{\vbox{
\hbox{\vbox{\small public 
class 
ExpressionException}}
\noindent\hbox{\vbox{{\bf extends} java.lang.Exception}}
}}}
\startsubsubsection{Constructors}{
\vskip -2em
\begin{itemize}
\item{\vskip -1.9ex 
\membername{ExpressionException}
{\tt public {\bf ExpressionException}( {\tt java.lang.Exception } {\bf e} )
\label{l793}\label{l794}}%end signature
}%end item
\divideents{ExpressionException}
\item{\vskip -1.9ex 
\membername{ExpressionException}
{\tt public {\bf ExpressionException}( {\tt java.lang.String } {\bf msg} )
\label{l795}\label{l796}}%end signature
}%end item
\end{itemize}
}
\startsubsubsection{Methods inherited from class {\tt java.lang.Exception}}{
\par{\small 
\refdefined{l454}}}
\startsubsubsection{Methods inherited from class {\tt java.lang.Throwable}}{
\par{\small 
\refdefined{l455}\vskip -2em
\begin{itemize}
\item{\vskip -1.9ex 
\membername{fillInStackTrace}
{\tt public synchronized native Throwable {\bf fillInStackTrace}(  )
}%end signature
}%end item
\divideents{getCause}
\item{\vskip -1.9ex 
\membername{getCause}
{\tt public Throwable {\bf getCause}(  )
}%end signature
}%end item
\divideents{getLocalizedMessage}
\item{\vskip -1.9ex 
\membername{getLocalizedMessage}
{\tt public String {\bf getLocalizedMessage}(  )
}%end signature
}%end item
\divideents{getMessage}
\item{\vskip -1.9ex 
\membername{getMessage}
{\tt public String {\bf getMessage}(  )
}%end signature
}%end item
\divideents{getStackTrace}
\item{\vskip -1.9ex 
\membername{getStackTrace}
{\tt public StackTraceElement {\bf getStackTrace}(  )
}%end signature
}%end item
\divideents{initCause}
\item{\vskip -1.9ex 
\membername{initCause}
{\tt public synchronized Throwable {\bf initCause}( {\tt java.lang.Throwable } {\bf arg0} )
}%end signature
}%end item
\divideents{printStackTrace}
\item{\vskip -1.9ex 
\membername{printStackTrace}
{\tt public void {\bf printStackTrace}(  )
}%end signature
}%end item
\divideents{printStackTrace}
\item{\vskip -1.9ex 
\membername{printStackTrace}
{\tt public void {\bf printStackTrace}( {\tt java.io.PrintStream } {\bf arg0} )
}%end signature
}%end item
\divideents{printStackTrace}
\item{\vskip -1.9ex 
\membername{printStackTrace}
{\tt public void {\bf printStackTrace}( {\tt java.io.PrintWriter } {\bf arg0} )
}%end signature
}%end item
\divideents{setStackTrace}
\item{\vskip -1.9ex 
\membername{setStackTrace}
{\tt public void {\bf setStackTrace}( {\tt java.lang.StackTraceElement []} {\bf arg0} )
}%end signature
}%end item
\divideents{toString}
\item{\vskip -1.9ex 
\membername{toString}
{\tt public String {\bf toString}(  )
}%end signature
}%end item
\end{itemize}
}}
}
}
}
\newpage
\def\packagename{uk.ac.ic.doc.neuralnets.commands}
\chapter{\bf Package uk.ac.ic.doc.neuralnets.commands}{
\vskip -.25in
\hbox to \hsize{\it Package Contents\hfil Page}
\rule{\hsize}{.7mm}
\vskip .13in
\hbox{\bf Classes}
\entityintro{Command}{l797}{Action that can be undone or redone.}
\entityintro{CommandControl}{l798}{Implements undo and redo functionality.}
\entityintro{CommandEvent}{l799}{...no description...}
\vskip .1in
\rule{\hsize}{.7mm}
\vskip .1in
\newpage
\section{Classes}{
\startsection{Class}{Command}{l797}{%
{\small Action that can be undone or redone.}
\vskip .1in 
\startsubsubsection{Declaration}{
\fbox{\vbox{
\hbox{\vbox{\small public abstract 
class 
Command}}
\noindent\hbox{\vbox{{\bf extends} java.lang.Object}}
\noindent\hbox{\vbox{{\bf implements} 
java.lang.Runnable}}
}}}
\startsubsubsection{Constructors}{
\vskip -2em
\begin{itemize}
\item{\vskip -1.9ex 
\membername{Command}
{\tt public {\bf Command}(  )
\label{l800}\label{l801}}%end signature
}%end item
\end{itemize}
}
\startsubsubsection{Methods}{
\vskip -2em
\begin{itemize}
\item{\vskip -1.9ex 
\membername{execute}
{\tt protected abstract void {\bf execute}(  )
\label{l802}\label{l803}}%end signature
}%end item
\divideents{isUndo}
\item{\vskip -1.9ex 
\membername{isUndo}
{\tt public boolean {\bf isUndo}(  )
\label{l804}\label{l805}}%end signature
\begin{itemize}
\sld
\item{
\sld
{\bf Usage}
  \begin{itemize}\isep
   \item{
Returns the value of whether the command is set to undo.
}%end item
  \end{itemize}
}
\item{{\bf Returns} - 
Boolean commands undo state. 
}%end item
\end{itemize}
}%end item
\divideents{run}
\item{\vskip -1.9ex 
\membername{run}
{\tt public void {\bf run}(  )
\label{l806}\label{l807}}%end signature
\begin{itemize}
\sld
\item{
\sld
{\bf Usage}
  \begin{itemize}\isep
   \item{
Runs the command, undone is undo state is true, else command executed.
}%end item
  \end{itemize}
}
\end{itemize}
}%end item
\divideents{setUndo}
\item{\vskip -1.9ex 
\membername{setUndo}
{\tt public void {\bf setUndo}( {\tt boolean } {\bf undo} )
\label{l808}\label{l809}}%end signature
\begin{itemize}
\sld
\item{
\sld
{\bf Usage}
  \begin{itemize}\isep
   \item{
Sets the commands state of undo.
}%end item
  \end{itemize}
}
\item{
\sld
{\bf Parameters}
\sld\isep
  \begin{itemize}
\sld\isep
   \item{
\sld
{\tt undo} - Boolean for undo state.}
  \end{itemize}
}%end item
\end{itemize}
}%end item
\divideents{undo}
\item{\vskip -1.9ex 
\membername{undo}
{\tt protected abstract void {\bf undo}(  )
\label{l810}\label{l811}}%end signature
}%end item
\end{itemize}
}
}
\startsection{Class}{CommandControl}{l798}{%
{\small Implements undo and redo functionality. The addCommand() method adds a new
 stack and runs it, and the undo() and redo() methods can be called from the
 GUI.}
\vskip .1in 
\startsubsubsection{Declaration}{
\fbox{\vbox{
\hbox{\vbox{\small public 
class 
CommandControl}}
\noindent\hbox{\vbox{{\bf extends} java.lang.Object}}
}}}
\startsubsubsection{Constructors}{
\vskip -2em
\begin{itemize}
\item{\vskip -1.9ex 
\membername{CommandControl}
{\tt public {\bf CommandControl}(  )
\label{l812}\label{l813}}%end signature
}%end item
\end{itemize}
}
\startsubsubsection{Methods}{
\vskip -2em
\begin{itemize}
\item{\vskip -1.9ex 
\membername{addCommand}
{\tt public void {\bf addCommand}( {\tt uk.ac.ic.doc.neuralnets.commands.Command } {\bf command} )
\label{l814}\label{l815}}%end signature
\begin{itemize}
\sld
\item{
\sld
{\bf Usage}
  \begin{itemize}\isep
   \item{
Executes a command and adds it to the stack so it can be undone and
 redone.
}%end item
  \end{itemize}
}
\item{
\sld
{\bf Parameters}
\sld\isep
  \begin{itemize}
\sld\isep
   \item{
\sld
{\tt command} - }
  \end{itemize}
}%end item
\end{itemize}
}%end item
\divideents{canRedo}
\item{\vskip -1.9ex 
\membername{canRedo}
{\tt public boolean {\bf canRedo}(  )
\label{l816}\label{l817}}%end signature
\begin{itemize}
\sld
\item{
\sld
{\bf Usage}
  \begin{itemize}\isep
   \item{
Returns boolean value of ability to redo.
}%end item
  \end{itemize}
}
\item{{\bf Returns} - 
Boolean of ability to redo. 
}%end item
\end{itemize}
}%end item
\divideents{canUndo}
\item{\vskip -1.9ex 
\membername{canUndo}
{\tt public boolean {\bf canUndo}(  )
\label{l818}\label{l819}}%end signature
\begin{itemize}
\sld
\item{
\sld
{\bf Usage}
  \begin{itemize}\isep
   \item{
Returns boolean value of ability to undo.
}%end item
  \end{itemize}
}
\item{{\bf Returns} - 
Boolean of ability to undo. 
}%end item
\end{itemize}
}%end item
\divideents{redo}
\item{\vskip -1.9ex 
\membername{redo}
{\tt public void {\bf redo}(  )
\label{l820}\label{l821}}%end signature
\begin{itemize}
\sld
\item{
\sld
{\bf Usage}
  \begin{itemize}\isep
   \item{
Redoes the last command that was undone.
}%end item
  \end{itemize}
}
\end{itemize}
}%end item
\divideents{reset}
\item{\vskip -1.9ex 
\membername{reset}
{\tt public void {\bf reset}(  )
\label{l822}\label{l823}}%end signature
}%end item
\divideents{stopDispatcher}
\item{\vskip -1.9ex 
\membername{stopDispatcher}
{\tt public void {\bf stopDispatcher}(  )
\label{l824}\label{l825}}%end signature
}%end item
\divideents{undo}
\item{\vskip -1.9ex 
\membername{undo}
{\tt public void {\bf undo}(  )
\label{l826}\label{l827}}%end signature
\begin{itemize}
\sld
\item{
\sld
{\bf Usage}
  \begin{itemize}\isep
   \item{
Undoes the most recent command.
}%end item
  \end{itemize}
}
\end{itemize}
}%end item
\end{itemize}
}
}
\startsection{Class}{CommandEvent}{l799}{%
\startsubsubsection{Declaration}{
\fbox{\vbox{
\hbox{\vbox{\small public 
class 
CommandEvent}}
\noindent\hbox{\vbox{{\bf extends} uk.ac.ic.doc.neuralnets.events.Event}}
}}}
\startsubsubsection{Constructors}{
\vskip -2em
\begin{itemize}
\item{\vskip -1.9ex 
\membername{CommandEvent}
{\tt public {\bf CommandEvent}(  )
\label{l828}\label{l829}}%end signature
}%end item
\end{itemize}
}
\startsubsubsection{Methods}{
\vskip -2em
\begin{itemize}
\item{\vskip -1.9ex 
\membername{toString}
{\tt public String {\bf toString}(  )
\label{l830}\label{l831}}%end signature
}%end item
\end{itemize}
}
\startsubsubsection{Methods inherited from class {\tt uk.ac.ic.doc.neuralnets.events.Event}}{
\par{\small 
\refdefined{l220}\vskip -2em
\begin{itemize}
\item{\vskip -1.9ex 
\membername{toString}
{\tt public abstract String {\bf toString}(  )
}%end signature
}%end item
\end{itemize}
}}
}
}
}
\newpage
\def\packagename{uk.ac.ic.doc.neuralnets.gui.graph.listener}
\chapter{\bf Package uk.ac.ic.doc.neuralnets.gui.graph.listener}{
\vskip -.25in
\hbox to \hsize{\it Package Contents\hfil Page}
\rule{\hsize}{.7mm}
\vskip .13in
\hbox{\bf Classes}
\entityintro{KeyboardPlugin}{l832}{...no description...}
\entityintro{MouseItemListener}{l833}{...no description...}
\entityintro{MousePlugin}{l834}{...no description...}
\vskip .1in
\rule{\hsize}{.7mm}
\vskip .1in
\newpage
\section{Classes}{
\startsection{Class}{KeyboardPlugin}{l832}{%
\startsubsubsection{Declaration}{
\fbox{\vbox{
\hbox{\vbox{\small public abstract 
class 
KeyboardPlugin}}
\noindent\hbox{\vbox{{\bf extends} java.lang.Object}}
\noindent\hbox{\vbox{{\bf implements} 
org.eclipse.swt.events.KeyListener, uk.ac.ic.doc.neuralnets.util.plugins.Plugin}}
}}}
\startsubsubsection{Constructors}{
\vskip -2em
\begin{itemize}
\item{\vskip -1.9ex 
\membername{KeyboardPlugin}
{\tt public {\bf KeyboardPlugin}(  )
\label{l835}\label{l836}}%end signature
}%end item
\end{itemize}
}
\startsubsubsection{Methods}{
\vskip -2em
\begin{itemize}
\item{\vskip -1.9ex 
\membername{getName}
{\tt public abstract String {\bf getName}(  )
\label{l837}\label{l838}}%end signature
}%end item
\divideents{keyPressed}
\item{\vskip -1.9ex 
\membername{keyPressed}
{\tt public void {\bf keyPressed}( {\tt org.eclipse.swt.events.KeyEvent } {\bf e} )
\label{l839}\label{l840}}%end signature
}%end item
\divideents{keyReleased}
\item{\vskip -1.9ex 
\membername{keyReleased}
{\tt public void {\bf keyReleased}( {\tt org.eclipse.swt.events.KeyEvent } {\bf e} )
\label{l841}\label{l842}}%end signature
}%end item
\divideents{setManager}
\item{\vskip -1.9ex 
\membername{setManager}
{\tt public void {\bf setManager}( {\tt uk.ac.ic.doc.neuralnets.coreui.ZoomingInterfaceManager } {\bf g} )
\label{l843}\label{l844}}%end signature
}%end item
\end{itemize}
}
}
\startsection{Class}{MouseItemListener}{l833}{%
\startsubsubsection{Declaration}{
\fbox{\vbox{
\hbox{\vbox{\small public 
class 
MouseItemListener}}
\noindent\hbox{\vbox{{\bf extends} java.lang.Object}}
\noindent\hbox{\vbox{{\bf implements} 
org.eclipse.swt.events.MouseListener}}
}}}
\startsubsubsection{Constructors}{
\vskip -2em
\begin{itemize}
\item{\vskip -1.9ex 
\membername{MouseItemListener}
{\tt public {\bf MouseItemListener}(  )
\label{l845}\label{l846}}%end signature
}%end item
\divideents{MouseItemListener}
\item{\vskip -1.9ex 
\membername{MouseItemListener}
{\tt public {\bf MouseItemListener}( {\tt org.eclipse.zest.core.widgets.Graph } {\bf g} )
\label{l847}\label{l848}}%end signature
}%end item
\end{itemize}
}
\startsubsubsection{Methods}{
\vskip -2em
\begin{itemize}
\item{\vskip -1.9ex 
\membername{getFigureAt}
{\tt protected IFigure {\bf getFigureAt}( {\tt int } {\bf x},
{\tt int } {\bf y} )
\label{l849}\label{l850}}%end signature
}%end item
\divideents{getGraph}
\item{\vskip -1.9ex 
\membername{getGraph}
{\tt public Graph {\bf getGraph}(  )
\label{l851}\label{l852}}%end signature
}%end item
\divideents{getItemAt}
\item{\vskip -1.9ex 
\membername{getItemAt}
{\tt protected GraphItem {\bf getItemAt}( {\tt int } {\bf x},
{\tt int } {\bf y} )
\label{l853}\label{l854}}%end signature
}%end item
\divideents{getItemFor}
\item{\vskip -1.9ex 
\membername{getItemFor}
{\tt protected GraphItem {\bf getItemFor}( {\tt org.eclipse.draw2d.IFigure } {\bf figure} )
\label{l855}\label{l856}}%end signature
\begin{itemize}
\sld
\item{
\sld
{\bf Usage}
  \begin{itemize}\isep
   \item{
This could be hideously slow, in theory. We're iterating over
 all the nodes, then all the edges. However, experimentally it is
 faster than the GUI update for a given size of network.
 
 We could store this data in a Map\textless IFigure,GraphItem\textgreater , but then
 there's a lot of housekeeping involved in keeping the map up to
 date - plus we end up with a big chunk of memory storing all the
 pointers again
}%end item
  \end{itemize}
}
\end{itemize}
}%end item
\divideents{handleClick}
\item{\vskip -1.9ex 
\membername{handleClick}
{\tt protected void {\bf handleClick}( {\tt org.eclipse.swt.events.MouseEvent } {\bf e},
{\tt org.eclipse.zest.core.widgets.GraphItem } {\bf i} )
\label{l857}\label{l858}}%end signature
}%end item
\divideents{handleDoubleClick}
\item{\vskip -1.9ex 
\membername{handleDoubleClick}
{\tt protected void {\bf handleDoubleClick}( {\tt org.eclipse.swt.events.MouseEvent } {\bf e},
{\tt org.eclipse.zest.core.widgets.GraphItem } {\bf i} )
\label{l859}\label{l860}}%end signature
}%end item
\divideents{handleDown}
\item{\vskip -1.9ex 
\membername{handleDown}
{\tt protected void {\bf handleDown}( {\tt org.eclipse.swt.events.MouseEvent } {\bf e},
{\tt org.eclipse.zest.core.widgets.GraphItem } {\bf i} )
\label{l861}\label{l862}}%end signature
}%end item
\divideents{handleUp}
\item{\vskip -1.9ex 
\membername{handleUp}
{\tt protected void {\bf handleUp}( {\tt org.eclipse.swt.events.MouseEvent } {\bf e},
{\tt org.eclipse.zest.core.widgets.GraphItem } {\bf i} )
\label{l863}\label{l864}}%end signature
}%end item
\divideents{mouseDoubleClick}
\item{\vskip -1.9ex 
\membername{mouseDoubleClick}
{\tt public void {\bf mouseDoubleClick}( {\tt org.eclipse.swt.events.MouseEvent } {\bf e} )
\label{l865}\label{l866}}%end signature
}%end item
\divideents{mouseDown}
\item{\vskip -1.9ex 
\membername{mouseDown}
{\tt public void {\bf mouseDown}( {\tt org.eclipse.swt.events.MouseEvent } {\bf e} )
\label{l867}\label{l868}}%end signature
}%end item
\divideents{mouseUp}
\item{\vskip -1.9ex 
\membername{mouseUp}
{\tt public void {\bf mouseUp}( {\tt org.eclipse.swt.events.MouseEvent } {\bf e} )
\label{l869}\label{l870}}%end signature
}%end item
\divideents{setGraph}
\item{\vskip -1.9ex 
\membername{setGraph}
{\tt public void {\bf setGraph}( {\tt org.eclipse.zest.core.widgets.Graph } {\bf g} )
\label{l871}\label{l872}}%end signature
}%end item
\end{itemize}
}
}
\startsection{Class}{MousePlugin}{l834}{%
\startsubsubsection{Declaration}{
\fbox{\vbox{
\hbox{\vbox{\small public abstract 
class 
MousePlugin}}
\noindent\hbox{\vbox{{\bf extends} uk.ac.ic.doc.neuralnets.gui.graph.listener.MouseItemListener}}
\noindent\hbox{\vbox{{\bf implements} 
uk.ac.ic.doc.neuralnets.util.plugins.Plugin}}
}}}
\startsubsubsection{Constructors}{
\vskip -2em
\begin{itemize}
\item{\vskip -1.9ex 
\membername{MousePlugin}
{\tt public {\bf MousePlugin}(  )
\label{l873}\label{l874}}%end signature
}%end item
\end{itemize}
}
\startsubsubsection{Methods}{
\vskip -2em
\begin{itemize}
\item{\vskip -1.9ex 
\membername{getName}
{\tt public abstract String {\bf getName}(  )
\label{l875}\label{l876}}%end signature
}%end item
\divideents{setManager}
\item{\vskip -1.9ex 
\membername{setManager}
{\tt public void {\bf setManager}( {\tt uk.ac.ic.doc.neuralnets.coreui.ZoomingInterfaceManager } {\bf g} )
\label{l877}\label{l878}}%end signature
}%end item
\end{itemize}
}
\startsubsubsection{Methods inherited from class {\tt uk.ac.ic.doc.neuralnets.gui.graph.listener.MouseItemListener}}{
\par{\small 
\refdefined{l833}\vskip -2em
\begin{itemize}
\item{\vskip -1.9ex 
\membername{getFigureAt}
{\tt protected IFigure {\bf getFigureAt}( {\tt int } {\bf x},
{\tt int } {\bf y} )
}%end signature
}%end item
\divideents{getGraph}
\item{\vskip -1.9ex 
\membername{getGraph}
{\tt public Graph {\bf getGraph}(  )
}%end signature
}%end item
\divideents{getItemAt}
\item{\vskip -1.9ex 
\membername{getItemAt}
{\tt protected GraphItem {\bf getItemAt}( {\tt int } {\bf x},
{\tt int } {\bf y} )
}%end signature
}%end item
\divideents{getItemFor}
\item{\vskip -1.9ex 
\membername{getItemFor}
{\tt protected GraphItem {\bf getItemFor}( {\tt org.eclipse.draw2d.IFigure } {\bf figure} )
}%end signature
\begin{itemize}
\sld
\item{
\sld
{\bf Usage}
  \begin{itemize}\isep
   \item{
This could be hideously slow, in theory. We're iterating over
 all the nodes, then all the edges. However, experimentally it is
 faster than the GUI update for a given size of network.
 
 We could store this data in a Map\textless IFigure,GraphItem\textgreater , but then
 there's a lot of housekeeping involved in keeping the map up to
 date - plus we end up with a big chunk of memory storing all the
 pointers again
}%end item
  \end{itemize}
}
\end{itemize}
}%end item
\divideents{handleClick}
\item{\vskip -1.9ex 
\membername{handleClick}
{\tt protected void {\bf handleClick}( {\tt org.eclipse.swt.events.MouseEvent } {\bf e},
{\tt org.eclipse.zest.core.widgets.GraphItem } {\bf i} )
}%end signature
}%end item
\divideents{handleDoubleClick}
\item{\vskip -1.9ex 
\membername{handleDoubleClick}
{\tt protected void {\bf handleDoubleClick}( {\tt org.eclipse.swt.events.MouseEvent } {\bf e},
{\tt org.eclipse.zest.core.widgets.GraphItem } {\bf i} )
}%end signature
}%end item
\divideents{handleDown}
\item{\vskip -1.9ex 
\membername{handleDown}
{\tt protected void {\bf handleDown}( {\tt org.eclipse.swt.events.MouseEvent } {\bf e},
{\tt org.eclipse.zest.core.widgets.GraphItem } {\bf i} )
}%end signature
}%end item
\divideents{handleUp}
\item{\vskip -1.9ex 
\membername{handleUp}
{\tt protected void {\bf handleUp}( {\tt org.eclipse.swt.events.MouseEvent } {\bf e},
{\tt org.eclipse.zest.core.widgets.GraphItem } {\bf i} )
}%end signature
}%end item
\divideents{mouseDoubleClick}
\item{\vskip -1.9ex 
\membername{mouseDoubleClick}
{\tt public void {\bf mouseDoubleClick}( {\tt org.eclipse.swt.events.MouseEvent } {\bf e} )
}%end signature
}%end item
\divideents{mouseDown}
\item{\vskip -1.9ex 
\membername{mouseDown}
{\tt public void {\bf mouseDown}( {\tt org.eclipse.swt.events.MouseEvent } {\bf e} )
}%end signature
}%end item
\divideents{mouseUp}
\item{\vskip -1.9ex 
\membername{mouseUp}
{\tt public void {\bf mouseUp}( {\tt org.eclipse.swt.events.MouseEvent } {\bf e} )
}%end signature
}%end item
\divideents{setGraph}
\item{\vskip -1.9ex 
\membername{setGraph}
{\tt public void {\bf setGraph}( {\tt org.eclipse.zest.core.widgets.Graph } {\bf g} )
}%end signature
}%end item
\end{itemize}
}}
}
}
}
\newpage
\def\packagename{uk.ac.ic.doc.neuralnets.expressions.ast}
\chapter{\bf Package uk.ac.ic.doc.neuralnets.expressions.ast}{
\vskip -.25in
\hbox to \hsize{\it Package Contents\hfil Page}
\rule{\hsize}{.7mm}
\vskip .13in
\hbox{\bf Classes}
\entityintro{ASTExpression}{l879}{An expression object with support for dynamically bound variables, parsing
 its contents into an abstract syntax tree.}
\entityintro{ASTExpressionFactory}{l880}{Factory for flyweight ASTExpression objects}
\entityintro{BinaryOperator}{l881}{Encodes an operator with two parameters, assumes infix notation when
 outputting this expression.}
\entityintro{Component}{l882}{The abstract super-type of all components of the abstract syntax tree.}
\entityintro{ExpressionASTLexer}{l883}{...no description...}
\entityintro{ExpressionASTParser}{l884}{...no description...}
\entityintro{Literal}{l885}{...no description...}
\entityintro{NoOpComponent}{l886}{Simple Component to perform no operation at all.}
\entityintro{NullaryOperator}{l887}{Component to be evaluated with no operators}
\entityintro{UnaryOperator}{l888}{Component that is evaluated with one operator only}
\entityintro{Variable}{l889}{A named variable Component, capable of being bound to any Double value.}
\vskip .1in
\rule{\hsize}{.7mm}
\vskip .1in
\newpage
\section{Classes}{
\startsection{Class}{ASTExpression}{l879}{%
{\small An expression object with support for dynamically bound variables, parsing
 its contents into an abstract syntax tree.}
\vskip .1in 
\startsubsubsection{Declaration}{
\fbox{\vbox{
\hbox{\vbox{\small public 
class 
ASTExpression}}
\noindent\hbox{\vbox{{\bf extends} java.lang.Object}}
}}}
\startsubsubsection{Constructors}{
\vskip -2em
\begin{itemize}
\item{\vskip -1.9ex 
\membername{ASTExpression}
{\tt public {\bf ASTExpression}( {\tt java.lang.Double } {\bf value} )
\label{l890}\label{l891}}%end signature
\begin{itemize}
\sld
\item{
\sld
{\bf Usage}
  \begin{itemize}\isep
   \item{
Create an Expression to encode the given value
}%end item
  \end{itemize}
}
\item{
\sld
{\bf Parameters}
\sld\isep
  \begin{itemize}
\sld\isep
   \item{
\sld
{\tt value} - The value returned by this Expression}
  \end{itemize}
}%end item
\end{itemize}
}%end item
\divideents{ASTExpression}
\item{\vskip -1.9ex 
\membername{ASTExpression}
{\tt public {\bf ASTExpression}( {\tt java.lang.String } {\bf expr} )
\label{l892}\label{l893}}%end signature
\begin{itemize}
\sld
\item{
\sld
{\bf Usage}
  \begin{itemize}\isep
   \item{
Create an Expression for the given string
}%end item
  \end{itemize}
}
\item{
\sld
{\bf Parameters}
\sld\isep
  \begin{itemize}
\sld\isep
   \item{
\sld
{\tt expr} - The expression to represent}
  \end{itemize}
}%end item
\end{itemize}
}%end item
\end{itemize}
}
\startsubsubsection{Methods}{
\vskip -2em
\begin{itemize}
\item{\vskip -1.9ex 
\membername{bind}
{\tt public void {\bf bind}( {\tt java.lang.Object } {\bf o} )
\label{l894}\label{l895}}%end signature
\begin{itemize}
\sld
\item{
\sld
{\bf Usage}
  \begin{itemize}\isep
   \item{
Bind variables according to BindVariable annotations present in this
 object, and all of its super-classes
}%end item
  \end{itemize}
}
\item{
\sld
{\bf Parameters}
\sld\isep
  \begin{itemize}
\sld\isep
   \item{
\sld
{\tt o} - The object to bind variables from}
  \end{itemize}
}%end item
\end{itemize}
}%end item
\divideents{bind}
\item{\vskip -1.9ex 
\membername{bind}
{\tt public void {\bf bind}( {\tt java.lang.String } {\bf var},
{\tt java.lang.Double } {\bf val} )
\label{l896}\label{l897}}%end signature
\begin{itemize}
\sld
\item{
\sld
{\bf Usage}
  \begin{itemize}\isep
   \item{
Manually bind a variable in the expression
}%end item
  \end{itemize}
}
\item{
\sld
{\bf Parameters}
\sld\isep
  \begin{itemize}
\sld\isep
   \item{
\sld
{\tt var} - The variable to bind}
   \item{
\sld
{\tt val} - The value to bind to}
  \end{itemize}
}%end item
\end{itemize}
}%end item
\divideents{bind}
\item{\vskip -1.9ex 
\membername{bind}
{\tt protected void {\bf bind}( {\tt java.lang.String } {\bf var},
{\tt java.lang.reflect.Method } {\bf m},
{\tt java.lang.Object } {\bf o} )
\label{l898}\label{l899}}%end signature
}%end item
\divideents{evaluate}
\item{\vskip -1.9ex 
\membername{evaluate}
{\tt public Double {\bf evaluate}(  )
\label{l900}\label{l901}}%end signature
\begin{itemize}
\sld
\item{
\sld
{\bf Usage}
  \begin{itemize}\isep
   \item{
Evaluate the expression after refreshing its current bindings
}%end item
  \end{itemize}
}
\item{{\bf Returns} - 
The value this expression evaluates to 
}%end item
\item{{\bf Exceptions}
  \begin{itemize}
\sld
   \item{\vskip -.6ex{\tt uk.ac.ic.doc.neuralnets.expressions.ExpressionException} - }
  \end{itemize}
}%end item
\end{itemize}
}%end item
\divideents{evaluate}
\item{\vskip -1.9ex 
\membername{evaluate}
{\tt public Double {\bf evaluate}( {\tt java.lang.Object } {\bf o} )
\label{l902}\label{l903}}%end signature
\begin{itemize}
\sld
\item{
\sld
{\bf Usage}
  \begin{itemize}\isep
   \item{
Re-bind variables, then evaluate the expression
}%end item
  \end{itemize}
}
\item{
\sld
{\bf Parameters}
\sld\isep
  \begin{itemize}
\sld\isep
   \item{
\sld
{\tt o} - The object to bind variables from}
  \end{itemize}
}%end item
\item{{\bf Returns} - 
The value this expression evaluates to 
}%end item
\item{{\bf Exceptions}
  \begin{itemize}
\sld
   \item{\vskip -.6ex{\tt uk.ac.ic.doc.neuralnets.expressions.ExpressionException} - }
  \end{itemize}
}%end item
\end{itemize}
}%end item
\divideents{evaluateThis}
\item{\vskip -1.9ex 
\membername{evaluateThis}
{\tt public Double {\bf evaluateThis}( {\tt java.lang.Object } {\bf o} )
\label{l904}\label{l905}}%end signature
\begin{itemize}
\sld
\item{
\sld
{\bf Usage}
  \begin{itemize}\isep
   \item{
Evaluate the expression after refreshing its current bindings from the
 supplied object. Will not seek new annotations.
}%end item
  \end{itemize}
}
\item{
\sld
{\bf Parameters}
\sld\isep
  \begin{itemize}
\sld\isep
   \item{
\sld
{\tt o} - The object to bind on to}
  \end{itemize}
}%end item
\item{{\bf Returns} - 
The value this expression evaluates to 
}%end item
\item{{\bf Exceptions}
  \begin{itemize}
\sld
   \item{\vskip -.6ex{\tt uk.ac.ic.doc.neuralnets.expressions.ExpressionException} - }
  \end{itemize}
}%end item
\end{itemize}
}%end item
\divideents{getExpression}
\item{\vskip -1.9ex 
\membername{getExpression}
{\tt public String {\bf getExpression}(  )
\label{l906}\label{l907}}%end signature
\begin{itemize}
\sld
\item{
\sld
{\bf Usage}
  \begin{itemize}\isep
   \item{
Answer the input expression
}%end item
  \end{itemize}
}
\item{{\bf Returns} - 
The mathematical expression encoded by this object 
}%end item
\end{itemize}
}%end item
\divideents{parse}
\item{\vskip -1.9ex 
\membername{parse}
{\tt protected Component {\bf parse}( {\tt java.lang.String } {\bf ex} )
\label{l908}\label{l909}}%end signature
}%end item
\divideents{toString}
\item{\vskip -1.9ex 
\membername{toString}
{\tt public String {\bf toString}(  )
\label{l910}\label{l911}}%end signature
}%end item
\end{itemize}
}
}
\startsection{Class}{ASTExpressionFactory}{l880}{%
{\small Factory for flyweight ASTExpression objects}
\vskip .1in 
\startsubsubsection{Declaration}{
\fbox{\vbox{
\hbox{\vbox{\small public 
class 
ASTExpressionFactory}}
\noindent\hbox{\vbox{{\bf extends} java.lang.Object}}
}}}
\startsubsubsection{Methods}{
\vskip -2em
\begin{itemize}
\item{\vskip -1.9ex 
\membername{flushCache}
{\tt public void {\bf flushCache}(  )
\label{l912}\label{l913}}%end signature
\begin{itemize}
\sld
\item{
\sld
{\bf Usage}
  \begin{itemize}\isep
   \item{
Clear the cache of expressions, preventing any further replication of
 old flyweights.
}%end item
  \end{itemize}
}
\end{itemize}
}%end item
\divideents{get}
\item{\vskip -1.9ex 
\membername{get}
{\tt public static ASTExpressionFactory {\bf get}(  )
\label{l914}\label{l915}}%end signature
\begin{itemize}
\sld
\item{
\sld
{\bf Usage}
  \begin{itemize}\isep
   \item{
Answer the instance of this singleton service
}%end item
  \end{itemize}
}
\item{{\bf Returns} - 
The ASTExpressionFactory 
}%end item
\end{itemize}
}%end item
\divideents{getExpression}
\item{\vskip -1.9ex 
\membername{getExpression}
{\tt public ASTExpression {\bf getExpression}( {\tt java.lang.Double } {\bf d} )
\label{l916}\label{l917}}%end signature
\begin{itemize}
\sld
\item{
\sld
{\bf Usage}
  \begin{itemize}\isep
   \item{
Convenience method to answer an expression for a simple Double value.
}%end item
  \end{itemize}
}
\item{
\sld
{\bf Parameters}
\sld\isep
  \begin{itemize}
\sld\isep
   \item{
\sld
{\tt d} - the Double to encode as an ASTExpression}
  \end{itemize}
}%end item
\item{{\bf Returns} - 
The ASTExpression flyweight for this Double 
}%end item
\item{{\bf Exceptions}
  \begin{itemize}
\sld
   \item{\vskip -.6ex{\tt uk.ac.ic.doc.neuralnets.expressions.ExpressionException} - }
  \end{itemize}
}%end item
\item{{\bf See Also}
  \begin{itemize}
   \item{{\tt uk.ac.ic.doc.neuralnets.expressions.ast.ASTExpression} {\small 
\refdefined{l879}}%end \small
}%end item
  \end{itemize}
}%end item
\end{itemize}
}%end item
\divideents{getExpression}
\item{\vskip -1.9ex 
\membername{getExpression}
{\tt public ASTExpression {\bf getExpression}( {\tt java.lang.String } {\bf expressionString} )
\label{l918}\label{l919}}%end signature
\begin{itemize}
\sld
\item{
\sld
{\bf Usage}
  \begin{itemize}\isep
   \item{
Return a flyweight ASTExpression respresenting the given input string.
 Attempts to do some disambiguation through removal of whitespace
 before seeking an equivalent expression. Does not attempt any
 re-ordering of expression components or more complex semantic
 equivalence tests.
}%end item
  \end{itemize}
}
\item{
\sld
{\bf Parameters}
\sld\isep
  \begin{itemize}
\sld\isep
   \item{
\sld
{\tt expressionString} - The expression to parse into an ASTExpression}
  \end{itemize}
}%end item
\item{{\bf Returns} - 
An ASTExpression object, pulled from cache wherever possible. 
}%end item
\item{{\bf Exceptions}
  \begin{itemize}
\sld
   \item{\vskip -.6ex{\tt uk.ac.ic.doc.neuralnets.expressions.ExpressionException} - }
  \end{itemize}
}%end item
\item{{\bf See Also}
  \begin{itemize}
   \item{{\tt uk.ac.ic.doc.neuralnets.expressions.ast.ASTExpression} {\small 
\refdefined{l879}}%end \small
}%end item
  \end{itemize}
}%end item
\end{itemize}
}%end item
\end{itemize}
}
}
\startsection{Class}{BinaryOperator}{l881}{%
{\small Encodes an operator with two parameters, assumes infix notation when
 outputting this expression.}
\vskip .1in 
\startsubsubsection{Declaration}{
\fbox{\vbox{
\hbox{\vbox{\small public abstract 
class 
BinaryOperator}}
\noindent\hbox{\vbox{{\bf extends} uk.ac.ic.doc.neuralnets.expressions.ast.Component}}
}}}
\startsubsubsection{Constructors}{
\vskip -2em
\begin{itemize}
\item{\vskip -1.9ex 
\membername{BinaryOperator}
{\tt public {\bf BinaryOperator}( {\tt uk.ac.ic.doc.neuralnets.expressions.ast.Component } {\bf l},
{\tt uk.ac.ic.doc.neuralnets.expressions.ast.Component } {\bf r},
{\tt java.lang.String } {\bf operation} )
\label{l920}\label{l921}}%end signature
}%end item
\end{itemize}
}
\startsubsubsection{Methods}{
\vskip -2em
\begin{itemize}
\item{\vskip -1.9ex 
\membername{evaluate}
{\tt public abstract Double {\bf evaluate}(  )
\label{l922}\label{l923}}%end signature
}%end item
\divideents{getExpression}
\item{\vskip -1.9ex 
\membername{getExpression}
{\tt public String {\bf getExpression}(  )
\label{l924}\label{l925}}%end signature
}%end item
\divideents{getOperation}
\item{\vskip -1.9ex 
\membername{getOperation}
{\tt public String {\bf getOperation}(  )
\label{l926}\label{l927}}%end signature
\begin{itemize}
\sld
\item{
\sld
{\bf Usage}
  \begin{itemize}\isep
   \item{
Answer the operation encoded by this BinaryOperator
}%end item
  \end{itemize}
}
\item{{\bf Returns} - 
The lexical form of the operation 
}%end item
\end{itemize}
}%end item
\divideents{getVariables}
\item{\vskip -1.9ex 
\membername{getVariables}
{\tt public Set {\bf getVariables}(  )
\label{l928}\label{l929}}%end signature
}%end item
\end{itemize}
}
\startsubsubsection{Methods inherited from class {\tt uk.ac.ic.doc.neuralnets.expressions.ast.Component}}{
\par{\small 
\refdefined{l882}\vskip -2em
\begin{itemize}
\item{\vskip -1.9ex 
\membername{bracket}
{\tt public String {\bf bracket}( {\tt uk.ac.ic.doc.neuralnets.expressions.ast.Component } {\bf c} )
}%end signature
\begin{itemize}
\sld
\item{
\sld
{\bf Usage}
  \begin{itemize}\isep
   \item{
A meethod to parenthesise the given child expression in the context of
 the current operation; applies mathematical order of operations rules.
}%end item
  \end{itemize}
}
\item{
\sld
{\bf Parameters}
\sld\isep
  \begin{itemize}
\sld\isep
   \item{
\sld
{\tt c} - The child component to parenthesise}
  \end{itemize}
}%end item
\item{{\bf Returns} - 
A String representation of the child, with or without
 parentheses, as deemed necessary. 
}%end item
\end{itemize}
}%end item
\divideents{evaluate}
\item{\vskip -1.9ex 
\membername{evaluate}
{\tt public abstract Double {\bf evaluate}(  )
}%end signature
\begin{itemize}
\sld
\item{
\sld
{\bf Usage}
  \begin{itemize}\isep
   \item{
Calculate the value of this expression sub-tree in its current bindings
 (if applicable)
}%end item
  \end{itemize}
}
\item{{\bf Returns} - 
A Double value of the output of evaluating this tree 
}%end item
\item{{\bf Exceptions}
  \begin{itemize}
\sld
   \item{\vskip -.6ex{\tt uk.ac.ic.doc.neuralnets.expressions.ExpressionException} - }
  \end{itemize}
}%end item
\end{itemize}
}%end item
\divideents{getExpression}
\item{\vskip -1.9ex 
\membername{getExpression}
{\tt public abstract String {\bf getExpression}(  )
}%end signature
\begin{itemize}
\sld
\item{
\sld
{\bf Usage}
  \begin{itemize}\isep
   \item{
Retrieve the original expression, re-formatted for user friendly output
}%end item
  \end{itemize}
}
\item{{\bf Returns} - 
A String representation of this expression tree; must be
 re-parsable by the ASTExpressionFactory. 
}%end item
\end{itemize}
}%end item
\divideents{getVariables}
\item{\vskip -1.9ex 
\membername{getVariables}
{\tt public abstract Set {\bf getVariables}(  )
}%end signature
\begin{itemize}
\sld
\item{
\sld
{\bf Usage}
  \begin{itemize}\isep
   \item{
Answer a set of the variable objects in this tree; this may include
 any instances of the Variable class, or any operations that return a
 different value for each evaluation, e.g. random operators, counters etc
}%end item
  \end{itemize}
}
\item{{\bf Returns} - 
A Set of the variable components 
}%end item
\item{{\bf See Also}
  \begin{itemize}
   \item{{\tt uk.ac.ic.doc.neuralnets.expressions.ast.Variable} {\small 
\refdefined{l889}}%end \small
}%end item
  \end{itemize}
}%end item
\end{itemize}
}%end item
\divideents{order}
\item{\vskip -1.9ex 
\membername{order}
{\tt public int {\bf order}( {\tt java.lang.String } {\bf op} )
}%end signature
\begin{itemize}
\sld
\item{
\sld
{\bf Usage}
  \begin{itemize}\isep
   \item{
Decide the internal ordering of the supplied operation; higher numbers
 represent a lower importance. Defaults to Integer.MAX\_VALUE if the
 operator is not recognised.
}%end item
  \end{itemize}
}
\item{
\sld
{\bf Parameters}
\sld\isep
  \begin{itemize}
\sld\isep
   \item{
\sld
{\tt op} - The operator to decide precedence of}
  \end{itemize}
}%end item
\item{{\bf Returns} - 
An integer value; lower values for greater precedence 
}%end item
\end{itemize}
}%end item
\end{itemize}
}}
}
\startsection{Class}{Component}{l882}{%
{\small The abstract super-type of all components of the abstract syntax tree.}
\vskip .1in 
\startsubsubsection{Declaration}{
\fbox{\vbox{
\hbox{\vbox{\small public abstract 
class 
Component}}
\noindent\hbox{\vbox{{\bf extends} java.lang.Object}}
}}}
\startsubsubsection{Constructors}{
\vskip -2em
\begin{itemize}
\item{\vskip -1.9ex 
\membername{Component}
{\tt public {\bf Component}(  )
\label{l930}\label{l931}}%end signature
}%end item
\end{itemize}
}
\startsubsubsection{Methods}{
\vskip -2em
\begin{itemize}
\item{\vskip -1.9ex 
\membername{bracket}
{\tt public String {\bf bracket}( {\tt uk.ac.ic.doc.neuralnets.expressions.ast.Component } {\bf c} )
\label{l932}\label{l933}}%end signature
\begin{itemize}
\sld
\item{
\sld
{\bf Usage}
  \begin{itemize}\isep
   \item{
A meethod to parenthesise the given child expression in the context of
 the current operation; applies mathematical order of operations rules.
}%end item
  \end{itemize}
}
\item{
\sld
{\bf Parameters}
\sld\isep
  \begin{itemize}
\sld\isep
   \item{
\sld
{\tt c} - The child component to parenthesise}
  \end{itemize}
}%end item
\item{{\bf Returns} - 
A String representation of the child, with or without
 parentheses, as deemed necessary. 
}%end item
\end{itemize}
}%end item
\divideents{evaluate}
\item{\vskip -1.9ex 
\membername{evaluate}
{\tt public abstract Double {\bf evaluate}(  )
\label{l934}\label{l935}}%end signature
\begin{itemize}
\sld
\item{
\sld
{\bf Usage}
  \begin{itemize}\isep
   \item{
Calculate the value of this expression sub-tree in its current bindings
 (if applicable)
}%end item
  \end{itemize}
}
\item{{\bf Returns} - 
A Double value of the output of evaluating this tree 
}%end item
\item{{\bf Exceptions}
  \begin{itemize}
\sld
   \item{\vskip -.6ex{\tt uk.ac.ic.doc.neuralnets.expressions.ExpressionException} - }
  \end{itemize}
}%end item
\end{itemize}
}%end item
\divideents{getExpression}
\item{\vskip -1.9ex 
\membername{getExpression}
{\tt public abstract String {\bf getExpression}(  )
\label{l936}\label{l937}}%end signature
\begin{itemize}
\sld
\item{
\sld
{\bf Usage}
  \begin{itemize}\isep
   \item{
Retrieve the original expression, re-formatted for user friendly output
}%end item
  \end{itemize}
}
\item{{\bf Returns} - 
A String representation of this expression tree; must be
 re-parsable by the ASTExpressionFactory. 
}%end item
\end{itemize}
}%end item
\divideents{getVariables}
\item{\vskip -1.9ex 
\membername{getVariables}
{\tt public abstract Set {\bf getVariables}(  )
\label{l938}\label{l939}}%end signature
\begin{itemize}
\sld
\item{
\sld
{\bf Usage}
  \begin{itemize}\isep
   \item{
Answer a set of the variable objects in this tree; this may include
 any instances of the Variable class, or any operations that return a
 different value for each evaluation, e.g. random operators, counters etc
}%end item
  \end{itemize}
}
\item{{\bf Returns} - 
A Set of the variable components 
}%end item
\item{{\bf See Also}
  \begin{itemize}
   \item{{\tt uk.ac.ic.doc.neuralnets.expressions.ast.Variable} {\small 
\refdefined{l889}}%end \small
}%end item
  \end{itemize}
}%end item
\end{itemize}
}%end item
\divideents{order}
\item{\vskip -1.9ex 
\membername{order}
{\tt public int {\bf order}( {\tt java.lang.String } {\bf op} )
\label{l940}\label{l941}}%end signature
\begin{itemize}
\sld
\item{
\sld
{\bf Usage}
  \begin{itemize}\isep
   \item{
Decide the internal ordering of the supplied operation; higher numbers
 represent a lower importance. Defaults to Integer.MAX\_VALUE if the
 operator is not recognised.
}%end item
  \end{itemize}
}
\item{
\sld
{\bf Parameters}
\sld\isep
  \begin{itemize}
\sld\isep
   \item{
\sld
{\tt op} - The operator to decide precedence of}
  \end{itemize}
}%end item
\item{{\bf Returns} - 
An integer value; lower values for greater precedence 
}%end item
\end{itemize}
}%end item
\end{itemize}
}
}
\startsection{Class}{ExpressionASTLexer}{l883}{%
\startsubsubsection{Declaration}{
\fbox{\vbox{
\hbox{\vbox{\small public 
class 
ExpressionASTLexer}}
\noindent\hbox{\vbox{{\bf extends} org.antlr.runtime.Lexer}}
}}}
\startsubsubsection{Fields}{
\begin{itemize}
\item{
public static final int MOD\begin{itemize}\item{\vskip -.9ex }\end{itemize}
}
\item{
public static final int GRAND\begin{itemize}\item{\vskip -.9ex }\end{itemize}
}
\item{
public static final int INT\begin{itemize}\item{\vskip -.9ex }\end{itemize}
}
\item{
public static final int COSH\begin{itemize}\item{\vskip -.9ex }\end{itemize}
}
\item{
public static final int MULT\begin{itemize}\item{\vskip -.9ex }\end{itemize}
}
\item{
public static final int MINUS\begin{itemize}\item{\vskip -.9ex }\end{itemize}
}
\item{
public static final int SQRT\begin{itemize}\item{\vskip -.9ex }\end{itemize}
}
\item{
public static final int EOF\begin{itemize}\item{\vskip -.9ex }\end{itemize}
}
\item{
public static final int SINH\begin{itemize}\item{\vskip -.9ex }\end{itemize}
}
\item{
public static final int LPAREN\begin{itemize}\item{\vskip -.9ex }\end{itemize}
}
\item{
public static final int RPAREN\begin{itemize}\item{\vskip -.9ex }\end{itemize}
}
\item{
public static final int TANH\begin{itemize}\item{\vskip -.9ex }\end{itemize}
}
\item{
public static final int WS\begin{itemize}\item{\vskip -.9ex }\end{itemize}
}
\item{
public static final int POW\begin{itemize}\item{\vskip -.9ex }\end{itemize}
}
\item{
public static final int NEWLINE\begin{itemize}\item{\vskip -.9ex }\end{itemize}
}
\item{
public static final int SIN\begin{itemize}\item{\vskip -.9ex }\end{itemize}
}
\item{
public static final int COS\begin{itemize}\item{\vskip -.9ex }\end{itemize}
}
\item{
public static final int TAN\begin{itemize}\item{\vskip -.9ex }\end{itemize}
}
\item{
public static final int RAND\begin{itemize}\item{\vskip -.9ex }\end{itemize}
}
\item{
public static final int DOUBLE\begin{itemize}\item{\vskip -.9ex }\end{itemize}
}
\item{
public static final int PLUS\begin{itemize}\item{\vskip -.9ex }\end{itemize}
}
\item{
public static final int VAR\begin{itemize}\item{\vskip -.9ex }\end{itemize}
}
\item{
public static final int DIV\begin{itemize}\item{\vskip -.9ex }\end{itemize}
}
\end{itemize}
}
\startsubsubsection{Constructors}{
\vskip -2em
\begin{itemize}
\item{\vskip -1.9ex 
\membername{ExpressionASTLexer}
{\tt public {\bf ExpressionASTLexer}(  )
\label{l942}\label{l943}}%end signature
}%end item
\divideents{ExpressionASTLexer}
\item{\vskip -1.9ex 
\membername{ExpressionASTLexer}
{\tt public {\bf ExpressionASTLexer}( {\tt org.antlr.runtime.CharStream } {\bf input} )
\label{l944}\label{l945}}%end signature
}%end item
\divideents{ExpressionASTLexer}
\item{\vskip -1.9ex 
\membername{ExpressionASTLexer}
{\tt public {\bf ExpressionASTLexer}( {\tt org.antlr.runtime.CharStream } {\bf input},
{\tt org.antlr.runtime.RecognizerSharedState } {\bf state} )
\label{l946}\label{l947}}%end signature
}%end item
\end{itemize}
}
\startsubsubsection{Methods}{
\vskip -2em
\begin{itemize}
\item{\vskip -1.9ex 
\membername{getGrammarFileName}
{\tt public String {\bf getGrammarFileName}(  )
\label{l948}\label{l949}}%end signature
}%end item
\divideents{mCOS}
\item{\vskip -1.9ex 
\membername{mCOS}
{\tt public final void {\bf mCOS}(  )
\label{l950}\label{l951}}%end signature
}%end item
\divideents{mCOSH}
\item{\vskip -1.9ex 
\membername{mCOSH}
{\tt public final void {\bf mCOSH}(  )
\label{l952}\label{l953}}%end signature
}%end item
\divideents{mDIV}
\item{\vskip -1.9ex 
\membername{mDIV}
{\tt public final void {\bf mDIV}(  )
\label{l954}\label{l955}}%end signature
}%end item
\divideents{mDOUBLE}
\item{\vskip -1.9ex 
\membername{mDOUBLE}
{\tt public final void {\bf mDOUBLE}(  )
\label{l956}\label{l957}}%end signature
}%end item
\divideents{mGRAND}
\item{\vskip -1.9ex 
\membername{mGRAND}
{\tt public final void {\bf mGRAND}(  )
\label{l958}\label{l959}}%end signature
}%end item
\divideents{mINT}
\item{\vskip -1.9ex 
\membername{mINT}
{\tt public final void {\bf mINT}(  )
\label{l960}\label{l961}}%end signature
}%end item
\divideents{mLPAREN}
\item{\vskip -1.9ex 
\membername{mLPAREN}
{\tt public final void {\bf mLPAREN}(  )
\label{l962}\label{l963}}%end signature
}%end item
\divideents{mMINUS}
\item{\vskip -1.9ex 
\membername{mMINUS}
{\tt public final void {\bf mMINUS}(  )
\label{l964}\label{l965}}%end signature
}%end item
\divideents{mMOD}
\item{\vskip -1.9ex 
\membername{mMOD}
{\tt public final void {\bf mMOD}(  )
\label{l966}\label{l967}}%end signature
}%end item
\divideents{mMULT}
\item{\vskip -1.9ex 
\membername{mMULT}
{\tt public final void {\bf mMULT}(  )
\label{l968}\label{l969}}%end signature
}%end item
\divideents{mNEWLINE}
\item{\vskip -1.9ex 
\membername{mNEWLINE}
{\tt public final void {\bf mNEWLINE}(  )
\label{l970}\label{l971}}%end signature
}%end item
\divideents{mPLUS}
\item{\vskip -1.9ex 
\membername{mPLUS}
{\tt public final void {\bf mPLUS}(  )
\label{l972}\label{l973}}%end signature
}%end item
\divideents{mPOW}
\item{\vskip -1.9ex 
\membername{mPOW}
{\tt public final void {\bf mPOW}(  )
\label{l974}\label{l975}}%end signature
}%end item
\divideents{mRAND}
\item{\vskip -1.9ex 
\membername{mRAND}
{\tt public final void {\bf mRAND}(  )
\label{l976}\label{l977}}%end signature
}%end item
\divideents{mRPAREN}
\item{\vskip -1.9ex 
\membername{mRPAREN}
{\tt public final void {\bf mRPAREN}(  )
\label{l978}\label{l979}}%end signature
}%end item
\divideents{mSIN}
\item{\vskip -1.9ex 
\membername{mSIN}
{\tt public final void {\bf mSIN}(  )
\label{l980}\label{l981}}%end signature
}%end item
\divideents{mSINH}
\item{\vskip -1.9ex 
\membername{mSINH}
{\tt public final void {\bf mSINH}(  )
\label{l982}\label{l983}}%end signature
}%end item
\divideents{mSQRT}
\item{\vskip -1.9ex 
\membername{mSQRT}
{\tt public final void {\bf mSQRT}(  )
\label{l984}\label{l985}}%end signature
}%end item
\divideents{mTAN}
\item{\vskip -1.9ex 
\membername{mTAN}
{\tt public final void {\bf mTAN}(  )
\label{l986}\label{l987}}%end signature
}%end item
\divideents{mTANH}
\item{\vskip -1.9ex 
\membername{mTANH}
{\tt public final void {\bf mTANH}(  )
\label{l988}\label{l989}}%end signature
}%end item
\divideents{mTokens}
\item{\vskip -1.9ex 
\membername{mTokens}
{\tt public void {\bf mTokens}(  )
\label{l990}\label{l991}}%end signature
}%end item
\divideents{mVAR}
\item{\vskip -1.9ex 
\membername{mVAR}
{\tt public final void {\bf mVAR}(  )
\label{l992}\label{l993}}%end signature
}%end item
\divideents{mWS}
\item{\vskip -1.9ex 
\membername{mWS}
{\tt public final void {\bf mWS}(  )
\label{l994}\label{l995}}%end signature
}%end item
\end{itemize}
}
\startsubsubsection{Methods inherited from class {\tt org.antlr.runtime.Lexer}}{
\par{\small 
\refdefined{l744}\vskip -2em
\begin{itemize}
\item{\vskip -1.9ex 
\membername{emit}
{\tt public Token {\bf emit}(  )
}%end signature
}%end item
\divideents{emit}
\item{\vskip -1.9ex 
\membername{emit}
{\tt public void {\bf emit}( {\tt org.antlr.runtime.Token } {\bf arg0} )
}%end signature
}%end item
\divideents{getCharErrorDisplay}
\item{\vskip -1.9ex 
\membername{getCharErrorDisplay}
{\tt public String {\bf getCharErrorDisplay}( {\tt int } {\bf arg0} )
}%end signature
}%end item
\divideents{getCharIndex}
\item{\vskip -1.9ex 
\membername{getCharIndex}
{\tt public int {\bf getCharIndex}(  )
}%end signature
}%end item
\divideents{getCharPositionInLine}
\item{\vskip -1.9ex 
\membername{getCharPositionInLine}
{\tt public int {\bf getCharPositionInLine}(  )
}%end signature
}%end item
\divideents{getCharStream}
\item{\vskip -1.9ex 
\membername{getCharStream}
{\tt public CharStream {\bf getCharStream}(  )
}%end signature
}%end item
\divideents{getErrorMessage}
\item{\vskip -1.9ex 
\membername{getErrorMessage}
{\tt public String {\bf getErrorMessage}( {\tt org.antlr.runtime.RecognitionException } {\bf arg0},
{\tt java.lang.String []} {\bf arg1} )
}%end signature
}%end item
\divideents{getLine}
\item{\vskip -1.9ex 
\membername{getLine}
{\tt public int {\bf getLine}(  )
}%end signature
}%end item
\divideents{getSourceName}
\item{\vskip -1.9ex 
\membername{getSourceName}
{\tt public String {\bf getSourceName}(  )
}%end signature
}%end item
\divideents{getText}
\item{\vskip -1.9ex 
\membername{getText}
{\tt public String {\bf getText}(  )
}%end signature
}%end item
\divideents{match}
\item{\vskip -1.9ex 
\membername{match}
{\tt public void {\bf match}( {\tt int } {\bf arg0} )
}%end signature
}%end item
\divideents{match}
\item{\vskip -1.9ex 
\membername{match}
{\tt public void {\bf match}( {\tt java.lang.String } {\bf arg0} )
}%end signature
}%end item
\divideents{matchAny}
\item{\vskip -1.9ex 
\membername{matchAny}
{\tt public void {\bf matchAny}(  )
}%end signature
}%end item
\divideents{matchRange}
\item{\vskip -1.9ex 
\membername{matchRange}
{\tt public void {\bf matchRange}( {\tt int } {\bf arg0},
{\tt int } {\bf arg1} )
}%end signature
}%end item
\divideents{mTokens}
\item{\vskip -1.9ex 
\membername{mTokens}
{\tt public abstract void {\bf mTokens}(  )
}%end signature
}%end item
\divideents{nextToken}
\item{\vskip -1.9ex 
\membername{nextToken}
{\tt public Token {\bf nextToken}(  )
}%end signature
}%end item
\divideents{recover}
\item{\vskip -1.9ex 
\membername{recover}
{\tt public void {\bf recover}( {\tt org.antlr.runtime.RecognitionException } {\bf arg0} )
}%end signature
}%end item
\divideents{reportError}
\item{\vskip -1.9ex 
\membername{reportError}
{\tt public void {\bf reportError}( {\tt org.antlr.runtime.RecognitionException } {\bf arg0} )
}%end signature
}%end item
\divideents{reset}
\item{\vskip -1.9ex 
\membername{reset}
{\tt public void {\bf reset}(  )
}%end signature
}%end item
\divideents{setCharStream}
\item{\vskip -1.9ex 
\membername{setCharStream}
{\tt public void {\bf setCharStream}( {\tt org.antlr.runtime.CharStream } {\bf arg0} )
}%end signature
}%end item
\divideents{setText}
\item{\vskip -1.9ex 
\membername{setText}
{\tt public void {\bf setText}( {\tt java.lang.String } {\bf arg0} )
}%end signature
}%end item
\divideents{skip}
\item{\vskip -1.9ex 
\membername{skip}
{\tt public void {\bf skip}(  )
}%end signature
}%end item
\divideents{traceIn}
\item{\vskip -1.9ex 
\membername{traceIn}
{\tt public void {\bf traceIn}( {\tt java.lang.String } {\bf arg0},
{\tt int } {\bf arg1} )
}%end signature
}%end item
\divideents{traceOut}
\item{\vskip -1.9ex 
\membername{traceOut}
{\tt public void {\bf traceOut}( {\tt java.lang.String } {\bf arg0},
{\tt int } {\bf arg1} )
}%end signature
}%end item
\end{itemize}
}}
\startsubsubsection{Methods inherited from class {\tt org.antlr.runtime.BaseRecognizer}}{
\par{\small 
\refdefined{l745}\vskip -2em
\begin{itemize}
\item{\vskip -1.9ex 
\membername{alreadyParsedRule}
{\tt public boolean {\bf alreadyParsedRule}( {\tt org.antlr.runtime.IntStream } {\bf arg0},
{\tt int } {\bf arg1} )
}%end signature
}%end item
\divideents{beginResync}
\item{\vskip -1.9ex 
\membername{beginResync}
{\tt public void {\bf beginResync}(  )
}%end signature
}%end item
\divideents{combineFollows}
\item{\vskip -1.9ex 
\membername{combineFollows}
{\tt protected BitSet {\bf combineFollows}( {\tt boolean } {\bf arg0} )
}%end signature
}%end item
\divideents{computeContextSensitiveRuleFOLLOW}
\item{\vskip -1.9ex 
\membername{computeContextSensitiveRuleFOLLOW}
{\tt protected BitSet {\bf computeContextSensitiveRuleFOLLOW}(  )
}%end signature
}%end item
\divideents{computeErrorRecoverySet}
\item{\vskip -1.9ex 
\membername{computeErrorRecoverySet}
{\tt protected BitSet {\bf computeErrorRecoverySet}(  )
}%end signature
}%end item
\divideents{consumeUntil}
\item{\vskip -1.9ex 
\membername{consumeUntil}
{\tt public void {\bf consumeUntil}( {\tt org.antlr.runtime.IntStream } {\bf arg0},
{\tt org.antlr.runtime.BitSet } {\bf arg1} )
}%end signature
}%end item
\divideents{consumeUntil}
\item{\vskip -1.9ex 
\membername{consumeUntil}
{\tt public void {\bf consumeUntil}( {\tt org.antlr.runtime.IntStream } {\bf arg0},
{\tt int } {\bf arg1} )
}%end signature
}%end item
\divideents{displayRecognitionError}
\item{\vskip -1.9ex 
\membername{displayRecognitionError}
{\tt public void {\bf displayRecognitionError}( {\tt java.lang.String []} {\bf arg0},
{\tt org.antlr.runtime.RecognitionException } {\bf arg1} )
}%end signature
}%end item
\divideents{emitErrorMessage}
\item{\vskip -1.9ex 
\membername{emitErrorMessage}
{\tt public void {\bf emitErrorMessage}( {\tt java.lang.String } {\bf arg0} )
}%end signature
}%end item
\divideents{endResync}
\item{\vskip -1.9ex 
\membername{endResync}
{\tt public void {\bf endResync}(  )
}%end signature
}%end item
\divideents{getBacktrackingLevel}
\item{\vskip -1.9ex 
\membername{getBacktrackingLevel}
{\tt public int {\bf getBacktrackingLevel}(  )
}%end signature
}%end item
\divideents{getCurrentInputSymbol}
\item{\vskip -1.9ex 
\membername{getCurrentInputSymbol}
{\tt protected Object {\bf getCurrentInputSymbol}( {\tt org.antlr.runtime.IntStream } {\bf arg0} )
}%end signature
}%end item
\divideents{getErrorHeader}
\item{\vskip -1.9ex 
\membername{getErrorHeader}
{\tt public String {\bf getErrorHeader}( {\tt org.antlr.runtime.RecognitionException } {\bf arg0} )
}%end signature
}%end item
\divideents{getErrorMessage}
\item{\vskip -1.9ex 
\membername{getErrorMessage}
{\tt public String {\bf getErrorMessage}( {\tt org.antlr.runtime.RecognitionException } {\bf arg0},
{\tt java.lang.String []} {\bf arg1} )
}%end signature
}%end item
\divideents{getGrammarFileName}
\item{\vskip -1.9ex 
\membername{getGrammarFileName}
{\tt public String {\bf getGrammarFileName}(  )
}%end signature
}%end item
\divideents{getMissingSymbol}
\item{\vskip -1.9ex 
\membername{getMissingSymbol}
{\tt protected Object {\bf getMissingSymbol}( {\tt org.antlr.runtime.IntStream } {\bf arg0},
{\tt org.antlr.runtime.RecognitionException } {\bf arg1},
{\tt int } {\bf arg2},
{\tt org.antlr.runtime.BitSet } {\bf arg3} )
}%end signature
}%end item
\divideents{getNumberOfSyntaxErrors}
\item{\vskip -1.9ex 
\membername{getNumberOfSyntaxErrors}
{\tt public int {\bf getNumberOfSyntaxErrors}(  )
}%end signature
}%end item
\divideents{getRuleInvocationStack}
\item{\vskip -1.9ex 
\membername{getRuleInvocationStack}
{\tt public List {\bf getRuleInvocationStack}(  )
}%end signature
}%end item
\divideents{getRuleInvocationStack}
\item{\vskip -1.9ex 
\membername{getRuleInvocationStack}
{\tt public static List {\bf getRuleInvocationStack}( {\tt java.lang.Throwable } {\bf arg0},
{\tt java.lang.String } {\bf arg1} )
}%end signature
}%end item
\divideents{getRuleMemoization}
\item{\vskip -1.9ex 
\membername{getRuleMemoization}
{\tt public int {\bf getRuleMemoization}( {\tt int } {\bf arg0},
{\tt int } {\bf arg1} )
}%end signature
}%end item
\divideents{getRuleMemoizationCacheSize}
\item{\vskip -1.9ex 
\membername{getRuleMemoizationCacheSize}
{\tt public int {\bf getRuleMemoizationCacheSize}(  )
}%end signature
}%end item
\divideents{getSourceName}
\item{\vskip -1.9ex 
\membername{getSourceName}
{\tt public abstract String {\bf getSourceName}(  )
}%end signature
}%end item
\divideents{getTokenErrorDisplay}
\item{\vskip -1.9ex 
\membername{getTokenErrorDisplay}
{\tt public String {\bf getTokenErrorDisplay}( {\tt org.antlr.runtime.Token } {\bf arg0} )
}%end signature
}%end item
\divideents{getTokenNames}
\item{\vskip -1.9ex 
\membername{getTokenNames}
{\tt public String {\bf getTokenNames}(  )
}%end signature
}%end item
\divideents{match}
\item{\vskip -1.9ex 
\membername{match}
{\tt public Object {\bf match}( {\tt org.antlr.runtime.IntStream } {\bf arg0},
{\tt int } {\bf arg1},
{\tt org.antlr.runtime.BitSet } {\bf arg2} )
}%end signature
}%end item
\divideents{matchAny}
\item{\vskip -1.9ex 
\membername{matchAny}
{\tt public void {\bf matchAny}( {\tt org.antlr.runtime.IntStream } {\bf arg0} )
}%end signature
}%end item
\divideents{memoize}
\item{\vskip -1.9ex 
\membername{memoize}
{\tt public void {\bf memoize}( {\tt org.antlr.runtime.IntStream } {\bf arg0},
{\tt int } {\bf arg1},
{\tt int } {\bf arg2} )
}%end signature
}%end item
\divideents{mismatch}
\item{\vskip -1.9ex 
\membername{mismatch}
{\tt protected void {\bf mismatch}( {\tt org.antlr.runtime.IntStream } {\bf arg0},
{\tt int } {\bf arg1},
{\tt org.antlr.runtime.BitSet } {\bf arg2} )
}%end signature
}%end item
\divideents{mismatchIsMissingToken}
\item{\vskip -1.9ex 
\membername{mismatchIsMissingToken}
{\tt public boolean {\bf mismatchIsMissingToken}( {\tt org.antlr.runtime.IntStream } {\bf arg0},
{\tt org.antlr.runtime.BitSet } {\bf arg1} )
}%end signature
}%end item
\divideents{mismatchIsUnwantedToken}
\item{\vskip -1.9ex 
\membername{mismatchIsUnwantedToken}
{\tt public boolean {\bf mismatchIsUnwantedToken}( {\tt org.antlr.runtime.IntStream } {\bf arg0},
{\tt int } {\bf arg1} )
}%end signature
}%end item
\divideents{pushFollow}
\item{\vskip -1.9ex 
\membername{pushFollow}
{\tt protected void {\bf pushFollow}( {\tt org.antlr.runtime.BitSet } {\bf arg0} )
}%end signature
}%end item
\divideents{recover}
\item{\vskip -1.9ex 
\membername{recover}
{\tt public void {\bf recover}( {\tt org.antlr.runtime.IntStream } {\bf arg0},
{\tt org.antlr.runtime.RecognitionException } {\bf arg1} )
}%end signature
}%end item
\divideents{recoverFromMismatchedSet}
\item{\vskip -1.9ex 
\membername{recoverFromMismatchedSet}
{\tt public Object {\bf recoverFromMismatchedSet}( {\tt org.antlr.runtime.IntStream } {\bf arg0},
{\tt org.antlr.runtime.RecognitionException } {\bf arg1},
{\tt org.antlr.runtime.BitSet } {\bf arg2} )
}%end signature
}%end item
\divideents{recoverFromMismatchedToken}
\item{\vskip -1.9ex 
\membername{recoverFromMismatchedToken}
{\tt protected Object {\bf recoverFromMismatchedToken}( {\tt org.antlr.runtime.IntStream } {\bf arg0},
{\tt int } {\bf arg1},
{\tt org.antlr.runtime.BitSet } {\bf arg2} )
}%end signature
}%end item
\divideents{reportError}
\item{\vskip -1.9ex 
\membername{reportError}
{\tt public void {\bf reportError}( {\tt org.antlr.runtime.RecognitionException } {\bf arg0} )
}%end signature
}%end item
\divideents{reset}
\item{\vskip -1.9ex 
\membername{reset}
{\tt public void {\bf reset}(  )
}%end signature
}%end item
\divideents{toStrings}
\item{\vskip -1.9ex 
\membername{toStrings}
{\tt public List {\bf toStrings}( {\tt java.util.List } {\bf arg0} )
}%end signature
}%end item
\divideents{traceIn}
\item{\vskip -1.9ex 
\membername{traceIn}
{\tt public void {\bf traceIn}( {\tt java.lang.String } {\bf arg0},
{\tt int } {\bf arg1},
{\tt java.lang.Object } {\bf arg2} )
}%end signature
}%end item
\divideents{traceOut}
\item{\vskip -1.9ex 
\membername{traceOut}
{\tt public void {\bf traceOut}( {\tt java.lang.String } {\bf arg0},
{\tt int } {\bf arg1},
{\tt java.lang.Object } {\bf arg2} )
}%end signature
}%end item
\end{itemize}
}}
}
\startsection{Class}{ExpressionASTParser}{l884}{%
\startsubsubsection{Declaration}{
\fbox{\vbox{
\hbox{\vbox{\small public 
class 
ExpressionASTParser}}
\noindent\hbox{\vbox{{\bf extends} org.antlr.runtime.Parser}}
}}}
\startsubsubsection{Fields}{
\begin{itemize}
\item{
public static final String tokenNames\begin{itemize}\item{\vskip -.9ex }\end{itemize}
}
\item{
public static final int MOD\begin{itemize}\item{\vskip -.9ex }\end{itemize}
}
\item{
public static final int INT\begin{itemize}\item{\vskip -.9ex }\end{itemize}
}
\item{
public static final int GRAND\begin{itemize}\item{\vskip -.9ex }\end{itemize}
}
\item{
public static final int COSH\begin{itemize}\item{\vskip -.9ex }\end{itemize}
}
\item{
public static final int MULT\begin{itemize}\item{\vskip -.9ex }\end{itemize}
}
\item{
public static final int MINUS\begin{itemize}\item{\vskip -.9ex }\end{itemize}
}
\item{
public static final int SQRT\begin{itemize}\item{\vskip -.9ex }\end{itemize}
}
\item{
public static final int EOF\begin{itemize}\item{\vskip -.9ex }\end{itemize}
}
\item{
public static final int SINH\begin{itemize}\item{\vskip -.9ex }\end{itemize}
}
\item{
public static final int LPAREN\begin{itemize}\item{\vskip -.9ex }\end{itemize}
}
\item{
public static final int RPAREN\begin{itemize}\item{\vskip -.9ex }\end{itemize}
}
\item{
public static final int TANH\begin{itemize}\item{\vskip -.9ex }\end{itemize}
}
\item{
public static final int WS\begin{itemize}\item{\vskip -.9ex }\end{itemize}
}
\item{
public static final int POW\begin{itemize}\item{\vskip -.9ex }\end{itemize}
}
\item{
public static final int NEWLINE\begin{itemize}\item{\vskip -.9ex }\end{itemize}
}
\item{
public static final int SIN\begin{itemize}\item{\vskip -.9ex }\end{itemize}
}
\item{
public static final int COS\begin{itemize}\item{\vskip -.9ex }\end{itemize}
}
\item{
public static final int RAND\begin{itemize}\item{\vskip -.9ex }\end{itemize}
}
\item{
public static final int TAN\begin{itemize}\item{\vskip -.9ex }\end{itemize}
}
\item{
public static final int DOUBLE\begin{itemize}\item{\vskip -.9ex }\end{itemize}
}
\item{
public static final int PLUS\begin{itemize}\item{\vskip -.9ex }\end{itemize}
}
\item{
public static final int VAR\begin{itemize}\item{\vskip -.9ex }\end{itemize}
}
\item{
public static final int DIV\begin{itemize}\item{\vskip -.9ex }\end{itemize}
}
\item{
public static final BitSet FOLLOW\_lowLevelExpr\_in\_getTree199\begin{itemize}\item{\vskip -.9ex }\end{itemize}
}
\item{
public static final BitSet FOLLOW\_NEWLINE\_in\_getTree201\begin{itemize}\item{\vskip -.9ex }\end{itemize}
}
\item{
public static final BitSet FOLLOW\_multLevelExpr\_in\_lowLevelExpr223\begin{itemize}\item{\vskip -.9ex }\end{itemize}
}
\item{
public static final BitSet FOLLOW\_PLUS\_in\_lowLevelExpr238\begin{itemize}\item{\vskip -.9ex }\end{itemize}
}
\item{
public static final BitSet FOLLOW\_multLevelExpr\_in\_lowLevelExpr242\begin{itemize}\item{\vskip -.9ex }\end{itemize}
}
\item{
public static final BitSet FOLLOW\_MINUS\_in\_lowLevelExpr257\begin{itemize}\item{\vskip -.9ex }\end{itemize}
}
\item{
public static final BitSet FOLLOW\_multLevelExpr\_in\_lowLevelExpr261\begin{itemize}\item{\vskip -.9ex }\end{itemize}
}
\item{
public static final BitSet FOLLOW\_powLevelExpr\_in\_multLevelExpr295\begin{itemize}\item{\vskip -.9ex }\end{itemize}
}
\item{
public static final BitSet FOLLOW\_MULT\_in\_multLevelExpr307\begin{itemize}\item{\vskip -.9ex }\end{itemize}
}
\item{
public static final BitSet FOLLOW\_powLevelExpr\_in\_multLevelExpr311\begin{itemize}\item{\vskip -.9ex }\end{itemize}
}
\item{
public static final BitSet FOLLOW\_DIV\_in\_multLevelExpr323\begin{itemize}\item{\vskip -.9ex }\end{itemize}
}
\item{
public static final BitSet FOLLOW\_powLevelExpr\_in\_multLevelExpr327\begin{itemize}\item{\vskip -.9ex }\end{itemize}
}
\item{
public static final BitSet FOLLOW\_MOD\_in\_multLevelExpr339\begin{itemize}\item{\vskip -.9ex }\end{itemize}
}
\item{
public static final BitSet FOLLOW\_powLevelExpr\_in\_multLevelExpr343\begin{itemize}\item{\vskip -.9ex }\end{itemize}
}
\item{
public static final BitSet FOLLOW\_unary\_in\_powLevelExpr372\begin{itemize}\item{\vskip -.9ex }\end{itemize}
}
\item{
public static final BitSet FOLLOW\_POW\_in\_powLevelExpr380\begin{itemize}\item{\vskip -.9ex }\end{itemize}
}
\item{
public static final BitSet FOLLOW\_unary\_in\_powLevelExpr384\begin{itemize}\item{\vskip -.9ex }\end{itemize}
}
\item{
public static final BitSet FOLLOW\_atom\_in\_unary408\begin{itemize}\item{\vskip -.9ex }\end{itemize}
}
\item{
public static final BitSet FOLLOW\_MINUS\_in\_unary415\begin{itemize}\item{\vskip -.9ex }\end{itemize}
}
\item{
public static final BitSet FOLLOW\_atom\_in\_unary419\begin{itemize}\item{\vskip -.9ex }\end{itemize}
}
\item{
public static final BitSet FOLLOW\_INT\_in\_atom440\begin{itemize}\item{\vskip -.9ex }\end{itemize}
}
\item{
public static final BitSet FOLLOW\_DOUBLE\_in\_atom447\begin{itemize}\item{\vskip -.9ex }\end{itemize}
}
\item{
public static final BitSet FOLLOW\_VAR\_in\_atom454\begin{itemize}\item{\vskip -.9ex }\end{itemize}
}
\item{
public static final BitSet FOLLOW\_LPAREN\_in\_atom464\begin{itemize}\item{\vskip -.9ex }\end{itemize}
}
\item{
public static final BitSet FOLLOW\_lowLevelExpr\_in\_atom466\begin{itemize}\item{\vskip -.9ex }\end{itemize}
}
\item{
public static final BitSet FOLLOW\_RPAREN\_in\_atom468\begin{itemize}\item{\vskip -.9ex }\end{itemize}
}
\item{
public static final BitSet FOLLOW\_SQRT\_in\_atom475\begin{itemize}\item{\vskip -.9ex }\end{itemize}
}
\item{
public static final BitSet FOLLOW\_LPAREN\_in\_atom477\begin{itemize}\item{\vskip -.9ex }\end{itemize}
}
\item{
public static final BitSet FOLLOW\_lowLevelExpr\_in\_atom481\begin{itemize}\item{\vskip -.9ex }\end{itemize}
}
\item{
public static final BitSet FOLLOW\_RPAREN\_in\_atom484\begin{itemize}\item{\vskip -.9ex }\end{itemize}
}
\item{
public static final BitSet FOLLOW\_RAND\_in\_atom490\begin{itemize}\item{\vskip -.9ex }\end{itemize}
}
\item{
public static final BitSet FOLLOW\_GRAND\_in\_atom498\begin{itemize}\item{\vskip -.9ex }\end{itemize}
}
\item{
public static final BitSet FOLLOW\_SINH\_in\_atom505\begin{itemize}\item{\vskip -.9ex }\end{itemize}
}
\item{
public static final BitSet FOLLOW\_LPAREN\_in\_atom507\begin{itemize}\item{\vskip -.9ex }\end{itemize}
}
\item{
public static final BitSet FOLLOW\_lowLevelExpr\_in\_atom511\begin{itemize}\item{\vskip -.9ex }\end{itemize}
}
\item{
public static final BitSet FOLLOW\_RPAREN\_in\_atom514\begin{itemize}\item{\vskip -.9ex }\end{itemize}
}
\item{
public static final BitSet FOLLOW\_COSH\_in\_atom519\begin{itemize}\item{\vskip -.9ex }\end{itemize}
}
\item{
public static final BitSet FOLLOW\_LPAREN\_in\_atom521\begin{itemize}\item{\vskip -.9ex }\end{itemize}
}
\item{
public static final BitSet FOLLOW\_lowLevelExpr\_in\_atom525\begin{itemize}\item{\vskip -.9ex }\end{itemize}
}
\item{
public static final BitSet FOLLOW\_RPAREN\_in\_atom528\begin{itemize}\item{\vskip -.9ex }\end{itemize}
}
\item{
public static final BitSet FOLLOW\_TANH\_in\_atom533\begin{itemize}\item{\vskip -.9ex }\end{itemize}
}
\item{
public static final BitSet FOLLOW\_LPAREN\_in\_atom535\begin{itemize}\item{\vskip -.9ex }\end{itemize}
}
\item{
public static final BitSet FOLLOW\_lowLevelExpr\_in\_atom539\begin{itemize}\item{\vskip -.9ex }\end{itemize}
}
\item{
public static final BitSet FOLLOW\_RPAREN\_in\_atom542\begin{itemize}\item{\vskip -.9ex }\end{itemize}
}
\item{
public static final BitSet FOLLOW\_SIN\_in\_atom547\begin{itemize}\item{\vskip -.9ex }\end{itemize}
}
\item{
public static final BitSet FOLLOW\_LPAREN\_in\_atom549\begin{itemize}\item{\vskip -.9ex }\end{itemize}
}
\item{
public static final BitSet FOLLOW\_lowLevelExpr\_in\_atom553\begin{itemize}\item{\vskip -.9ex }\end{itemize}
}
\item{
public static final BitSet FOLLOW\_RPAREN\_in\_atom556\begin{itemize}\item{\vskip -.9ex }\end{itemize}
}
\item{
public static final BitSet FOLLOW\_COS\_in\_atom561\begin{itemize}\item{\vskip -.9ex }\end{itemize}
}
\item{
public static final BitSet FOLLOW\_LPAREN\_in\_atom563\begin{itemize}\item{\vskip -.9ex }\end{itemize}
}
\item{
public static final BitSet FOLLOW\_lowLevelExpr\_in\_atom567\begin{itemize}\item{\vskip -.9ex }\end{itemize}
}
\item{
public static final BitSet FOLLOW\_RPAREN\_in\_atom570\begin{itemize}\item{\vskip -.9ex }\end{itemize}
}
\item{
public static final BitSet FOLLOW\_TAN\_in\_atom575\begin{itemize}\item{\vskip -.9ex }\end{itemize}
}
\item{
public static final BitSet FOLLOW\_LPAREN\_in\_atom577\begin{itemize}\item{\vskip -.9ex }\end{itemize}
}
\item{
public static final BitSet FOLLOW\_lowLevelExpr\_in\_atom581\begin{itemize}\item{\vskip -.9ex }\end{itemize}
}
\item{
public static final BitSet FOLLOW\_RPAREN\_in\_atom584\begin{itemize}\item{\vskip -.9ex }\end{itemize}
}
\end{itemize}
}
\startsubsubsection{Constructors}{
\vskip -2em
\begin{itemize}
\item{\vskip -1.9ex 
\membername{ExpressionASTParser}
{\tt public {\bf ExpressionASTParser}( {\tt org.antlr.runtime.TokenStream } {\bf input} )
\label{l996}\label{l997}}%end signature
}%end item
\divideents{ExpressionASTParser}
\item{\vskip -1.9ex 
\membername{ExpressionASTParser}
{\tt public {\bf ExpressionASTParser}( {\tt org.antlr.runtime.TokenStream } {\bf input},
{\tt org.antlr.runtime.RecognizerSharedState } {\bf state} )
\label{l998}\label{l999}}%end signature
}%end item
\end{itemize}
}
\startsubsubsection{Methods}{
\vskip -2em
\begin{itemize}
\item{\vskip -1.9ex 
\membername{atom}
{\tt public final Component {\bf atom}(  )
\label{l1000}\label{l1001}}%end signature
}%end item
\divideents{getGrammarFileName}
\item{\vskip -1.9ex 
\membername{getGrammarFileName}
{\tt public String {\bf getGrammarFileName}(  )
\label{l1002}\label{l1003}}%end signature
}%end item
\divideents{getTokenNames}
\item{\vskip -1.9ex 
\membername{getTokenNames}
{\tt public String {\bf getTokenNames}(  )
\label{l1004}\label{l1005}}%end signature
}%end item
\divideents{getTree}
\item{\vskip -1.9ex 
\membername{getTree}
{\tt public final Component {\bf getTree}(  )
\label{l1006}\label{l1007}}%end signature
}%end item
\divideents{getVariables}
\item{\vskip -1.9ex 
\membername{getVariables}
{\tt public Map {\bf getVariables}(  )
\label{l1008}\label{l1009}}%end signature
}%end item
\divideents{lowLevelExpr}
\item{\vskip -1.9ex 
\membername{lowLevelExpr}
{\tt public final Component {\bf lowLevelExpr}(  )
\label{l1010}\label{l1011}}%end signature
}%end item
\divideents{multLevelExpr}
\item{\vskip -1.9ex 
\membername{multLevelExpr}
{\tt public final Component {\bf multLevelExpr}(  )
\label{l1012}\label{l1013}}%end signature
}%end item
\divideents{powLevelExpr}
\item{\vskip -1.9ex 
\membername{powLevelExpr}
{\tt public final Component {\bf powLevelExpr}(  )
\label{l1014}\label{l1015}}%end signature
}%end item
\divideents{unary}
\item{\vskip -1.9ex 
\membername{unary}
{\tt public final Component {\bf unary}(  )
\label{l1016}\label{l1017}}%end signature
}%end item
\end{itemize}
}
\startsubsubsection{Methods inherited from class {\tt org.antlr.runtime.Parser}}{
\par{\small 
\refdefined{l772}\vskip -2em
\begin{itemize}
\item{\vskip -1.9ex 
\membername{getCurrentInputSymbol}
{\tt protected Object {\bf getCurrentInputSymbol}( {\tt org.antlr.runtime.IntStream } {\bf arg0} )
}%end signature
}%end item
\divideents{getMissingSymbol}
\item{\vskip -1.9ex 
\membername{getMissingSymbol}
{\tt protected Object {\bf getMissingSymbol}( {\tt org.antlr.runtime.IntStream } {\bf arg0},
{\tt org.antlr.runtime.RecognitionException } {\bf arg1},
{\tt int } {\bf arg2},
{\tt org.antlr.runtime.BitSet } {\bf arg3} )
}%end signature
}%end item
\divideents{getSourceName}
\item{\vskip -1.9ex 
\membername{getSourceName}
{\tt public String {\bf getSourceName}(  )
}%end signature
}%end item
\divideents{getTokenStream}
\item{\vskip -1.9ex 
\membername{getTokenStream}
{\tt public TokenStream {\bf getTokenStream}(  )
}%end signature
}%end item
\divideents{reset}
\item{\vskip -1.9ex 
\membername{reset}
{\tt public void {\bf reset}(  )
}%end signature
}%end item
\divideents{setTokenStream}
\item{\vskip -1.9ex 
\membername{setTokenStream}
{\tt public void {\bf setTokenStream}( {\tt org.antlr.runtime.TokenStream } {\bf arg0} )
}%end signature
}%end item
\divideents{traceIn}
\item{\vskip -1.9ex 
\membername{traceIn}
{\tt public void {\bf traceIn}( {\tt java.lang.String } {\bf arg0},
{\tt int } {\bf arg1} )
}%end signature
}%end item
\divideents{traceOut}
\item{\vskip -1.9ex 
\membername{traceOut}
{\tt public void {\bf traceOut}( {\tt java.lang.String } {\bf arg0},
{\tt int } {\bf arg1} )
}%end signature
}%end item
\end{itemize}
}}
\startsubsubsection{Methods inherited from class {\tt org.antlr.runtime.BaseRecognizer}}{
\par{\small 
\refdefined{l745}\vskip -2em
\begin{itemize}
\item{\vskip -1.9ex 
\membername{alreadyParsedRule}
{\tt public boolean {\bf alreadyParsedRule}( {\tt org.antlr.runtime.IntStream } {\bf arg0},
{\tt int } {\bf arg1} )
}%end signature
}%end item
\divideents{beginResync}
\item{\vskip -1.9ex 
\membername{beginResync}
{\tt public void {\bf beginResync}(  )
}%end signature
}%end item
\divideents{combineFollows}
\item{\vskip -1.9ex 
\membername{combineFollows}
{\tt protected BitSet {\bf combineFollows}( {\tt boolean } {\bf arg0} )
}%end signature
}%end item
\divideents{computeContextSensitiveRuleFOLLOW}
\item{\vskip -1.9ex 
\membername{computeContextSensitiveRuleFOLLOW}
{\tt protected BitSet {\bf computeContextSensitiveRuleFOLLOW}(  )
}%end signature
}%end item
\divideents{computeErrorRecoverySet}
\item{\vskip -1.9ex 
\membername{computeErrorRecoverySet}
{\tt protected BitSet {\bf computeErrorRecoverySet}(  )
}%end signature
}%end item
\divideents{consumeUntil}
\item{\vskip -1.9ex 
\membername{consumeUntil}
{\tt public void {\bf consumeUntil}( {\tt org.antlr.runtime.IntStream } {\bf arg0},
{\tt org.antlr.runtime.BitSet } {\bf arg1} )
}%end signature
}%end item
\divideents{consumeUntil}
\item{\vskip -1.9ex 
\membername{consumeUntil}
{\tt public void {\bf consumeUntil}( {\tt org.antlr.runtime.IntStream } {\bf arg0},
{\tt int } {\bf arg1} )
}%end signature
}%end item
\divideents{displayRecognitionError}
\item{\vskip -1.9ex 
\membername{displayRecognitionError}
{\tt public void {\bf displayRecognitionError}( {\tt java.lang.String []} {\bf arg0},
{\tt org.antlr.runtime.RecognitionException } {\bf arg1} )
}%end signature
}%end item
\divideents{emitErrorMessage}
\item{\vskip -1.9ex 
\membername{emitErrorMessage}
{\tt public void {\bf emitErrorMessage}( {\tt java.lang.String } {\bf arg0} )
}%end signature
}%end item
\divideents{endResync}
\item{\vskip -1.9ex 
\membername{endResync}
{\tt public void {\bf endResync}(  )
}%end signature
}%end item
\divideents{getBacktrackingLevel}
\item{\vskip -1.9ex 
\membername{getBacktrackingLevel}
{\tt public int {\bf getBacktrackingLevel}(  )
}%end signature
}%end item
\divideents{getCurrentInputSymbol}
\item{\vskip -1.9ex 
\membername{getCurrentInputSymbol}
{\tt protected Object {\bf getCurrentInputSymbol}( {\tt org.antlr.runtime.IntStream } {\bf arg0} )
}%end signature
}%end item
\divideents{getErrorHeader}
\item{\vskip -1.9ex 
\membername{getErrorHeader}
{\tt public String {\bf getErrorHeader}( {\tt org.antlr.runtime.RecognitionException } {\bf arg0} )
}%end signature
}%end item
\divideents{getErrorMessage}
\item{\vskip -1.9ex 
\membername{getErrorMessage}
{\tt public String {\bf getErrorMessage}( {\tt org.antlr.runtime.RecognitionException } {\bf arg0},
{\tt java.lang.String []} {\bf arg1} )
}%end signature
}%end item
\divideents{getGrammarFileName}
\item{\vskip -1.9ex 
\membername{getGrammarFileName}
{\tt public String {\bf getGrammarFileName}(  )
}%end signature
}%end item
\divideents{getMissingSymbol}
\item{\vskip -1.9ex 
\membername{getMissingSymbol}
{\tt protected Object {\bf getMissingSymbol}( {\tt org.antlr.runtime.IntStream } {\bf arg0},
{\tt org.antlr.runtime.RecognitionException } {\bf arg1},
{\tt int } {\bf arg2},
{\tt org.antlr.runtime.BitSet } {\bf arg3} )
}%end signature
}%end item
\divideents{getNumberOfSyntaxErrors}
\item{\vskip -1.9ex 
\membername{getNumberOfSyntaxErrors}
{\tt public int {\bf getNumberOfSyntaxErrors}(  )
}%end signature
}%end item
\divideents{getRuleInvocationStack}
\item{\vskip -1.9ex 
\membername{getRuleInvocationStack}
{\tt public List {\bf getRuleInvocationStack}(  )
}%end signature
}%end item
\divideents{getRuleInvocationStack}
\item{\vskip -1.9ex 
\membername{getRuleInvocationStack}
{\tt public static List {\bf getRuleInvocationStack}( {\tt java.lang.Throwable } {\bf arg0},
{\tt java.lang.String } {\bf arg1} )
}%end signature
}%end item
\divideents{getRuleMemoization}
\item{\vskip -1.9ex 
\membername{getRuleMemoization}
{\tt public int {\bf getRuleMemoization}( {\tt int } {\bf arg0},
{\tt int } {\bf arg1} )
}%end signature
}%end item
\divideents{getRuleMemoizationCacheSize}
\item{\vskip -1.9ex 
\membername{getRuleMemoizationCacheSize}
{\tt public int {\bf getRuleMemoizationCacheSize}(  )
}%end signature
}%end item
\divideents{getSourceName}
\item{\vskip -1.9ex 
\membername{getSourceName}
{\tt public abstract String {\bf getSourceName}(  )
}%end signature
}%end item
\divideents{getTokenErrorDisplay}
\item{\vskip -1.9ex 
\membername{getTokenErrorDisplay}
{\tt public String {\bf getTokenErrorDisplay}( {\tt org.antlr.runtime.Token } {\bf arg0} )
}%end signature
}%end item
\divideents{getTokenNames}
\item{\vskip -1.9ex 
\membername{getTokenNames}
{\tt public String {\bf getTokenNames}(  )
}%end signature
}%end item
\divideents{match}
\item{\vskip -1.9ex 
\membername{match}
{\tt public Object {\bf match}( {\tt org.antlr.runtime.IntStream } {\bf arg0},
{\tt int } {\bf arg1},
{\tt org.antlr.runtime.BitSet } {\bf arg2} )
}%end signature
}%end item
\divideents{matchAny}
\item{\vskip -1.9ex 
\membername{matchAny}
{\tt public void {\bf matchAny}( {\tt org.antlr.runtime.IntStream } {\bf arg0} )
}%end signature
}%end item
\divideents{memoize}
\item{\vskip -1.9ex 
\membername{memoize}
{\tt public void {\bf memoize}( {\tt org.antlr.runtime.IntStream } {\bf arg0},
{\tt int } {\bf arg1},
{\tt int } {\bf arg2} )
}%end signature
}%end item
\divideents{mismatch}
\item{\vskip -1.9ex 
\membername{mismatch}
{\tt protected void {\bf mismatch}( {\tt org.antlr.runtime.IntStream } {\bf arg0},
{\tt int } {\bf arg1},
{\tt org.antlr.runtime.BitSet } {\bf arg2} )
}%end signature
}%end item
\divideents{mismatchIsMissingToken}
\item{\vskip -1.9ex 
\membername{mismatchIsMissingToken}
{\tt public boolean {\bf mismatchIsMissingToken}( {\tt org.antlr.runtime.IntStream } {\bf arg0},
{\tt org.antlr.runtime.BitSet } {\bf arg1} )
}%end signature
}%end item
\divideents{mismatchIsUnwantedToken}
\item{\vskip -1.9ex 
\membername{mismatchIsUnwantedToken}
{\tt public boolean {\bf mismatchIsUnwantedToken}( {\tt org.antlr.runtime.IntStream } {\bf arg0},
{\tt int } {\bf arg1} )
}%end signature
}%end item
\divideents{pushFollow}
\item{\vskip -1.9ex 
\membername{pushFollow}
{\tt protected void {\bf pushFollow}( {\tt org.antlr.runtime.BitSet } {\bf arg0} )
}%end signature
}%end item
\divideents{recover}
\item{\vskip -1.9ex 
\membername{recover}
{\tt public void {\bf recover}( {\tt org.antlr.runtime.IntStream } {\bf arg0},
{\tt org.antlr.runtime.RecognitionException } {\bf arg1} )
}%end signature
}%end item
\divideents{recoverFromMismatchedSet}
\item{\vskip -1.9ex 
\membername{recoverFromMismatchedSet}
{\tt public Object {\bf recoverFromMismatchedSet}( {\tt org.antlr.runtime.IntStream } {\bf arg0},
{\tt org.antlr.runtime.RecognitionException } {\bf arg1},
{\tt org.antlr.runtime.BitSet } {\bf arg2} )
}%end signature
}%end item
\divideents{recoverFromMismatchedToken}
\item{\vskip -1.9ex 
\membername{recoverFromMismatchedToken}
{\tt protected Object {\bf recoverFromMismatchedToken}( {\tt org.antlr.runtime.IntStream } {\bf arg0},
{\tt int } {\bf arg1},
{\tt org.antlr.runtime.BitSet } {\bf arg2} )
}%end signature
}%end item
\divideents{reportError}
\item{\vskip -1.9ex 
\membername{reportError}
{\tt public void {\bf reportError}( {\tt org.antlr.runtime.RecognitionException } {\bf arg0} )
}%end signature
}%end item
\divideents{reset}
\item{\vskip -1.9ex 
\membername{reset}
{\tt public void {\bf reset}(  )
}%end signature
}%end item
\divideents{toStrings}
\item{\vskip -1.9ex 
\membername{toStrings}
{\tt public List {\bf toStrings}( {\tt java.util.List } {\bf arg0} )
}%end signature
}%end item
\divideents{traceIn}
\item{\vskip -1.9ex 
\membername{traceIn}
{\tt public void {\bf traceIn}( {\tt java.lang.String } {\bf arg0},
{\tt int } {\bf arg1},
{\tt java.lang.Object } {\bf arg2} )
}%end signature
}%end item
\divideents{traceOut}
\item{\vskip -1.9ex 
\membername{traceOut}
{\tt public void {\bf traceOut}( {\tt java.lang.String } {\bf arg0},
{\tt int } {\bf arg1},
{\tt java.lang.Object } {\bf arg2} )
}%end signature
}%end item
\end{itemize}
}}
}
\startsection{Class}{Literal}{l885}{%
\startsubsubsection{Declaration}{
\fbox{\vbox{
\hbox{\vbox{\small public 
class 
Literal}}
\noindent\hbox{\vbox{{\bf extends} uk.ac.ic.doc.neuralnets.expressions.ast.Component}}
}}}
\startsubsubsection{Constructors}{
\vskip -2em
\begin{itemize}
\item{\vskip -1.9ex 
\membername{Literal}
{\tt public {\bf Literal}( {\tt java.lang.Double } {\bf d} )
\label{l1018}\label{l1019}}%end signature
}%end item
\divideents{Literal}
\item{\vskip -1.9ex 
\membername{Literal}
{\tt public {\bf Literal}( {\tt java.lang.String } {\bf val} )
\label{l1020}\label{l1021}}%end signature
}%end item
\end{itemize}
}
\startsubsubsection{Methods}{
\vskip -2em
\begin{itemize}
\item{\vskip -1.9ex 
\membername{evaluate}
{\tt public Double {\bf evaluate}(  )
\label{l1022}\label{l1023}}%end signature
}%end item
\divideents{getExpression}
\item{\vskip -1.9ex 
\membername{getExpression}
{\tt public String {\bf getExpression}(  )
\label{l1024}\label{l1025}}%end signature
}%end item
\divideents{getVariables}
\item{\vskip -1.9ex 
\membername{getVariables}
{\tt public Set {\bf getVariables}(  )
\label{l1026}\label{l1027}}%end signature
}%end item
\end{itemize}
}
\startsubsubsection{Methods inherited from class {\tt uk.ac.ic.doc.neuralnets.expressions.ast.Component}}{
\par{\small 
\refdefined{l882}\vskip -2em
\begin{itemize}
\item{\vskip -1.9ex 
\membername{bracket}
{\tt public String {\bf bracket}( {\tt uk.ac.ic.doc.neuralnets.expressions.ast.Component } {\bf c} )
}%end signature
\begin{itemize}
\sld
\item{
\sld
{\bf Usage}
  \begin{itemize}\isep
   \item{
A meethod to parenthesise the given child expression in the context of
 the current operation; applies mathematical order of operations rules.
}%end item
  \end{itemize}
}
\item{
\sld
{\bf Parameters}
\sld\isep
  \begin{itemize}
\sld\isep
   \item{
\sld
{\tt c} - The child component to parenthesise}
  \end{itemize}
}%end item
\item{{\bf Returns} - 
A String representation of the child, with or without
 parentheses, as deemed necessary. 
}%end item
\end{itemize}
}%end item
\divideents{evaluate}
\item{\vskip -1.9ex 
\membername{evaluate}
{\tt public abstract Double {\bf evaluate}(  )
}%end signature
\begin{itemize}
\sld
\item{
\sld
{\bf Usage}
  \begin{itemize}\isep
   \item{
Calculate the value of this expression sub-tree in its current bindings
 (if applicable)
}%end item
  \end{itemize}
}
\item{{\bf Returns} - 
A Double value of the output of evaluating this tree 
}%end item
\item{{\bf Exceptions}
  \begin{itemize}
\sld
   \item{\vskip -.6ex{\tt uk.ac.ic.doc.neuralnets.expressions.ExpressionException} - }
  \end{itemize}
}%end item
\end{itemize}
}%end item
\divideents{getExpression}
\item{\vskip -1.9ex 
\membername{getExpression}
{\tt public abstract String {\bf getExpression}(  )
}%end signature
\begin{itemize}
\sld
\item{
\sld
{\bf Usage}
  \begin{itemize}\isep
   \item{
Retrieve the original expression, re-formatted for user friendly output
}%end item
  \end{itemize}
}
\item{{\bf Returns} - 
A String representation of this expression tree; must be
 re-parsable by the ASTExpressionFactory. 
}%end item
\end{itemize}
}%end item
\divideents{getVariables}
\item{\vskip -1.9ex 
\membername{getVariables}
{\tt public abstract Set {\bf getVariables}(  )
}%end signature
\begin{itemize}
\sld
\item{
\sld
{\bf Usage}
  \begin{itemize}\isep
   \item{
Answer a set of the variable objects in this tree; this may include
 any instances of the Variable class, or any operations that return a
 different value for each evaluation, e.g. random operators, counters etc
}%end item
  \end{itemize}
}
\item{{\bf Returns} - 
A Set of the variable components 
}%end item
\item{{\bf See Also}
  \begin{itemize}
   \item{{\tt uk.ac.ic.doc.neuralnets.expressions.ast.Variable} {\small 
\refdefined{l889}}%end \small
}%end item
  \end{itemize}
}%end item
\end{itemize}
}%end item
\divideents{order}
\item{\vskip -1.9ex 
\membername{order}
{\tt public int {\bf order}( {\tt java.lang.String } {\bf op} )
}%end signature
\begin{itemize}
\sld
\item{
\sld
{\bf Usage}
  \begin{itemize}\isep
   \item{
Decide the internal ordering of the supplied operation; higher numbers
 represent a lower importance. Defaults to Integer.MAX\_VALUE if the
 operator is not recognised.
}%end item
  \end{itemize}
}
\item{
\sld
{\bf Parameters}
\sld\isep
  \begin{itemize}
\sld\isep
   \item{
\sld
{\tt op} - The operator to decide precedence of}
  \end{itemize}
}%end item
\item{{\bf Returns} - 
An integer value; lower values for greater precedence 
}%end item
\end{itemize}
}%end item
\end{itemize}
}}
}
\startsection{Class}{NoOpComponent}{l886}{%
{\small Simple Component to perform no operation at all. Must have a sub-component
 under it in order to be evaluated.}
\vskip .1in 
\startsubsubsection{Declaration}{
\fbox{\vbox{
\hbox{\vbox{\small public 
class 
NoOpComponent}}
\noindent\hbox{\vbox{{\bf extends} uk.ac.ic.doc.neuralnets.expressions.ast.Component}}
}}}
\startsubsubsection{Constructors}{
\vskip -2em
\begin{itemize}
\item{\vskip -1.9ex 
\membername{NoOpComponent}
{\tt public {\bf NoOpComponent}( {\tt uk.ac.ic.doc.neuralnets.expressions.ast.Component } {\bf sub} )
\label{l1028}\label{l1029}}%end signature
}%end item
\end{itemize}
}
\startsubsubsection{Methods}{
\vskip -2em
\begin{itemize}
\item{\vskip -1.9ex 
\membername{evaluate}
{\tt public Double {\bf evaluate}(  )
\label{l1030}\label{l1031}}%end signature
}%end item
\divideents{getExpression}
\item{\vskip -1.9ex 
\membername{getExpression}
{\tt public String {\bf getExpression}(  )
\label{l1032}\label{l1033}}%end signature
}%end item
\divideents{getVariables}
\item{\vskip -1.9ex 
\membername{getVariables}
{\tt public Set {\bf getVariables}(  )
\label{l1034}\label{l1035}}%end signature
}%end item
\end{itemize}
}
\startsubsubsection{Methods inherited from class {\tt uk.ac.ic.doc.neuralnets.expressions.ast.Component}}{
\par{\small 
\refdefined{l882}\vskip -2em
\begin{itemize}
\item{\vskip -1.9ex 
\membername{bracket}
{\tt public String {\bf bracket}( {\tt uk.ac.ic.doc.neuralnets.expressions.ast.Component } {\bf c} )
}%end signature
\begin{itemize}
\sld
\item{
\sld
{\bf Usage}
  \begin{itemize}\isep
   \item{
A meethod to parenthesise the given child expression in the context of
 the current operation; applies mathematical order of operations rules.
}%end item
  \end{itemize}
}
\item{
\sld
{\bf Parameters}
\sld\isep
  \begin{itemize}
\sld\isep
   \item{
\sld
{\tt c} - The child component to parenthesise}
  \end{itemize}
}%end item
\item{{\bf Returns} - 
A String representation of the child, with or without
 parentheses, as deemed necessary. 
}%end item
\end{itemize}
}%end item
\divideents{evaluate}
\item{\vskip -1.9ex 
\membername{evaluate}
{\tt public abstract Double {\bf evaluate}(  )
}%end signature
\begin{itemize}
\sld
\item{
\sld
{\bf Usage}
  \begin{itemize}\isep
   \item{
Calculate the value of this expression sub-tree in its current bindings
 (if applicable)
}%end item
  \end{itemize}
}
\item{{\bf Returns} - 
A Double value of the output of evaluating this tree 
}%end item
\item{{\bf Exceptions}
  \begin{itemize}
\sld
   \item{\vskip -.6ex{\tt uk.ac.ic.doc.neuralnets.expressions.ExpressionException} - }
  \end{itemize}
}%end item
\end{itemize}
}%end item
\divideents{getExpression}
\item{\vskip -1.9ex 
\membername{getExpression}
{\tt public abstract String {\bf getExpression}(  )
}%end signature
\begin{itemize}
\sld
\item{
\sld
{\bf Usage}
  \begin{itemize}\isep
   \item{
Retrieve the original expression, re-formatted for user friendly output
}%end item
  \end{itemize}
}
\item{{\bf Returns} - 
A String representation of this expression tree; must be
 re-parsable by the ASTExpressionFactory. 
}%end item
\end{itemize}
}%end item
\divideents{getVariables}
\item{\vskip -1.9ex 
\membername{getVariables}
{\tt public abstract Set {\bf getVariables}(  )
}%end signature
\begin{itemize}
\sld
\item{
\sld
{\bf Usage}
  \begin{itemize}\isep
   \item{
Answer a set of the variable objects in this tree; this may include
 any instances of the Variable class, or any operations that return a
 different value for each evaluation, e.g. random operators, counters etc
}%end item
  \end{itemize}
}
\item{{\bf Returns} - 
A Set of the variable components 
}%end item
\item{{\bf See Also}
  \begin{itemize}
   \item{{\tt uk.ac.ic.doc.neuralnets.expressions.ast.Variable} {\small 
\refdefined{l889}}%end \small
}%end item
  \end{itemize}
}%end item
\end{itemize}
}%end item
\divideents{order}
\item{\vskip -1.9ex 
\membername{order}
{\tt public int {\bf order}( {\tt java.lang.String } {\bf op} )
}%end signature
\begin{itemize}
\sld
\item{
\sld
{\bf Usage}
  \begin{itemize}\isep
   \item{
Decide the internal ordering of the supplied operation; higher numbers
 represent a lower importance. Defaults to Integer.MAX\_VALUE if the
 operator is not recognised.
}%end item
  \end{itemize}
}
\item{
\sld
{\bf Parameters}
\sld\isep
  \begin{itemize}
\sld\isep
   \item{
\sld
{\tt op} - The operator to decide precedence of}
  \end{itemize}
}%end item
\item{{\bf Returns} - 
An integer value; lower values for greater precedence 
}%end item
\end{itemize}
}%end item
\end{itemize}
}}
}
\startsection{Class}{NullaryOperator}{l887}{%
{\small Component to be evaluated with no operators}
\vskip .1in 
\startsubsubsection{Declaration}{
\fbox{\vbox{
\hbox{\vbox{\small public abstract 
class 
NullaryOperator}}
\noindent\hbox{\vbox{{\bf extends} uk.ac.ic.doc.neuralnets.expressions.ast.Component}}
}}}
\startsubsubsection{Constructors}{
\vskip -2em
\begin{itemize}
\item{\vskip -1.9ex 
\membername{NullaryOperator}
{\tt public {\bf NullaryOperator}( {\tt java.lang.String } {\bf operation} )
\label{l1036}\label{l1037}}%end signature
}%end item
\end{itemize}
}
\startsubsubsection{Methods}{
\vskip -2em
\begin{itemize}
\item{\vskip -1.9ex 
\membername{evaluate}
{\tt public abstract Double {\bf evaluate}(  )
\label{l1038}\label{l1039}}%end signature
}%end item
\divideents{getExpression}
\item{\vskip -1.9ex 
\membername{getExpression}
{\tt public String {\bf getExpression}(  )
\label{l1040}\label{l1041}}%end signature
}%end item
\divideents{getVariables}
\item{\vskip -1.9ex 
\membername{getVariables}
{\tt public Set {\bf getVariables}(  )
\label{l1042}\label{l1043}}%end signature
}%end item
\end{itemize}
}
\startsubsubsection{Methods inherited from class {\tt uk.ac.ic.doc.neuralnets.expressions.ast.Component}}{
\par{\small 
\refdefined{l882}\vskip -2em
\begin{itemize}
\item{\vskip -1.9ex 
\membername{bracket}
{\tt public String {\bf bracket}( {\tt uk.ac.ic.doc.neuralnets.expressions.ast.Component } {\bf c} )
}%end signature
\begin{itemize}
\sld
\item{
\sld
{\bf Usage}
  \begin{itemize}\isep
   \item{
A meethod to parenthesise the given child expression in the context of
 the current operation; applies mathematical order of operations rules.
}%end item
  \end{itemize}
}
\item{
\sld
{\bf Parameters}
\sld\isep
  \begin{itemize}
\sld\isep
   \item{
\sld
{\tt c} - The child component to parenthesise}
  \end{itemize}
}%end item
\item{{\bf Returns} - 
A String representation of the child, with or without
 parentheses, as deemed necessary. 
}%end item
\end{itemize}
}%end item
\divideents{evaluate}
\item{\vskip -1.9ex 
\membername{evaluate}
{\tt public abstract Double {\bf evaluate}(  )
}%end signature
\begin{itemize}
\sld
\item{
\sld
{\bf Usage}
  \begin{itemize}\isep
   \item{
Calculate the value of this expression sub-tree in its current bindings
 (if applicable)
}%end item
  \end{itemize}
}
\item{{\bf Returns} - 
A Double value of the output of evaluating this tree 
}%end item
\item{{\bf Exceptions}
  \begin{itemize}
\sld
   \item{\vskip -.6ex{\tt uk.ac.ic.doc.neuralnets.expressions.ExpressionException} - }
  \end{itemize}
}%end item
\end{itemize}
}%end item
\divideents{getExpression}
\item{\vskip -1.9ex 
\membername{getExpression}
{\tt public abstract String {\bf getExpression}(  )
}%end signature
\begin{itemize}
\sld
\item{
\sld
{\bf Usage}
  \begin{itemize}\isep
   \item{
Retrieve the original expression, re-formatted for user friendly output
}%end item
  \end{itemize}
}
\item{{\bf Returns} - 
A String representation of this expression tree; must be
 re-parsable by the ASTExpressionFactory. 
}%end item
\end{itemize}
}%end item
\divideents{getVariables}
\item{\vskip -1.9ex 
\membername{getVariables}
{\tt public abstract Set {\bf getVariables}(  )
}%end signature
\begin{itemize}
\sld
\item{
\sld
{\bf Usage}
  \begin{itemize}\isep
   \item{
Answer a set of the variable objects in this tree; this may include
 any instances of the Variable class, or any operations that return a
 different value for each evaluation, e.g. random operators, counters etc
}%end item
  \end{itemize}
}
\item{{\bf Returns} - 
A Set of the variable components 
}%end item
\item{{\bf See Also}
  \begin{itemize}
   \item{{\tt uk.ac.ic.doc.neuralnets.expressions.ast.Variable} {\small 
\refdefined{l889}}%end \small
}%end item
  \end{itemize}
}%end item
\end{itemize}
}%end item
\divideents{order}
\item{\vskip -1.9ex 
\membername{order}
{\tt public int {\bf order}( {\tt java.lang.String } {\bf op} )
}%end signature
\begin{itemize}
\sld
\item{
\sld
{\bf Usage}
  \begin{itemize}\isep
   \item{
Decide the internal ordering of the supplied operation; higher numbers
 represent a lower importance. Defaults to Integer.MAX\_VALUE if the
 operator is not recognised.
}%end item
  \end{itemize}
}
\item{
\sld
{\bf Parameters}
\sld\isep
  \begin{itemize}
\sld\isep
   \item{
\sld
{\tt op} - The operator to decide precedence of}
  \end{itemize}
}%end item
\item{{\bf Returns} - 
An integer value; lower values for greater precedence 
}%end item
\end{itemize}
}%end item
\end{itemize}
}}
}
\startsection{Class}{UnaryOperator}{l888}{%
{\small Component that is evaluated with one operator only}
\vskip .1in 
\startsubsubsection{Declaration}{
\fbox{\vbox{
\hbox{\vbox{\small public abstract 
class 
UnaryOperator}}
\noindent\hbox{\vbox{{\bf extends} uk.ac.ic.doc.neuralnets.expressions.ast.Component}}
}}}
\startsubsubsection{Constructors}{
\vskip -2em
\begin{itemize}
\item{\vskip -1.9ex 
\membername{UnaryOperator}
{\tt public {\bf UnaryOperator}( {\tt uk.ac.ic.doc.neuralnets.expressions.ast.Component } {\bf c},
{\tt java.lang.String } {\bf operation} )
\label{l1044}\label{l1045}}%end signature
}%end item
\end{itemize}
}
\startsubsubsection{Methods}{
\vskip -2em
\begin{itemize}
\item{\vskip -1.9ex 
\membername{evaluate}
{\tt public abstract Double {\bf evaluate}(  )
\label{l1046}\label{l1047}}%end signature
}%end item
\divideents{getExpression}
\item{\vskip -1.9ex 
\membername{getExpression}
{\tt public String {\bf getExpression}(  )
\label{l1048}\label{l1049}}%end signature
}%end item
\divideents{getVariables}
\item{\vskip -1.9ex 
\membername{getVariables}
{\tt public Set {\bf getVariables}(  )
\label{l1050}\label{l1051}}%end signature
}%end item
\end{itemize}
}
\startsubsubsection{Methods inherited from class {\tt uk.ac.ic.doc.neuralnets.expressions.ast.Component}}{
\par{\small 
\refdefined{l882}\vskip -2em
\begin{itemize}
\item{\vskip -1.9ex 
\membername{bracket}
{\tt public String {\bf bracket}( {\tt uk.ac.ic.doc.neuralnets.expressions.ast.Component } {\bf c} )
}%end signature
\begin{itemize}
\sld
\item{
\sld
{\bf Usage}
  \begin{itemize}\isep
   \item{
A meethod to parenthesise the given child expression in the context of
 the current operation; applies mathematical order of operations rules.
}%end item
  \end{itemize}
}
\item{
\sld
{\bf Parameters}
\sld\isep
  \begin{itemize}
\sld\isep
   \item{
\sld
{\tt c} - The child component to parenthesise}
  \end{itemize}
}%end item
\item{{\bf Returns} - 
A String representation of the child, with or without
 parentheses, as deemed necessary. 
}%end item
\end{itemize}
}%end item
\divideents{evaluate}
\item{\vskip -1.9ex 
\membername{evaluate}
{\tt public abstract Double {\bf evaluate}(  )
}%end signature
\begin{itemize}
\sld
\item{
\sld
{\bf Usage}
  \begin{itemize}\isep
   \item{
Calculate the value of this expression sub-tree in its current bindings
 (if applicable)
}%end item
  \end{itemize}
}
\item{{\bf Returns} - 
A Double value of the output of evaluating this tree 
}%end item
\item{{\bf Exceptions}
  \begin{itemize}
\sld
   \item{\vskip -.6ex{\tt uk.ac.ic.doc.neuralnets.expressions.ExpressionException} - }
  \end{itemize}
}%end item
\end{itemize}
}%end item
\divideents{getExpression}
\item{\vskip -1.9ex 
\membername{getExpression}
{\tt public abstract String {\bf getExpression}(  )
}%end signature
\begin{itemize}
\sld
\item{
\sld
{\bf Usage}
  \begin{itemize}\isep
   \item{
Retrieve the original expression, re-formatted for user friendly output
}%end item
  \end{itemize}
}
\item{{\bf Returns} - 
A String representation of this expression tree; must be
 re-parsable by the ASTExpressionFactory. 
}%end item
\end{itemize}
}%end item
\divideents{getVariables}
\item{\vskip -1.9ex 
\membername{getVariables}
{\tt public abstract Set {\bf getVariables}(  )
}%end signature
\begin{itemize}
\sld
\item{
\sld
{\bf Usage}
  \begin{itemize}\isep
   \item{
Answer a set of the variable objects in this tree; this may include
 any instances of the Variable class, or any operations that return a
 different value for each evaluation, e.g. random operators, counters etc
}%end item
  \end{itemize}
}
\item{{\bf Returns} - 
A Set of the variable components 
}%end item
\item{{\bf See Also}
  \begin{itemize}
   \item{{\tt uk.ac.ic.doc.neuralnets.expressions.ast.Variable} {\small 
\refdefined{l889}}%end \small
}%end item
  \end{itemize}
}%end item
\end{itemize}
}%end item
\divideents{order}
\item{\vskip -1.9ex 
\membername{order}
{\tt public int {\bf order}( {\tt java.lang.String } {\bf op} )
}%end signature
\begin{itemize}
\sld
\item{
\sld
{\bf Usage}
  \begin{itemize}\isep
   \item{
Decide the internal ordering of the supplied operation; higher numbers
 represent a lower importance. Defaults to Integer.MAX\_VALUE if the
 operator is not recognised.
}%end item
  \end{itemize}
}
\item{
\sld
{\bf Parameters}
\sld\isep
  \begin{itemize}
\sld\isep
   \item{
\sld
{\tt op} - The operator to decide precedence of}
  \end{itemize}
}%end item
\item{{\bf Returns} - 
An integer value; lower values for greater precedence 
}%end item
\end{itemize}
}%end item
\end{itemize}
}}
}
\startsection{Class}{Variable}{l889}{%
{\small A named variable Component, capable of being bound to any Double value.}
\vskip .1in 
\startsubsubsection{Declaration}{
\fbox{\vbox{
\hbox{\vbox{\small public 
class 
Variable}}
\noindent\hbox{\vbox{{\bf extends} uk.ac.ic.doc.neuralnets.expressions.ast.Component}}
}}}
\startsubsubsection{Constructors}{
\vskip -2em
\begin{itemize}
\item{\vskip -1.9ex 
\membername{Variable}
{\tt public {\bf Variable}( {\tt java.lang.String } {\bf name} )
\label{l1052}\label{l1053}}%end signature
}%end item
\end{itemize}
}
\startsubsubsection{Methods}{
\vskip -2em
\begin{itemize}
\item{\vskip -1.9ex 
\membername{bind}
{\tt public void {\bf bind}( {\tt java.lang.Double } {\bf val} )
\label{l1054}\label{l1055}}%end signature
\begin{itemize}
\sld
\item{
\sld
{\bf Usage}
  \begin{itemize}\isep
   \item{
Bind this variable to the given value
}%end item
  \end{itemize}
}
\item{
\sld
{\bf Parameters}
\sld\isep
  \begin{itemize}
\sld\isep
   \item{
\sld
{\tt val} - The value to bind this Variable component to}
  \end{itemize}
}%end item
\end{itemize}
}%end item
\divideents{evaluate}
\item{\vskip -1.9ex 
\membername{evaluate}
{\tt public Double {\bf evaluate}(  )
\label{l1056}\label{l1057}}%end signature
}%end item
\divideents{getExpression}
\item{\vskip -1.9ex 
\membername{getExpression}
{\tt public String {\bf getExpression}(  )
\label{l1058}\label{l1059}}%end signature
}%end item
\divideents{getVariables}
\item{\vskip -1.9ex 
\membername{getVariables}
{\tt public Set {\bf getVariables}(  )
\label{l1060}\label{l1061}}%end signature
}%end item
\end{itemize}
}
\startsubsubsection{Methods inherited from class {\tt uk.ac.ic.doc.neuralnets.expressions.ast.Component}}{
\par{\small 
\refdefined{l882}\vskip -2em
\begin{itemize}
\item{\vskip -1.9ex 
\membername{bracket}
{\tt public String {\bf bracket}( {\tt uk.ac.ic.doc.neuralnets.expressions.ast.Component } {\bf c} )
}%end signature
\begin{itemize}
\sld
\item{
\sld
{\bf Usage}
  \begin{itemize}\isep
   \item{
A meethod to parenthesise the given child expression in the context of
 the current operation; applies mathematical order of operations rules.
}%end item
  \end{itemize}
}
\item{
\sld
{\bf Parameters}
\sld\isep
  \begin{itemize}
\sld\isep
   \item{
\sld
{\tt c} - The child component to parenthesise}
  \end{itemize}
}%end item
\item{{\bf Returns} - 
A String representation of the child, with or without
 parentheses, as deemed necessary. 
}%end item
\end{itemize}
}%end item
\divideents{evaluate}
\item{\vskip -1.9ex 
\membername{evaluate}
{\tt public abstract Double {\bf evaluate}(  )
}%end signature
\begin{itemize}
\sld
\item{
\sld
{\bf Usage}
  \begin{itemize}\isep
   \item{
Calculate the value of this expression sub-tree in its current bindings
 (if applicable)
}%end item
  \end{itemize}
}
\item{{\bf Returns} - 
A Double value of the output of evaluating this tree 
}%end item
\item{{\bf Exceptions}
  \begin{itemize}
\sld
   \item{\vskip -.6ex{\tt uk.ac.ic.doc.neuralnets.expressions.ExpressionException} - }
  \end{itemize}
}%end item
\end{itemize}
}%end item
\divideents{getExpression}
\item{\vskip -1.9ex 
\membername{getExpression}
{\tt public abstract String {\bf getExpression}(  )
}%end signature
\begin{itemize}
\sld
\item{
\sld
{\bf Usage}
  \begin{itemize}\isep
   \item{
Retrieve the original expression, re-formatted for user friendly output
}%end item
  \end{itemize}
}
\item{{\bf Returns} - 
A String representation of this expression tree; must be
 re-parsable by the ASTExpressionFactory. 
}%end item
\end{itemize}
}%end item
\divideents{getVariables}
\item{\vskip -1.9ex 
\membername{getVariables}
{\tt public abstract Set {\bf getVariables}(  )
}%end signature
\begin{itemize}
\sld
\item{
\sld
{\bf Usage}
  \begin{itemize}\isep
   \item{
Answer a set of the variable objects in this tree; this may include
 any instances of the Variable class, or any operations that return a
 different value for each evaluation, e.g. random operators, counters etc
}%end item
  \end{itemize}
}
\item{{\bf Returns} - 
A Set of the variable components 
}%end item
\item{{\bf See Also}
  \begin{itemize}
   \item{{\tt uk.ac.ic.doc.neuralnets.expressions.ast.Variable} {\small 
\refdefined{l889}}%end \small
}%end item
  \end{itemize}
}%end item
\end{itemize}
}%end item
\divideents{order}
\item{\vskip -1.9ex 
\membername{order}
{\tt public int {\bf order}( {\tt java.lang.String } {\bf op} )
}%end signature
\begin{itemize}
\sld
\item{
\sld
{\bf Usage}
  \begin{itemize}\isep
   \item{
Decide the internal ordering of the supplied operation; higher numbers
 represent a lower importance. Defaults to Integer.MAX\_VALUE if the
 operator is not recognised.
}%end item
  \end{itemize}
}
\item{
\sld
{\bf Parameters}
\sld\isep
  \begin{itemize}
\sld\isep
   \item{
\sld
{\tt op} - The operator to decide precedence of}
  \end{itemize}
}%end item
\item{{\bf Returns} - 
An integer value; lower values for greater precedence 
}%end item
\end{itemize}
}%end item
\end{itemize}
}}
}
}
}
\newpage
\def\packagename{uk.ac.ic.doc.neuralnets.graph.neural}
\chapter{\bf Package uk.ac.ic.doc.neuralnets.graph.neural}{
\vskip -.25in
\hbox to \hsize{\it Package Contents\hfil Page}
\rule{\hsize}{.7mm}
\vskip .13in
\hbox{\bf Interfaces}
\entityintro{Persistable}{l1062}{...no description...}
\vskip .13in
\hbox{\bf Classes}
\entityintro{EdgeBase}{l1063}{...no description...}
\entityintro{EdgeDecoration}{l1064}{...no description...}
\entityintro{EdgeSpecification}{l1065}{Default EdgeSpecification}
\entityintro{NetworkBridge}{l1066}{Models a connection between two NeuralNetworks as a bundle of synapses}
\entityintro{NeuralNetwork}{l226}{...no description...}
\entityintro{NeuralNetworkSimulationEvent}{l1067}{...no description...}
\entityintro{NeuralNetworkTickEvent}{l1068}{...no description...}
\entityintro{Neurone}{l225}{...no description...}
\entityintro{NeuroneTypeConfig}{l1069}{Configurator to load Statisticians}
\entityintro{NeuroneTypes}{l1070}{Container object for the Neurone Types created by NeuroneTypeConfig}
\entityintro{NewNeuroneTypeEvent}{l1071}{Indicates a new neurone type has been created}
\entityintro{NodeBase}{l514}{Basic Node implementation; should suffice for most Node purposes}
\entityintro{NodeChargeUpdateEvent}{l1072}{...no description...}
\entityintro{NodeFired}{l1073}{...no description...}
\entityintro{NodeSpecification}{l283}{Default NodeSpecification}
\entityintro{Perceptron}{l1074}{...no description...}
\entityintro{SpikingNeurone}{l1075}{...no description...}
\entityintro{Synapse}{l1076}{...no description...}
\vskip .1in
\rule{\hsize}{.7mm}
\vskip .1in
\newpage
\section{Interfaces}{
\startsection{Interface}{Persistable}{l1062}{%
\startsubsubsection{Declaration}{
\fbox{\vbox{
\hbox{\vbox{\small public interface 
Persistable}}
\noindent\hbox{\vbox{{\bf implements} 
java.lang.annotation.Annotation}}
}}}
}
}
\section{Classes}{
\startsection{Class}{EdgeBase}{l1063}{%
\startsubsubsection{Declaration}{
\fbox{\vbox{
\hbox{\vbox{\small public abstract 
class 
EdgeBase}}
\noindent\hbox{\vbox{{\bf extends} java.lang.Object}}
\noindent\hbox{\vbox{{\bf implements} 
uk.ac.ic.doc.neuralnets.graph.Edge}}
}}}
\startsubsubsection{Serializable Fields}{
\begin{itemize}
\item{
private int id\begin{itemize}\item{\vskip -.9ex }\end{itemize}
}
\end{itemize}
}
\startsubsubsection{Constructors}{
\vskip -2em
\begin{itemize}
\item{\vskip -1.9ex 
\membername{EdgeBase}
{\tt public {\bf EdgeBase}( {\tt uk.ac.ic.doc.neuralnets.graph.Node } {\bf start},
{\tt uk.ac.ic.doc.neuralnets.graph.Node } {\bf end} )
\label{l1077}\label{l1078}}%end signature
}%end item
\end{itemize}
}
\startsubsubsection{Methods}{
\vskip -2em
\begin{itemize}
\item{\vskip -1.9ex 
\membername{getEnd}
{\tt public Node {\bf getEnd}(  )
\label{l1079}\label{l1080}}%end signature
}%end item
\divideents{getFreshID}
\item{\vskip -1.9ex 
\membername{getFreshID}
{\tt public void {\bf getFreshID}(  )
\label{l1081}\label{l1082}}%end signature
}%end item
\divideents{getID}
\item{\vskip -1.9ex 
\membername{getID}
{\tt public int {\bf getID}(  )
\label{l1083}\label{l1084}}%end signature
}%end item
\divideents{getStart}
\item{\vskip -1.9ex 
\membername{getStart}
{\tt public Node {\bf getStart}(  )
\label{l1085}\label{l1086}}%end signature
}%end item
\divideents{setID}
\item{\vskip -1.9ex 
\membername{setID}
{\tt public void {\bf setID}( {\tt int } {\bf id} )
\label{l1087}\label{l1088}}%end signature
}%end item
\divideents{setStart}
\item{\vskip -1.9ex 
\membername{setStart}
{\tt public Edge {\bf setStart}( {\tt uk.ac.ic.doc.neuralnets.graph.Node } {\bf start} )
\label{l1089}\label{l1090}}%end signature
}%end item
\divideents{setTo}
\item{\vskip -1.9ex 
\membername{setTo}
{\tt public Edge {\bf setTo}( {\tt uk.ac.ic.doc.neuralnets.graph.Node } {\bf end} )
\label{l1091}\label{l1092}}%end signature
}%end item
\divideents{tick}
\item{\vskip -1.9ex 
\membername{tick}
{\tt public void {\bf tick}(  )
\label{l1093}\label{l1094}}%end signature
}%end item
\divideents{toString}
\item{\vskip -1.9ex 
\membername{toString}
{\tt public String {\bf toString}(  )
\label{l1095}\label{l1096}}%end signature
}%end item
\end{itemize}
}
}
\startsection{Class}{EdgeDecoration}{l1064}{%
\startsubsubsection{Declaration}{
\fbox{\vbox{
\hbox{\vbox{\small public abstract 
class 
EdgeDecoration}}
\noindent\hbox{\vbox{{\bf extends} java.lang.Object}}
\noindent\hbox{\vbox{{\bf implements} 
uk.ac.ic.doc.neuralnets.util.plugins.Plugin, java.io.Serializable}}
}}}
\startsubsubsection{Constructors}{
\vskip -2em
\begin{itemize}
\item{\vskip -1.9ex 
\membername{EdgeDecoration}
{\tt public {\bf EdgeDecoration}(  )
\label{l1097}\label{l1098}}%end signature
}%end item
\end{itemize}
}
\startsubsubsection{Methods}{
\vskip -2em
\begin{itemize}
\item{\vskip -1.9ex 
\membername{getFigure}
{\tt public abstract Object {\bf getFigure}(  )
\label{l1099}\label{l1100}}%end signature
}%end item
\divideents{getName}
\item{\vskip -1.9ex 
\membername{getName}
{\tt public abstract String {\bf getName}(  )
\label{l1101}\label{l1102}}%end signature
}%end item
\end{itemize}
}
}
\startsection{Class}{EdgeSpecification}{l1065}{%
{\small Default EdgeSpecification}
\vskip .1in 
\startsubsubsection{Declaration}{
\fbox{\vbox{
\hbox{\vbox{\small public 
class 
EdgeSpecification}}
\noindent\hbox{\vbox{{\bf extends} java.lang.Object}}
\noindent\hbox{\vbox{{\bf implements} 
java.io.Serializable}}
}}}
\startsubsubsection{Constructors}{
\vskip -2em
\begin{itemize}
\item{\vskip -1.9ex 
\membername{EdgeSpecification}
{\tt public {\bf EdgeSpecification}(  )
\label{l1103}\label{l1104}}%end signature
}%end item
\end{itemize}
}
\startsubsubsection{Methods}{
\vskip -2em
\begin{itemize}
\item{\vskip -1.9ex 
\membername{getEnd}
{\tt public Node {\bf getEnd}(  )
\label{l1105}\label{l1106}}%end signature
\begin{itemize}
\sld
\item{
\sld
{\bf Usage}
  \begin{itemize}\isep
   \item{
Get the end of the edge.
}%end item
  \end{itemize}
}
\item{{\bf Returns} - 
The end. 
}%end item
\end{itemize}
}%end item
\divideents{getStart}
\item{\vskip -1.9ex 
\membername{getStart}
{\tt public Node {\bf getStart}(  )
\label{l1107}\label{l1108}}%end signature
\begin{itemize}
\sld
\item{
\sld
{\bf Usage}
  \begin{itemize}\isep
   \item{
Get the start of the edge.
}%end item
  \end{itemize}
}
\item{{\bf Returns} - 
The start. 
}%end item
\end{itemize}
}%end item
\divideents{getWeight}
\item{\vskip -1.9ex 
\membername{getWeight}
{\tt public double {\bf getWeight}(  )
\label{l1109}\label{l1110}}%end signature
\begin{itemize}
\sld
\item{
\sld
{\bf Usage}
  \begin{itemize}\isep
   \item{
Returns a random weight.
}%end item
  \end{itemize}
}
\item{{\bf Returns} - 
Random weight: 0 \textless  w \textless  1 
}%end item
\end{itemize}
}%end item
\end{itemize}
}
}
\startsection{Class}{NetworkBridge}{l1066}{%
{\small Models a connection between two NeuralNetworks as a bundle of synapses}
\vskip .1in 
\startsubsubsection{Declaration}{
\fbox{\vbox{
\hbox{\vbox{\small public 
class 
NetworkBridge}}
\noindent\hbox{\vbox{{\bf extends} uk.ac.ic.doc.neuralnets.graph.neural.EdgeBase}}
}}}
\startsubsubsection{Serializable Fields}{
\begin{itemize}
\item{
private Set bundle\begin{itemize}\item{\vskip -.9ex }\end{itemize}
}
\end{itemize}
}
\startsubsubsection{Constructors}{
\vskip -2em
\begin{itemize}
\item{\vskip -1.9ex 
\membername{NetworkBridge}
{\tt public {\bf NetworkBridge}(  )
\label{l1111}\label{l1112}}%end signature
}%end item
\divideents{NetworkBridge}
\item{\vskip -1.9ex 
\membername{NetworkBridge}
{\tt public {\bf NetworkBridge}( {\tt uk.ac.ic.doc.neuralnets.graph.neural.NeuralNetwork } {\bf start},
{\tt uk.ac.ic.doc.neuralnets.graph.neural.NeuralNetwork } {\bf end} )
\label{l1113}\label{l1114}}%end signature
}%end item
\end{itemize}
}
\startsubsubsection{Methods}{
\vskip -2em
\begin{itemize}
\item{\vskip -1.9ex 
\membername{connect}
{\tt public Edge {\bf connect}( {\tt uk.ac.ic.doc.neuralnets.graph.Edge } {\bf e} )
\label{l1115}\label{l1116}}%end signature
}%end item
\divideents{getBundle}
\item{\vskip -1.9ex 
\membername{getBundle}
{\tt public Collection {\bf getBundle}(  )
\label{l1117}\label{l1118}}%end signature
}%end item
\divideents{toString}
\item{\vskip -1.9ex 
\membername{toString}
{\tt public String {\bf toString}(  )
\label{l1119}\label{l1120}}%end signature
}%end item
\end{itemize}
}
\startsubsubsection{Methods inherited from class {\tt uk.ac.ic.doc.neuralnets.graph.neural.EdgeBase}}{
\par{\small 
\refdefined{l1063}\vskip -2em
\begin{itemize}
\item{\vskip -1.9ex 
\membername{getEnd}
{\tt public Node {\bf getEnd}(  )
}%end signature
}%end item
\divideents{getFreshID}
\item{\vskip -1.9ex 
\membername{getFreshID}
{\tt public void {\bf getFreshID}(  )
}%end signature
}%end item
\divideents{getID}
\item{\vskip -1.9ex 
\membername{getID}
{\tt public int {\bf getID}(  )
}%end signature
}%end item
\divideents{getStart}
\item{\vskip -1.9ex 
\membername{getStart}
{\tt public Node {\bf getStart}(  )
}%end signature
}%end item
\divideents{setID}
\item{\vskip -1.9ex 
\membername{setID}
{\tt public void {\bf setID}( {\tt int } {\bf id} )
}%end signature
}%end item
\divideents{setStart}
\item{\vskip -1.9ex 
\membername{setStart}
{\tt public Edge {\bf setStart}( {\tt uk.ac.ic.doc.neuralnets.graph.Node } {\bf start} )
}%end signature
}%end item
\divideents{setTo}
\item{\vskip -1.9ex 
\membername{setTo}
{\tt public Edge {\bf setTo}( {\tt uk.ac.ic.doc.neuralnets.graph.Node } {\bf end} )
}%end signature
}%end item
\divideents{tick}
\item{\vskip -1.9ex 
\membername{tick}
{\tt public void {\bf tick}(  )
}%end signature
}%end item
\divideents{toString}
\item{\vskip -1.9ex 
\membername{toString}
{\tt public String {\bf toString}(  )
}%end signature
}%end item
\end{itemize}
}}
}
\startsection{Class}{NeuralNetwork}{l226}{%
\startsubsubsection{Declaration}{
\fbox{\vbox{
\hbox{\vbox{\small public 
class 
NeuralNetwork}}
\noindent\hbox{\vbox{{\bf extends} uk.ac.ic.doc.neuralnets.graph.Graph}}
\noindent\hbox{\vbox{{\bf implements} 
uk.ac.ic.doc.neuralnets.graph.Node, uk.ac.ic.doc.neuralnets.graph.Saveable}}
}}}
\startsubsubsection{Serializable Fields}{
\begin{itemize}
\item{
private Set in\begin{itemize}\item{\vskip -.9ex }\end{itemize}
}
\item{
private Set out\begin{itemize}\item{\vskip -.9ex }\end{itemize}
}
\item{
private Map metadata\begin{itemize}\item{\vskip -.9ex }\end{itemize}
}
\item{
private int xpos\begin{itemize}\item{\vskip -.9ex }\end{itemize}
}
\item{
private int ypos\begin{itemize}\item{\vskip -.9ex }\end{itemize}
}
\item{
private int zpos\begin{itemize}\item{\vskip -.9ex }\end{itemize}
}
\item{
private int ticks\begin{itemize}\item{\vskip -.9ex }\end{itemize}
}
\end{itemize}
}
\startsubsubsection{Constructors}{
\vskip -2em
\begin{itemize}
\item{\vskip -1.9ex 
\membername{NeuralNetwork}
{\tt public {\bf NeuralNetwork}(  )
\label{l1121}\label{l1122}}%end signature
}%end item
\end{itemize}
}
\startsubsubsection{Methods}{
\vskip -2em
\begin{itemize}
\item{\vskip -1.9ex 
\membername{connect}
{\tt public Node {\bf connect}( {\tt uk.ac.ic.doc.neuralnets.graph.neural.NetworkBridge } {\bf e} )
\label{l1123}\label{l1124}}%end signature
}%end item
\divideents{getIncoming}
\item{\vskip -1.9ex 
\membername{getIncoming}
{\tt public Collection {\bf getIncoming}(  )
\label{l1125}\label{l1126}}%end signature
}%end item
\divideents{getMetadata}
\item{\vskip -1.9ex 
\membername{getMetadata}
{\tt public String {\bf getMetadata}( {\tt java.lang.String } {\bf key} )
\label{l1127}\label{l1128}}%end signature
}%end item
\divideents{getOutgoing}
\item{\vskip -1.9ex 
\membername{getOutgoing}
{\tt public Collection {\bf getOutgoing}(  )
\label{l1129}\label{l1130}}%end signature
}%end item
\divideents{getTicks}
\item{\vskip -1.9ex 
\membername{getTicks}
{\tt public int {\bf getTicks}(  )
\label{l1131}\label{l1132}}%end signature
}%end item
\divideents{getX}
\item{\vskip -1.9ex 
\membername{getX}
{\tt public int {\bf getX}(  )
\label{l1133}\label{l1134}}%end signature
}%end item
\divideents{getY}
\item{\vskip -1.9ex 
\membername{getY}
{\tt public int {\bf getY}(  )
\label{l1135}\label{l1136}}%end signature
}%end item
\divideents{getZ}
\item{\vskip -1.9ex 
\membername{getZ}
{\tt public int {\bf getZ}(  )
\label{l1137}\label{l1138}}%end signature
}%end item
\divideents{resetTicks}
\item{\vskip -1.9ex 
\membername{resetTicks}
{\tt public void {\bf resetTicks}(  )
\label{l1139}\label{l1140}}%end signature
}%end item
\divideents{setMetadata}
\item{\vskip -1.9ex 
\membername{setMetadata}
{\tt public Node {\bf setMetadata}( {\tt java.lang.String } {\bf key},
{\tt java.lang.String } {\bf item} )
\label{l1141}\label{l1142}}%end signature
}%end item
\divideents{setPos}
\item{\vskip -1.9ex 
\membername{setPos}
{\tt public void {\bf setPos}( {\tt int } {\bf x},
{\tt int } {\bf y},
{\tt int } {\bf z} )
\label{l1143}\label{l1144}}%end signature
}%end item
\divideents{tick}
\item{\vskip -1.9ex 
\membername{tick}
{\tt public Node {\bf tick}(  )
\label{l1145}\label{l1146}}%end signature
}%end item
\divideents{type}
\item{\vskip -1.9ex 
\membername{type}
{\tt protected String {\bf type}(  )
\label{l1147}\label{l1148}}%end signature
}%end item
\end{itemize}
}
\startsubsubsection{Methods inherited from class {\tt uk.ac.ic.doc.neuralnets.graph.Graph}}{
\par{\small 
\refdefined{l507}\vskip -2em
\begin{itemize}
\item{\vskip -1.9ex 
\membername{addAllNodes}
{\tt public Graph {\bf addAllNodes}( {\tt java.util.Collection } {\bf ns} )
}%end signature
\begin{itemize}
\sld
\item{
\sld
{\bf Usage}
  \begin{itemize}\isep
   \item{
Adds a collection of nodes to the graph, only if that collection doesn't
 contain itself.
}%end item
  \end{itemize}
}
\item{
\sld
{\bf Parameters}
\sld\isep
  \begin{itemize}
\sld\isep
   \item{
\sld
{\tt ns} - Collection of nodes to add.}
  \end{itemize}
}%end item
\item{{\bf Returns} - 
Itself with the nodes added or not added. 
}%end item
\end{itemize}
}%end item
\divideents{addEdge}
\item{\vskip -1.9ex 
\membername{addEdge}
{\tt public Graph {\bf addEdge}( {\tt uk.ac.ic.doc.neuralnets.graph.Edge } {\bf e} )
}%end signature
\begin{itemize}
\sld
\item{
\sld
{\bf Usage}
  \begin{itemize}\isep
   \item{
Adds an edge to the graph and adds its start and end nodes to the graph.
}%end item
  \end{itemize}
}
\item{
\sld
{\bf Parameters}
\sld\isep
  \begin{itemize}
\sld\isep
   \item{
\sld
{\tt e} - Edge to add.}
  \end{itemize}
}%end item
\item{{\bf Returns} - 
Itself 
}%end item
\end{itemize}
}%end item
\divideents{addNode}
\item{\vskip -1.9ex 
\membername{addNode}
{\tt public Graph {\bf addNode}( {\tt uk.ac.ic.doc.neuralnets.graph.Node } {\bf n} )
}%end signature
\begin{itemize}
\sld
\item{
\sld
{\bf Usage}
  \begin{itemize}\isep
   \item{
Adds input node to the graph as long as input node is not itself, returns
 itself.
}%end item
  \end{itemize}
}
\item{
\sld
{\bf Parameters}
\sld\isep
  \begin{itemize}
\sld\isep
   \item{
\sld
{\tt n} - Node to add.}
  \end{itemize}
}%end item
\item{{\bf Returns} - 
Itself with the node added or not added. 
}%end item
\end{itemize}
}%end item
\divideents{forEachEdge}
\item{\vskip -1.9ex 
\membername{forEachEdge}
{\tt public Graph {\bf forEachEdge}( {\tt uk.ac.ic.doc.neuralnets.graph.Graph.Command } {\bf c} )
}%end signature
\begin{itemize}
\sld
\item{
\sld
{\bf Usage}
  \begin{itemize}\isep
   \item{
Conducts a command on each edge within the graph.
}%end item
  \end{itemize}
}
\item{
\sld
{\bf Parameters}
\sld\isep
  \begin{itemize}
\sld\isep
   \item{
\sld
{\tt c} - Command to execute.}
  \end{itemize}
}%end item
\item{{\bf Returns} - 
Itself. 
}%end item
\end{itemize}
}%end item
\divideents{forEachNode}
\item{\vskip -1.9ex 
\membername{forEachNode}
{\tt public Graph {\bf forEachNode}( {\tt uk.ac.ic.doc.neuralnets.graph.Graph.Command } {\bf c} )
}%end signature
\begin{itemize}
\sld
\item{
\sld
{\bf Usage}
  \begin{itemize}\isep
   \item{
Conducts a command on each node within the graph.
}%end item
  \end{itemize}
}
\item{
\sld
{\bf Parameters}
\sld\isep
  \begin{itemize}
\sld\isep
   \item{
\sld
{\tt c} - Command to execute.}
  \end{itemize}
}%end item
\item{{\bf Returns} - 
Itself. 
}%end item
\end{itemize}
}%end item
\divideents{getEdges}
\item{\vskip -1.9ex 
\membername{getEdges}
{\tt public Collection {\bf getEdges}(  )
}%end signature
\begin{itemize}
\sld
\item{
\sld
{\bf Usage}
  \begin{itemize}\isep
   \item{
Gets the edges from within.
}%end item
  \end{itemize}
}
\item{{\bf Returns} - 
The edges. 
}%end item
\end{itemize}
}%end item
\divideents{getFreshID}
\item{\vskip -1.9ex 
\membername{getFreshID}
{\tt public void {\bf getFreshID}(  )
}%end signature
\begin{itemize}
\sld
\item{
\sld
{\bf Usage}
  \begin{itemize}\isep
   \item{
Sets the id of the object to a new fresh id.
}%end item
  \end{itemize}
}
\end{itemize}
}%end item
\divideents{getID}
\item{\vskip -1.9ex 
\membername{getID}
{\tt public int {\bf getID}(  )
}%end signature
\begin{itemize}
\sld
\item{
\sld
{\bf Usage}
  \begin{itemize}\isep
   \item{
Gets the id of the object.
}%end item
  \end{itemize}
}
\item{{\bf Returns} - 
The id. 
}%end item
\end{itemize}
}%end item
\divideents{getNodes}
\item{\vskip -1.9ex 
\membername{getNodes}
{\tt public Collection {\bf getNodes}(  )
}%end signature
\begin{itemize}
\sld
\item{
\sld
{\bf Usage}
  \begin{itemize}\isep
   \item{
Gets the nodes from within.
}%end item
  \end{itemize}
}
\item{{\bf Returns} - 
The nodes. 
}%end item
\end{itemize}
}%end item
\divideents{merge}
\item{\vskip -1.9ex 
\membername{merge}
{\tt public Graph {\bf merge}( {\tt uk.ac.ic.doc.neuralnets.graph.Graph } {\bf o} )
}%end signature
\begin{itemize}
\sld
\item{
\sld
{\bf Usage}
  \begin{itemize}\isep
   \item{
Merges one graph with its self, as all the edges and nodes.
}%end item
  \end{itemize}
}
\item{
\sld
{\bf Parameters}
\sld\isep
  \begin{itemize}
\sld\isep
   \item{
\sld
{\tt o} - Graph to merge with.}
  \end{itemize}
}%end item
\item{{\bf Returns} - 
Itself 
}%end item
\end{itemize}
}%end item
\divideents{setID}
\item{\vskip -1.9ex 
\membername{setID}
{\tt public void {\bf setID}( {\tt int } {\bf id} )
}%end signature
\begin{itemize}
\sld
\item{
\sld
{\bf Usage}
  \begin{itemize}\isep
   \item{
Sets the id of the object to parameter.
}%end item
  \end{itemize}
}
\item{
\sld
{\bf Parameters}
\sld\isep
  \begin{itemize}
\sld\isep
   \item{
\sld
{\tt int} - New id.}
  \end{itemize}
}%end item
\end{itemize}
}%end item
\divideents{toString}
\item{\vskip -1.9ex 
\membername{toString}
{\tt public String {\bf toString}(  )
}%end signature
}%end item
\divideents{type}
\item{\vskip -1.9ex 
\membername{type}
{\tt protected String {\bf type}(  )
}%end signature
\begin{itemize}
\sld
\item{
\sld
{\bf Usage}
  \begin{itemize}\isep
   \item{
Returns the object type.
}%end item
  \end{itemize}
}
\item{{\bf Returns} - 
Object type. 
}%end item
\end{itemize}
}%end item
\end{itemize}
}}
}
\startsection{Class}{NeuralNetworkSimulationEvent}{l1067}{%
\startsubsubsection{Declaration}{
\fbox{\vbox{
\hbox{\vbox{\small public 
class 
NeuralNetworkSimulationEvent}}
\noindent\hbox{\vbox{{\bf extends} uk.ac.ic.doc.neuralnets.events.RevalidateStatisticiansEvent}}
}}}
\startsubsubsection{Constructors}{
\vskip -2em
\begin{itemize}
\item{\vskip -1.9ex 
\membername{NeuralNetworkSimulationEvent}
{\tt public {\bf NeuralNetworkSimulationEvent}(  )
\label{l1149}\label{l1150}}%end signature
}%end item
\divideents{NeuralNetworkSimulationEvent}
\item{\vskip -1.9ex 
\membername{NeuralNetworkSimulationEvent}
{\tt public {\bf NeuralNetworkSimulationEvent}( {\tt boolean } {\bf b} )
\label{l1151}\label{l1152}}%end signature
}%end item
\end{itemize}
}
\startsubsubsection{Methods}{
\vskip -2em
\begin{itemize}
\item{\vskip -1.9ex 
\membername{started}
{\tt public boolean {\bf started}(  )
\label{l1153}\label{l1154}}%end signature
}%end item
\divideents{toString}
\item{\vskip -1.9ex 
\membername{toString}
{\tt public String {\bf toString}(  )
\label{l1155}\label{l1156}}%end signature
}%end item
\end{itemize}
}
\startsubsubsection{Methods inherited from class {\tt uk.ac.ic.doc.neuralnets.events.RevalidateStatisticiansEvent}}{
\par{\small 
\refdefined{l1157}\vskip -2em
\begin{itemize}
\item{\vskip -1.9ex 
\membername{toString}
{\tt public String {\bf toString}(  )
}%end signature
}%end item
\end{itemize}
}}
\startsubsubsection{Methods inherited from class {\tt uk.ac.ic.doc.neuralnets.events.Event}}{
\par{\small 
\refdefined{l220}\vskip -2em
\begin{itemize}
\item{\vskip -1.9ex 
\membername{toString}
{\tt public abstract String {\bf toString}(  )
}%end signature
}%end item
\end{itemize}
}}
}
\startsection{Class}{NeuralNetworkTickEvent}{l1068}{%
\startsubsubsection{Declaration}{
\fbox{\vbox{
\hbox{\vbox{\small public 
class 
NeuralNetworkTickEvent}}
\noindent\hbox{\vbox{{\bf extends} uk.ac.ic.doc.neuralnets.events.Event}}
}}}
\startsubsubsection{Constructors}{
\vskip -2em
\begin{itemize}
\item{\vskip -1.9ex 
\membername{NeuralNetworkTickEvent}
{\tt public {\bf NeuralNetworkTickEvent}( {\tt int } {\bf ticks} )
\label{l1158}\label{l1159}}%end signature
}%end item
\end{itemize}
}
\startsubsubsection{Methods}{
\vskip -2em
\begin{itemize}
\item{\vskip -1.9ex 
\membername{toString}
{\tt public String {\bf toString}(  )
\label{l1160}\label{l1161}}%end signature
}%end item
\end{itemize}
}
\startsubsubsection{Methods inherited from class {\tt uk.ac.ic.doc.neuralnets.events.Event}}{
\par{\small 
\refdefined{l220}\vskip -2em
\begin{itemize}
\item{\vskip -1.9ex 
\membername{toString}
{\tt public abstract String {\bf toString}(  )
}%end signature
}%end item
\end{itemize}
}}
}
\startsection{Class}{Neurone}{l225}{%
\startsubsubsection{Declaration}{
\fbox{\vbox{
\hbox{\vbox{\small public 
class 
Neurone}}
\noindent\hbox{\vbox{{\bf extends} uk.ac.ic.doc.neuralnets.graph.neural.NodeBase}}
}}}
\startsubsubsection{Serializable Fields}{
\begin{itemize}
\item{
private String squashString\begin{itemize}\item{\vskip -.9ex }\end{itemize}
}
\end{itemize}
}
\startsubsubsection{Constructors}{
\vskip -2em
\begin{itemize}
\item{\vskip -1.9ex 
\membername{Neurone}
{\tt public {\bf Neurone}(  )
\label{l1162}\label{l1163}}%end signature
}%end item
\end{itemize}
}
\startsubsubsection{Methods}{
\vskip -2em
\begin{itemize}
\item{\vskip -1.9ex 
\membername{charge}
{\tt public Neurone {\bf charge}( {\tt double } {\bf amt} )
\label{l1164}\label{l1165}}%end signature
}%end item
\divideents{getCharge}
\item{\vskip -1.9ex 
\membername{getCharge}
{\tt public double {\bf getCharge}(  )
\label{l1166}\label{l1167}}%end signature
}%end item
\divideents{getCurrentCharge}
\item{\vskip -1.9ex 
\membername{getCurrentCharge}
{\tt public Double {\bf getCurrentCharge}(  )
\label{l1168}\label{l1169}}%end signature
}%end item
\divideents{getEdgeDecoration}
\item{\vskip -1.9ex 
\membername{getEdgeDecoration}
{\tt public EdgeDecoration {\bf getEdgeDecoration}(  )
\label{l1170}\label{l1171}}%end signature
}%end item
\divideents{getFreshID}
\item{\vskip -1.9ex 
\membername{getFreshID}
{\tt public void {\bf getFreshID}(  )
\label{l1172}\label{l1173}}%end signature
}%end item
\divideents{getID}
\item{\vskip -1.9ex 
\membername{getID}
{\tt public int {\bf getID}(  )
\label{l1174}\label{l1175}}%end signature
}%end item
\divideents{getSquashFunction}
\item{\vskip -1.9ex 
\membername{getSquashFunction}
{\tt public ASTExpression {\bf getSquashFunction}(  )
\label{l1176}\label{l1177}}%end signature
}%end item
\divideents{getTrigger}
\item{\vskip -1.9ex 
\membername{getTrigger}
{\tt public double {\bf getTrigger}(  )
\label{l1178}\label{l1179}}%end signature
}%end item
\divideents{reset}
\item{\vskip -1.9ex 
\membername{reset}
{\tt public void {\bf reset}(  )
\label{l1180}\label{l1181}}%end signature
}%end item
\divideents{setCharge}
\item{\vskip -1.9ex 
\membername{setCharge}
{\tt public void {\bf setCharge}( {\tt double } {\bf charge} )
\label{l1182}\label{l1183}}%end signature
}%end item
\divideents{setEdgeDecoration}
\item{\vskip -1.9ex 
\membername{setEdgeDecoration}
{\tt public void {\bf setEdgeDecoration}( {\tt uk.ac.ic.doc.neuralnets.graph.neural.EdgeDecoration } {\bf ed} )
\label{l1184}\label{l1185}}%end signature
}%end item
\divideents{setID}
\item{\vskip -1.9ex 
\membername{setID}
{\tt public void {\bf setID}( {\tt int } {\bf id} )
\label{l1186}\label{l1187}}%end signature
}%end item
\divideents{setInitialCharge}
\item{\vskip -1.9ex 
\membername{setInitialCharge}
{\tt public void {\bf setInitialCharge}( {\tt uk.ac.ic.doc.neuralnets.expressions.ast.ASTExpression } {\bf c} )
\label{l1188}\label{l1189}}%end signature
}%end item
\divideents{setSquashFunction}
\item{\vskip -1.9ex 
\membername{setSquashFunction}
{\tt public void {\bf setSquashFunction}( {\tt uk.ac.ic.doc.neuralnets.expressions.ast.ASTExpression } {\bf e} )
\label{l1190}\label{l1191}}%end signature
}%end item
\divideents{setTrigger}
\item{\vskip -1.9ex 
\membername{setTrigger}
{\tt public void {\bf setTrigger}( {\tt uk.ac.ic.doc.neuralnets.expressions.ast.ASTExpression } {\bf t} )
\label{l1192}\label{l1193}}%end signature
}%end item
\divideents{setTrigger}
\item{\vskip -1.9ex 
\membername{setTrigger}
{\tt public void {\bf setTrigger}( {\tt double } {\bf d} )
\label{l1194}\label{l1195}}%end signature
}%end item
\divideents{tick}
\item{\vskip -1.9ex 
\membername{tick}
{\tt public Node {\bf tick}(  )
\label{l1196}\label{l1197}}%end signature
\begin{itemize}
\sld
\item{
\sld
{\bf Usage}
  \begin{itemize}\isep
   \item{
Ticks the neurone one step forward. Fires the neurone is appropriate.
}%end item
  \end{itemize}
}
\item{{\bf Returns} - 
Itself. 
}%end item
\end{itemize}
}%end item
\divideents{toString}
\item{\vskip -1.9ex 
\membername{toString}
{\tt public String {\bf toString}(  )
\label{l1198}\label{l1199}}%end signature
}%end item
\end{itemize}
}
\startsubsubsection{Methods inherited from class {\tt uk.ac.ic.doc.neuralnets.graph.neural.NodeBase}}{
\par{\small 
\refdefined{l514}\vskip -2em
\begin{itemize}
\item{\vskip -1.9ex 
\membername{connect}
{\tt public Node {\bf connect}( {\tt uk.ac.ic.doc.neuralnets.graph.Edge } {\bf e} )
}%end signature
\begin{itemize}
\sld
\item{
\sld
{\bf Usage}
  \begin{itemize}\isep
   \item{
Connect this node up with the input edge.
}%end item
  \end{itemize}
}
\end{itemize}
}%end item
\divideents{getIncoming}
\item{\vskip -1.9ex 
\membername{getIncoming}
{\tt public Collection {\bf getIncoming}(  )
}%end signature
\begin{itemize}
\sld
\item{
\sld
{\bf Usage}
  \begin{itemize}\isep
   \item{
Get incoming edges.
}%end item
  \end{itemize}
}
\end{itemize}
}%end item
\divideents{getMetadata}
\item{\vskip -1.9ex 
\membername{getMetadata}
{\tt public String {\bf getMetadata}( {\tt java.lang.String } {\bf key} )
}%end signature
\begin{itemize}
\sld
\item{
\sld
{\bf Usage}
  \begin{itemize}\isep
   \item{
Returns the meta data for the key input.
}%end item
  \end{itemize}
}
\item{
\sld
{\bf Parameters}
\sld\isep
  \begin{itemize}
\sld\isep
   \item{
\sld
{\tt key} - To look for.}
  \end{itemize}
}%end item
\item{{\bf Returns} - 
item Found. 
}%end item
\end{itemize}
}%end item
\divideents{getOutgoing}
\item{\vskip -1.9ex 
\membername{getOutgoing}
{\tt public Collection {\bf getOutgoing}(  )
}%end signature
\begin{itemize}
\sld
\item{
\sld
{\bf Usage}
  \begin{itemize}\isep
   \item{
Get outgoing edges.
}%end item
  \end{itemize}
}
\end{itemize}
}%end item
\divideents{getX}
\item{\vskip -1.9ex 
\membername{getX}
{\tt public int {\bf getX}(  )
}%end signature
\begin{itemize}
\sld
\item{
\sld
{\bf Usage}
  \begin{itemize}\isep
   \item{
Returns the position of the node on the x axis.
}%end item
  \end{itemize}
}
\item{{\bf Returns} - 
x axis position. 
}%end item
\end{itemize}
}%end item
\divideents{getY}
\item{\vskip -1.9ex 
\membername{getY}
{\tt public int {\bf getY}(  )
}%end signature
\begin{itemize}
\sld
\item{
\sld
{\bf Usage}
  \begin{itemize}\isep
   \item{
Returns the position of the node on the y axis.
}%end item
  \end{itemize}
}
\item{{\bf Returns} - 
y axis position. 
}%end item
\end{itemize}
}%end item
\divideents{getZ}
\item{\vskip -1.9ex 
\membername{getZ}
{\tt public int {\bf getZ}(  )
}%end signature
\begin{itemize}
\sld
\item{
\sld
{\bf Usage}
  \begin{itemize}\isep
   \item{
Returns the position of the node on the z axis.
}%end item
  \end{itemize}
}
\item{{\bf Returns} - 
z axis position. 
}%end item
\end{itemize}
}%end item
\divideents{setMetadata}
\item{\vskip -1.9ex 
\membername{setMetadata}
{\tt public Node {\bf setMetadata}( {\tt java.lang.String } {\bf key},
{\tt java.lang.String } {\bf item} )
}%end signature
\begin{itemize}
\sld
\item{
\sld
{\bf Usage}
  \begin{itemize}\isep
   \item{
Set meta data for the object.
}%end item
  \end{itemize}
}
\item{
\sld
{\bf Parameters}
\sld\isep
  \begin{itemize}
\sld\isep
   \item{
\sld
{\tt key} - String key}
   \item{
\sld
{\tt item} - String item}
  \end{itemize}
}%end item
\end{itemize}
}%end item
\divideents{setPos}
\item{\vskip -1.9ex 
\membername{setPos}
{\tt public void {\bf setPos}( {\tt int } {\bf x},
{\tt int } {\bf y},
{\tt int } {\bf z} )
}%end signature
\begin{itemize}
\sld
\item{
\sld
{\bf Usage}
  \begin{itemize}\isep
   \item{
Sets the position of the node.
}%end item
  \end{itemize}
}
\item{
\sld
{\bf Parameters}
\sld\isep
  \begin{itemize}
\sld\isep
   \item{
\sld
{\tt x} - Position on x axis.}
   \item{
\sld
{\tt y} - Position on y axis.}
   \item{
\sld
{\tt z} - Position on z axis.}
  \end{itemize}
}%end item
\end{itemize}
}%end item
\divideents{setX}
\item{\vskip -1.9ex 
\membername{setX}
{\tt public void {\bf setX}( {\tt int } {\bf x} )
}%end signature
\begin{itemize}
\sld
\item{
\sld
{\bf Usage}
  \begin{itemize}\isep
   \item{
Sets the position of the node on the x axis.
}%end item
  \end{itemize}
}
\item{
\sld
{\bf Parameters}
\sld\isep
  \begin{itemize}
\sld\isep
   \item{
\sld
{\tt x} - Position on x axis.}
  \end{itemize}
}%end item
\end{itemize}
}%end item
\divideents{setY}
\item{\vskip -1.9ex 
\membername{setY}
{\tt public void {\bf setY}( {\tt int } {\bf y} )
}%end signature
\begin{itemize}
\sld
\item{
\sld
{\bf Usage}
  \begin{itemize}\isep
   \item{
Sets the position of the node on the y axis.
}%end item
  \end{itemize}
}
\item{
\sld
{\bf Parameters}
\sld\isep
  \begin{itemize}
\sld\isep
   \item{
\sld
{\tt y} - Position on y axis.}
  \end{itemize}
}%end item
\end{itemize}
}%end item
\divideents{setZ}
\item{\vskip -1.9ex 
\membername{setZ}
{\tt public void {\bf setZ}( {\tt int } {\bf z} )
}%end signature
\begin{itemize}
\sld
\item{
\sld
{\bf Usage}
  \begin{itemize}\isep
   \item{
Sets the position of the node on the z axis.
}%end item
  \end{itemize}
}
\item{
\sld
{\bf Parameters}
\sld\isep
  \begin{itemize}
\sld\isep
   \item{
\sld
{\tt z} - Position on z axis.}
  \end{itemize}
}%end item
\end{itemize}
}%end item
\divideents{tick}
\item{\vskip -1.9ex 
\membername{tick}
{\tt public abstract Node {\bf tick}(  )
}%end signature
}%end item
\divideents{toString}
\item{\vskip -1.9ex 
\membername{toString}
{\tt public abstract String {\bf toString}(  )
}%end signature
}%end item
\end{itemize}
}}
}
\startsection{Class}{NeuroneTypeConfig}{l1069}{%
{\small Configurator to load Statisticians}
\vskip .1in 
\startsubsubsection{Declaration}{
\fbox{\vbox{
\hbox{\vbox{\small public 
class 
NeuroneTypeConfig}}
\noindent\hbox{\vbox{{\bf extends} java.lang.Object}}
\noindent\hbox{\vbox{{\bf implements} 
uk.ac.ic.doc.neuralnets.util.configuration.Configurator}}
}}}
\startsubsubsection{Constructors}{
\vskip -2em
\begin{itemize}
\item{\vskip -1.9ex 
\membername{NeuroneTypeConfig}
{\tt public {\bf NeuroneTypeConfig}(  )
\label{l1200}\label{l1201}}%end signature
}%end item
\end{itemize}
}
\startsubsubsection{Methods}{
\vskip -2em
\begin{itemize}
\item{\vskip -1.9ex 
\membername{commitConfiguration}
{\tt public void {\bf commitConfiguration}(  )
\label{l1202}\label{l1203}}%end signature
}%end item
\divideents{configure}
\item{\vskip -1.9ex 
\membername{configure}
{\tt public void {\bf configure}(  )
\label{l1204}\label{l1205}}%end signature
}%end item
\divideents{getName}
\item{\vskip -1.9ex 
\membername{getName}
{\tt public String {\bf getName}(  )
\label{l1206}\label{l1207}}%end signature
}%end item
\end{itemize}
}
}
\startsection{Class}{NeuroneTypes}{l1070}{%
{\small Container object for the Neurone Types created by NeuroneTypeConfig}
\vskip .1in 
\startsubsubsection{Declaration}{
\fbox{\vbox{
\hbox{\vbox{\small public 
class 
NeuroneTypes}}
\noindent\hbox{\vbox{{\bf extends} java.lang.Object}}
}}}
\startsubsubsection{Fields}{
\begin{itemize}
\item{
public static final String EDGE\_DECORATION\_NAME\begin{itemize}\item{\vskip -.9ex Magic keyword for edge decoration}\end{itemize}
}
\item{
public static final Map nodeTypes\begin{itemize}\item{\vskip -.9ex Map from node type name to class}\end{itemize}
}
\item{
public static final Map nodeDecorations\begin{itemize}\item{\vskip -.9ex Map from type name to edge decoration}\end{itemize}
}
\item{
public static final Map nodeParams\begin{itemize}\item{\vskip -.9ex Map from type name to list of the parameters}\end{itemize}
}
\item{
public static final Map paramValues\begin{itemize}\item{\vskip -.9ex Map from type name to list of the default parameter values}\end{itemize}
}
\end{itemize}
}
\startsubsubsection{Constructors}{
\vskip -2em
\begin{itemize}
\item{\vskip -1.9ex 
\membername{NeuroneTypes}
{\tt public {\bf NeuroneTypes}(  )
\label{l1208}\label{l1209}}%end signature
}%end item
\end{itemize}
}
\startsubsubsection{Methods}{
\vskip -2em
\begin{itemize}
\item{\vskip -1.9ex 
\membername{specFor}
{\tt public static NodeSpecification {\bf specFor}( {\tt java.lang.String } {\bf name} )
\label{l1210}\label{l1211}}%end signature
\begin{itemize}
\sld
\item{
\sld
{\bf Usage}
  \begin{itemize}\isep
   \item{
Build a NodeSpecification for the specified Neurone type
}%end item
  \end{itemize}
}
\item{
\sld
{\bf Parameters}
\sld\isep
  \begin{itemize}
\sld\isep
   \item{
\sld
{\tt name} - The name of the Neurone (assumed to exist in nodeTypes)}
  \end{itemize}
}%end item
\item{{\bf Returns} - 
The NodeSpecification for the given Neurone type 
}%end item
\end{itemize}
}%end item
\end{itemize}
}
}
\startsection{Class}{NewNeuroneTypeEvent}{l1071}{%
{\small Indicates a new neurone type has been created}
\vskip .1in 
\startsubsubsection{Declaration}{
\fbox{\vbox{
\hbox{\vbox{\small public 
class 
NewNeuroneTypeEvent}}
\noindent\hbox{\vbox{{\bf extends} uk.ac.ic.doc.neuralnets.events.Event}}
}}}
\startsubsubsection{Constructors}{
\vskip -2em
\begin{itemize}
\item{\vskip -1.9ex 
\membername{NewNeuroneTypeEvent}
{\tt public {\bf NewNeuroneTypeEvent}( {\tt java.lang.String } {\bf name} )
\label{l1212}\label{l1213}}%end signature
}%end item
\end{itemize}
}
\startsubsubsection{Methods}{
\vskip -2em
\begin{itemize}
\item{\vskip -1.9ex 
\membername{getName}
{\tt public String {\bf getName}(  )
\label{l1214}\label{l1215}}%end signature
}%end item
\divideents{toString}
\item{\vskip -1.9ex 
\membername{toString}
{\tt public String {\bf toString}(  )
\label{l1216}\label{l1217}}%end signature
}%end item
\end{itemize}
}
\startsubsubsection{Methods inherited from class {\tt uk.ac.ic.doc.neuralnets.events.Event}}{
\par{\small 
\refdefined{l220}\vskip -2em
\begin{itemize}
\item{\vskip -1.9ex 
\membername{toString}
{\tt public abstract String {\bf toString}(  )
}%end signature
}%end item
\end{itemize}
}}
}
\startsection{Class}{NodeBase}{l514}{%
{\small Basic Node implementation; should suffice for most Node purposes}
\vskip .1in 
\startsubsubsection{Declaration}{
\fbox{\vbox{
\hbox{\vbox{\small public abstract 
class 
NodeBase}}
\noindent\hbox{\vbox{{\bf extends} java.lang.Object}}
\noindent\hbox{\vbox{{\bf implements} 
uk.ac.ic.doc.neuralnets.graph.Node}}
}}}
\startsubsubsection{Serializable Fields}{
\begin{itemize}
\item{
private Map metadata\begin{itemize}\item{\vskip -.9ex }\end{itemize}
}
\item{
private int xpos\begin{itemize}\item{\vskip -.9ex }\end{itemize}
}
\item{
private int ypos\begin{itemize}\item{\vskip -.9ex }\end{itemize}
}
\item{
private int zpos\begin{itemize}\item{\vskip -.9ex }\end{itemize}
}
\end{itemize}
}
\startsubsubsection{Constructors}{
\vskip -2em
\begin{itemize}
\item{\vskip -1.9ex 
\membername{NodeBase}
{\tt protected {\bf NodeBase}(  )
\label{l1218}\label{l1219}}%end signature
}%end item
\divideents{NodeBase}
\item{\vskip -1.9ex 
\membername{NodeBase}
{\tt protected {\bf NodeBase}( {\tt java.util.Set } {\bf in},
{\tt java.util.Set } {\bf out} )
\label{l1220}\label{l1221}}%end signature
}%end item
\end{itemize}
}
\startsubsubsection{Methods}{
\vskip -2em
\begin{itemize}
\item{\vskip -1.9ex 
\membername{connect}
{\tt public Node {\bf connect}( {\tt uk.ac.ic.doc.neuralnets.graph.Edge } {\bf e} )
\label{l1222}\label{l1223}}%end signature
\begin{itemize}
\sld
\item{
\sld
{\bf Usage}
  \begin{itemize}\isep
   \item{
Connect this node up with the input edge.
}%end item
  \end{itemize}
}
\end{itemize}
}%end item
\divideents{getIncoming}
\item{\vskip -1.9ex 
\membername{getIncoming}
{\tt public Collection {\bf getIncoming}(  )
\label{l1224}\label{l1225}}%end signature
\begin{itemize}
\sld
\item{
\sld
{\bf Usage}
  \begin{itemize}\isep
   \item{
Get incoming edges.
}%end item
  \end{itemize}
}
\end{itemize}
}%end item
\divideents{getMetadata}
\item{\vskip -1.9ex 
\membername{getMetadata}
{\tt public String {\bf getMetadata}( {\tt java.lang.String } {\bf key} )
\label{l1226}\label{l1227}}%end signature
\begin{itemize}
\sld
\item{
\sld
{\bf Usage}
  \begin{itemize}\isep
   \item{
Returns the meta data for the key input.
}%end item
  \end{itemize}
}
\item{
\sld
{\bf Parameters}
\sld\isep
  \begin{itemize}
\sld\isep
   \item{
\sld
{\tt key} - To look for.}
  \end{itemize}
}%end item
\item{{\bf Returns} - 
item Found. 
}%end item
\end{itemize}
}%end item
\divideents{getOutgoing}
\item{\vskip -1.9ex 
\membername{getOutgoing}
{\tt public Collection {\bf getOutgoing}(  )
\label{l1228}\label{l1229}}%end signature
\begin{itemize}
\sld
\item{
\sld
{\bf Usage}
  \begin{itemize}\isep
   \item{
Get outgoing edges.
}%end item
  \end{itemize}
}
\end{itemize}
}%end item
\divideents{getX}
\item{\vskip -1.9ex 
\membername{getX}
{\tt public int {\bf getX}(  )
\label{l1230}\label{l1231}}%end signature
\begin{itemize}
\sld
\item{
\sld
{\bf Usage}
  \begin{itemize}\isep
   \item{
Returns the position of the node on the x axis.
}%end item
  \end{itemize}
}
\item{{\bf Returns} - 
x axis position. 
}%end item
\end{itemize}
}%end item
\divideents{getY}
\item{\vskip -1.9ex 
\membername{getY}
{\tt public int {\bf getY}(  )
\label{l1232}\label{l1233}}%end signature
\begin{itemize}
\sld
\item{
\sld
{\bf Usage}
  \begin{itemize}\isep
   \item{
Returns the position of the node on the y axis.
}%end item
  \end{itemize}
}
\item{{\bf Returns} - 
y axis position. 
}%end item
\end{itemize}
}%end item
\divideents{getZ}
\item{\vskip -1.9ex 
\membername{getZ}
{\tt public int {\bf getZ}(  )
\label{l1234}\label{l1235}}%end signature
\begin{itemize}
\sld
\item{
\sld
{\bf Usage}
  \begin{itemize}\isep
   \item{
Returns the position of the node on the z axis.
}%end item
  \end{itemize}
}
\item{{\bf Returns} - 
z axis position. 
}%end item
\end{itemize}
}%end item
\divideents{setMetadata}
\item{\vskip -1.9ex 
\membername{setMetadata}
{\tt public Node {\bf setMetadata}( {\tt java.lang.String } {\bf key},
{\tt java.lang.String } {\bf item} )
\label{l1236}\label{l1237}}%end signature
\begin{itemize}
\sld
\item{
\sld
{\bf Usage}
  \begin{itemize}\isep
   \item{
Set meta data for the object.
}%end item
  \end{itemize}
}
\item{
\sld
{\bf Parameters}
\sld\isep
  \begin{itemize}
\sld\isep
   \item{
\sld
{\tt key} - String key}
   \item{
\sld
{\tt item} - String item}
  \end{itemize}
}%end item
\end{itemize}
}%end item
\divideents{setPos}
\item{\vskip -1.9ex 
\membername{setPos}
{\tt public void {\bf setPos}( {\tt int } {\bf x},
{\tt int } {\bf y},
{\tt int } {\bf z} )
\label{l1238}\label{l1239}}%end signature
\begin{itemize}
\sld
\item{
\sld
{\bf Usage}
  \begin{itemize}\isep
   \item{
Sets the position of the node.
}%end item
  \end{itemize}
}
\item{
\sld
{\bf Parameters}
\sld\isep
  \begin{itemize}
\sld\isep
   \item{
\sld
{\tt x} - Position on x axis.}
   \item{
\sld
{\tt y} - Position on y axis.}
   \item{
\sld
{\tt z} - Position on z axis.}
  \end{itemize}
}%end item
\end{itemize}
}%end item
\divideents{setX}
\item{\vskip -1.9ex 
\membername{setX}
{\tt public void {\bf setX}( {\tt int } {\bf x} )
\label{l1240}\label{l1241}}%end signature
\begin{itemize}
\sld
\item{
\sld
{\bf Usage}
  \begin{itemize}\isep
   \item{
Sets the position of the node on the x axis.
}%end item
  \end{itemize}
}
\item{
\sld
{\bf Parameters}
\sld\isep
  \begin{itemize}
\sld\isep
   \item{
\sld
{\tt x} - Position on x axis.}
  \end{itemize}
}%end item
\end{itemize}
}%end item
\divideents{setY}
\item{\vskip -1.9ex 
\membername{setY}
{\tt public void {\bf setY}( {\tt int } {\bf y} )
\label{l1242}\label{l1243}}%end signature
\begin{itemize}
\sld
\item{
\sld
{\bf Usage}
  \begin{itemize}\isep
   \item{
Sets the position of the node on the y axis.
}%end item
  \end{itemize}
}
\item{
\sld
{\bf Parameters}
\sld\isep
  \begin{itemize}
\sld\isep
   \item{
\sld
{\tt y} - Position on y axis.}
  \end{itemize}
}%end item
\end{itemize}
}%end item
\divideents{setZ}
\item{\vskip -1.9ex 
\membername{setZ}
{\tt public void {\bf setZ}( {\tt int } {\bf z} )
\label{l1244}\label{l1245}}%end signature
\begin{itemize}
\sld
\item{
\sld
{\bf Usage}
  \begin{itemize}\isep
   \item{
Sets the position of the node on the z axis.
}%end item
  \end{itemize}
}
\item{
\sld
{\bf Parameters}
\sld\isep
  \begin{itemize}
\sld\isep
   \item{
\sld
{\tt z} - Position on z axis.}
  \end{itemize}
}%end item
\end{itemize}
}%end item
\divideents{tick}
\item{\vskip -1.9ex 
\membername{tick}
{\tt public abstract Node {\bf tick}(  )
\label{l1246}\label{l1247}}%end signature
}%end item
\divideents{toString}
\item{\vskip -1.9ex 
\membername{toString}
{\tt public abstract String {\bf toString}(  )
\label{l1248}\label{l1249}}%end signature
}%end item
\end{itemize}
}
}
\startsection{Class}{NodeChargeUpdateEvent}{l1072}{%
\startsubsubsection{Declaration}{
\fbox{\vbox{
\hbox{\vbox{\small public 
class 
NodeChargeUpdateEvent}}
\noindent\hbox{\vbox{{\bf extends} uk.ac.ic.doc.neuralnets.events.SingletonEvent}}
}}}
\startsubsubsection{Constructors}{
\vskip -2em
\begin{itemize}
\item{\vskip -1.9ex 
\membername{NodeChargeUpdateEvent}
{\tt public {\bf NodeChargeUpdateEvent}( {\tt uk.ac.ic.doc.neuralnets.graph.neural.Neurone } {\bf n} )
\label{l1250}\label{l1251}}%end signature
}%end item
\end{itemize}
}
\startsubsubsection{Methods}{
\vskip -2em
\begin{itemize}
\item{\vskip -1.9ex 
\membername{equals}
{\tt public boolean {\bf equals}( {\tt java.lang.Object } {\bf o} )
\label{l1252}\label{l1253}}%end signature
}%end item
\divideents{getNeurone}
\item{\vskip -1.9ex 
\membername{getNeurone}
{\tt public Neurone {\bf getNeurone}(  )
\label{l1254}\label{l1255}}%end signature
}%end item
\divideents{toString}
\item{\vskip -1.9ex 
\membername{toString}
{\tt public String {\bf toString}(  )
\label{l1256}\label{l1257}}%end signature
}%end item
\end{itemize}
}
\startsubsubsection{Methods inherited from class {\tt uk.ac.ic.doc.neuralnets.events.SingletonEvent}}{
\par{\small 
\refdefined{l1258}\vskip -2em
\begin{itemize}
\item{\vskip -1.9ex 
\membername{equals}
{\tt public abstract boolean {\bf equals}( {\tt java.lang.Object } {\bf o} )
}%end signature
}%end item
\end{itemize}
}}
\startsubsubsection{Methods inherited from class {\tt uk.ac.ic.doc.neuralnets.events.Event}}{
\par{\small 
\refdefined{l220}\vskip -2em
\begin{itemize}
\item{\vskip -1.9ex 
\membername{toString}
{\tt public abstract String {\bf toString}(  )
}%end signature
}%end item
\end{itemize}
}}
}
\startsection{Class}{NodeFired}{l1073}{%
\startsubsubsection{Declaration}{
\fbox{\vbox{
\hbox{\vbox{\small public 
class 
NodeFired}}
\noindent\hbox{\vbox{{\bf extends} uk.ac.ic.doc.neuralnets.events.NumericalEvent}}
}}}
\startsubsubsection{Constructors}{
\vskip -2em
\begin{itemize}
\item{\vskip -1.9ex 
\membername{NodeFired}
{\tt public {\bf NodeFired}( {\tt uk.ac.ic.doc.neuralnets.graph.Node } {\bf node},
{\tt int } {\bf tick} )
\label{l1259}\label{l1260}}%end signature
}%end item
\end{itemize}
}
\startsubsubsection{Methods}{
\vskip -2em
\begin{itemize}
\item{\vskip -1.9ex 
\membername{get}
{\tt public double {\bf get}( {\tt int } {\bf idx} )
\label{l1261}\label{l1262}}%end signature
}%end item
\divideents{getNode}
\item{\vskip -1.9ex 
\membername{getNode}
{\tt public Node {\bf getNode}(  )
\label{l1263}\label{l1264}}%end signature
}%end item
\divideents{getTick}
\item{\vskip -1.9ex 
\membername{getTick}
{\tt public int {\bf getTick}(  )
\label{l1265}\label{l1266}}%end signature
}%end item
\divideents{numPoints}
\item{\vskip -1.9ex 
\membername{numPoints}
{\tt public double {\bf numPoints}(  )
\label{l1267}\label{l1268}}%end signature
}%end item
\divideents{push}
\item{\vskip -1.9ex 
\membername{push}
{\tt public void {\bf push}( {\tt uk.ac.ic.doc.neuralnets.events.NumericalStatistician } {\bf s} )
\label{l1269}\label{l1270}}%end signature
}%end item
\divideents{toString}
\item{\vskip -1.9ex 
\membername{toString}
{\tt public String {\bf toString}(  )
\label{l1271}\label{l1272}}%end signature
}%end item
\end{itemize}
}
\startsubsubsection{Methods inherited from class {\tt uk.ac.ic.doc.neuralnets.events.NumericalEvent}}{
\par{\small 
\refdefined{l1273}\vskip -2em
\begin{itemize}
\item{\vskip -1.9ex 
\membername{get}
{\tt public abstract double {\bf get}( {\tt int } {\bf idx} )
}%end signature
}%end item
\divideents{numPoints}
\item{\vskip -1.9ex 
\membername{numPoints}
{\tt public abstract double {\bf numPoints}(  )
}%end signature
}%end item
\divideents{push}
\item{\vskip -1.9ex 
\membername{push}
{\tt public abstract void {\bf push}( {\tt uk.ac.ic.doc.neuralnets.events.NumericalStatistician } {\bf s} )
}%end signature
}%end item
\end{itemize}
}}
\startsubsubsection{Methods inherited from class {\tt uk.ac.ic.doc.neuralnets.events.Event}}{
\par{\small 
\refdefined{l220}\vskip -2em
\begin{itemize}
\item{\vskip -1.9ex 
\membername{toString}
{\tt public abstract String {\bf toString}(  )
}%end signature
}%end item
\end{itemize}
}}
}
\startsection{Class}{NodeSpecification}{l283}{%
{\small Default NodeSpecification}
\vskip .1in 
\startsubsubsection{Declaration}{
\fbox{\vbox{
\hbox{\vbox{\small public 
class 
NodeSpecification}}
\noindent\hbox{\vbox{{\bf extends} java.lang.Object}}
\noindent\hbox{\vbox{{\bf implements} 
java.io.Serializable}}
}}}
\startsubsubsection{Serializable Fields}{
\begin{itemize}
\item{
private Map parameters\begin{itemize}\item{\vskip -.9ex }\end{itemize}
}
\item{
private Class target\begin{itemize}\item{\vskip -.9ex }\end{itemize}
}
\item{
private EdgeDecoration ed\begin{itemize}\item{\vskip -.9ex }\end{itemize}
}
\item{
private String name\begin{itemize}\item{\vskip -.9ex }\end{itemize}
}
\end{itemize}
}
\startsubsubsection{Constructors}{
\vskip -2em
\begin{itemize}
\item{\vskip -1.9ex 
\membername{NodeSpecification}
{\tt public {\bf NodeSpecification}(  )
\label{l1274}\label{l1275}}%end signature
}%end item
\divideents{NodeSpecification}
\item{\vskip -1.9ex 
\membername{NodeSpecification}
{\tt public {\bf NodeSpecification}( {\tt java.lang.Class } {\bf target} )
\label{l1276}\label{l1277}}%end signature
}%end item
\end{itemize}
}
\startsubsubsection{Methods}{
\vskip -2em
\begin{itemize}
\item{\vskip -1.9ex 
\membername{get}
{\tt public ASTExpression {\bf get}( {\tt java.lang.String } {\bf param} )
\label{l1278}\label{l1279}}%end signature
\begin{itemize}
\sld
\item{
\sld
{\bf Usage}
  \begin{itemize}\isep
   \item{
Get the AST expression for input parameter.
}%end item
  \end{itemize}
}
\item{
\sld
{\bf Parameters}
\sld\isep
  \begin{itemize}
\sld\isep
   \item{
\sld
{\tt param} - String}
  \end{itemize}
}%end item
\item{{\bf Returns} - 
AST expression 
}%end item
\end{itemize}
}%end item
\divideents{getEdgeDecoration}
\item{\vskip -1.9ex 
\membername{getEdgeDecoration}
{\tt public EdgeDecoration {\bf getEdgeDecoration}(  )
\label{l1280}\label{l1281}}%end signature
\begin{itemize}
\sld
\item{
\sld
{\bf Usage}
  \begin{itemize}\isep
   \item{
Get the edge decoration for the node specification.
}%end item
  \end{itemize}
}
\item{{\bf Returns} - 
The edge decoration. 
}%end item
\end{itemize}
}%end item
\divideents{getName}
\item{\vskip -1.9ex 
\membername{getName}
{\tt public String {\bf getName}(  )
\label{l1282}\label{l1283}}%end signature
\begin{itemize}
\sld
\item{
\sld
{\bf Usage}
  \begin{itemize}\isep
   \item{
Get the name of the node specification.
}%end item
  \end{itemize}
}
\item{{\bf Returns} - 
The name. 
}%end item
\end{itemize}
}%end item
\divideents{getParameters}
\item{\vskip -1.9ex 
\membername{getParameters}
{\tt public Set {\bf getParameters}(  )
\label{l1284}\label{l1285}}%end signature
\begin{itemize}
\sld
\item{
\sld
{\bf Usage}
  \begin{itemize}\isep
   \item{
Get the parameter key set.
}%end item
  \end{itemize}
}
\item{{\bf Returns} - 
Parameter key set. 
}%end item
\end{itemize}
}%end item
\divideents{getTarget}
\item{\vskip -1.9ex 
\membername{getTarget}
{\tt public Class {\bf getTarget}(  )
\label{l1286}\label{l1287}}%end signature
\begin{itemize}
\sld
\item{
\sld
{\bf Usage}
  \begin{itemize}\isep
   \item{
Get target of node specification.
}%end item
  \end{itemize}
}
\item{{\bf Returns} - 
Target 
}%end item
\end{itemize}
}%end item
\divideents{set}
\item{\vskip -1.9ex 
\membername{set}
{\tt public NodeSpecification {\bf set}( {\tt java.lang.String } {\bf param},
{\tt uk.ac.ic.doc.neuralnets.expressions.ast.ASTExpression } {\bf target} )
\label{l1288}\label{l1289}}%end signature
\begin{itemize}
\sld
\item{
\sld
{\bf Usage}
  \begin{itemize}\isep
   \item{
Set a parameter to an AST expresion.
}%end item
  \end{itemize}
}
\item{
\sld
{\bf Parameters}
\sld\isep
  \begin{itemize}
\sld\isep
   \item{
\sld
{\tt param} - Parameter name}
   \item{
\sld
{\tt target} - AST expression value.}
  \end{itemize}
}%end item
\item{{\bf Returns} - 
Itself. 
}%end item
\end{itemize}
}%end item
\divideents{setEdgeDecoration}
\item{\vskip -1.9ex 
\membername{setEdgeDecoration}
{\tt public void {\bf setEdgeDecoration}( {\tt uk.ac.ic.doc.neuralnets.graph.neural.EdgeDecoration } {\bf ed} )
\label{l1290}\label{l1291}}%end signature
\begin{itemize}
\sld
\item{
\sld
{\bf Usage}
  \begin{itemize}\isep
   \item{
Set the edge decorator for the node specification.
}%end item
  \end{itemize}
}
\item{
\sld
{\bf Parameters}
\sld\isep
  \begin{itemize}
\sld\isep
   \item{
\sld
{\tt ed} - The edge decoration.}
  \end{itemize}
}%end item
\end{itemize}
}%end item
\divideents{setName}
\item{\vskip -1.9ex 
\membername{setName}
{\tt public void {\bf setName}( {\tt java.lang.String } {\bf n} )
\label{l1292}\label{l1293}}%end signature
\begin{itemize}
\sld
\item{
\sld
{\bf Usage}
  \begin{itemize}\isep
   \item{
Set name of node specification.
}%end item
  \end{itemize}
}
\item{
\sld
{\bf Parameters}
\sld\isep
  \begin{itemize}
\sld\isep
   \item{
\sld
{\tt n} - Name}
  \end{itemize}
}%end item
\end{itemize}
}%end item
\end{itemize}
}
}
\startsection{Class}{Perceptron}{l1074}{%
\startsubsubsection{Declaration}{
\fbox{\vbox{
\hbox{\vbox{\small public 
class 
Perceptron}}
\noindent\hbox{\vbox{{\bf extends} uk.ac.ic.doc.neuralnets.graph.neural.Neurone}}
}}}
\startsubsubsection{Constructors}{
\vskip -2em
\begin{itemize}
\item{\vskip -1.9ex 
\membername{Perceptron}
{\tt public {\bf Perceptron}(  )
\label{l1294}\label{l1295}}%end signature
}%end item
\end{itemize}
}
\startsubsubsection{Methods}{
\vskip -2em
\begin{itemize}
\item{\vskip -1.9ex 
\membername{getCharge}
{\tt public double {\bf getCharge}(  )
\label{l1296}\label{l1297}}%end signature
}%end item
\divideents{tick}
\item{\vskip -1.9ex 
\membername{tick}
{\tt public Node {\bf tick}(  )
\label{l1298}\label{l1299}}%end signature
}%end item
\divideents{toString}
\item{\vskip -1.9ex 
\membername{toString}
{\tt public String {\bf toString}(  )
\label{l1300}\label{l1301}}%end signature
}%end item
\end{itemize}
}
\startsubsubsection{Methods inherited from class {\tt uk.ac.ic.doc.neuralnets.graph.neural.Neurone}}{
\par{\small 
\refdefined{l225}\vskip -2em
\begin{itemize}
\item{\vskip -1.9ex 
\membername{charge}
{\tt public Neurone {\bf charge}( {\tt double } {\bf amt} )
}%end signature
}%end item
\divideents{getCharge}
\item{\vskip -1.9ex 
\membername{getCharge}
{\tt public double {\bf getCharge}(  )
}%end signature
}%end item
\divideents{getCurrentCharge}
\item{\vskip -1.9ex 
\membername{getCurrentCharge}
{\tt public Double {\bf getCurrentCharge}(  )
}%end signature
}%end item
\divideents{getEdgeDecoration}
\item{\vskip -1.9ex 
\membername{getEdgeDecoration}
{\tt public EdgeDecoration {\bf getEdgeDecoration}(  )
}%end signature
}%end item
\divideents{getFreshID}
\item{\vskip -1.9ex 
\membername{getFreshID}
{\tt public void {\bf getFreshID}(  )
}%end signature
}%end item
\divideents{getID}
\item{\vskip -1.9ex 
\membername{getID}
{\tt public int {\bf getID}(  )
}%end signature
}%end item
\divideents{getSquashFunction}
\item{\vskip -1.9ex 
\membername{getSquashFunction}
{\tt public ASTExpression {\bf getSquashFunction}(  )
}%end signature
}%end item
\divideents{getTrigger}
\item{\vskip -1.9ex 
\membername{getTrigger}
{\tt public double {\bf getTrigger}(  )
}%end signature
}%end item
\divideents{reset}
\item{\vskip -1.9ex 
\membername{reset}
{\tt public void {\bf reset}(  )
}%end signature
}%end item
\divideents{setCharge}
\item{\vskip -1.9ex 
\membername{setCharge}
{\tt public void {\bf setCharge}( {\tt double } {\bf charge} )
}%end signature
}%end item
\divideents{setEdgeDecoration}
\item{\vskip -1.9ex 
\membername{setEdgeDecoration}
{\tt public void {\bf setEdgeDecoration}( {\tt uk.ac.ic.doc.neuralnets.graph.neural.EdgeDecoration } {\bf ed} )
}%end signature
}%end item
\divideents{setID}
\item{\vskip -1.9ex 
\membername{setID}
{\tt public void {\bf setID}( {\tt int } {\bf id} )
}%end signature
}%end item
\divideents{setInitialCharge}
\item{\vskip -1.9ex 
\membername{setInitialCharge}
{\tt public void {\bf setInitialCharge}( {\tt uk.ac.ic.doc.neuralnets.expressions.ast.ASTExpression } {\bf c} )
}%end signature
}%end item
\divideents{setSquashFunction}
\item{\vskip -1.9ex 
\membername{setSquashFunction}
{\tt public void {\bf setSquashFunction}( {\tt uk.ac.ic.doc.neuralnets.expressions.ast.ASTExpression } {\bf e} )
}%end signature
}%end item
\divideents{setTrigger}
\item{\vskip -1.9ex 
\membername{setTrigger}
{\tt public void {\bf setTrigger}( {\tt uk.ac.ic.doc.neuralnets.expressions.ast.ASTExpression } {\bf t} )
}%end signature
}%end item
\divideents{setTrigger}
\item{\vskip -1.9ex 
\membername{setTrigger}
{\tt public void {\bf setTrigger}( {\tt double } {\bf d} )
}%end signature
}%end item
\divideents{tick}
\item{\vskip -1.9ex 
\membername{tick}
{\tt public Node {\bf tick}(  )
}%end signature
\begin{itemize}
\sld
\item{
\sld
{\bf Usage}
  \begin{itemize}\isep
   \item{
Ticks the neurone one step forward. Fires the neurone is appropriate.
}%end item
  \end{itemize}
}
\item{{\bf Returns} - 
Itself. 
}%end item
\end{itemize}
}%end item
\divideents{toString}
\item{\vskip -1.9ex 
\membername{toString}
{\tt public String {\bf toString}(  )
}%end signature
}%end item
\end{itemize}
}}
\startsubsubsection{Methods inherited from class {\tt uk.ac.ic.doc.neuralnets.graph.neural.NodeBase}}{
\par{\small 
\refdefined{l514}\vskip -2em
\begin{itemize}
\item{\vskip -1.9ex 
\membername{connect}
{\tt public Node {\bf connect}( {\tt uk.ac.ic.doc.neuralnets.graph.Edge } {\bf e} )
}%end signature
\begin{itemize}
\sld
\item{
\sld
{\bf Usage}
  \begin{itemize}\isep
   \item{
Connect this node up with the input edge.
}%end item
  \end{itemize}
}
\end{itemize}
}%end item
\divideents{getIncoming}
\item{\vskip -1.9ex 
\membername{getIncoming}
{\tt public Collection {\bf getIncoming}(  )
}%end signature
\begin{itemize}
\sld
\item{
\sld
{\bf Usage}
  \begin{itemize}\isep
   \item{
Get incoming edges.
}%end item
  \end{itemize}
}
\end{itemize}
}%end item
\divideents{getMetadata}
\item{\vskip -1.9ex 
\membername{getMetadata}
{\tt public String {\bf getMetadata}( {\tt java.lang.String } {\bf key} )
}%end signature
\begin{itemize}
\sld
\item{
\sld
{\bf Usage}
  \begin{itemize}\isep
   \item{
Returns the meta data for the key input.
}%end item
  \end{itemize}
}
\item{
\sld
{\bf Parameters}
\sld\isep
  \begin{itemize}
\sld\isep
   \item{
\sld
{\tt key} - To look for.}
  \end{itemize}
}%end item
\item{{\bf Returns} - 
item Found. 
}%end item
\end{itemize}
}%end item
\divideents{getOutgoing}
\item{\vskip -1.9ex 
\membername{getOutgoing}
{\tt public Collection {\bf getOutgoing}(  )
}%end signature
\begin{itemize}
\sld
\item{
\sld
{\bf Usage}
  \begin{itemize}\isep
   \item{
Get outgoing edges.
}%end item
  \end{itemize}
}
\end{itemize}
}%end item
\divideents{getX}
\item{\vskip -1.9ex 
\membername{getX}
{\tt public int {\bf getX}(  )
}%end signature
\begin{itemize}
\sld
\item{
\sld
{\bf Usage}
  \begin{itemize}\isep
   \item{
Returns the position of the node on the x axis.
}%end item
  \end{itemize}
}
\item{{\bf Returns} - 
x axis position. 
}%end item
\end{itemize}
}%end item
\divideents{getY}
\item{\vskip -1.9ex 
\membername{getY}
{\tt public int {\bf getY}(  )
}%end signature
\begin{itemize}
\sld
\item{
\sld
{\bf Usage}
  \begin{itemize}\isep
   \item{
Returns the position of the node on the y axis.
}%end item
  \end{itemize}
}
\item{{\bf Returns} - 
y axis position. 
}%end item
\end{itemize}
}%end item
\divideents{getZ}
\item{\vskip -1.9ex 
\membername{getZ}
{\tt public int {\bf getZ}(  )
}%end signature
\begin{itemize}
\sld
\item{
\sld
{\bf Usage}
  \begin{itemize}\isep
   \item{
Returns the position of the node on the z axis.
}%end item
  \end{itemize}
}
\item{{\bf Returns} - 
z axis position. 
}%end item
\end{itemize}
}%end item
\divideents{setMetadata}
\item{\vskip -1.9ex 
\membername{setMetadata}
{\tt public Node {\bf setMetadata}( {\tt java.lang.String } {\bf key},
{\tt java.lang.String } {\bf item} )
}%end signature
\begin{itemize}
\sld
\item{
\sld
{\bf Usage}
  \begin{itemize}\isep
   \item{
Set meta data for the object.
}%end item
  \end{itemize}
}
\item{
\sld
{\bf Parameters}
\sld\isep
  \begin{itemize}
\sld\isep
   \item{
\sld
{\tt key} - String key}
   \item{
\sld
{\tt item} - String item}
  \end{itemize}
}%end item
\end{itemize}
}%end item
\divideents{setPos}
\item{\vskip -1.9ex 
\membername{setPos}
{\tt public void {\bf setPos}( {\tt int } {\bf x},
{\tt int } {\bf y},
{\tt int } {\bf z} )
}%end signature
\begin{itemize}
\sld
\item{
\sld
{\bf Usage}
  \begin{itemize}\isep
   \item{
Sets the position of the node.
}%end item
  \end{itemize}
}
\item{
\sld
{\bf Parameters}
\sld\isep
  \begin{itemize}
\sld\isep
   \item{
\sld
{\tt x} - Position on x axis.}
   \item{
\sld
{\tt y} - Position on y axis.}
   \item{
\sld
{\tt z} - Position on z axis.}
  \end{itemize}
}%end item
\end{itemize}
}%end item
\divideents{setX}
\item{\vskip -1.9ex 
\membername{setX}
{\tt public void {\bf setX}( {\tt int } {\bf x} )
}%end signature
\begin{itemize}
\sld
\item{
\sld
{\bf Usage}
  \begin{itemize}\isep
   \item{
Sets the position of the node on the x axis.
}%end item
  \end{itemize}
}
\item{
\sld
{\bf Parameters}
\sld\isep
  \begin{itemize}
\sld\isep
   \item{
\sld
{\tt x} - Position on x axis.}
  \end{itemize}
}%end item
\end{itemize}
}%end item
\divideents{setY}
\item{\vskip -1.9ex 
\membername{setY}
{\tt public void {\bf setY}( {\tt int } {\bf y} )
}%end signature
\begin{itemize}
\sld
\item{
\sld
{\bf Usage}
  \begin{itemize}\isep
   \item{
Sets the position of the node on the y axis.
}%end item
  \end{itemize}
}
\item{
\sld
{\bf Parameters}
\sld\isep
  \begin{itemize}
\sld\isep
   \item{
\sld
{\tt y} - Position on y axis.}
  \end{itemize}
}%end item
\end{itemize}
}%end item
\divideents{setZ}
\item{\vskip -1.9ex 
\membername{setZ}
{\tt public void {\bf setZ}( {\tt int } {\bf z} )
}%end signature
\begin{itemize}
\sld
\item{
\sld
{\bf Usage}
  \begin{itemize}\isep
   \item{
Sets the position of the node on the z axis.
}%end item
  \end{itemize}
}
\item{
\sld
{\bf Parameters}
\sld\isep
  \begin{itemize}
\sld\isep
   \item{
\sld
{\tt z} - Position on z axis.}
  \end{itemize}
}%end item
\end{itemize}
}%end item
\divideents{tick}
\item{\vskip -1.9ex 
\membername{tick}
{\tt public abstract Node {\bf tick}(  )
}%end signature
}%end item
\divideents{toString}
\item{\vskip -1.9ex 
\membername{toString}
{\tt public abstract String {\bf toString}(  )
}%end signature
}%end item
\end{itemize}
}}
}
\startsection{Class}{SpikingNeurone}{l1075}{%
\startsubsubsection{Declaration}{
\fbox{\vbox{
\hbox{\vbox{\small public 
class 
SpikingNeurone}}
\noindent\hbox{\vbox{{\bf extends} uk.ac.ic.doc.neuralnets.graph.neural.Neurone}}
}}}
\startsubsubsection{Serializable Fields}{
\begin{itemize}
\item{
private double recoveryScale\begin{itemize}\item{\vskip -.9ex }\end{itemize}
}
\item{
private double recoverySensitivity\begin{itemize}\item{\vskip -.9ex }\end{itemize}
}
\item{
private double psr\begin{itemize}\item{\vskip -.9ex }\end{itemize}
}
\item{
private double u\begin{itemize}\item{\vskip -.9ex }\end{itemize}
}
\item{
private double psrRecovery\begin{itemize}\item{\vskip -.9ex }\end{itemize}
}
\item{
private double chargeUp\begin{itemize}\item{\vskip -.9ex }\end{itemize}
}
\item{
private String thalamicString\begin{itemize}\item{\vskip -.9ex }\end{itemize}
}
\item{
private String synapticDelayString\begin{itemize}\item{\vskip -.9ex }\end{itemize}
}
\item{
private int fired\begin{itemize}\item{\vskip -.9ex }\end{itemize}
}
\item{
private List delays\begin{itemize}\item{\vskip -.9ex }\end{itemize}
}
\item{
private Synapse outbound\begin{itemize}\item{\vskip -.9ex }\end{itemize}
}
\end{itemize}
}
\startsubsubsection{Constructors}{
\vskip -2em
\begin{itemize}
\item{\vskip -1.9ex 
\membername{SpikingNeurone}
{\tt public {\bf SpikingNeurone}(  )
\label{l1302}\label{l1303}}%end signature
}%end item
\end{itemize}
}
\startsubsubsection{Methods}{
\vskip -2em
\begin{itemize}
\item{\vskip -1.9ex 
\membername{charge}
{\tt public Neurone {\bf charge}( {\tt double } {\bf amt} )
\label{l1304}\label{l1305}}%end signature
}%end item
\divideents{getPostSpikeReset}
\item{\vskip -1.9ex 
\membername{getPostSpikeReset}
{\tt public Double {\bf getPostSpikeReset}(  )
\label{l1306}\label{l1307}}%end signature
}%end item
\divideents{getPSRRecovery}
\item{\vskip -1.9ex 
\membername{getPSRRecovery}
{\tt public Double {\bf getPSRRecovery}(  )
\label{l1308}\label{l1309}}%end signature
}%end item
\divideents{getRecoveryScale}
\item{\vskip -1.9ex 
\membername{getRecoveryScale}
{\tt public Double {\bf getRecoveryScale}(  )
\label{l1310}\label{l1311}}%end signature
}%end item
\divideents{getRecoverySensitivity}
\item{\vskip -1.9ex 
\membername{getRecoverySensitivity}
{\tt public Double {\bf getRecoverySensitivity}(  )
\label{l1312}\label{l1313}}%end signature
}%end item
\divideents{setPostSpikeReset}
\item{\vskip -1.9ex 
\membername{setPostSpikeReset}
{\tt public void {\bf setPostSpikeReset}( {\tt uk.ac.ic.doc.neuralnets.expressions.ast.ASTExpression } {\bf e} )
\label{l1314}\label{l1315}}%end signature
}%end item
\divideents{setPSRRecovery}
\item{\vskip -1.9ex 
\membername{setPSRRecovery}
{\tt public void {\bf setPSRRecovery}( {\tt uk.ac.ic.doc.neuralnets.expressions.ast.ASTExpression } {\bf e} )
\label{l1316}\label{l1317}}%end signature
}%end item
\divideents{setRecoveryScale}
\item{\vskip -1.9ex 
\membername{setRecoveryScale}
{\tt public void {\bf setRecoveryScale}( {\tt uk.ac.ic.doc.neuralnets.expressions.ast.ASTExpression } {\bf e} )
\label{l1318}\label{l1319}}%end signature
}%end item
\divideents{setRecoverySensitivity}
\item{\vskip -1.9ex 
\membername{setRecoverySensitivity}
{\tt public void {\bf setRecoverySensitivity}( {\tt uk.ac.ic.doc.neuralnets.expressions.ast.ASTExpression } {\bf e} )
\label{l1320}\label{l1321}}%end signature
}%end item
\divideents{setSynapticDelay}
\item{\vskip -1.9ex 
\membername{setSynapticDelay}
{\tt public void {\bf setSynapticDelay}( {\tt uk.ac.ic.doc.neuralnets.expressions.ast.ASTExpression } {\bf e} )
\label{l1322}\label{l1323}}%end signature
}%end item
\divideents{setThalamicInput}
\item{\vskip -1.9ex 
\membername{setThalamicInput}
{\tt public void {\bf setThalamicInput}( {\tt uk.ac.ic.doc.neuralnets.expressions.ast.ASTExpression } {\bf e} )
\label{l1324}\label{l1325}}%end signature
}%end item
\divideents{tick}
\item{\vskip -1.9ex 
\membername{tick}
{\tt public Node {\bf tick}(  )
\label{l1326}\label{l1327}}%end signature
}%end item
\divideents{toString}
\item{\vskip -1.9ex 
\membername{toString}
{\tt public String {\bf toString}(  )
\label{l1328}\label{l1329}}%end signature
}%end item
\end{itemize}
}
\startsubsubsection{Methods inherited from class {\tt uk.ac.ic.doc.neuralnets.graph.neural.Neurone}}{
\par{\small 
\refdefined{l225}\vskip -2em
\begin{itemize}
\item{\vskip -1.9ex 
\membername{charge}
{\tt public Neurone {\bf charge}( {\tt double } {\bf amt} )
}%end signature
}%end item
\divideents{getCharge}
\item{\vskip -1.9ex 
\membername{getCharge}
{\tt public double {\bf getCharge}(  )
}%end signature
}%end item
\divideents{getCurrentCharge}
\item{\vskip -1.9ex 
\membername{getCurrentCharge}
{\tt public Double {\bf getCurrentCharge}(  )
}%end signature
}%end item
\divideents{getEdgeDecoration}
\item{\vskip -1.9ex 
\membername{getEdgeDecoration}
{\tt public EdgeDecoration {\bf getEdgeDecoration}(  )
}%end signature
}%end item
\divideents{getFreshID}
\item{\vskip -1.9ex 
\membername{getFreshID}
{\tt public void {\bf getFreshID}(  )
}%end signature
}%end item
\divideents{getID}
\item{\vskip -1.9ex 
\membername{getID}
{\tt public int {\bf getID}(  )
}%end signature
}%end item
\divideents{getSquashFunction}
\item{\vskip -1.9ex 
\membername{getSquashFunction}
{\tt public ASTExpression {\bf getSquashFunction}(  )
}%end signature
}%end item
\divideents{getTrigger}
\item{\vskip -1.9ex 
\membername{getTrigger}
{\tt public double {\bf getTrigger}(  )
}%end signature
}%end item
\divideents{reset}
\item{\vskip -1.9ex 
\membername{reset}
{\tt public void {\bf reset}(  )
}%end signature
}%end item
\divideents{setCharge}
\item{\vskip -1.9ex 
\membername{setCharge}
{\tt public void {\bf setCharge}( {\tt double } {\bf charge} )
}%end signature
}%end item
\divideents{setEdgeDecoration}
\item{\vskip -1.9ex 
\membername{setEdgeDecoration}
{\tt public void {\bf setEdgeDecoration}( {\tt uk.ac.ic.doc.neuralnets.graph.neural.EdgeDecoration } {\bf ed} )
}%end signature
}%end item
\divideents{setID}
\item{\vskip -1.9ex 
\membername{setID}
{\tt public void {\bf setID}( {\tt int } {\bf id} )
}%end signature
}%end item
\divideents{setInitialCharge}
\item{\vskip -1.9ex 
\membername{setInitialCharge}
{\tt public void {\bf setInitialCharge}( {\tt uk.ac.ic.doc.neuralnets.expressions.ast.ASTExpression } {\bf c} )
}%end signature
}%end item
\divideents{setSquashFunction}
\item{\vskip -1.9ex 
\membername{setSquashFunction}
{\tt public void {\bf setSquashFunction}( {\tt uk.ac.ic.doc.neuralnets.expressions.ast.ASTExpression } {\bf e} )
}%end signature
}%end item
\divideents{setTrigger}
\item{\vskip -1.9ex 
\membername{setTrigger}
{\tt public void {\bf setTrigger}( {\tt uk.ac.ic.doc.neuralnets.expressions.ast.ASTExpression } {\bf t} )
}%end signature
}%end item
\divideents{setTrigger}
\item{\vskip -1.9ex 
\membername{setTrigger}
{\tt public void {\bf setTrigger}( {\tt double } {\bf d} )
}%end signature
}%end item
\divideents{tick}
\item{\vskip -1.9ex 
\membername{tick}
{\tt public Node {\bf tick}(  )
}%end signature
\begin{itemize}
\sld
\item{
\sld
{\bf Usage}
  \begin{itemize}\isep
   \item{
Ticks the neurone one step forward. Fires the neurone is appropriate.
}%end item
  \end{itemize}
}
\item{{\bf Returns} - 
Itself. 
}%end item
\end{itemize}
}%end item
\divideents{toString}
\item{\vskip -1.9ex 
\membername{toString}
{\tt public String {\bf toString}(  )
}%end signature
}%end item
\end{itemize}
}}
\startsubsubsection{Methods inherited from class {\tt uk.ac.ic.doc.neuralnets.graph.neural.NodeBase}}{
\par{\small 
\refdefined{l514}\vskip -2em
\begin{itemize}
\item{\vskip -1.9ex 
\membername{connect}
{\tt public Node {\bf connect}( {\tt uk.ac.ic.doc.neuralnets.graph.Edge } {\bf e} )
}%end signature
\begin{itemize}
\sld
\item{
\sld
{\bf Usage}
  \begin{itemize}\isep
   \item{
Connect this node up with the input edge.
}%end item
  \end{itemize}
}
\end{itemize}
}%end item
\divideents{getIncoming}
\item{\vskip -1.9ex 
\membername{getIncoming}
{\tt public Collection {\bf getIncoming}(  )
}%end signature
\begin{itemize}
\sld
\item{
\sld
{\bf Usage}
  \begin{itemize}\isep
   \item{
Get incoming edges.
}%end item
  \end{itemize}
}
\end{itemize}
}%end item
\divideents{getMetadata}
\item{\vskip -1.9ex 
\membername{getMetadata}
{\tt public String {\bf getMetadata}( {\tt java.lang.String } {\bf key} )
}%end signature
\begin{itemize}
\sld
\item{
\sld
{\bf Usage}
  \begin{itemize}\isep
   \item{
Returns the meta data for the key input.
}%end item
  \end{itemize}
}
\item{
\sld
{\bf Parameters}
\sld\isep
  \begin{itemize}
\sld\isep
   \item{
\sld
{\tt key} - To look for.}
  \end{itemize}
}%end item
\item{{\bf Returns} - 
item Found. 
}%end item
\end{itemize}
}%end item
\divideents{getOutgoing}
\item{\vskip -1.9ex 
\membername{getOutgoing}
{\tt public Collection {\bf getOutgoing}(  )
}%end signature
\begin{itemize}
\sld
\item{
\sld
{\bf Usage}
  \begin{itemize}\isep
   \item{
Get outgoing edges.
}%end item
  \end{itemize}
}
\end{itemize}
}%end item
\divideents{getX}
\item{\vskip -1.9ex 
\membername{getX}
{\tt public int {\bf getX}(  )
}%end signature
\begin{itemize}
\sld
\item{
\sld
{\bf Usage}
  \begin{itemize}\isep
   \item{
Returns the position of the node on the x axis.
}%end item
  \end{itemize}
}
\item{{\bf Returns} - 
x axis position. 
}%end item
\end{itemize}
}%end item
\divideents{getY}
\item{\vskip -1.9ex 
\membername{getY}
{\tt public int {\bf getY}(  )
}%end signature
\begin{itemize}
\sld
\item{
\sld
{\bf Usage}
  \begin{itemize}\isep
   \item{
Returns the position of the node on the y axis.
}%end item
  \end{itemize}
}
\item{{\bf Returns} - 
y axis position. 
}%end item
\end{itemize}
}%end item
\divideents{getZ}
\item{\vskip -1.9ex 
\membername{getZ}
{\tt public int {\bf getZ}(  )
}%end signature
\begin{itemize}
\sld
\item{
\sld
{\bf Usage}
  \begin{itemize}\isep
   \item{
Returns the position of the node on the z axis.
}%end item
  \end{itemize}
}
\item{{\bf Returns} - 
z axis position. 
}%end item
\end{itemize}
}%end item
\divideents{setMetadata}
\item{\vskip -1.9ex 
\membername{setMetadata}
{\tt public Node {\bf setMetadata}( {\tt java.lang.String } {\bf key},
{\tt java.lang.String } {\bf item} )
}%end signature
\begin{itemize}
\sld
\item{
\sld
{\bf Usage}
  \begin{itemize}\isep
   \item{
Set meta data for the object.
}%end item
  \end{itemize}
}
\item{
\sld
{\bf Parameters}
\sld\isep
  \begin{itemize}
\sld\isep
   \item{
\sld
{\tt key} - String key}
   \item{
\sld
{\tt item} - String item}
  \end{itemize}
}%end item
\end{itemize}
}%end item
\divideents{setPos}
\item{\vskip -1.9ex 
\membername{setPos}
{\tt public void {\bf setPos}( {\tt int } {\bf x},
{\tt int } {\bf y},
{\tt int } {\bf z} )
}%end signature
\begin{itemize}
\sld
\item{
\sld
{\bf Usage}
  \begin{itemize}\isep
   \item{
Sets the position of the node.
}%end item
  \end{itemize}
}
\item{
\sld
{\bf Parameters}
\sld\isep
  \begin{itemize}
\sld\isep
   \item{
\sld
{\tt x} - Position on x axis.}
   \item{
\sld
{\tt y} - Position on y axis.}
   \item{
\sld
{\tt z} - Position on z axis.}
  \end{itemize}
}%end item
\end{itemize}
}%end item
\divideents{setX}
\item{\vskip -1.9ex 
\membername{setX}
{\tt public void {\bf setX}( {\tt int } {\bf x} )
}%end signature
\begin{itemize}
\sld
\item{
\sld
{\bf Usage}
  \begin{itemize}\isep
   \item{
Sets the position of the node on the x axis.
}%end item
  \end{itemize}
}
\item{
\sld
{\bf Parameters}
\sld\isep
  \begin{itemize}
\sld\isep
   \item{
\sld
{\tt x} - Position on x axis.}
  \end{itemize}
}%end item
\end{itemize}
}%end item
\divideents{setY}
\item{\vskip -1.9ex 
\membername{setY}
{\tt public void {\bf setY}( {\tt int } {\bf y} )
}%end signature
\begin{itemize}
\sld
\item{
\sld
{\bf Usage}
  \begin{itemize}\isep
   \item{
Sets the position of the node on the y axis.
}%end item
  \end{itemize}
}
\item{
\sld
{\bf Parameters}
\sld\isep
  \begin{itemize}
\sld\isep
   \item{
\sld
{\tt y} - Position on y axis.}
  \end{itemize}
}%end item
\end{itemize}
}%end item
\divideents{setZ}
\item{\vskip -1.9ex 
\membername{setZ}
{\tt public void {\bf setZ}( {\tt int } {\bf z} )
}%end signature
\begin{itemize}
\sld
\item{
\sld
{\bf Usage}
  \begin{itemize}\isep
   \item{
Sets the position of the node on the z axis.
}%end item
  \end{itemize}
}
\item{
\sld
{\bf Parameters}
\sld\isep
  \begin{itemize}
\sld\isep
   \item{
\sld
{\tt z} - Position on z axis.}
  \end{itemize}
}%end item
\end{itemize}
}%end item
\divideents{tick}
\item{\vskip -1.9ex 
\membername{tick}
{\tt public abstract Node {\bf tick}(  )
}%end signature
}%end item
\divideents{toString}
\item{\vskip -1.9ex 
\membername{toString}
{\tt public abstract String {\bf toString}(  )
}%end signature
}%end item
\end{itemize}
}}
}
\startsection{Class}{Synapse}{l1076}{%
\startsubsubsection{Declaration}{
\fbox{\vbox{
\hbox{\vbox{\small public 
class 
Synapse}}
\noindent\hbox{\vbox{{\bf extends} uk.ac.ic.doc.neuralnets.graph.neural.EdgeBase}}
}}}
\startsubsubsection{Serializable Fields}{
\begin{itemize}
\item{
private double weight\begin{itemize}\item{\vskip -.9ex }\end{itemize}
}
\item{
private int delay\begin{itemize}\item{\vskip -.9ex }\end{itemize}
}
\end{itemize}
}
\startsubsubsection{Constructors}{
\vskip -2em
\begin{itemize}
\item{\vskip -1.9ex 
\membername{Synapse}
{\tt public {\bf Synapse}(  )
\label{l1330}\label{l1331}}%end signature
}%end item
\divideents{Synapse}
\item{\vskip -1.9ex 
\membername{Synapse}
{\tt public {\bf Synapse}( {\tt double } {\bf weight},
{\tt uk.ac.ic.doc.neuralnets.graph.neural.Neurone } {\bf start},
{\tt uk.ac.ic.doc.neuralnets.graph.neural.Neurone } {\bf end} )
\label{l1332}\label{l1333}}%end signature
}%end item
\divideents{Synapse}
\item{\vskip -1.9ex 
\membername{Synapse}
{\tt public {\bf Synapse}( {\tt uk.ac.ic.doc.neuralnets.graph.neural.Neurone } {\bf start},
{\tt uk.ac.ic.doc.neuralnets.graph.neural.Neurone } {\bf end} )
\label{l1334}\label{l1335}}%end signature
}%end item
\end{itemize}
}
\startsubsubsection{Methods}{
\vskip -2em
\begin{itemize}
\item{\vskip -1.9ex 
\membername{fire}
{\tt public Synapse {\bf fire}( {\tt double } {\bf amt} )
\label{l1336}\label{l1337}}%end signature
}%end item
\divideents{getDelay}
\item{\vskip -1.9ex 
\membername{getDelay}
{\tt public int {\bf getDelay}(  )
\label{l1338}\label{l1339}}%end signature
}%end item
\divideents{getWeight}
\item{\vskip -1.9ex 
\membername{getWeight}
{\tt public double {\bf getWeight}(  )
\label{l1340}\label{l1341}}%end signature
}%end item
\divideents{setDelay}
\item{\vskip -1.9ex 
\membername{setDelay}
{\tt public Synapse {\bf setDelay}( {\tt int } {\bf d} )
\label{l1342}\label{l1343}}%end signature
}%end item
\divideents{setWeight}
\item{\vskip -1.9ex 
\membername{setWeight}
{\tt public Synapse {\bf setWeight}( {\tt double } {\bf weight} )
\label{l1344}\label{l1345}}%end signature
}%end item
\divideents{toString}
\item{\vskip -1.9ex 
\membername{toString}
{\tt public String {\bf toString}(  )
\label{l1346}\label{l1347}}%end signature
}%end item
\end{itemize}
}
\startsubsubsection{Methods inherited from class {\tt uk.ac.ic.doc.neuralnets.graph.neural.EdgeBase}}{
\par{\small 
\refdefined{l1063}\vskip -2em
\begin{itemize}
\item{\vskip -1.9ex 
\membername{getEnd}
{\tt public Node {\bf getEnd}(  )
}%end signature
}%end item
\divideents{getFreshID}
\item{\vskip -1.9ex 
\membername{getFreshID}
{\tt public void {\bf getFreshID}(  )
}%end signature
}%end item
\divideents{getID}
\item{\vskip -1.9ex 
\membername{getID}
{\tt public int {\bf getID}(  )
}%end signature
}%end item
\divideents{getStart}
\item{\vskip -1.9ex 
\membername{getStart}
{\tt public Node {\bf getStart}(  )
}%end signature
}%end item
\divideents{setID}
\item{\vskip -1.9ex 
\membername{setID}
{\tt public void {\bf setID}( {\tt int } {\bf id} )
}%end signature
}%end item
\divideents{setStart}
\item{\vskip -1.9ex 
\membername{setStart}
{\tt public Edge {\bf setStart}( {\tt uk.ac.ic.doc.neuralnets.graph.Node } {\bf start} )
}%end signature
}%end item
\divideents{setTo}
\item{\vskip -1.9ex 
\membername{setTo}
{\tt public Edge {\bf setTo}( {\tt uk.ac.ic.doc.neuralnets.graph.Node } {\bf end} )
}%end signature
}%end item
\divideents{tick}
\item{\vskip -1.9ex 
\membername{tick}
{\tt public void {\bf tick}(  )
}%end signature
}%end item
\divideents{toString}
\item{\vskip -1.9ex 
\membername{toString}
{\tt public String {\bf toString}(  )
}%end signature
}%end item
\end{itemize}
}}
}
}
}
\newpage
\def\packagename{uk.ac.ic.doc.neuralnets.graph}
\chapter{\bf Package uk.ac.ic.doc.neuralnets.graph}{
\vskip -.25in
\hbox to \hsize{\it Package Contents\hfil Page}
\rule{\hsize}{.7mm}
\vskip .13in
\hbox{\bf Interfaces}
\entityintro{Edge}{l227}{...no description...}
\entityintro{Graph.Command}{l1348}{...no description...}
\entityintro{Identifiable}{l1349}{...no description...}
\entityintro{Node}{l228}{...no description...}
\entityintro{Saveable}{l1350}{...no description...}
\vskip .13in
\hbox{\bf Classes}
\entityintro{Graph}{l507}{...no description...}
\entityintro{GraphStreamer}{l1351}{...no description...}
\entityintro{Metadata}{l1352}{Constants for use in setting and getting metadata
 Useful to keep all in one place, should be inlined by compiler too.}
\vskip .1in
\rule{\hsize}{.7mm}
\vskip .1in
\newpage
\section{Interfaces}{
\startsection{Interface}{Edge}{l227}{%
\startsubsubsection{Declaration}{
\fbox{\vbox{
\hbox{\vbox{\small public interface 
Edge}}
\noindent\hbox{\vbox{{\bf implements} 
java.io.Serializable, Identifiable}}
}}}
\startsubsubsection{Methods}{
\vskip -2em
\begin{itemize}
\item{\vskip -1.9ex 
\membername{getEnd}
{\tt public Node {\bf getEnd}(  )
\label{l1353}\label{l1354}}%end signature
}%end item
\divideents{getStart}
\item{\vskip -1.9ex 
\membername{getStart}
{\tt public Node {\bf getStart}(  )
\label{l1355}\label{l1356}}%end signature
}%end item
\divideents{setStart}
\item{\vskip -1.9ex 
\membername{setStart}
{\tt public Edge {\bf setStart}( {\tt uk.ac.ic.doc.neuralnets.graph.Node } {\bf start} )
\label{l1357}\label{l1358}}%end signature
}%end item
\divideents{setTo}
\item{\vskip -1.9ex 
\membername{setTo}
{\tt public Edge {\bf setTo}( {\tt uk.ac.ic.doc.neuralnets.graph.Node } {\bf end} )
\label{l1359}\label{l1360}}%end signature
}%end item
\divideents{tick}
\item{\vskip -1.9ex 
\membername{tick}
{\tt public void {\bf tick}(  )
\label{l1361}\label{l1362}}%end signature
}%end item
\end{itemize}
}
}
\startsection{Interface}{Graph.Command}{l1348}{%
\startsubsubsection{Declaration}{
\fbox{\vbox{
\hbox{\vbox{\small public static interface 
Graph.Command}}
}}}
\startsubsubsection{Methods}{
\vskip -2em
\begin{itemize}
\item{\vskip -1.9ex 
\membername{exec}
{\tt public void {\bf exec}( {\tt java.lang.Object } {\bf input} )
\label{l1363}\label{l1364}}%end signature
}%end item
\end{itemize}
}
}
\startsection{Interface}{Identifiable}{l1349}{%
\startsubsubsection{Declaration}{
\fbox{\vbox{
\hbox{\vbox{\small public interface 
Identifiable}}
}}}
\startsubsubsection{Methods}{
\vskip -2em
\begin{itemize}
\item{\vskip -1.9ex 
\membername{getFreshID}
{\tt public void {\bf getFreshID}(  )
\label{l1365}\label{l1366}}%end signature
}%end item
\divideents{getID}
\item{\vskip -1.9ex 
\membername{getID}
{\tt public int {\bf getID}(  )
\label{l1367}\label{l1368}}%end signature
}%end item
\divideents{setID}
\item{\vskip -1.9ex 
\membername{setID}
{\tt public void {\bf setID}( {\tt int } {\bf id} )
\label{l1369}\label{l1370}}%end signature
}%end item
\end{itemize}
}
}
\startsection{Interface}{Node}{l228}{%
\startsubsubsection{Declaration}{
\fbox{\vbox{
\hbox{\vbox{\small public interface 
Node}}
\noindent\hbox{\vbox{{\bf implements} 
java.io.Serializable, Identifiable}}
}}}
\startsubsubsection{Methods}{
\vskip -2em
\begin{itemize}
\item{\vskip -1.9ex 
\membername{connect}
{\tt public Node {\bf connect}( {\tt uk.ac.ic.doc.neuralnets.graph.Edge } {\bf e} )
\label{l1371}\label{l1372}}%end signature
}%end item
\divideents{getIncoming}
\item{\vskip -1.9ex 
\membername{getIncoming}
{\tt public Collection {\bf getIncoming}(  )
\label{l1373}\label{l1374}}%end signature
}%end item
\divideents{getMetadata}
\item{\vskip -1.9ex 
\membername{getMetadata}
{\tt public String {\bf getMetadata}( {\tt java.lang.String } {\bf key} )
\label{l1375}\label{l1376}}%end signature
}%end item
\divideents{getOutgoing}
\item{\vskip -1.9ex 
\membername{getOutgoing}
{\tt public Collection {\bf getOutgoing}(  )
\label{l1377}\label{l1378}}%end signature
}%end item
\divideents{getX}
\item{\vskip -1.9ex 
\membername{getX}
{\tt public int {\bf getX}(  )
\label{l1379}\label{l1380}}%end signature
}%end item
\divideents{getY}
\item{\vskip -1.9ex 
\membername{getY}
{\tt public int {\bf getY}(  )
\label{l1381}\label{l1382}}%end signature
}%end item
\divideents{getZ}
\item{\vskip -1.9ex 
\membername{getZ}
{\tt public int {\bf getZ}(  )
\label{l1383}\label{l1384}}%end signature
}%end item
\divideents{setMetadata}
\item{\vskip -1.9ex 
\membername{setMetadata}
{\tt public Node {\bf setMetadata}( {\tt java.lang.String } {\bf key},
{\tt java.lang.String } {\bf item} )
\label{l1385}\label{l1386}}%end signature
}%end item
\divideents{setPos}
\item{\vskip -1.9ex 
\membername{setPos}
{\tt public void {\bf setPos}( {\tt int } {\bf x},
{\tt int } {\bf y},
{\tt int } {\bf z} )
\label{l1387}\label{l1388}}%end signature
}%end item
\divideents{tick}
\item{\vskip -1.9ex 
\membername{tick}
{\tt public Node {\bf tick}(  )
\label{l1389}\label{l1390}}%end signature
\begin{itemize}
\sld
\item{
\sld
{\bf Usage}
  \begin{itemize}\isep
   \item{
States that this node has advanced one "tick" in time
}%end item
  \end{itemize}
}
\end{itemize}
}%end item
\end{itemize}
}
}
\startsection{Interface}{Saveable}{l1350}{%
\startsubsubsection{Declaration}{
\fbox{\vbox{
\hbox{\vbox{\small public interface 
Saveable}}
\noindent\hbox{\vbox{{\bf implements} 
java.io.Serializable}}
}}}
}
}
\section{Classes}{
\startsection{Class}{Graph}{l507}{%
\startsubsubsection{Declaration}{
\fbox{\vbox{
\hbox{\vbox{\small public 
class 
Graph}}
\noindent\hbox{\vbox{{\bf extends} java.lang.Object}}
\noindent\hbox{\vbox{{\bf implements} 
java.io.Serializable, Identifiable}}
}}}
\startsubsubsection{Serializable Fields}{
\begin{itemize}
\item{
private int id\begin{itemize}\item{\vskip -.9ex }\end{itemize}
}
\end{itemize}
}
\startsubsubsection{Constructors}{
\vskip -2em
\begin{itemize}
\item{\vskip -1.9ex 
\membername{Graph}
{\tt public {\bf Graph}(  )
\label{l1391}\label{l1392}}%end signature
}%end item
\end{itemize}
}
\startsubsubsection{Methods}{
\vskip -2em
\begin{itemize}
\item{\vskip -1.9ex 
\membername{addAllNodes}
{\tt public Graph {\bf addAllNodes}( {\tt java.util.Collection } {\bf ns} )
\label{l1393}\label{l1394}}%end signature
\begin{itemize}
\sld
\item{
\sld
{\bf Usage}
  \begin{itemize}\isep
   \item{
Adds a collection of nodes to the graph, only if that collection doesn't
 contain itself.
}%end item
  \end{itemize}
}
\item{
\sld
{\bf Parameters}
\sld\isep
  \begin{itemize}
\sld\isep
   \item{
\sld
{\tt ns} - Collection of nodes to add.}
  \end{itemize}
}%end item
\item{{\bf Returns} - 
Itself with the nodes added or not added. 
}%end item
\end{itemize}
}%end item
\divideents{addEdge}
\item{\vskip -1.9ex 
\membername{addEdge}
{\tt public Graph {\bf addEdge}( {\tt uk.ac.ic.doc.neuralnets.graph.Edge } {\bf e} )
\label{l1395}\label{l1396}}%end signature
\begin{itemize}
\sld
\item{
\sld
{\bf Usage}
  \begin{itemize}\isep
   \item{
Adds an edge to the graph and adds its start and end nodes to the graph.
}%end item
  \end{itemize}
}
\item{
\sld
{\bf Parameters}
\sld\isep
  \begin{itemize}
\sld\isep
   \item{
\sld
{\tt e} - Edge to add.}
  \end{itemize}
}%end item
\item{{\bf Returns} - 
Itself 
}%end item
\end{itemize}
}%end item
\divideents{addNode}
\item{\vskip -1.9ex 
\membername{addNode}
{\tt public Graph {\bf addNode}( {\tt uk.ac.ic.doc.neuralnets.graph.Node } {\bf n} )
\label{l1397}\label{l1398}}%end signature
\begin{itemize}
\sld
\item{
\sld
{\bf Usage}
  \begin{itemize}\isep
   \item{
Adds input node to the graph as long as input node is not itself, returns
 itself.
}%end item
  \end{itemize}
}
\item{
\sld
{\bf Parameters}
\sld\isep
  \begin{itemize}
\sld\isep
   \item{
\sld
{\tt n} - Node to add.}
  \end{itemize}
}%end item
\item{{\bf Returns} - 
Itself with the node added or not added. 
}%end item
\end{itemize}
}%end item
\divideents{forEachEdge}
\item{\vskip -1.9ex 
\membername{forEachEdge}
{\tt public Graph {\bf forEachEdge}( {\tt uk.ac.ic.doc.neuralnets.graph.Graph.Command } {\bf c} )
\label{l1399}\label{l1400}}%end signature
\begin{itemize}
\sld
\item{
\sld
{\bf Usage}
  \begin{itemize}\isep
   \item{
Conducts a command on each edge within the graph.
}%end item
  \end{itemize}
}
\item{
\sld
{\bf Parameters}
\sld\isep
  \begin{itemize}
\sld\isep
   \item{
\sld
{\tt c} - Command to execute.}
  \end{itemize}
}%end item
\item{{\bf Returns} - 
Itself. 
}%end item
\end{itemize}
}%end item
\divideents{forEachNode}
\item{\vskip -1.9ex 
\membername{forEachNode}
{\tt public Graph {\bf forEachNode}( {\tt uk.ac.ic.doc.neuralnets.graph.Graph.Command } {\bf c} )
\label{l1401}\label{l1402}}%end signature
\begin{itemize}
\sld
\item{
\sld
{\bf Usage}
  \begin{itemize}\isep
   \item{
Conducts a command on each node within the graph.
}%end item
  \end{itemize}
}
\item{
\sld
{\bf Parameters}
\sld\isep
  \begin{itemize}
\sld\isep
   \item{
\sld
{\tt c} - Command to execute.}
  \end{itemize}
}%end item
\item{{\bf Returns} - 
Itself. 
}%end item
\end{itemize}
}%end item
\divideents{getEdges}
\item{\vskip -1.9ex 
\membername{getEdges}
{\tt public Collection {\bf getEdges}(  )
\label{l1403}\label{l1404}}%end signature
\begin{itemize}
\sld
\item{
\sld
{\bf Usage}
  \begin{itemize}\isep
   \item{
Gets the edges from within.
}%end item
  \end{itemize}
}
\item{{\bf Returns} - 
The edges. 
}%end item
\end{itemize}
}%end item
\divideents{getFreshID}
\item{\vskip -1.9ex 
\membername{getFreshID}
{\tt public void {\bf getFreshID}(  )
\label{l1405}\label{l1406}}%end signature
\begin{itemize}
\sld
\item{
\sld
{\bf Usage}
  \begin{itemize}\isep
   \item{
Sets the id of the object to a new fresh id.
}%end item
  \end{itemize}
}
\end{itemize}
}%end item
\divideents{getID}
\item{\vskip -1.9ex 
\membername{getID}
{\tt public int {\bf getID}(  )
\label{l1407}\label{l1408}}%end signature
\begin{itemize}
\sld
\item{
\sld
{\bf Usage}
  \begin{itemize}\isep
   \item{
Gets the id of the object.
}%end item
  \end{itemize}
}
\item{{\bf Returns} - 
The id. 
}%end item
\end{itemize}
}%end item
\divideents{getNodes}
\item{\vskip -1.9ex 
\membername{getNodes}
{\tt public Collection {\bf getNodes}(  )
\label{l1409}\label{l1410}}%end signature
\begin{itemize}
\sld
\item{
\sld
{\bf Usage}
  \begin{itemize}\isep
   \item{
Gets the nodes from within.
}%end item
  \end{itemize}
}
\item{{\bf Returns} - 
The nodes. 
}%end item
\end{itemize}
}%end item
\divideents{merge}
\item{\vskip -1.9ex 
\membername{merge}
{\tt public Graph {\bf merge}( {\tt uk.ac.ic.doc.neuralnets.graph.Graph } {\bf o} )
\label{l1411}\label{l1412}}%end signature
\begin{itemize}
\sld
\item{
\sld
{\bf Usage}
  \begin{itemize}\isep
   \item{
Merges one graph with its self, as all the edges and nodes.
}%end item
  \end{itemize}
}
\item{
\sld
{\bf Parameters}
\sld\isep
  \begin{itemize}
\sld\isep
   \item{
\sld
{\tt o} - Graph to merge with.}
  \end{itemize}
}%end item
\item{{\bf Returns} - 
Itself 
}%end item
\end{itemize}
}%end item
\divideents{setID}
\item{\vskip -1.9ex 
\membername{setID}
{\tt public void {\bf setID}( {\tt int } {\bf id} )
\label{l1413}\label{l1414}}%end signature
\begin{itemize}
\sld
\item{
\sld
{\bf Usage}
  \begin{itemize}\isep
   \item{
Sets the id of the object to parameter.
}%end item
  \end{itemize}
}
\item{
\sld
{\bf Parameters}
\sld\isep
  \begin{itemize}
\sld\isep
   \item{
\sld
{\tt int} - New id.}
  \end{itemize}
}%end item
\end{itemize}
}%end item
\divideents{toString}
\item{\vskip -1.9ex 
\membername{toString}
{\tt public String {\bf toString}(  )
\label{l1415}\label{l1416}}%end signature
}%end item
\divideents{type}
\item{\vskip -1.9ex 
\membername{type}
{\tt protected String {\bf type}(  )
\label{l1417}\label{l1418}}%end signature
\begin{itemize}
\sld
\item{
\sld
{\bf Usage}
  \begin{itemize}\isep
   \item{
Returns the object type.
}%end item
  \end{itemize}
}
\item{{\bf Returns} - 
Object type. 
}%end item
\end{itemize}
}%end item
\end{itemize}
}
}
\startsection{Class}{GraphStreamer}{l1351}{%
\startsubsubsection{Declaration}{
\fbox{\vbox{
\hbox{\vbox{\small public 
class 
GraphStreamer}}
\noindent\hbox{\vbox{{\bf extends} java.lang.Object}}
}}}
\startsubsubsection{Constructors}{
\vskip -2em
\begin{itemize}
\item{\vskip -1.9ex 
\membername{GraphStreamer}
{\tt public {\bf GraphStreamer}( {\tt uk.ac.ic.doc.neuralnets.graph.Graph } {\bf g},
{\tt uk.ac.ic.doc.neuralnets.util.Transformer } {\bf edgeMaker},
{\tt uk.ac.ic.doc.neuralnets.util.Transformer } {\bf nodeMaker} )
\label{l1419}\label{l1420}}%end signature
}%end item
\end{itemize}
}
\startsubsubsection{Methods}{
\vskip -2em
\begin{itemize}
\item{\vskip -1.9ex 
\membername{getEdgeIterator}
{\tt public Iterator {\bf getEdgeIterator}(  )
\label{l1421}\label{l1422}}%end signature
\begin{itemize}
\sld
\item{
\sld
{\bf Usage}
  \begin{itemize}\isep
   \item{
Returns an iterator for the edges that are contained in the GraphStreamer
}%end item
  \end{itemize}
}
\item{{\bf Returns} - 
Iterator of edges. 
}%end item
\end{itemize}
}%end item
\divideents{getNodeIterator}
\item{\vskip -1.9ex 
\membername{getNodeIterator}
{\tt public Iterator {\bf getNodeIterator}(  )
\label{l1423}\label{l1424}}%end signature
\begin{itemize}
\sld
\item{
\sld
{\bf Usage}
  \begin{itemize}\isep
   \item{
Returns an iterator for the nodes that are contained in the GraphStreamer
}%end item
  \end{itemize}
}
\item{{\bf Returns} - 
Iterator of nodes. 
}%end item
\end{itemize}
}%end item
\end{itemize}
}
}
\startsection{Class}{Metadata}{l1352}{%
{\small Constants for use in setting and getting metadata
 Useful to keep all in one place, should be inlined by compiler too.}
\vskip .1in 
\startsubsubsection{Declaration}{
\fbox{\vbox{
\hbox{\vbox{\small public 
class 
Metadata}}
\noindent\hbox{\vbox{{\bf extends} java.lang.Object}}
}}}
\startsubsubsection{Fields}{
\begin{itemize}
\item{
public static final String X\_POS\begin{itemize}\item{\vskip -.9ex }\end{itemize}
}
\item{
public static final String Y\_POS\begin{itemize}\item{\vskip -.9ex }\end{itemize}
}
\end{itemize}
}
\startsubsubsection{Constructors}{
\vskip -2em
\begin{itemize}
\item{\vskip -1.9ex 
\membername{Metadata}
{\tt public {\bf Metadata}(  )
\label{l1425}\label{l1426}}%end signature
}%end item
\end{itemize}
}
}
}
}
\newpage
\def\packagename{uk.ac.ic.doc.neuralnets.coreui}
\chapter{\bf Package uk.ac.ic.doc.neuralnets.coreui}{
\vskip -.25in
\hbox to \hsize{\it Package Contents\hfil Page}
\rule{\hsize}{.7mm}
\vskip .13in
\hbox{\bf Classes}
\entityintro{InterfaceManager}{l107}{...no description...}
\entityintro{ZoomingInterfaceManager}{l106}{...no description...}
\vskip .1in
\rule{\hsize}{.7mm}
\vskip .1in
\newpage
\section{Classes}{
\startsection{Class}{InterfaceManager}{l107}{%
\startsubsubsection{Declaration}{
\fbox{\vbox{
\hbox{\vbox{\small public abstract 
class 
InterfaceManager}}
\noindent\hbox{\vbox{{\bf extends} java.lang.Object}}
}}}
\startsubsubsection{Constructors}{
\vskip -2em
\begin{itemize}
\item{\vskip -1.9ex 
\membername{InterfaceManager}
{\tt public {\bf InterfaceManager}(  )
\label{l1427}\label{l1428}}%end signature
}%end item
\end{itemize}
}
\startsubsubsection{Methods}{
\vskip -2em
\begin{itemize}
\item{\vskip -1.9ex 
\membername{addConnection}
{\tt public void {\bf addConnection}( {\tt uk.ac.ic.doc.neuralnets.graph.Edge } {\bf e} )
\label{l1429}\label{l1430}}%end signature
\begin{itemize}
\sld
\item{
\sld
{\bf Usage}
  \begin{itemize}\isep
   \item{
Adds the given edge to the current view, and redraws the screen as
 necessary.
}%end item
  \end{itemize}
}
\item{
\sld
{\bf Parameters}
\sld\isep
  \begin{itemize}
\sld\isep
   \item{
\sld
{\tt e} - }
  \end{itemize}
}%end item
\end{itemize}
}%end item
\divideents{addNetwork}
\item{\vskip -1.9ex 
\membername{addNetwork}
{\tt public void {\bf addNetwork}( {\tt uk.ac.ic.doc.neuralnets.graph.neural.NeuralNetwork } {\bf n} )
\label{l1431}\label{l1432}}%end signature
\begin{itemize}
\sld
\item{
\sld
{\bf Usage}
  \begin{itemize}\isep
   \item{
Adds the given neural network to the current view, and redraws the screen
 as necessary.
}%end item
  \end{itemize}
}
\item{
\sld
{\bf Parameters}
\sld\isep
  \begin{itemize}
\sld\isep
   \item{
\sld
{\tt n} - the neural network to add to the current section of the neural
            network}
  \end{itemize}
}%end item
\end{itemize}
}%end item
\divideents{addNeurone}
\item{\vskip -1.9ex 
\membername{addNeurone}
{\tt public void {\bf addNeurone}( {\tt uk.ac.ic.doc.neuralnets.graph.neural.Neurone } {\bf n} )
\label{l1433}\label{l1434}}%end signature
\begin{itemize}
\sld
\item{
\sld
{\bf Usage}
  \begin{itemize}\isep
   \item{
Adds the given neurone to the current view, and redraws the screen
 as necessary.
}%end item
  \end{itemize}
}
\item{
\sld
{\bf Parameters}
\sld\isep
  \begin{itemize}
\sld\isep
   \item{
\sld
{\tt n} - the neurone to add to the current section of the neural
            network}
  \end{itemize}
}%end item
\end{itemize}
}%end item
\divideents{addNode}
\item{\vskip -1.9ex 
\membername{addNode}
{\tt public void {\bf addNode}( {\tt uk.ac.ic.doc.neuralnets.graph.Node } {\bf n} )
\label{l1435}\label{l1436}}%end signature
\begin{itemize}
\sld
\item{
\sld
{\bf Usage}
  \begin{itemize}\isep
   \item{
Adds the given node to the current view, and redraws the screen
 as necessary.
}%end item
  \end{itemize}
}
\item{
\sld
{\bf Parameters}
\sld\isep
  \begin{itemize}
\sld\isep
   \item{
\sld
{\tt n} - the node to add to the current section of the neural
            network}
  \end{itemize}
}%end item
\end{itemize}
}%end item
\divideents{addNode}
\item{\vskip -1.9ex 
\membername{addNode}
{\tt public void {\bf addNode}( {\tt uk.ac.ic.doc.neuralnets.graph.neural.NodeSpecification } {\bf spec} )
\label{l1437}\label{l1438}}%end signature
\begin{itemize}
\sld
\item{
\sld
{\bf Usage}
  \begin{itemize}\isep
   \item{
Creates a node from the give specification, adds to the current view, and
 redraws the screen as necessary.
}%end item
  \end{itemize}
}
\item{
\sld
{\bf Parameters}
\sld\isep
  \begin{itemize}
\sld\isep
   \item{
\sld
{\tt spec} - the specification of the node to add to the current section of
            the neural network}
  \end{itemize}
}%end item
\end{itemize}
}%end item
\divideents{getCommandControl}
\item{\vskip -1.9ex 
\membername{getCommandControl}
{\tt public CommandControl {\bf getCommandControl}(  )
\label{l1439}\label{l1440}}%end signature
\begin{itemize}
\sld
\item{
\sld
{\bf Usage}
  \begin{itemize}\isep
   \item{
Gets the command control used by the GUIManager. This object handles the
 undo and redo stacks as commands are executed and undone.
}%end item
  \end{itemize}
}
\item{{\bf Returns} - 
the CommandControl object used by the GUIManager 
}%end item
\end{itemize}
}%end item
\divideents{getCurrentNetwork}
\item{\vskip -1.9ex 
\membername{getCurrentNetwork}
{\tt public abstract NeuralNetwork {\bf getCurrentNetwork}(  )
\label{l1441}\label{l1442}}%end signature
\begin{itemize}
\sld
\item{
\sld
{\bf Usage}
  \begin{itemize}\isep
   \item{
Returns the neural network layer currently being viewed in the
 GUIManager.
}%end item
  \end{itemize}
}
\item{{\bf Returns} - 
the current neural network layer 
}%end item
\end{itemize}
}%end item
\divideents{getGraph}
\item{\vskip -1.9ex 
\membername{getGraph}
{\tt public abstract Object {\bf getGraph}(  )
\label{l1443}\label{l1444}}%end signature
\begin{itemize}
\sld
\item{
\sld
{\bf Usage}
  \begin{itemize}\isep
   \item{
Returns the Graph representation used by this UI Manager.
}%end item
  \end{itemize}
}
\item{{\bf Returns} - 
the Graph that the Manager draws onto 
}%end item
\end{itemize}
}%end item
\divideents{getNode}
\item{\vskip -1.9ex 
\membername{getNode}
{\tt public abstract Object {\bf getNode}( {\tt uk.ac.ic.doc.neuralnets.graph.neural.Neurone } {\bf n} )
\label{l1445}\label{l1446}}%end signature
\begin{itemize}
\sld
\item{
\sld
{\bf Usage}
  \begin{itemize}\isep
   \item{
Finds the GUINode in the GUI corresponding to the given Neurone and
 returns it. Returns null if the given Neurone is not loaded in the GUI.
}%end item
  \end{itemize}
}
\item{
\sld
{\bf Parameters}
\sld\isep
  \begin{itemize}
\sld\isep
   \item{
\sld
{\tt n} - the Neurone to look up in the GUI}
  \end{itemize}
}%end item
\item{{\bf Returns} - 
the GUINode in the GUI corresponding to the given Neurone 
}%end item
\end{itemize}
}%end item
\divideents{getRootNetwork}
\item{\vskip -1.9ex 
\membername{getRootNetwork}
{\tt public NeuralNetwork {\bf getRootNetwork}(  )
\label{l1447}\label{l1448}}%end signature
\begin{itemize}
\sld
\item{
\sld
{\bf Usage}
  \begin{itemize}\isep
   \item{
Gets the root of the layered neural network stored in the GUIManager.
}%end item
  \end{itemize}
}
\item{{\bf Returns} - 
the root of the main neural network 
}%end item
\end{itemize}
}%end item
\divideents{getSaveLocation}
\item{\vskip -1.9ex 
\membername{getSaveLocation}
{\tt public FileSpecification {\bf getSaveLocation}(  )
\label{l1449}\label{l1450}}%end signature
\begin{itemize}
\sld
\item{
\sld
{\bf Usage}
  \begin{itemize}\isep
   \item{
Gets the location to save the network to, or null if no such location
 exists.
}%end item
  \end{itemize}
}
\item{{\bf Returns} - 
the network's save location, or null if none exists 
}%end item
\end{itemize}
}%end item
\divideents{getUtils}
\item{\vskip -1.9ex 
\membername{getUtils}
{\tt public InteractionUtils {\bf getUtils}(  )
\label{l1451}\label{l1452}}%end signature
\begin{itemize}
\sld
\item{
\sld
{\bf Usage}
  \begin{itemize}\isep
   \item{
Returns the GUIManager's interaction utilities.
}%end item
  \end{itemize}
}
\item{{\bf Returns} - 
the InteractionUtils object used by the GUIManager 
}%end item
\end{itemize}
}%end item
\divideents{persistLocations}
\item{\vskip -1.9ex 
\membername{persistLocations}
{\tt public abstract void {\bf persistLocations}(  )
\label{l1453}\label{l1454}}%end signature
\begin{itemize}
\sld
\item{
\sld
{\bf Usage}
  \begin{itemize}\isep
   \item{
Pushes down the locations of all Nodes to the model. Allows positions
 to be persisted to storage and reloaded.
}%end item
  \end{itemize}
}
\end{itemize}
}%end item
\divideents{redrawCurrentView}
\item{\vskip -1.9ex 
\membername{redrawCurrentView}
{\tt public abstract void {\bf redrawCurrentView}(  )
\label{l1455}\label{l1456}}%end signature
\begin{itemize}
\sld
\item{
\sld
{\bf Usage}
  \begin{itemize}\isep
   \item{
Draws the current view of the graph. Imports the current network layer
 from the internal model and applies the current layout.
}%end item
  \end{itemize}
}
\end{itemize}
}%end item
\divideents{remove}
\item{\vskip -1.9ex 
\membername{remove}
{\tt public abstract void {\bf remove}( {\tt java.lang.Object } {\bf i} )
\label{l1457}\label{l1458}}%end signature
\begin{itemize}
\sld
\item{
\sld
{\bf Usage}
  \begin{itemize}\isep
   \item{
Removes the given GraphItem from the view.
}%end item
  \end{itemize}
}
\item{
\sld
{\bf Parameters}
\sld\isep
  \begin{itemize}
\sld\isep
   \item{
\sld
{\tt i} - the graphitem to be removed from the view}
  \end{itemize}
}%end item
\end{itemize}
}%end item
\divideents{removeNetwork}
\item{\vskip -1.9ex 
\membername{removeNetwork}
{\tt public void {\bf removeNetwork}( {\tt uk.ac.ic.doc.neuralnets.graph.neural.NeuralNetwork } {\bf n} )
\label{l1459}\label{l1460}}%end signature
\begin{itemize}
\sld
\item{
\sld
{\bf Usage}
  \begin{itemize}\isep
   \item{
Removes the given neural network from the current view, and redraws the
 screen as necessary.
}%end item
  \end{itemize}
}
\item{
\sld
{\bf Parameters}
\sld\isep
  \begin{itemize}
\sld\isep
   \item{
\sld
{\tt n} - the neural network to remove from the current section of the neural
            network}
  \end{itemize}
}%end item
\end{itemize}
}%end item
\divideents{reset}
\item{\vskip -1.9ex 
\membername{reset}
{\tt protected abstract void {\bf reset}(  )
\label{l1461}\label{l1462}}%end signature
\begin{itemize}
\sld
\item{
\sld
{\bf Usage}
  \begin{itemize}\isep
   \item{
Reset the current manager, e.g. when a new network is loaded
}%end item
  \end{itemize}
}
\end{itemize}
}%end item
\divideents{setNetwork}
\item{\vskip -1.9ex 
\membername{setNetwork}
{\tt public void {\bf setNetwork}( {\tt uk.ac.ic.doc.neuralnets.graph.neural.NeuralNetwork } {\bf network},
{\tt uk.ac.ic.doc.neuralnets.persistence.FileSpecification } {\bf location} )
\label{l1463}\label{l1464}}%end signature
\begin{itemize}
\sld
\item{
\sld
{\bf Usage}
  \begin{itemize}\isep
   \item{
Loads the given neural network into the GUIManager, from the given
 location.
}%end item
  \end{itemize}
}
\item{
\sld
{\bf Parameters}
\sld\isep
  \begin{itemize}
\sld\isep
   \item{
\sld
{\tt network} - the network to be loaded into the GUIManager}
   \item{
\sld
{\tt location} - the location to load the network from}
  \end{itemize}
}%end item
\end{itemize}
}%end item
\divideents{setSaveLocation}
\item{\vskip -1.9ex 
\membername{setSaveLocation}
{\tt public void {\bf setSaveLocation}( {\tt uk.ac.ic.doc.neuralnets.persistence.FileSpecification } {\bf saveLoc} )
\label{l1465}\label{l1466}}%end signature
\begin{itemize}
\sld
\item{
\sld
{\bf Usage}
  \begin{itemize}\isep
   \item{
Sets the network's save location.
}%end item
  \end{itemize}
}
\item{
\sld
{\bf Parameters}
\sld\isep
  \begin{itemize}
\sld\isep
   \item{
\sld
{\tt saveLoc} - }
  \end{itemize}
}%end item
\end{itemize}
}%end item
\divideents{updateInterfaceHints}
\item{\vskip -1.9ex 
\membername{updateInterfaceHints}
{\tt public abstract void {\bf updateInterfaceHints}(  )
\label{l1467}\label{l1468}}%end signature
\begin{itemize}
\sld
\item{
\sld
{\bf Usage}
  \begin{itemize}\isep
   \item{
Updates the tooltips or other UI hints of all graph elements 
 in the current view.
}%end item
  \end{itemize}
}
\end{itemize}
}%end item
\end{itemize}
}
}
\startsection{Class}{ZoomingInterfaceManager}{l106}{%
\startsubsubsection{Declaration}{
\fbox{\vbox{
\hbox{\vbox{\small public abstract 
class 
ZoomingInterfaceManager}}
\noindent\hbox{\vbox{{\bf extends} uk.ac.ic.doc.neuralnets.coreui.InterfaceManager}}
}}}
\startsubsubsection{Constructors}{
\vskip -2em
\begin{itemize}
\item{\vskip -1.9ex 
\membername{ZoomingInterfaceManager}
{\tt public {\bf ZoomingInterfaceManager}(  )
\label{l1469}\label{l1470}}%end signature
}%end item
\end{itemize}
}
\startsubsubsection{Methods}{
\vskip -2em
\begin{itemize}
\item{\vskip -1.9ex 
\membername{canZoomIn}
{\tt public abstract boolean {\bf canZoomIn}(  )
\label{l1471}\label{l1472}}%end signature
\begin{itemize}
\sld
\item{
\sld
{\bf Usage}
  \begin{itemize}\isep
   \item{
Checks whether or not it is possible to zoom in. It is only possible to
 zoom in if exactly one internal network layer is selected.
}%end item
  \end{itemize}
}
\item{{\bf Returns} - 
whether or not it is possible to zoom in 
}%end item
\end{itemize}
}%end item
\divideents{canZoomOut}
\item{\vskip -1.9ex 
\membername{canZoomOut}
{\tt public abstract boolean {\bf canZoomOut}(  )
\label{l1473}\label{l1474}}%end signature
\begin{itemize}
\sld
\item{
\sld
{\bf Usage}
  \begin{itemize}\isep
   \item{
Checks whether or not it is possible to zoom out. It is always possible
 to zoom out unless the current view is the root network.
}%end item
  \end{itemize}
}
\item{{\bf Returns} - 
whether or not it is possible to zoom out 
}%end item
\end{itemize}
}%end item
\divideents{getZoomIDs}
\item{\vskip -1.9ex 
\membername{getZoomIDs}
{\tt public abstract Stack {\bf getZoomIDs}(  )
\label{l1475}\label{l1476}}%end signature
\begin{itemize}
\sld
\item{
\sld
{\bf Usage}
  \begin{itemize}\isep
   \item{
Returns a stack containing the IDs of each network layer that has
 currently been zoomed into. This can be used to trace the current zoom
 path from the root of the neural network.
}%end item
  \end{itemize}
}
\item{{\bf Returns} - 
a stack of IDs of each network layer that is currently zoomed
         into 
}%end item
\end{itemize}
}%end item
\divideents{getZoomLevels}
\item{\vskip -1.9ex 
\membername{getZoomLevels}
{\tt public abstract Stack {\bf getZoomLevels}(  )
\label{l1477}\label{l1478}}%end signature
\begin{itemize}
\sld
\item{
\sld
{\bf Usage}
  \begin{itemize}\isep
   \item{
Returns a stack containing each network layer that has currently been
 zoomed into, starting with the root network.
}%end item
  \end{itemize}
}
\item{{\bf Returns} - 
a stack containing each network layer that has currently been
         zoomed into. 
}%end item
\end{itemize}
}%end item
\divideents{zoomIn}
\item{\vskip -1.9ex 
\membername{zoomIn}
{\tt public abstract void {\bf zoomIn}( {\tt uk.ac.ic.doc.neuralnets.graph.neural.NeuralNetwork } {\bf n} )
\label{l1479}\label{l1480}}%end signature
\begin{itemize}
\sld
\item{
\sld
{\bf Usage}
  \begin{itemize}\isep
   \item{
Zooms into the selected network layer. Clears the current view, and
 instead shows the contents of the selected network layer.
}%end item
  \end{itemize}
}
\item{
\sld
{\bf Parameters}
\sld\isep
  \begin{itemize}
\sld\isep
   \item{
\sld
{\tt n} - the network to zoom into.}
  \end{itemize}
}%end item
\end{itemize}
}%end item
\divideents{zoomOut}
\item{\vskip -1.9ex 
\membername{zoomOut}
{\tt public abstract void {\bf zoomOut}(  )
\label{l1481}\label{l1482}}%end signature
\begin{itemize}
\sld
\item{
\sld
{\bf Usage}
  \begin{itemize}\isep
   \item{
Zooms out one layer. Clears the current view, and instead shows the
 contents of the current layer's parent. If the current view is the root
 network, then nothing happens as it is not possible to zoom out further.
}%end item
  \end{itemize}
}
\end{itemize}
}%end item
\end{itemize}
}
\startsubsubsection{Methods inherited from class {\tt uk.ac.ic.doc.neuralnets.coreui.InterfaceManager}}{
\par{\small 
\refdefined{l107}\vskip -2em
\begin{itemize}
\item{\vskip -1.9ex 
\membername{addConnection}
{\tt public void {\bf addConnection}( {\tt uk.ac.ic.doc.neuralnets.graph.Edge } {\bf e} )
}%end signature
\begin{itemize}
\sld
\item{
\sld
{\bf Usage}
  \begin{itemize}\isep
   \item{
Adds the given edge to the current view, and redraws the screen as
 necessary.
}%end item
  \end{itemize}
}
\item{
\sld
{\bf Parameters}
\sld\isep
  \begin{itemize}
\sld\isep
   \item{
\sld
{\tt e} - }
  \end{itemize}
}%end item
\end{itemize}
}%end item
\divideents{addNetwork}
\item{\vskip -1.9ex 
\membername{addNetwork}
{\tt public void {\bf addNetwork}( {\tt uk.ac.ic.doc.neuralnets.graph.neural.NeuralNetwork } {\bf n} )
}%end signature
\begin{itemize}
\sld
\item{
\sld
{\bf Usage}
  \begin{itemize}\isep
   \item{
Adds the given neural network to the current view, and redraws the screen
 as necessary.
}%end item
  \end{itemize}
}
\item{
\sld
{\bf Parameters}
\sld\isep
  \begin{itemize}
\sld\isep
   \item{
\sld
{\tt n} - the neural network to add to the current section of the neural
            network}
  \end{itemize}
}%end item
\end{itemize}
}%end item
\divideents{addNeurone}
\item{\vskip -1.9ex 
\membername{addNeurone}
{\tt public void {\bf addNeurone}( {\tt uk.ac.ic.doc.neuralnets.graph.neural.Neurone } {\bf n} )
}%end signature
\begin{itemize}
\sld
\item{
\sld
{\bf Usage}
  \begin{itemize}\isep
   \item{
Adds the given neurone to the current view, and redraws the screen
 as necessary.
}%end item
  \end{itemize}
}
\item{
\sld
{\bf Parameters}
\sld\isep
  \begin{itemize}
\sld\isep
   \item{
\sld
{\tt n} - the neurone to add to the current section of the neural
            network}
  \end{itemize}
}%end item
\end{itemize}
}%end item
\divideents{addNode}
\item{\vskip -1.9ex 
\membername{addNode}
{\tt public void {\bf addNode}( {\tt uk.ac.ic.doc.neuralnets.graph.Node } {\bf n} )
}%end signature
\begin{itemize}
\sld
\item{
\sld
{\bf Usage}
  \begin{itemize}\isep
   \item{
Adds the given node to the current view, and redraws the screen
 as necessary.
}%end item
  \end{itemize}
}
\item{
\sld
{\bf Parameters}
\sld\isep
  \begin{itemize}
\sld\isep
   \item{
\sld
{\tt n} - the node to add to the current section of the neural
            network}
  \end{itemize}
}%end item
\end{itemize}
}%end item
\divideents{addNode}
\item{\vskip -1.9ex 
\membername{addNode}
{\tt public void {\bf addNode}( {\tt uk.ac.ic.doc.neuralnets.graph.neural.NodeSpecification } {\bf spec} )
}%end signature
\begin{itemize}
\sld
\item{
\sld
{\bf Usage}
  \begin{itemize}\isep
   \item{
Creates a node from the give specification, adds to the current view, and
 redraws the screen as necessary.
}%end item
  \end{itemize}
}
\item{
\sld
{\bf Parameters}
\sld\isep
  \begin{itemize}
\sld\isep
   \item{
\sld
{\tt spec} - the specification of the node to add to the current section of
            the neural network}
  \end{itemize}
}%end item
\end{itemize}
}%end item
\divideents{getCommandControl}
\item{\vskip -1.9ex 
\membername{getCommandControl}
{\tt public CommandControl {\bf getCommandControl}(  )
}%end signature
\begin{itemize}
\sld
\item{
\sld
{\bf Usage}
  \begin{itemize}\isep
   \item{
Gets the command control used by the GUIManager. This object handles the
 undo and redo stacks as commands are executed and undone.
}%end item
  \end{itemize}
}
\item{{\bf Returns} - 
the CommandControl object used by the GUIManager 
}%end item
\end{itemize}
}%end item
\divideents{getCurrentNetwork}
\item{\vskip -1.9ex 
\membername{getCurrentNetwork}
{\tt public abstract NeuralNetwork {\bf getCurrentNetwork}(  )
}%end signature
\begin{itemize}
\sld
\item{
\sld
{\bf Usage}
  \begin{itemize}\isep
   \item{
Returns the neural network layer currently being viewed in the
 GUIManager.
}%end item
  \end{itemize}
}
\item{{\bf Returns} - 
the current neural network layer 
}%end item
\end{itemize}
}%end item
\divideents{getGraph}
\item{\vskip -1.9ex 
\membername{getGraph}
{\tt public abstract Object {\bf getGraph}(  )
}%end signature
\begin{itemize}
\sld
\item{
\sld
{\bf Usage}
  \begin{itemize}\isep
   \item{
Returns the Graph representation used by this UI Manager.
}%end item
  \end{itemize}
}
\item{{\bf Returns} - 
the Graph that the Manager draws onto 
}%end item
\end{itemize}
}%end item
\divideents{getNode}
\item{\vskip -1.9ex 
\membername{getNode}
{\tt public abstract Object {\bf getNode}( {\tt uk.ac.ic.doc.neuralnets.graph.neural.Neurone } {\bf n} )
}%end signature
\begin{itemize}
\sld
\item{
\sld
{\bf Usage}
  \begin{itemize}\isep
   \item{
Finds the GUINode in the GUI corresponding to the given Neurone and
 returns it. Returns null if the given Neurone is not loaded in the GUI.
}%end item
  \end{itemize}
}
\item{
\sld
{\bf Parameters}
\sld\isep
  \begin{itemize}
\sld\isep
   \item{
\sld
{\tt n} - the Neurone to look up in the GUI}
  \end{itemize}
}%end item
\item{{\bf Returns} - 
the GUINode in the GUI corresponding to the given Neurone 
}%end item
\end{itemize}
}%end item
\divideents{getRootNetwork}
\item{\vskip -1.9ex 
\membername{getRootNetwork}
{\tt public NeuralNetwork {\bf getRootNetwork}(  )
}%end signature
\begin{itemize}
\sld
\item{
\sld
{\bf Usage}
  \begin{itemize}\isep
   \item{
Gets the root of the layered neural network stored in the GUIManager.
}%end item
  \end{itemize}
}
\item{{\bf Returns} - 
the root of the main neural network 
}%end item
\end{itemize}
}%end item
\divideents{getSaveLocation}
\item{\vskip -1.9ex 
\membername{getSaveLocation}
{\tt public FileSpecification {\bf getSaveLocation}(  )
}%end signature
\begin{itemize}
\sld
\item{
\sld
{\bf Usage}
  \begin{itemize}\isep
   \item{
Gets the location to save the network to, or null if no such location
 exists.
}%end item
  \end{itemize}
}
\item{{\bf Returns} - 
the network's save location, or null if none exists 
}%end item
\end{itemize}
}%end item
\divideents{getUtils}
\item{\vskip -1.9ex 
\membername{getUtils}
{\tt public InteractionUtils {\bf getUtils}(  )
}%end signature
\begin{itemize}
\sld
\item{
\sld
{\bf Usage}
  \begin{itemize}\isep
   \item{
Returns the GUIManager's interaction utilities.
}%end item
  \end{itemize}
}
\item{{\bf Returns} - 
the InteractionUtils object used by the GUIManager 
}%end item
\end{itemize}
}%end item
\divideents{persistLocations}
\item{\vskip -1.9ex 
\membername{persistLocations}
{\tt public abstract void {\bf persistLocations}(  )
}%end signature
\begin{itemize}
\sld
\item{
\sld
{\bf Usage}
  \begin{itemize}\isep
   \item{
Pushes down the locations of all Nodes to the model. Allows positions
 to be persisted to storage and reloaded.
}%end item
  \end{itemize}
}
\end{itemize}
}%end item
\divideents{redrawCurrentView}
\item{\vskip -1.9ex 
\membername{redrawCurrentView}
{\tt public abstract void {\bf redrawCurrentView}(  )
}%end signature
\begin{itemize}
\sld
\item{
\sld
{\bf Usage}
  \begin{itemize}\isep
   \item{
Draws the current view of the graph. Imports the current network layer
 from the internal model and applies the current layout.
}%end item
  \end{itemize}
}
\end{itemize}
}%end item
\divideents{remove}
\item{\vskip -1.9ex 
\membername{remove}
{\tt public abstract void {\bf remove}( {\tt java.lang.Object } {\bf i} )
}%end signature
\begin{itemize}
\sld
\item{
\sld
{\bf Usage}
  \begin{itemize}\isep
   \item{
Removes the given GraphItem from the view.
}%end item
  \end{itemize}
}
\item{
\sld
{\bf Parameters}
\sld\isep
  \begin{itemize}
\sld\isep
   \item{
\sld
{\tt i} - the graphitem to be removed from the view}
  \end{itemize}
}%end item
\end{itemize}
}%end item
\divideents{removeNetwork}
\item{\vskip -1.9ex 
\membername{removeNetwork}
{\tt public void {\bf removeNetwork}( {\tt uk.ac.ic.doc.neuralnets.graph.neural.NeuralNetwork } {\bf n} )
}%end signature
\begin{itemize}
\sld
\item{
\sld
{\bf Usage}
  \begin{itemize}\isep
   \item{
Removes the given neural network from the current view, and redraws the
 screen as necessary.
}%end item
  \end{itemize}
}
\item{
\sld
{\bf Parameters}
\sld\isep
  \begin{itemize}
\sld\isep
   \item{
\sld
{\tt n} - the neural network to remove from the current section of the neural
            network}
  \end{itemize}
}%end item
\end{itemize}
}%end item
\divideents{reset}
\item{\vskip -1.9ex 
\membername{reset}
{\tt protected abstract void {\bf reset}(  )
}%end signature
\begin{itemize}
\sld
\item{
\sld
{\bf Usage}
  \begin{itemize}\isep
   \item{
Reset the current manager, e.g. when a new network is loaded
}%end item
  \end{itemize}
}
\end{itemize}
}%end item
\divideents{setNetwork}
\item{\vskip -1.9ex 
\membername{setNetwork}
{\tt public void {\bf setNetwork}( {\tt uk.ac.ic.doc.neuralnets.graph.neural.NeuralNetwork } {\bf network},
{\tt uk.ac.ic.doc.neuralnets.persistence.FileSpecification } {\bf location} )
}%end signature
\begin{itemize}
\sld
\item{
\sld
{\bf Usage}
  \begin{itemize}\isep
   \item{
Loads the given neural network into the GUIManager, from the given
 location.
}%end item
  \end{itemize}
}
\item{
\sld
{\bf Parameters}
\sld\isep
  \begin{itemize}
\sld\isep
   \item{
\sld
{\tt network} - the network to be loaded into the GUIManager}
   \item{
\sld
{\tt location} - the location to load the network from}
  \end{itemize}
}%end item
\end{itemize}
}%end item
\divideents{setSaveLocation}
\item{\vskip -1.9ex 
\membername{setSaveLocation}
{\tt public void {\bf setSaveLocation}( {\tt uk.ac.ic.doc.neuralnets.persistence.FileSpecification } {\bf saveLoc} )
}%end signature
\begin{itemize}
\sld
\item{
\sld
{\bf Usage}
  \begin{itemize}\isep
   \item{
Sets the network's save location.
}%end item
  \end{itemize}
}
\item{
\sld
{\bf Parameters}
\sld\isep
  \begin{itemize}
\sld\isep
   \item{
\sld
{\tt saveLoc} - }
  \end{itemize}
}%end item
\end{itemize}
}%end item
\divideents{updateInterfaceHints}
\item{\vskip -1.9ex 
\membername{updateInterfaceHints}
{\tt public abstract void {\bf updateInterfaceHints}(  )
}%end signature
\begin{itemize}
\sld
\item{
\sld
{\bf Usage}
  \begin{itemize}\isep
   \item{
Updates the tooltips or other UI hints of all graph elements 
 in the current view.
}%end item
  \end{itemize}
}
\end{itemize}
}%end item
\end{itemize}
}}
}
}
}
\newpage
\def\packagename{uk.ac.ic.doc.neuralnets.events}
\chapter{\bf Package uk.ac.ic.doc.neuralnets.events}{
\vskip -.25in
\hbox to \hsize{\it Package Contents\hfil Page}
\rule{\hsize}{.7mm}
\vskip .13in
\hbox{\bf Interfaces}
\entityintro{EventHandler}{l1483}{Basic interface for EventHandlers}
\vskip .13in
\hbox{\bf Classes}
\entityintro{Event}{l220}{...no description...}
\entityintro{EventManager}{l1484}{...no description...}
\entityintro{GraphUpdateEvent}{l1485}{...no description...}
\entityintro{NumericalEvent}{l1273}{...no description...}
\entityintro{NumericalStatistician}{l1486}{...no description...}
\entityintro{RevalidateStatisticiansEvent}{l1157}{...no description...}
\entityintro{SingletonEvent}{l1258}{...no description...}
\vskip .1in
\rule{\hsize}{.7mm}
\vskip .1in
\newpage
\section{Interfaces}{
\startsection{Interface}{EventHandler}{l1483}{%
{\small Basic interface for EventHandlers}
\vskip .1in 
\startsubsubsection{Declaration}{
\fbox{\vbox{
\hbox{\vbox{\small public interface 
EventHandler}}
\noindent\hbox{\vbox{{\bf implements} 
uk.ac.ic.doc.neuralnets.util.plugins.Plugin}}
}}}
\startsubsubsection{Methods}{
\vskip -2em
\begin{itemize}
\item{\vskip -1.9ex 
\membername{flush}
{\tt public void {\bf flush}(  )
\label{l1487}\label{l1488}}%end signature
\begin{itemize}
\sld
\item{
\sld
{\bf Usage}
  \begin{itemize}\isep
   \item{
Instructs this handler to flush its buffers of data (usually
 indicating that execution has completed)
}%end item
  \end{itemize}
}
\end{itemize}
}%end item
\divideents{handle}
\item{\vskip -1.9ex 
\membername{handle}
{\tt public void {\bf handle}( {\tt uk.ac.ic.doc.neuralnets.events.Event } {\bf e} )
\label{l1489}\label{l1490}}%end signature
\begin{itemize}
\sld
\item{
\sld
{\bf Usage}
  \begin{itemize}\isep
   \item{
Fires an event at this handler
}%end item
  \end{itemize}
}
\item{
\sld
{\bf Parameters}
\sld\isep
  \begin{itemize}
\sld\isep
   \item{
\sld
{\tt e} - The event which has occurred}
  \end{itemize}
}%end item
\end{itemize}
}%end item
\divideents{isValid}
\item{\vskip -1.9ex 
\membername{isValid}
{\tt public boolean {\bf isValid}(  )
\label{l1491}\label{l1492}}%end signature
\begin{itemize}
\sld
\item{
\sld
{\bf Usage}
  \begin{itemize}\isep
   \item{
Answers whether or not this handler is valid for execution. If not,
 when a new Neural Network run begins the Statistician may be re-created
 by the StatisticsManager.
}%end item
  \end{itemize}
}
\item{{\bf Returns} - 
True iff this Statistician may process new input 
}%end item
\end{itemize}
}%end item
\end{itemize}
}
}
}
\section{Classes}{
\startsection{Class}{Event}{l220}{%
\startsubsubsection{Declaration}{
\fbox{\vbox{
\hbox{\vbox{\small public abstract 
class 
Event}}
\noindent\hbox{\vbox{{\bf extends} java.lang.Object}}
}}}
\startsubsubsection{Constructors}{
\vskip -2em
\begin{itemize}
\item{\vskip -1.9ex 
\membername{Event}
{\tt public {\bf Event}(  )
\label{l1493}\label{l1494}}%end signature
}%end item
\end{itemize}
}
\startsubsubsection{Methods}{
\vskip -2em
\begin{itemize}
\item{\vskip -1.9ex 
\membername{toString}
{\tt public abstract String {\bf toString}(  )
\label{l1495}\label{l1496}}%end signature
}%end item
\end{itemize}
}
}
\startsection{Class}{EventManager}{l1484}{%
\startsubsubsection{Declaration}{
\fbox{\vbox{
\hbox{\vbox{\small public 
class 
EventManager}}
\noindent\hbox{\vbox{{\bf extends} java.lang.Object}}
}}}
\startsubsubsection{Methods}{
\vskip -2em
\begin{itemize}
\item{\vskip -1.9ex 
\membername{deregisterAsync}
{\tt public void {\bf deregisterAsync}( {\tt java.lang.Class } {\bf c},
{\tt uk.ac.ic.doc.neuralnets.events.EventHandler } {\bf s} )
\label{l1497}\label{l1498}}%end signature
}%end item
\divideents{deregisterSynchro}
\item{\vskip -1.9ex 
\membername{deregisterSynchro}
{\tt public void {\bf deregisterSynchro}( {\tt java.lang.Class } {\bf c},
{\tt uk.ac.ic.doc.neuralnets.events.EventHandler } {\bf s} )
\label{l1499}\label{l1500}}%end signature
}%end item
\divideents{fire}
\item{\vskip -1.9ex 
\membername{fire}
{\tt public void {\bf fire}( {\tt uk.ac.ic.doc.neuralnets.events.Event } {\bf e} )
\label{l1501}\label{l1502}}%end signature
}%end item
\divideents{flush}
\item{\vskip -1.9ex 
\membername{flush}
{\tt public boolean {\bf flush}( {\tt java.lang.Class } {\bf e} )
\label{l1503}\label{l1504}}%end signature
}%end item
\divideents{flushAll}
\item{\vskip -1.9ex 
\membername{flushAll}
{\tt public void {\bf flushAll}(  )
\label{l1505}\label{l1506}}%end signature
}%end item
\divideents{get}
\item{\vskip -1.9ex 
\membername{get}
{\tt public static EventManager {\bf get}(  )
\label{l1507}\label{l1508}}%end signature
}%end item
\divideents{getUniqueID}
\item{\vskip -1.9ex 
\membername{getUniqueID}
{\tt public synchronized int {\bf getUniqueID}(  )
\label{l1509}\label{l1510}}%end signature
}%end item
\divideents{handle}
\item{\vskip -1.9ex 
\membername{handle}
{\tt protected void {\bf handle}( {\tt java.lang.Class } {\bf c},
{\tt uk.ac.ic.doc.neuralnets.events.Event } {\bf e},
{\tt java.util.Map } {\bf handlers} )
\label{l1511}\label{l1512}}%end signature
}%end item
\divideents{registerAsync}
\item{\vskip -1.9ex 
\membername{registerAsync}
{\tt public void {\bf registerAsync}( {\tt java.lang.Class } {\bf c},
{\tt uk.ac.ic.doc.neuralnets.events.EventHandler } {\bf s} )
\label{l1513}\label{l1514}}%end signature
}%end item
\divideents{registerSynchro}
\item{\vskip -1.9ex 
\membername{registerSynchro}
{\tt public void {\bf registerSynchro}( {\tt java.lang.Class } {\bf c},
{\tt uk.ac.ic.doc.neuralnets.events.EventHandler } {\bf s} )
\label{l1515}\label{l1516}}%end signature
}%end item
\end{itemize}
}
}
\startsection{Class}{GraphUpdateEvent}{l1485}{%
\startsubsubsection{Declaration}{
\fbox{\vbox{
\hbox{\vbox{\small public 
class 
GraphUpdateEvent}}
\noindent\hbox{\vbox{{\bf extends} uk.ac.ic.doc.neuralnets.events.Event}}
}}}
\startsubsubsection{Constructors}{
\vskip -2em
\begin{itemize}
\item{\vskip -1.9ex 
\membername{GraphUpdateEvent}
{\tt public {\bf GraphUpdateEvent}(  )
\label{l1517}\label{l1518}}%end signature
}%end item
\end{itemize}
}
\startsubsubsection{Methods}{
\vskip -2em
\begin{itemize}
\item{\vskip -1.9ex 
\membername{toString}
{\tt public String {\bf toString}(  )
\label{l1519}\label{l1520}}%end signature
}%end item
\end{itemize}
}
\startsubsubsection{Methods inherited from class {\tt uk.ac.ic.doc.neuralnets.events.Event}}{
\par{\small 
\refdefined{l220}\vskip -2em
\begin{itemize}
\item{\vskip -1.9ex 
\membername{toString}
{\tt public abstract String {\bf toString}(  )
}%end signature
}%end item
\end{itemize}
}}
}
\startsection{Class}{NumericalEvent}{l1273}{%
\startsubsubsection{Declaration}{
\fbox{\vbox{
\hbox{\vbox{\small public abstract 
class 
NumericalEvent}}
\noindent\hbox{\vbox{{\bf extends} uk.ac.ic.doc.neuralnets.events.Event}}
}}}
\startsubsubsection{Constructors}{
\vskip -2em
\begin{itemize}
\item{\vskip -1.9ex 
\membername{NumericalEvent}
{\tt public {\bf NumericalEvent}(  )
\label{l1521}\label{l1522}}%end signature
}%end item
\end{itemize}
}
\startsubsubsection{Methods}{
\vskip -2em
\begin{itemize}
\item{\vskip -1.9ex 
\membername{get}
{\tt public abstract double {\bf get}( {\tt int } {\bf idx} )
\label{l1523}\label{l1524}}%end signature
}%end item
\divideents{numPoints}
\item{\vskip -1.9ex 
\membername{numPoints}
{\tt public abstract double {\bf numPoints}(  )
\label{l1525}\label{l1526}}%end signature
}%end item
\divideents{push}
\item{\vskip -1.9ex 
\membername{push}
{\tt public abstract void {\bf push}( {\tt uk.ac.ic.doc.neuralnets.events.NumericalStatistician } {\bf s} )
\label{l1527}\label{l1528}}%end signature
}%end item
\end{itemize}
}
\startsubsubsection{Methods inherited from class {\tt uk.ac.ic.doc.neuralnets.events.Event}}{
\par{\small 
\refdefined{l220}\vskip -2em
\begin{itemize}
\item{\vskip -1.9ex 
\membername{toString}
{\tt public abstract String {\bf toString}(  )
}%end signature
}%end item
\end{itemize}
}}
}
\startsection{Class}{NumericalStatistician}{l1486}{%
\startsubsubsection{Declaration}{
\fbox{\vbox{
\hbox{\vbox{\small public abstract 
class 
NumericalStatistician}}
\noindent\hbox{\vbox{{\bf extends} java.lang.Object}}
\noindent\hbox{\vbox{{\bf implements} 
EventHandler}}
}}}
\startsubsubsection{Constructors}{
\vskip -2em
\begin{itemize}
\item{\vskip -1.9ex 
\membername{NumericalStatistician}
{\tt public {\bf NumericalStatistician}(  )
\label{l1529}\label{l1530}}%end signature
}%end item
\end{itemize}
}
\startsubsubsection{Methods}{
\vskip -2em
\begin{itemize}
\item{\vskip -1.9ex 
\membername{handle}
{\tt public void {\bf handle}( {\tt uk.ac.ic.doc.neuralnets.events.Event } {\bf e} )
\label{l1531}\label{l1532}}%end signature
}%end item
\divideents{handle}
\item{\vskip -1.9ex 
\membername{handle}
{\tt public void {\bf handle}( {\tt java.lang.Integer []} {\bf vs} )
\label{l1533}\label{l1534}}%end signature
}%end item
\divideents{handle}
\item{\vskip -1.9ex 
\membername{handle}
{\tt public void {\bf handle}( {\tt java.util.List } {\bf vs} )
\label{l1535}\label{l1536}}%end signature
}%end item
\divideents{handle}
\item{\vskip -1.9ex 
\membername{handle}
{\tt public void {\bf handle}( {\tt uk.ac.ic.doc.neuralnets.events.NumericalEvent } {\bf e} )
\label{l1537}\label{l1538}}%end signature
}%end item
\divideents{isValid}
\item{\vskip -1.9ex 
\membername{isValid}
{\tt public boolean {\bf isValid}(  )
\label{l1539}\label{l1540}}%end signature
}%end item
\divideents{saveAs}
\item{\vskip -1.9ex 
\membername{saveAs}
{\tt public void {\bf saveAs}( {\tt java.lang.String } {\bf file} )
\label{l1541}\label{l1542}}%end signature
}%end item
\end{itemize}
}
}
\startsection{Class}{RevalidateStatisticiansEvent}{l1157}{%
\startsubsubsection{Declaration}{
\fbox{\vbox{
\hbox{\vbox{\small public 
class 
RevalidateStatisticiansEvent}}
\noindent\hbox{\vbox{{\bf extends} uk.ac.ic.doc.neuralnets.events.Event}}
}}}
\startsubsubsection{Constructors}{
\vskip -2em
\begin{itemize}
\item{\vskip -1.9ex 
\membername{RevalidateStatisticiansEvent}
{\tt public {\bf RevalidateStatisticiansEvent}(  )
\label{l1543}\label{l1544}}%end signature
}%end item
\end{itemize}
}
\startsubsubsection{Methods}{
\vskip -2em
\begin{itemize}
\item{\vskip -1.9ex 
\membername{toString}
{\tt public String {\bf toString}(  )
\label{l1545}\label{l1546}}%end signature
}%end item
\end{itemize}
}
\startsubsubsection{Methods inherited from class {\tt uk.ac.ic.doc.neuralnets.events.Event}}{
\par{\small 
\refdefined{l220}\vskip -2em
\begin{itemize}
\item{\vskip -1.9ex 
\membername{toString}
{\tt public abstract String {\bf toString}(  )
}%end signature
}%end item
\end{itemize}
}}
}
\startsection{Class}{SingletonEvent}{l1258}{%
\startsubsubsection{Declaration}{
\fbox{\vbox{
\hbox{\vbox{\small public abstract 
class 
SingletonEvent}}
\noindent\hbox{\vbox{{\bf extends} uk.ac.ic.doc.neuralnets.events.Event}}
}}}
\startsubsubsection{Constructors}{
\vskip -2em
\begin{itemize}
\item{\vskip -1.9ex 
\membername{SingletonEvent}
{\tt public {\bf SingletonEvent}(  )
\label{l1547}\label{l1548}}%end signature
}%end item
\end{itemize}
}
\startsubsubsection{Methods}{
\vskip -2em
\begin{itemize}
\item{\vskip -1.9ex 
\membername{equals}
{\tt public abstract boolean {\bf equals}( {\tt java.lang.Object } {\bf o} )
\label{l1549}\label{l1550}}%end signature
}%end item
\end{itemize}
}
\startsubsubsection{Methods inherited from class {\tt uk.ac.ic.doc.neuralnets.events.Event}}{
\par{\small 
\refdefined{l220}\vskip -2em
\begin{itemize}
\item{\vskip -1.9ex 
\membername{toString}
{\tt public abstract String {\bf toString}(  )
}%end signature
}%end item
\end{itemize}
}}
}
}
}
\newpage
\def\packagename{uk.ac.ic.doc.neuralnets.util.reflect}
\chapter{\bf Package uk.ac.ic.doc.neuralnets.util.reflect}{
\vskip -.25in
\hbox to \hsize{\it Package Contents\hfil Page}
\rule{\hsize}{.7mm}
\vskip .13in
\hbox{\bf Classes}
\entityintro{MethodPseudoAccessor}{l1551}{...no description...}
\entityintro{ReflectionHelper}{l1552}{Used to perform potentially unsafe reflection - e.g.}
\vskip .1in
\rule{\hsize}{.7mm}
\vskip .1in
\newpage
\section{Classes}{
\startsection{Class}{MethodPseudoAccessor}{l1551}{%
\startsubsubsection{Declaration}{
\fbox{\vbox{
\hbox{\vbox{\small public 
class 
MethodPseudoAccessor}}
\noindent\hbox{\vbox{{\bf extends} java.lang.Object}}
\noindent\hbox{\vbox{{\bf implements} 
sun.reflect.FieldAccessor}}
}}}
\startsubsubsection{Constructors}{
\vskip -2em
\begin{itemize}
\item{\vskip -1.9ex 
\membername{MethodPseudoAccessor}
{\tt public {\bf MethodPseudoAccessor}( {\tt java.lang.Class } {\bf c},
{\tt java.lang.String } {\bf f} )
\label{l1553}\label{l1554}}%end signature
}%end item
\divideents{MethodPseudoAccessor}
\item{\vskip -1.9ex 
\membername{MethodPseudoAccessor}
{\tt public {\bf MethodPseudoAccessor}( {\tt java.lang.reflect.Field } {\bf f} )
\label{l1555}\label{l1556}}%end signature
}%end item
\end{itemize}
}
\startsubsubsection{Methods}{
\vskip -2em
\begin{itemize}
\item{\vskip -1.9ex 
\membername{get}
{\tt public Object {\bf get}( {\tt java.lang.Object } {\bf o} )
\label{l1557}\label{l1558}}%end signature
}%end item
\divideents{getBoolean}
\item{\vskip -1.9ex 
\membername{getBoolean}
{\tt public boolean {\bf getBoolean}( {\tt java.lang.Object } {\bf o} )
\label{l1559}\label{l1560}}%end signature
}%end item
\divideents{getByte}
\item{\vskip -1.9ex 
\membername{getByte}
{\tt public byte {\bf getByte}( {\tt java.lang.Object } {\bf o} )
\label{l1561}\label{l1562}}%end signature
}%end item
\divideents{getChar}
\item{\vskip -1.9ex 
\membername{getChar}
{\tt public char {\bf getChar}( {\tt java.lang.Object } {\bf o} )
\label{l1563}\label{l1564}}%end signature
}%end item
\divideents{getDouble}
\item{\vskip -1.9ex 
\membername{getDouble}
{\tt public double {\bf getDouble}( {\tt java.lang.Object } {\bf o} )
\label{l1565}\label{l1566}}%end signature
}%end item
\divideents{getFloat}
\item{\vskip -1.9ex 
\membername{getFloat}
{\tt public float {\bf getFloat}( {\tt java.lang.Object } {\bf o} )
\label{l1567}\label{l1568}}%end signature
}%end item
\divideents{getInt}
\item{\vskip -1.9ex 
\membername{getInt}
{\tt public int {\bf getInt}( {\tt java.lang.Object } {\bf o} )
\label{l1569}\label{l1570}}%end signature
}%end item
\divideents{getLong}
\item{\vskip -1.9ex 
\membername{getLong}
{\tt public long {\bf getLong}( {\tt java.lang.Object } {\bf o} )
\label{l1571}\label{l1572}}%end signature
}%end item
\divideents{getShort}
\item{\vskip -1.9ex 
\membername{getShort}
{\tt public short {\bf getShort}( {\tt java.lang.Object } {\bf o} )
\label{l1573}\label{l1574}}%end signature
}%end item
\divideents{set}
\item{\vskip -1.9ex 
\membername{set}
{\tt public void {\bf set}( {\tt java.lang.Object } {\bf o},
{\tt java.lang.Object } {\bf v} )
\label{l1575}\label{l1576}}%end signature
}%end item
\divideents{setBoolean}
\item{\vskip -1.9ex 
\membername{setBoolean}
{\tt public void {\bf setBoolean}( {\tt java.lang.Object } {\bf o},
{\tt boolean } {\bf b} )
\label{l1577}\label{l1578}}%end signature
}%end item
\divideents{setByte}
\item{\vskip -1.9ex 
\membername{setByte}
{\tt public void {\bf setByte}( {\tt java.lang.Object } {\bf o},
{\tt byte } {\bf b} )
\label{l1579}\label{l1580}}%end signature
}%end item
\divideents{setChar}
\item{\vskip -1.9ex 
\membername{setChar}
{\tt public void {\bf setChar}( {\tt java.lang.Object } {\bf o},
{\tt char } {\bf c} )
\label{l1581}\label{l1582}}%end signature
}%end item
\divideents{setDouble}
\item{\vskip -1.9ex 
\membername{setDouble}
{\tt public void {\bf setDouble}( {\tt java.lang.Object } {\bf o},
{\tt double } {\bf d} )
\label{l1583}\label{l1584}}%end signature
}%end item
\divideents{setFloat}
\item{\vskip -1.9ex 
\membername{setFloat}
{\tt public void {\bf setFloat}( {\tt java.lang.Object } {\bf o},
{\tt float } {\bf f} )
\label{l1585}\label{l1586}}%end signature
}%end item
\divideents{setInt}
\item{\vskip -1.9ex 
\membername{setInt}
{\tt public void {\bf setInt}( {\tt java.lang.Object } {\bf o},
{\tt int } {\bf i} )
\label{l1587}\label{l1588}}%end signature
}%end item
\divideents{setLong}
\item{\vskip -1.9ex 
\membername{setLong}
{\tt public void {\bf setLong}( {\tt java.lang.Object } {\bf o},
{\tt long } {\bf l} )
\label{l1589}\label{l1590}}%end signature
}%end item
\divideents{setShort}
\item{\vskip -1.9ex 
\membername{setShort}
{\tt public void {\bf setShort}( {\tt java.lang.Object } {\bf o},
{\tt short } {\bf s} )
\label{l1591}\label{l1592}}%end signature
}%end item
\end{itemize}
}
}
\startsection{Class}{ReflectionHelper}{l1552}{%
{\small Used to perform potentially unsafe reflection - e.g. setting private fields,
 or getting Fields that backend to Methods.}
\vskip .1in 
\startsubsubsection{Declaration}{
\fbox{\vbox{
\hbox{\vbox{\small public 
class 
ReflectionHelper}}
\noindent\hbox{\vbox{{\bf extends} java.lang.Object}}
}}}
\startsubsubsection{Constructors}{
\vskip -2em
\begin{itemize}
\item{\vskip -1.9ex 
\membername{ReflectionHelper}
{\tt public {\bf ReflectionHelper}(  )
\label{l1593}\label{l1594}}%end signature
}%end item
\end{itemize}
}
\startsubsubsection{Methods}{
\vskip -2em
\begin{itemize}
\item{\vskip -1.9ex 
\membername{getMethodField}
{\tt public static final Field {\bf getMethodField}( {\tt java.lang.String } {\bf m},
{\tt java.lang.Class } {\bf c} )
\label{l1595}\label{l1596}}%end signature
\begin{itemize}
\sld
\item{
\sld
{\bf Usage}
  \begin{itemize}\isep
   \item{
Get a Field object which backends data access to the given method name,
 from the supplied class
}%end item
  \end{itemize}
}
\item{
\sld
{\bf Parameters}
\sld\isep
  \begin{itemize}
\sld\isep
   \item{
\sld
{\tt m} - The name of the method}
   \item{
\sld
{\tt c} - The class to get the method from}
  \end{itemize}
}%end item
\item{{\bf Returns} - 
a Field with an accessor that backends to the requested Method 
}%end item
\item{{\bf Exceptions}
  \begin{itemize}
\sld
   \item{\vskip -.6ex{\tt java.lang.NoSuchMethodException} - }
   \item{\vskip -.6ex{\tt java.lang.IllegalArgumentException} - }
   \item{\vskip -.6ex{\tt java.lang.IllegalAccessException} - }
  \end{itemize}
}%end item
\end{itemize}
}%end item
\divideents{getReflectionFactory}
\item{\vskip -1.9ex 
\membername{getReflectionFactory}
{\tt public static final ReflectionFactory {\bf getReflectionFactory}(  )
\label{l1597}\label{l1598}}%end signature
\begin{itemize}
\sld
\item{
\sld
{\bf Usage}
  \begin{itemize}\isep
   \item{
Get the Sun-JVM-specific ReflectionFactory object (in an unsafe manner). 
 This allows us to assign values to and read from private Fields
}%end item
  \end{itemize}
}
\item{{\bf Returns} - 
the ReflectionFactory 
}%end item
\end{itemize}
}%end item
\divideents{set}
\item{\vskip -1.9ex 
\membername{set}
{\tt public static final void {\bf set}( {\tt java.lang.Class } {\bf c},
{\tt java.lang.String } {\bf fi},
{\tt java.lang.Object } {\bf target},
{\tt java.lang.Object } {\bf v} )
\label{l1599}\label{l1600}}%end signature
\begin{itemize}
\sld
\item{
\sld
{\bf Usage}
  \begin{itemize}\isep
   \item{
Find the requested Field declared in the given class, and set its value
 (irrespective of the field's modifiers)
}%end item
  \end{itemize}
}
\item{
\sld
{\bf Parameters}
\sld\isep
  \begin{itemize}
\sld\isep
   \item{
\sld
{\tt c} - The Class to look in}
   \item{
\sld
{\tt fi} - The field name to seek}
   \item{
\sld
{\tt target} - The target object}
   \item{
\sld
{\tt v} - The value to set the field to}
  \end{itemize}
}%end item
\item{{\bf Exceptions}
  \begin{itemize}
\sld
   \item{\vskip -.6ex{\tt java.lang.IllegalArgumentException} - }
   \item{\vskip -.6ex{\tt java.lang.IllegalAccessException} - }
  \end{itemize}
}%end item
\end{itemize}
}%end item
\divideents{set}
\item{\vskip -1.9ex 
\membername{set}
{\tt public static final void {\bf set}( {\tt java.lang.reflect.Field } {\bf f},
{\tt java.lang.Object } {\bf target},
{\tt java.lang.Object } {\bf v} )
\label{l1601}\label{l1602}}%end signature
\begin{itemize}
\sld
\item{
\sld
{\bf Usage}
  \begin{itemize}\isep
   \item{
Set the given field on target to value, irrespective of its modifiers
}%end item
  \end{itemize}
}
\item{
\sld
{\bf Parameters}
\sld\isep
  \begin{itemize}
\sld\isep
   \item{
\sld
{\tt f} - The Field to set}
   \item{
\sld
{\tt target} - The object to set it on}
   \item{
\sld
{\tt v} - The value to set the field to}
  \end{itemize}
}%end item
\item{{\bf Exceptions}
  \begin{itemize}
\sld
   \item{\vskip -.6ex{\tt java.lang.IllegalArgumentException} - }
   \item{\vskip -.6ex{\tt java.lang.IllegalAccessException} - }
  \end{itemize}
}%end item
\end{itemize}
}%end item
\divideents{set}
\item{\vskip -1.9ex 
\membername{set}
{\tt public static final void {\bf set}( {\tt java.lang.String } {\bf fi},
{\tt java.lang.Object } {\bf target},
{\tt java.lang.Object } {\bf v} )
\label{l1603}\label{l1604}}%end signature
\begin{itemize}
\sld
\item{
\sld
{\bf Usage}
  \begin{itemize}\isep
   \item{
Find the requested Field declared in the target object's class, and set 
 its value (irrespective of the field's modifiers)
}%end item
  \end{itemize}
}
\item{
\sld
{\bf Parameters}
\sld\isep
  \begin{itemize}
\sld\isep
   \item{
\sld
{\tt fi} - The field name to seek}
   \item{
\sld
{\tt target} - The target object}
   \item{
\sld
{\tt v} - The value to set the field to}
  \end{itemize}
}%end item
\item{{\bf Exceptions}
  \begin{itemize}
\sld
   \item{\vskip -.6ex{\tt java.lang.IllegalArgumentException} - }
   \item{\vskip -.6ex{\tt java.lang.IllegalAccessException} - }
  \end{itemize}
}%end item
\end{itemize}
}%end item
\end{itemize}
}
}
}
}
\newpage
\def\packagename{uk.ac.ic.doc.neuralnets.gui.graph}
\chapter{\bf Package uk.ac.ic.doc.neuralnets.gui.graph}{
\vskip -.25in
\hbox to \hsize{\it Package Contents\hfil Page}
\rule{\hsize}{.7mm}
\vskip .13in
\hbox{\bf Interfaces}
\entityintro{NodeContainer}{l1605}{Objects of this type contain a model Node.}
\vskip .13in
\hbox{\bf Classes}
\entityintro{CachingLayout}{l1606}{...no description...}
\entityintro{GUIAnchor}{l1607}{GUIAnchor acts as both a source and sink in a network to show what it
 connects to and what connects to it.}
\entityintro{GUIBridge}{l1608}{Connection between two GUI Networks containing links connecting nodes between
 each network}
\entityintro{GUIEdge}{l1609}{Represent a Synapse in the Zest graph.}
\entityintro{GUINetwork}{l1610}{...no description...}
\entityintro{GUINode}{l1611}{Represents a Neurone in the Zest graph.}
\vskip .1in
\rule{\hsize}{.7mm}
\vskip .1in
\newpage
\section{Interfaces}{
\startsection{Interface}{NodeContainer}{l1605}{%
{\small Objects of this type contain a model Node.}
\vskip .1in 
\startsubsubsection{Declaration}{
\fbox{\vbox{
\hbox{\vbox{\small public interface 
NodeContainer}}
}}}
\startsubsubsection{Methods}{
\vskip -2em
\begin{itemize}
\item{\vskip -1.9ex 
\membername{getNode}
{\tt public Node {\bf getNode}(  )
\label{l1612}\label{l1613}}%end signature
\begin{itemize}
\sld
\item{
\sld
{\bf Usage}
  \begin{itemize}\isep
   \item{
Get the node contained in the container.
}%end item
  \end{itemize}
}
\item{{\bf Returns} - 
the contained node 
}%end item
\end{itemize}
}%end item
\divideents{setNode}
\item{\vskip -1.9ex 
\membername{setNode}
{\tt public void {\bf setNode}( {\tt uk.ac.ic.doc.neuralnets.graph.Node } {\bf n} )
\label{l1614}\label{l1615}}%end signature
\begin{itemize}
\sld
\item{
\sld
{\bf Usage}
  \begin{itemize}\isep
   \item{
Set the node contained in the container.
}%end item
  \end{itemize}
}
\item{
\sld
{\bf Parameters}
\sld\isep
  \begin{itemize}
\sld\isep
   \item{
\sld
{\tt n} - }
  \end{itemize}
}%end item
\end{itemize}
}%end item
\end{itemize}
}
}
}
\section{Classes}{
\startsection{Class}{CachingLayout}{l1606}{%
\startsubsubsection{Declaration}{
\fbox{\vbox{
\hbox{\vbox{\small public 
class 
CachingLayout}}
\noindent\hbox{\vbox{{\bf extends} java.lang.Object}}
\noindent\hbox{\vbox{{\bf implements} 
org.eclipse.zest.layouts.LayoutAlgorithm}}
}}}
\startsubsubsection{Constructors}{
\vskip -2em
\begin{itemize}
\item{\vskip -1.9ex 
\membername{CachingLayout}
{\tt public {\bf CachingLayout}(  )
\label{l1616}\label{l1617}}%end signature
}%end item
\divideents{CachingLayout}
\item{\vskip -1.9ex 
\membername{CachingLayout}
{\tt public {\bf CachingLayout}( {\tt org.eclipse.zest.layouts.LayoutAlgorithm } {\bf child} )
\label{l1618}\label{l1619}}%end signature
}%end item
\divideents{CachingLayout}
\item{\vskip -1.9ex 
\membername{CachingLayout}
{\tt public {\bf CachingLayout}( {\tt org.eclipse.zest.layouts.LayoutAlgorithm } {\bf child},
{\tt boolean } {\bf useCache} )
\label{l1620}\label{l1621}}%end signature
}%end item
\end{itemize}
}
\startsubsubsection{Methods}{
\vskip -2em
\begin{itemize}
\item{\vskip -1.9ex 
\membername{addEntity}
{\tt public void {\bf addEntity}( {\tt org.eclipse.zest.layouts.LayoutEntity } {\bf entity} )
\label{l1622}\label{l1623}}%end signature
}%end item
\divideents{addProgressListener}
\item{\vskip -1.9ex 
\membername{addProgressListener}
{\tt public void {\bf addProgressListener}( {\tt org.eclipse.zest.layouts.progress.ProgressListener } {\bf listener} )
\label{l1624}\label{l1625}}%end signature
}%end item
\divideents{addRelationship}
\item{\vskip -1.9ex 
\membername{addRelationship}
{\tt public void {\bf addRelationship}( {\tt org.eclipse.zest.layouts.LayoutRelationship } {\bf relationship} )
\label{l1626}\label{l1627}}%end signature
}%end item
\divideents{applyLayout}
\item{\vskip -1.9ex 
\membername{applyLayout}
{\tt public void {\bf applyLayout}( {\tt org.eclipse.zest.layouts.LayoutEntity []} {\bf entitiesToLayout},
{\tt org.eclipse.zest.layouts.LayoutRelationship []} {\bf relationshipsToConsider},
{\tt double } {\bf x},
{\tt double } {\bf y},
{\tt double } {\bf width},
{\tt double } {\bf height},
{\tt boolean } {\bf asynchronous},
{\tt boolean } {\bf continuous} )
\label{l1628}\label{l1629}}%end signature
}%end item
\divideents{getEntityAspectRatio}
\item{\vskip -1.9ex 
\membername{getEntityAspectRatio}
{\tt public double {\bf getEntityAspectRatio}(  )
\label{l1630}\label{l1631}}%end signature
}%end item
\divideents{getStyle}
\item{\vskip -1.9ex 
\membername{getStyle}
{\tt public int {\bf getStyle}(  )
\label{l1632}\label{l1633}}%end signature
}%end item
\divideents{isRunning}
\item{\vskip -1.9ex 
\membername{isRunning}
{\tt public boolean {\bf isRunning}(  )
\label{l1634}\label{l1635}}%end signature
}%end item
\divideents{removeEntity}
\item{\vskip -1.9ex 
\membername{removeEntity}
{\tt public void {\bf removeEntity}( {\tt org.eclipse.zest.layouts.LayoutEntity } {\bf entity} )
\label{l1636}\label{l1637}}%end signature
}%end item
\divideents{removeProgressListener}
\item{\vskip -1.9ex 
\membername{removeProgressListener}
{\tt public void {\bf removeProgressListener}( {\tt org.eclipse.zest.layouts.progress.ProgressListener } {\bf listener} )
\label{l1638}\label{l1639}}%end signature
}%end item
\divideents{removeRelationship}
\item{\vskip -1.9ex 
\membername{removeRelationship}
{\tt public void {\bf removeRelationship}( {\tt org.eclipse.zest.layouts.LayoutRelationship } {\bf relationship} )
\label{l1640}\label{l1641}}%end signature
}%end item
\divideents{removeRelationships}
\item{\vskip -1.9ex 
\membername{removeRelationships}
{\tt public void {\bf removeRelationships}( {\tt java.util.List } {\bf relationships} )
\label{l1642}\label{l1643}}%end signature
}%end item
\divideents{setChildAlgorithm}
\item{\vskip -1.9ex 
\membername{setChildAlgorithm}
{\tt public void {\bf setChildAlgorithm}( {\tt org.eclipse.zest.layouts.LayoutAlgorithm } {\bf child} )
\label{l1644}\label{l1645}}%end signature
}%end item
\divideents{setComparator}
\item{\vskip -1.9ex 
\membername{setComparator}
{\tt public void {\bf setComparator}( {\tt java.util.Comparator } {\bf comparator} )
\label{l1646}\label{l1647}}%end signature
}%end item
\divideents{setEntityAspectRatio}
\item{\vskip -1.9ex 
\membername{setEntityAspectRatio}
{\tt public void {\bf setEntityAspectRatio}( {\tt double } {\bf ratio} )
\label{l1648}\label{l1649}}%end signature
}%end item
\divideents{setFilter}
\item{\vskip -1.9ex 
\membername{setFilter}
{\tt public void {\bf setFilter}( {\tt org.eclipse.zest.layouts.Filter } {\bf filter} )
\label{l1650}\label{l1651}}%end signature
}%end item
\divideents{setStyle}
\item{\vskip -1.9ex 
\membername{setStyle}
{\tt public void {\bf setStyle}( {\tt int } {\bf style} )
\label{l1652}\label{l1653}}%end signature
}%end item
\divideents{stop}
\item{\vskip -1.9ex 
\membername{stop}
{\tt public void {\bf stop}(  )
\label{l1654}\label{l1655}}%end signature
}%end item
\end{itemize}
}
}
\startsection{Class}{GUIAnchor}{l1607}{%
{\small GUIAnchor acts as both a source and sink in a network to show what it
 connects to and what connects to it.}
\vskip .1in 
\startsubsubsection{Declaration}{
\fbox{\vbox{
\hbox{\vbox{\small public 
class 
GUIAnchor}}
\noindent\hbox{\vbox{{\bf extends} org.eclipse.zest.core.widgets.GraphNode}}
\noindent\hbox{\vbox{{\bf implements} 
NodeContainer}}
}}}
\startsubsubsection{Constructors}{
\vskip -2em
\begin{itemize}
\item{\vskip -1.9ex 
\membername{GUIAnchor}
{\tt public {\bf GUIAnchor}( {\tt boolean } {\bf isSink},
{\tt uk.ac.ic.doc.neuralnets.graph.neural.NeuralNetwork } {\bf network},
{\tt org.eclipse.zest.core.widgets.IContainer } {\bf graphModel},
{\tt int } {\bf style} )
\label{l1656}\label{l1657}}%end signature
\begin{itemize}
\sld
\item{
\sld
{\bf Usage}
  \begin{itemize}\isep
   \item{
Creates a GUI Anchor.
}%end item
  \end{itemize}
}
\item{
\sld
{\bf Parameters}
\sld\isep
  \begin{itemize}
\sld\isep
   \item{
\sld
{\tt isSink} - It is a Sink Node if true, Source Node if false}
   \item{
\sld
{\tt network} - Network to add Anchor to}
   \item{
\sld
{\tt graphModel} - Graph to insert Anchor into}
   \item{
\sld
{\tt style} - Style of Anchor}
  \end{itemize}
}%end item
\end{itemize}
}%end item
\end{itemize}
}
\startsubsubsection{Methods}{
\vskip -2em
\begin{itemize}
\item{\vskip -1.9ex 
\membername{createFigureForModel}
{\tt protected IFigure {\bf createFigureForModel}(  )
\label{l1658}\label{l1659}}%end signature
}%end item
\divideents{createToolTip}
\item{\vskip -1.9ex 
\membername{createToolTip}
{\tt public void {\bf createToolTip}(  )
\label{l1660}\label{l1661}}%end signature
}%end item
\divideents{getNode}
\item{\vskip -1.9ex 
\membername{getNode}
{\tt public Node {\bf getNode}(  )
\label{l1662}\label{l1663}}%end signature
}%end item
\divideents{highlight}
\item{\vskip -1.9ex 
\membername{highlight}
{\tt public void {\bf highlight}(  )
\label{l1664}\label{l1665}}%end signature
\begin{itemize}
\sld
\item{
\sld
{\bf Usage}
  \begin{itemize}\isep
   \item{
Highlights the anchor node.
}%end item
  \end{itemize}
}
\end{itemize}
}%end item
\divideents{isSink}
\item{\vskip -1.9ex 
\membername{isSink}
{\tt public boolean {\bf isSink}(  )
\label{l1666}\label{l1667}}%end signature
}%end item
\divideents{setNode}
\item{\vskip -1.9ex 
\membername{setNode}
{\tt public void {\bf setNode}( {\tt uk.ac.ic.doc.neuralnets.graph.Node } {\bf network} )
\label{l1668}\label{l1669}}%end signature
}%end item
\divideents{unhighlight}
\item{\vskip -1.9ex 
\membername{unhighlight}
{\tt public void {\bf unhighlight}(  )
\label{l1670}\label{l1671}}%end signature
\begin{itemize}
\sld
\item{
\sld
{\bf Usage}
  \begin{itemize}\isep
   \item{
Unhighlights the anchor node.
}%end item
  \end{itemize}
}
\end{itemize}
}%end item
\end{itemize}
}
\startsubsubsection{Methods inherited from class {\tt org.eclipse.zest.core.widgets.GraphNode}}{
\par{\small 
\refdefined{l1672}\vskip -2em
\begin{itemize}
\item{\vskip -1.9ex 
\membername{cacheLabel}
{\tt public boolean {\bf cacheLabel}(  )
}%end signature
}%end item
\divideents{createFigureForModel}
\item{\vskip -1.9ex 
\membername{createFigureForModel}
{\tt protected IFigure {\bf createFigureForModel}(  )
}%end signature
}%end item
\divideents{dispose}
\item{\vskip -1.9ex 
\membername{dispose}
{\tt public void {\bf dispose}(  )
}%end signature
}%end item
\divideents{fishEye}
\item{\vskip -1.9ex 
\membername{fishEye}
{\tt protected IFigure {\bf fishEye}( {\tt boolean } {\bf arg0},
{\tt boolean } {\bf arg1} )
}%end signature
}%end item
\divideents{getBackgroundColor}
\item{\vskip -1.9ex 
\membername{getBackgroundColor}
{\tt public Color {\bf getBackgroundColor}(  )
}%end signature
}%end item
\divideents{getBorderColor}
\item{\vskip -1.9ex 
\membername{getBorderColor}
{\tt public Color {\bf getBorderColor}(  )
}%end signature
}%end item
\divideents{getBorderHighlightColor}
\item{\vskip -1.9ex 
\membername{getBorderHighlightColor}
{\tt public Color {\bf getBorderHighlightColor}(  )
}%end signature
}%end item
\divideents{getBorderWidth}
\item{\vskip -1.9ex 
\membername{getBorderWidth}
{\tt public int {\bf getBorderWidth}(  )
}%end signature
}%end item
\divideents{getFont}
\item{\vskip -1.9ex 
\membername{getFont}
{\tt public Font {\bf getFont}(  )
}%end signature
}%end item
\divideents{getForegroundColor}
\item{\vskip -1.9ex 
\membername{getForegroundColor}
{\tt public Color {\bf getForegroundColor}(  )
}%end signature
}%end item
\divideents{getGraphModel}
\item{\vskip -1.9ex 
\membername{getGraphModel}
{\tt public Graph {\bf getGraphModel}(  )
}%end signature
}%end item
\divideents{getHighlightColor}
\item{\vskip -1.9ex 
\membername{getHighlightColor}
{\tt public Color {\bf getHighlightColor}(  )
}%end signature
}%end item
\divideents{getItemType}
\item{\vskip -1.9ex 
\membername{getItemType}
{\tt public int {\bf getItemType}(  )
}%end signature
}%end item
\divideents{getLayoutEntity}
\item{\vskip -1.9ex 
\membername{getLayoutEntity}
{\tt public LayoutEntity {\bf getLayoutEntity}(  )
}%end signature
}%end item
\divideents{getLocation}
\item{\vskip -1.9ex 
\membername{getLocation}
{\tt public Point {\bf getLocation}(  )
}%end signature
}%end item
\divideents{getNodeFigure}
\item{\vskip -1.9ex 
\membername{getNodeFigure}
{\tt public IFigure {\bf getNodeFigure}(  )
}%end signature
}%end item
\divideents{getNodeStyle}
\item{\vskip -1.9ex 
\membername{getNodeStyle}
{\tt public int {\bf getNodeStyle}(  )
}%end signature
}%end item
\divideents{getSize}
\item{\vskip -1.9ex 
\membername{getSize}
{\tt public Dimension {\bf getSize}(  )
}%end signature
}%end item
\divideents{getSourceConnections}
\item{\vskip -1.9ex 
\membername{getSourceConnections}
{\tt public List {\bf getSourceConnections}(  )
}%end signature
}%end item
\divideents{getStyle}
\item{\vskip -1.9ex 
\membername{getStyle}
{\tt public int {\bf getStyle}(  )
}%end signature
}%end item
\divideents{getTargetConnections}
\item{\vskip -1.9ex 
\membername{getTargetConnections}
{\tt public List {\bf getTargetConnections}(  )
}%end signature
}%end item
\divideents{getTooltip}
\item{\vskip -1.9ex 
\membername{getTooltip}
{\tt public IFigure {\bf getTooltip}(  )
}%end signature
}%end item
\divideents{highlight}
\item{\vskip -1.9ex 
\membername{highlight}
{\tt public void {\bf highlight}(  )
}%end signature
}%end item
\divideents{initFigure}
\item{\vskip -1.9ex 
\membername{initFigure}
{\tt protected void {\bf initFigure}(  )
}%end signature
}%end item
\divideents{initModel}
\item{\vskip -1.9ex 
\membername{initModel}
{\tt protected void {\bf initModel}( {\tt org.eclipse.zest.core.widgets.IContainer } {\bf arg0},
{\tt java.lang.String } {\bf arg1},
{\tt org.eclipse.swt.graphics.Image } {\bf arg2} )
}%end signature
}%end item
\divideents{isDisposed}
\item{\vskip -1.9ex 
\membername{isDisposed}
{\tt public boolean {\bf isDisposed}(  )
}%end signature
}%end item
\divideents{isSelected}
\item{\vskip -1.9ex 
\membername{isSelected}
{\tt public boolean {\bf isSelected}(  )
}%end signature
}%end item
\divideents{isSizeFixed}
\item{\vskip -1.9ex 
\membername{isSizeFixed}
{\tt public boolean {\bf isSizeFixed}(  )
}%end signature
}%end item
\divideents{isVisible}
\item{\vskip -1.9ex 
\membername{isVisible}
{\tt public boolean {\bf isVisible}(  )
}%end signature
}%end item
\divideents{refreshLocation}
\item{\vskip -1.9ex 
\membername{refreshLocation}
{\tt protected void {\bf refreshLocation}(  )
}%end signature
}%end item
\divideents{setBackgroundColor}
\item{\vskip -1.9ex 
\membername{setBackgroundColor}
{\tt public void {\bf setBackgroundColor}( {\tt org.eclipse.swt.graphics.Color } {\bf arg0} )
}%end signature
}%end item
\divideents{setBorderColor}
\item{\vskip -1.9ex 
\membername{setBorderColor}
{\tt public void {\bf setBorderColor}( {\tt org.eclipse.swt.graphics.Color } {\bf arg0} )
}%end signature
}%end item
\divideents{setBorderHighlightColor}
\item{\vskip -1.9ex 
\membername{setBorderHighlightColor}
{\tt public void {\bf setBorderHighlightColor}( {\tt org.eclipse.swt.graphics.Color } {\bf arg0} )
}%end signature
}%end item
\divideents{setBorderWidth}
\item{\vskip -1.9ex 
\membername{setBorderWidth}
{\tt public void {\bf setBorderWidth}( {\tt int } {\bf arg0} )
}%end signature
}%end item
\divideents{setCacheLabel}
\item{\vskip -1.9ex 
\membername{setCacheLabel}
{\tt public void {\bf setCacheLabel}( {\tt boolean } {\bf arg0} )
}%end signature
}%end item
\divideents{setFont}
\item{\vskip -1.9ex 
\membername{setFont}
{\tt public void {\bf setFont}( {\tt org.eclipse.swt.graphics.Font } {\bf arg0} )
}%end signature
}%end item
\divideents{setForegroundColor}
\item{\vskip -1.9ex 
\membername{setForegroundColor}
{\tt public void {\bf setForegroundColor}( {\tt org.eclipse.swt.graphics.Color } {\bf arg0} )
}%end signature
}%end item
\divideents{setHighlightColor}
\item{\vskip -1.9ex 
\membername{setHighlightColor}
{\tt public void {\bf setHighlightColor}( {\tt org.eclipse.swt.graphics.Color } {\bf arg0} )
}%end signature
}%end item
\divideents{setImage}
\item{\vskip -1.9ex 
\membername{setImage}
{\tt public void {\bf setImage}( {\tt org.eclipse.swt.graphics.Image } {\bf arg0} )
}%end signature
}%end item
\divideents{setLocation}
\item{\vskip -1.9ex 
\membername{setLocation}
{\tt public void {\bf setLocation}( {\tt double } {\bf arg0},
{\tt double } {\bf arg1} )
}%end signature
}%end item
\divideents{setNodeStyle}
\item{\vskip -1.9ex 
\membername{setNodeStyle}
{\tt public void {\bf setNodeStyle}( {\tt int } {\bf arg0} )
}%end signature
}%end item
\divideents{setSize}
\item{\vskip -1.9ex 
\membername{setSize}
{\tt public void {\bf setSize}( {\tt double } {\bf arg0},
{\tt double } {\bf arg1} )
}%end signature
}%end item
\divideents{setText}
\item{\vskip -1.9ex 
\membername{setText}
{\tt public void {\bf setText}( {\tt java.lang.String } {\bf arg0} )
}%end signature
}%end item
\divideents{setTooltip}
\item{\vskip -1.9ex 
\membername{setTooltip}
{\tt public void {\bf setTooltip}( {\tt org.eclipse.draw2d.IFigure } {\bf arg0} )
}%end signature
}%end item
\divideents{setVisible}
\item{\vskip -1.9ex 
\membername{setVisible}
{\tt public void {\bf setVisible}( {\tt boolean } {\bf arg0} )
}%end signature
}%end item
\divideents{toString}
\item{\vskip -1.9ex 
\membername{toString}
{\tt public String {\bf toString}(  )
}%end signature
}%end item
\divideents{unhighlight}
\item{\vskip -1.9ex 
\membername{unhighlight}
{\tt public void {\bf unhighlight}(  )
}%end signature
}%end item
\divideents{updateFigureForModel}
\item{\vskip -1.9ex 
\membername{updateFigureForModel}
{\tt protected void {\bf updateFigureForModel}( {\tt org.eclipse.draw2d.IFigure } {\bf arg0} )
}%end signature
}%end item
\end{itemize}
}}
\startsubsubsection{Methods inherited from class {\tt org.eclipse.zest.core.widgets.GraphItem}}{
\par{\small 
\refdefined{l1673}\vskip -2em
\begin{itemize}
\item{\vskip -1.9ex 
\membername{checkStyle}
{\tt protected boolean {\bf checkStyle}( {\tt int } {\bf arg0} )
}%end signature
}%end item
\divideents{dispose}
\item{\vskip -1.9ex 
\membername{dispose}
{\tt public void {\bf dispose}(  )
}%end signature
}%end item
\divideents{getGraphModel}
\item{\vskip -1.9ex 
\membername{getGraphModel}
{\tt public abstract Graph {\bf getGraphModel}(  )
}%end signature
}%end item
\divideents{getItemType}
\item{\vskip -1.9ex 
\membername{getItemType}
{\tt public abstract int {\bf getItemType}(  )
}%end signature
}%end item
\divideents{highlight}
\item{\vskip -1.9ex 
\membername{highlight}
{\tt public abstract void {\bf highlight}(  )
}%end signature
}%end item
\divideents{isVisible}
\item{\vskip -1.9ex 
\membername{isVisible}
{\tt public abstract boolean {\bf isVisible}(  )
}%end signature
}%end item
\divideents{setVisible}
\item{\vskip -1.9ex 
\membername{setVisible}
{\tt public abstract void {\bf setVisible}( {\tt boolean } {\bf arg0} )
}%end signature
}%end item
\divideents{unhighlight}
\item{\vskip -1.9ex 
\membername{unhighlight}
{\tt public abstract void {\bf unhighlight}(  )
}%end signature
}%end item
\end{itemize}
}}
\startsubsubsection{Methods inherited from class {\tt org.eclipse.swt.widgets.Item}}{
\par{\small 
\refdefined{l1674}\vskip -2em
\begin{itemize}
\item{\vskip -1.9ex 
\membername{checkSubclass}
{\tt protected void {\bf checkSubclass}(  )
}%end signature
}%end item
\divideents{getImage}
\item{\vskip -1.9ex 
\membername{getImage}
{\tt public Image {\bf getImage}(  )
}%end signature
}%end item
\divideents{getText}
\item{\vskip -1.9ex 
\membername{getText}
{\tt public String {\bf getText}(  )
}%end signature
}%end item
\divideents{setImage}
\item{\vskip -1.9ex 
\membername{setImage}
{\tt public void {\bf setImage}( {\tt org.eclipse.swt.graphics.Image } {\bf arg0} )
}%end signature
}%end item
\divideents{setText}
\item{\vskip -1.9ex 
\membername{setText}
{\tt public void {\bf setText}( {\tt java.lang.String } {\bf arg0} )
}%end signature
}%end item
\end{itemize}
}}
\startsubsubsection{Methods inherited from class {\tt org.eclipse.swt.widgets.Widget}}{
\par{\small 
\refdefined{l1675}\vskip -2em
\begin{itemize}
\item{\vskip -1.9ex 
\membername{addDisposeListener}
{\tt public void {\bf addDisposeListener}( {\tt org.eclipse.swt.events.DisposeListener } {\bf arg0} )
}%end signature
}%end item
\divideents{addListener}
\item{\vskip -1.9ex 
\membername{addListener}
{\tt public void {\bf addListener}( {\tt int } {\bf arg0},
{\tt org.eclipse.swt.widgets.Listener } {\bf arg1} )
}%end signature
}%end item
\divideents{checkSubclass}
\item{\vskip -1.9ex 
\membername{checkSubclass}
{\tt protected void {\bf checkSubclass}(  )
}%end signature
}%end item
\divideents{checkWidget}
\item{\vskip -1.9ex 
\membername{checkWidget}
{\tt protected void {\bf checkWidget}(  )
}%end signature
}%end item
\divideents{dispose}
\item{\vskip -1.9ex 
\membername{dispose}
{\tt public void {\bf dispose}(  )
}%end signature
}%end item
\divideents{getData}
\item{\vskip -1.9ex 
\membername{getData}
{\tt public Object {\bf getData}(  )
}%end signature
}%end item
\divideents{getData}
\item{\vskip -1.9ex 
\membername{getData}
{\tt public Object {\bf getData}( {\tt java.lang.String } {\bf arg0} )
}%end signature
}%end item
\divideents{getDisplay}
\item{\vskip -1.9ex 
\membername{getDisplay}
{\tt public Display {\bf getDisplay}(  )
}%end signature
}%end item
\divideents{getListeners}
\item{\vskip -1.9ex 
\membername{getListeners}
{\tt public Listener {\bf getListeners}( {\tt int } {\bf arg0} )
}%end signature
}%end item
\divideents{getStyle}
\item{\vskip -1.9ex 
\membername{getStyle}
{\tt public int {\bf getStyle}(  )
}%end signature
}%end item
\divideents{isDisposed}
\item{\vskip -1.9ex 
\membername{isDisposed}
{\tt public boolean {\bf isDisposed}(  )
}%end signature
}%end item
\divideents{isListening}
\item{\vskip -1.9ex 
\membername{isListening}
{\tt public boolean {\bf isListening}( {\tt int } {\bf arg0} )
}%end signature
}%end item
\divideents{notifyListeners}
\item{\vskip -1.9ex 
\membername{notifyListeners}
{\tt public void {\bf notifyListeners}( {\tt int } {\bf arg0},
{\tt org.eclipse.swt.widgets.Event } {\bf arg1} )
}%end signature
}%end item
\divideents{removeDisposeListener}
\item{\vskip -1.9ex 
\membername{removeDisposeListener}
{\tt public void {\bf removeDisposeListener}( {\tt org.eclipse.swt.events.DisposeListener } {\bf arg0} )
}%end signature
}%end item
\divideents{removeListener}
\item{\vskip -1.9ex 
\membername{removeListener}
{\tt public void {\bf removeListener}( {\tt int } {\bf arg0},
{\tt org.eclipse.swt.widgets.Listener } {\bf arg1} )
}%end signature
}%end item
\divideents{removeListener}
\item{\vskip -1.9ex 
\membername{removeListener}
{\tt protected void {\bf removeListener}( {\tt int } {\bf arg0},
{\tt org.eclipse.swt.internal.SWTEventListener } {\bf arg1} )
}%end signature
}%end item
\divideents{setData}
\item{\vskip -1.9ex 
\membername{setData}
{\tt public void {\bf setData}( {\tt java.lang.Object } {\bf arg0} )
}%end signature
}%end item
\divideents{setData}
\item{\vskip -1.9ex 
\membername{setData}
{\tt public void {\bf setData}( {\tt java.lang.String } {\bf arg0},
{\tt java.lang.Object } {\bf arg1} )
}%end signature
}%end item
\divideents{toString}
\item{\vskip -1.9ex 
\membername{toString}
{\tt public String {\bf toString}(  )
}%end signature
}%end item
\end{itemize}
}}
}
\startsection{Class}{GUIBridge}{l1608}{%
{\small Connection between two GUI Networks containing links connecting nodes between
 each network}
\vskip .1in 
\startsubsubsection{Declaration}{
\fbox{\vbox{
\hbox{\vbox{\small public 
class 
GUIBridge}}
\noindent\hbox{\vbox{{\bf extends} org.eclipse.zest.core.widgets.GraphConnection}}
}}}
\startsubsubsection{Constructors}{
\vskip -2em
\begin{itemize}
\item{\vskip -1.9ex 
\membername{GUIBridge}
{\tt public {\bf GUIBridge}( {\tt uk.ac.ic.doc.neuralnets.graph.neural.NetworkBridge } {\bf bridge},
{\tt org.eclipse.zest.core.widgets.Graph } {\bf graphModel},
{\tt int } {\bf style},
{\tt org.eclipse.zest.core.widgets.GraphNode } {\bf source},
{\tt org.eclipse.zest.core.widgets.GraphNode } {\bf destination} )
\label{l1676}\label{l1677}}%end signature
\begin{itemize}
\sld
\item{
\sld
{\bf Usage}
  \begin{itemize}\isep
   \item{
Create GUI Bridge that connects two GUI Networks in the UI.
}%end item
  \end{itemize}
}
\item{
\sld
{\bf Parameters}
\sld\isep
  \begin{itemize}
\sld\isep
   \item{
\sld
{\tt bridge} - Network Bridge between the neural networks}
   \item{
\sld
{\tt graphModel} - Graph that the bridge is inserted into}
   \item{
\sld
{\tt style} - Style of edge}
   \item{
\sld
{\tt source} - Start point of bridge}
   \item{
\sld
{\tt destination} - End point of bridge}
  \end{itemize}
}%end item
\end{itemize}
}%end item
\end{itemize}
}
\startsubsubsection{Methods}{
\vskip -2em
\begin{itemize}
\item{\vskip -1.9ex 
\membername{createToolTip}
{\tt public void {\bf createToolTip}(  )
\label{l1678}\label{l1679}}%end signature
}%end item
\divideents{getBridge}
\item{\vskip -1.9ex 
\membername{getBridge}
{\tt public NetworkBridge {\bf getBridge}(  )
\label{l1680}\label{l1681}}%end signature
}%end item
\divideents{setBridge}
\item{\vskip -1.9ex 
\membername{setBridge}
{\tt public void {\bf setBridge}( {\tt uk.ac.ic.doc.neuralnets.graph.neural.NetworkBridge } {\bf bridge} )
\label{l1682}\label{l1683}}%end signature
}%end item
\end{itemize}
}
\startsubsubsection{Methods inherited from class {\tt org.eclipse.zest.core.widgets.GraphConnection}}{
\par{\small 
\refdefined{l1684}\vskip -2em
\begin{itemize}
\item{\vskip -1.9ex 
\membername{changeLineColor}
{\tt public void {\bf changeLineColor}( {\tt org.eclipse.swt.graphics.Color } {\bf arg0} )
}%end signature
}%end item
\divideents{dispose}
\item{\vskip -1.9ex 
\membername{dispose}
{\tt public void {\bf dispose}(  )
}%end signature
}%end item
\divideents{getConnectionFigure}
\item{\vskip -1.9ex 
\membername{getConnectionFigure}
{\tt public Connection {\bf getConnectionFigure}(  )
}%end signature
}%end item
\divideents{getConnectionStyle}
\item{\vskip -1.9ex 
\membername{getConnectionStyle}
{\tt public int {\bf getConnectionStyle}(  )
}%end signature
}%end item
\divideents{getDestination}
\item{\vskip -1.9ex 
\membername{getDestination}
{\tt public GraphNode {\bf getDestination}(  )
}%end signature
}%end item
\divideents{getExternalConnection}
\item{\vskip -1.9ex 
\membername{getExternalConnection}
{\tt public Object {\bf getExternalConnection}(  )
}%end signature
}%end item
\divideents{getFont}
\item{\vskip -1.9ex 
\membername{getFont}
{\tt public Font {\bf getFont}(  )
}%end signature
}%end item
\divideents{getGraphModel}
\item{\vskip -1.9ex 
\membername{getGraphModel}
{\tt public Graph {\bf getGraphModel}(  )
}%end signature
}%end item
\divideents{getHighlightColor}
\item{\vskip -1.9ex 
\membername{getHighlightColor}
{\tt public Color {\bf getHighlightColor}(  )
}%end signature
}%end item
\divideents{getItemType}
\item{\vskip -1.9ex 
\membername{getItemType}
{\tt public int {\bf getItemType}(  )
}%end signature
}%end item
\divideents{getLayoutRelationship}
\item{\vskip -1.9ex 
\membername{getLayoutRelationship}
{\tt public LayoutRelationship {\bf getLayoutRelationship}(  )
}%end signature
}%end item
\divideents{getLineColor}
\item{\vskip -1.9ex 
\membername{getLineColor}
{\tt public Color {\bf getLineColor}(  )
}%end signature
}%end item
\divideents{getLineStyle}
\item{\vskip -1.9ex 
\membername{getLineStyle}
{\tt public int {\bf getLineStyle}(  )
}%end signature
}%end item
\divideents{getLineWidth}
\item{\vskip -1.9ex 
\membername{getLineWidth}
{\tt public int {\bf getLineWidth}(  )
}%end signature
}%end item
\divideents{getSource}
\item{\vskip -1.9ex 
\membername{getSource}
{\tt public GraphNode {\bf getSource}(  )
}%end signature
}%end item
\divideents{getTooltip}
\item{\vskip -1.9ex 
\membername{getTooltip}
{\tt public IFigure {\bf getTooltip}(  )
}%end signature
}%end item
\divideents{getWeightInLayout}
\item{\vskip -1.9ex 
\membername{getWeightInLayout}
{\tt public double {\bf getWeightInLayout}(  )
}%end signature
}%end item
\divideents{highlight}
\item{\vskip -1.9ex 
\membername{highlight}
{\tt public void {\bf highlight}(  )
}%end signature
}%end item
\divideents{isDisposed}
\item{\vskip -1.9ex 
\membername{isDisposed}
{\tt public boolean {\bf isDisposed}(  )
}%end signature
}%end item
\divideents{isHighlighted}
\item{\vskip -1.9ex 
\membername{isHighlighted}
{\tt public boolean {\bf isHighlighted}(  )
}%end signature
}%end item
\divideents{isVisible}
\item{\vskip -1.9ex 
\membername{isVisible}
{\tt public boolean {\bf isVisible}(  )
}%end signature
}%end item
\divideents{setConnectionStyle}
\item{\vskip -1.9ex 
\membername{setConnectionStyle}
{\tt public void {\bf setConnectionStyle}( {\tt int } {\bf arg0} )
}%end signature
}%end item
\divideents{setFont}
\item{\vskip -1.9ex 
\membername{setFont}
{\tt public void {\bf setFont}( {\tt org.eclipse.swt.graphics.Font } {\bf arg0} )
}%end signature
}%end item
\divideents{setHighlightColor}
\item{\vskip -1.9ex 
\membername{setHighlightColor}
{\tt public void {\bf setHighlightColor}( {\tt org.eclipse.swt.graphics.Color } {\bf arg0} )
}%end signature
}%end item
\divideents{setLineColor}
\item{\vskip -1.9ex 
\membername{setLineColor}
{\tt public void {\bf setLineColor}( {\tt org.eclipse.swt.graphics.Color } {\bf arg0} )
}%end signature
}%end item
\divideents{setLineStyle}
\item{\vskip -1.9ex 
\membername{setLineStyle}
{\tt public void {\bf setLineStyle}( {\tt int } {\bf arg0} )
}%end signature
}%end item
\divideents{setLineWidth}
\item{\vskip -1.9ex 
\membername{setLineWidth}
{\tt public void {\bf setLineWidth}( {\tt int } {\bf arg0} )
}%end signature
}%end item
\divideents{setText}
\item{\vskip -1.9ex 
\membername{setText}
{\tt public void {\bf setText}( {\tt java.lang.String } {\bf arg0} )
}%end signature
}%end item
\divideents{setTooltip}
\item{\vskip -1.9ex 
\membername{setTooltip}
{\tt public void {\bf setTooltip}( {\tt org.eclipse.draw2d.IFigure } {\bf arg0} )
}%end signature
}%end item
\divideents{setVisible}
\item{\vskip -1.9ex 
\membername{setVisible}
{\tt public void {\bf setVisible}( {\tt boolean } {\bf arg0} )
}%end signature
}%end item
\divideents{setWeight}
\item{\vskip -1.9ex 
\membername{setWeight}
{\tt public void {\bf setWeight}( {\tt double } {\bf arg0} )
}%end signature
}%end item
\divideents{toString}
\item{\vskip -1.9ex 
\membername{toString}
{\tt public String {\bf toString}(  )
}%end signature
}%end item
\divideents{unhighlight}
\item{\vskip -1.9ex 
\membername{unhighlight}
{\tt public void {\bf unhighlight}(  )
}%end signature
}%end item
\end{itemize}
}}
\startsubsubsection{Methods inherited from class {\tt org.eclipse.zest.core.widgets.GraphItem}}{
\par{\small 
\refdefined{l1673}\vskip -2em
\begin{itemize}
\item{\vskip -1.9ex 
\membername{checkStyle}
{\tt protected boolean {\bf checkStyle}( {\tt int } {\bf arg0} )
}%end signature
}%end item
\divideents{dispose}
\item{\vskip -1.9ex 
\membername{dispose}
{\tt public void {\bf dispose}(  )
}%end signature
}%end item
\divideents{getGraphModel}
\item{\vskip -1.9ex 
\membername{getGraphModel}
{\tt public abstract Graph {\bf getGraphModel}(  )
}%end signature
}%end item
\divideents{getItemType}
\item{\vskip -1.9ex 
\membername{getItemType}
{\tt public abstract int {\bf getItemType}(  )
}%end signature
}%end item
\divideents{highlight}
\item{\vskip -1.9ex 
\membername{highlight}
{\tt public abstract void {\bf highlight}(  )
}%end signature
}%end item
\divideents{isVisible}
\item{\vskip -1.9ex 
\membername{isVisible}
{\tt public abstract boolean {\bf isVisible}(  )
}%end signature
}%end item
\divideents{setVisible}
\item{\vskip -1.9ex 
\membername{setVisible}
{\tt public abstract void {\bf setVisible}( {\tt boolean } {\bf arg0} )
}%end signature
}%end item
\divideents{unhighlight}
\item{\vskip -1.9ex 
\membername{unhighlight}
{\tt public abstract void {\bf unhighlight}(  )
}%end signature
}%end item
\end{itemize}
}}
\startsubsubsection{Methods inherited from class {\tt org.eclipse.swt.widgets.Item}}{
\par{\small 
\refdefined{l1674}\vskip -2em
\begin{itemize}
\item{\vskip -1.9ex 
\membername{checkSubclass}
{\tt protected void {\bf checkSubclass}(  )
}%end signature
}%end item
\divideents{getImage}
\item{\vskip -1.9ex 
\membername{getImage}
{\tt public Image {\bf getImage}(  )
}%end signature
}%end item
\divideents{getText}
\item{\vskip -1.9ex 
\membername{getText}
{\tt public String {\bf getText}(  )
}%end signature
}%end item
\divideents{setImage}
\item{\vskip -1.9ex 
\membername{setImage}
{\tt public void {\bf setImage}( {\tt org.eclipse.swt.graphics.Image } {\bf arg0} )
}%end signature
}%end item
\divideents{setText}
\item{\vskip -1.9ex 
\membername{setText}
{\tt public void {\bf setText}( {\tt java.lang.String } {\bf arg0} )
}%end signature
}%end item
\end{itemize}
}}
\startsubsubsection{Methods inherited from class {\tt org.eclipse.swt.widgets.Widget}}{
\par{\small 
\refdefined{l1675}\vskip -2em
\begin{itemize}
\item{\vskip -1.9ex 
\membername{addDisposeListener}
{\tt public void {\bf addDisposeListener}( {\tt org.eclipse.swt.events.DisposeListener } {\bf arg0} )
}%end signature
}%end item
\divideents{addListener}
\item{\vskip -1.9ex 
\membername{addListener}
{\tt public void {\bf addListener}( {\tt int } {\bf arg0},
{\tt org.eclipse.swt.widgets.Listener } {\bf arg1} )
}%end signature
}%end item
\divideents{checkSubclass}
\item{\vskip -1.9ex 
\membername{checkSubclass}
{\tt protected void {\bf checkSubclass}(  )
}%end signature
}%end item
\divideents{checkWidget}
\item{\vskip -1.9ex 
\membername{checkWidget}
{\tt protected void {\bf checkWidget}(  )
}%end signature
}%end item
\divideents{dispose}
\item{\vskip -1.9ex 
\membername{dispose}
{\tt public void {\bf dispose}(  )
}%end signature
}%end item
\divideents{getData}
\item{\vskip -1.9ex 
\membername{getData}
{\tt public Object {\bf getData}(  )
}%end signature
}%end item
\divideents{getData}
\item{\vskip -1.9ex 
\membername{getData}
{\tt public Object {\bf getData}( {\tt java.lang.String } {\bf arg0} )
}%end signature
}%end item
\divideents{getDisplay}
\item{\vskip -1.9ex 
\membername{getDisplay}
{\tt public Display {\bf getDisplay}(  )
}%end signature
}%end item
\divideents{getListeners}
\item{\vskip -1.9ex 
\membername{getListeners}
{\tt public Listener {\bf getListeners}( {\tt int } {\bf arg0} )
}%end signature
}%end item
\divideents{getStyle}
\item{\vskip -1.9ex 
\membername{getStyle}
{\tt public int {\bf getStyle}(  )
}%end signature
}%end item
\divideents{isDisposed}
\item{\vskip -1.9ex 
\membername{isDisposed}
{\tt public boolean {\bf isDisposed}(  )
}%end signature
}%end item
\divideents{isListening}
\item{\vskip -1.9ex 
\membername{isListening}
{\tt public boolean {\bf isListening}( {\tt int } {\bf arg0} )
}%end signature
}%end item
\divideents{notifyListeners}
\item{\vskip -1.9ex 
\membername{notifyListeners}
{\tt public void {\bf notifyListeners}( {\tt int } {\bf arg0},
{\tt org.eclipse.swt.widgets.Event } {\bf arg1} )
}%end signature
}%end item
\divideents{removeDisposeListener}
\item{\vskip -1.9ex 
\membername{removeDisposeListener}
{\tt public void {\bf removeDisposeListener}( {\tt org.eclipse.swt.events.DisposeListener } {\bf arg0} )
}%end signature
}%end item
\divideents{removeListener}
\item{\vskip -1.9ex 
\membername{removeListener}
{\tt public void {\bf removeListener}( {\tt int } {\bf arg0},
{\tt org.eclipse.swt.widgets.Listener } {\bf arg1} )
}%end signature
}%end item
\divideents{removeListener}
\item{\vskip -1.9ex 
\membername{removeListener}
{\tt protected void {\bf removeListener}( {\tt int } {\bf arg0},
{\tt org.eclipse.swt.internal.SWTEventListener } {\bf arg1} )
}%end signature
}%end item
\divideents{setData}
\item{\vskip -1.9ex 
\membername{setData}
{\tt public void {\bf setData}( {\tt java.lang.Object } {\bf arg0} )
}%end signature
}%end item
\divideents{setData}
\item{\vskip -1.9ex 
\membername{setData}
{\tt public void {\bf setData}( {\tt java.lang.String } {\bf arg0},
{\tt java.lang.Object } {\bf arg1} )
}%end signature
}%end item
\divideents{toString}
\item{\vskip -1.9ex 
\membername{toString}
{\tt public String {\bf toString}(  )
}%end signature
}%end item
\end{itemize}
}}
}
\startsection{Class}{GUIEdge}{l1609}{%
{\small Represent a Synapse in the Zest graph.}
\vskip .1in 
\startsubsubsection{Declaration}{
\fbox{\vbox{
\hbox{\vbox{\small public 
class 
GUIEdge}}
\noindent\hbox{\vbox{{\bf extends} org.eclipse.zest.core.widgets.GraphConnection}}
}}}
\startsubsubsection{Constructors}{
\vskip -2em
\begin{itemize}
\item{\vskip -1.9ex 
\membername{GUIEdge}
{\tt public {\bf GUIEdge}( {\tt uk.ac.ic.doc.neuralnets.graph.Edge } {\bf edge},
{\tt org.eclipse.zest.core.widgets.Graph } {\bf graphModel},
{\tt int } {\bf style},
{\tt org.eclipse.zest.core.widgets.GraphNode } {\bf source},
{\tt org.eclipse.zest.core.widgets.GraphNode } {\bf destination} )
\label{l1685}\label{l1686}}%end signature
\begin{itemize}
\sld
\item{
\sld
{\bf Usage}
  \begin{itemize}\isep
   \item{
Creates a new edge in the specified graph for a Synapse. The edge
 decoration is set through the node specification, essentially ignoring
 the specified edge style.
}%end item
  \end{itemize}
}
\item{
\sld
{\bf Parameters}
\sld\isep
  \begin{itemize}
\sld\isep
   \item{
\sld
{\tt edge} - - the synapse to represent.}
   \item{
\sld
{\tt graphModel} - - the graph into which to insert the edge}
   \item{
\sld
{\tt style} - - the style of the edge (see ZestStyles) - this is overridden}
   \item{
\sld
{\tt source} - - the start point of the edge.}
   \item{
\sld
{\tt destination} - - the end point of the edge.}
  \end{itemize}
}%end item
\end{itemize}
}%end item
\end{itemize}
}
\startsubsubsection{Methods}{
\vskip -2em
\begin{itemize}
\item{\vskip -1.9ex 
\membername{createToolTip}
{\tt public void {\bf createToolTip}(  )
\label{l1687}\label{l1688}}%end signature
}%end item
\divideents{getEdge}
\item{\vskip -1.9ex 
\membername{getEdge}
{\tt public Edge {\bf getEdge}(  )
\label{l1689}\label{l1690}}%end signature
\begin{itemize}
\sld
\item{
\sld
{\bf Usage}
  \begin{itemize}\isep
   \item{
Get the Synapse represented.
}%end item
  \end{itemize}
}
\item{{\bf Returns} - 
the synapse edge. 
}%end item
\end{itemize}
}%end item
\divideents{highlight}
\item{\vskip -1.9ex 
\membername{highlight}
{\tt public void {\bf highlight}(  )
\label{l1691}\label{l1692}}%end signature
\begin{itemize}
\sld
\item{
\sld
{\bf Usage}
  \begin{itemize}\isep
   \item{
Unhighlight the edge
}%end item
  \end{itemize}
}
\end{itemize}
}%end item
\divideents{setEdge}
\item{\vskip -1.9ex 
\membername{setEdge}
{\tt public void {\bf setEdge}( {\tt uk.ac.ic.doc.neuralnets.graph.Edge } {\bf edge} )
\label{l1693}\label{l1694}}%end signature
\begin{itemize}
\sld
\item{
\sld
{\bf Usage}
  \begin{itemize}\isep
   \item{
Set the Synapse represented.
}%end item
  \end{itemize}
}
\item{
\sld
{\bf Parameters}
\sld\isep
  \begin{itemize}
\sld\isep
   \item{
\sld
{\tt edge} - - synapse to represent.}
  \end{itemize}
}%end item
\end{itemize}
}%end item
\divideents{unhighlight}
\item{\vskip -1.9ex 
\membername{unhighlight}
{\tt public void {\bf unhighlight}(  )
\label{l1695}\label{l1696}}%end signature
\begin{itemize}
\sld
\item{
\sld
{\bf Usage}
  \begin{itemize}\isep
   \item{
Highlight the edge.
}%end item
  \end{itemize}
}
\end{itemize}
}%end item
\end{itemize}
}
\startsubsubsection{Methods inherited from class {\tt org.eclipse.zest.core.widgets.GraphConnection}}{
\par{\small 
\refdefined{l1684}\vskip -2em
\begin{itemize}
\item{\vskip -1.9ex 
\membername{changeLineColor}
{\tt public void {\bf changeLineColor}( {\tt org.eclipse.swt.graphics.Color } {\bf arg0} )
}%end signature
}%end item
\divideents{dispose}
\item{\vskip -1.9ex 
\membername{dispose}
{\tt public void {\bf dispose}(  )
}%end signature
}%end item
\divideents{getConnectionFigure}
\item{\vskip -1.9ex 
\membername{getConnectionFigure}
{\tt public Connection {\bf getConnectionFigure}(  )
}%end signature
}%end item
\divideents{getConnectionStyle}
\item{\vskip -1.9ex 
\membername{getConnectionStyle}
{\tt public int {\bf getConnectionStyle}(  )
}%end signature
}%end item
\divideents{getDestination}
\item{\vskip -1.9ex 
\membername{getDestination}
{\tt public GraphNode {\bf getDestination}(  )
}%end signature
}%end item
\divideents{getExternalConnection}
\item{\vskip -1.9ex 
\membername{getExternalConnection}
{\tt public Object {\bf getExternalConnection}(  )
}%end signature
}%end item
\divideents{getFont}
\item{\vskip -1.9ex 
\membername{getFont}
{\tt public Font {\bf getFont}(  )
}%end signature
}%end item
\divideents{getGraphModel}
\item{\vskip -1.9ex 
\membername{getGraphModel}
{\tt public Graph {\bf getGraphModel}(  )
}%end signature
}%end item
\divideents{getHighlightColor}
\item{\vskip -1.9ex 
\membername{getHighlightColor}
{\tt public Color {\bf getHighlightColor}(  )
}%end signature
}%end item
\divideents{getItemType}
\item{\vskip -1.9ex 
\membername{getItemType}
{\tt public int {\bf getItemType}(  )
}%end signature
}%end item
\divideents{getLayoutRelationship}
\item{\vskip -1.9ex 
\membername{getLayoutRelationship}
{\tt public LayoutRelationship {\bf getLayoutRelationship}(  )
}%end signature
}%end item
\divideents{getLineColor}
\item{\vskip -1.9ex 
\membername{getLineColor}
{\tt public Color {\bf getLineColor}(  )
}%end signature
}%end item
\divideents{getLineStyle}
\item{\vskip -1.9ex 
\membername{getLineStyle}
{\tt public int {\bf getLineStyle}(  )
}%end signature
}%end item
\divideents{getLineWidth}
\item{\vskip -1.9ex 
\membername{getLineWidth}
{\tt public int {\bf getLineWidth}(  )
}%end signature
}%end item
\divideents{getSource}
\item{\vskip -1.9ex 
\membername{getSource}
{\tt public GraphNode {\bf getSource}(  )
}%end signature
}%end item
\divideents{getTooltip}
\item{\vskip -1.9ex 
\membername{getTooltip}
{\tt public IFigure {\bf getTooltip}(  )
}%end signature
}%end item
\divideents{getWeightInLayout}
\item{\vskip -1.9ex 
\membername{getWeightInLayout}
{\tt public double {\bf getWeightInLayout}(  )
}%end signature
}%end item
\divideents{highlight}
\item{\vskip -1.9ex 
\membername{highlight}
{\tt public void {\bf highlight}(  )
}%end signature
}%end item
\divideents{isDisposed}
\item{\vskip -1.9ex 
\membername{isDisposed}
{\tt public boolean {\bf isDisposed}(  )
}%end signature
}%end item
\divideents{isHighlighted}
\item{\vskip -1.9ex 
\membername{isHighlighted}
{\tt public boolean {\bf isHighlighted}(  )
}%end signature
}%end item
\divideents{isVisible}
\item{\vskip -1.9ex 
\membername{isVisible}
{\tt public boolean {\bf isVisible}(  )
}%end signature
}%end item
\divideents{setConnectionStyle}
\item{\vskip -1.9ex 
\membername{setConnectionStyle}
{\tt public void {\bf setConnectionStyle}( {\tt int } {\bf arg0} )
}%end signature
}%end item
\divideents{setFont}
\item{\vskip -1.9ex 
\membername{setFont}
{\tt public void {\bf setFont}( {\tt org.eclipse.swt.graphics.Font } {\bf arg0} )
}%end signature
}%end item
\divideents{setHighlightColor}
\item{\vskip -1.9ex 
\membername{setHighlightColor}
{\tt public void {\bf setHighlightColor}( {\tt org.eclipse.swt.graphics.Color } {\bf arg0} )
}%end signature
}%end item
\divideents{setLineColor}
\item{\vskip -1.9ex 
\membername{setLineColor}
{\tt public void {\bf setLineColor}( {\tt org.eclipse.swt.graphics.Color } {\bf arg0} )
}%end signature
}%end item
\divideents{setLineStyle}
\item{\vskip -1.9ex 
\membername{setLineStyle}
{\tt public void {\bf setLineStyle}( {\tt int } {\bf arg0} )
}%end signature
}%end item
\divideents{setLineWidth}
\item{\vskip -1.9ex 
\membername{setLineWidth}
{\tt public void {\bf setLineWidth}( {\tt int } {\bf arg0} )
}%end signature
}%end item
\divideents{setText}
\item{\vskip -1.9ex 
\membername{setText}
{\tt public void {\bf setText}( {\tt java.lang.String } {\bf arg0} )
}%end signature
}%end item
\divideents{setTooltip}
\item{\vskip -1.9ex 
\membername{setTooltip}
{\tt public void {\bf setTooltip}( {\tt org.eclipse.draw2d.IFigure } {\bf arg0} )
}%end signature
}%end item
\divideents{setVisible}
\item{\vskip -1.9ex 
\membername{setVisible}
{\tt public void {\bf setVisible}( {\tt boolean } {\bf arg0} )
}%end signature
}%end item
\divideents{setWeight}
\item{\vskip -1.9ex 
\membername{setWeight}
{\tt public void {\bf setWeight}( {\tt double } {\bf arg0} )
}%end signature
}%end item
\divideents{toString}
\item{\vskip -1.9ex 
\membername{toString}
{\tt public String {\bf toString}(  )
}%end signature
}%end item
\divideents{unhighlight}
\item{\vskip -1.9ex 
\membername{unhighlight}
{\tt public void {\bf unhighlight}(  )
}%end signature
}%end item
\end{itemize}
}}
\startsubsubsection{Methods inherited from class {\tt org.eclipse.zest.core.widgets.GraphItem}}{
\par{\small 
\refdefined{l1673}\vskip -2em
\begin{itemize}
\item{\vskip -1.9ex 
\membername{checkStyle}
{\tt protected boolean {\bf checkStyle}( {\tt int } {\bf arg0} )
}%end signature
}%end item
\divideents{dispose}
\item{\vskip -1.9ex 
\membername{dispose}
{\tt public void {\bf dispose}(  )
}%end signature
}%end item
\divideents{getGraphModel}
\item{\vskip -1.9ex 
\membername{getGraphModel}
{\tt public abstract Graph {\bf getGraphModel}(  )
}%end signature
}%end item
\divideents{getItemType}
\item{\vskip -1.9ex 
\membername{getItemType}
{\tt public abstract int {\bf getItemType}(  )
}%end signature
}%end item
\divideents{highlight}
\item{\vskip -1.9ex 
\membername{highlight}
{\tt public abstract void {\bf highlight}(  )
}%end signature
}%end item
\divideents{isVisible}
\item{\vskip -1.9ex 
\membername{isVisible}
{\tt public abstract boolean {\bf isVisible}(  )
}%end signature
}%end item
\divideents{setVisible}
\item{\vskip -1.9ex 
\membername{setVisible}
{\tt public abstract void {\bf setVisible}( {\tt boolean } {\bf arg0} )
}%end signature
}%end item
\divideents{unhighlight}
\item{\vskip -1.9ex 
\membername{unhighlight}
{\tt public abstract void {\bf unhighlight}(  )
}%end signature
}%end item
\end{itemize}
}}
\startsubsubsection{Methods inherited from class {\tt org.eclipse.swt.widgets.Item}}{
\par{\small 
\refdefined{l1674}\vskip -2em
\begin{itemize}
\item{\vskip -1.9ex 
\membername{checkSubclass}
{\tt protected void {\bf checkSubclass}(  )
}%end signature
}%end item
\divideents{getImage}
\item{\vskip -1.9ex 
\membername{getImage}
{\tt public Image {\bf getImage}(  )
}%end signature
}%end item
\divideents{getText}
\item{\vskip -1.9ex 
\membername{getText}
{\tt public String {\bf getText}(  )
}%end signature
}%end item
\divideents{setImage}
\item{\vskip -1.9ex 
\membername{setImage}
{\tt public void {\bf setImage}( {\tt org.eclipse.swt.graphics.Image } {\bf arg0} )
}%end signature
}%end item
\divideents{setText}
\item{\vskip -1.9ex 
\membername{setText}
{\tt public void {\bf setText}( {\tt java.lang.String } {\bf arg0} )
}%end signature
}%end item
\end{itemize}
}}
\startsubsubsection{Methods inherited from class {\tt org.eclipse.swt.widgets.Widget}}{
\par{\small 
\refdefined{l1675}\vskip -2em
\begin{itemize}
\item{\vskip -1.9ex 
\membername{addDisposeListener}
{\tt public void {\bf addDisposeListener}( {\tt org.eclipse.swt.events.DisposeListener } {\bf arg0} )
}%end signature
}%end item
\divideents{addListener}
\item{\vskip -1.9ex 
\membername{addListener}
{\tt public void {\bf addListener}( {\tt int } {\bf arg0},
{\tt org.eclipse.swt.widgets.Listener } {\bf arg1} )
}%end signature
}%end item
\divideents{checkSubclass}
\item{\vskip -1.9ex 
\membername{checkSubclass}
{\tt protected void {\bf checkSubclass}(  )
}%end signature
}%end item
\divideents{checkWidget}
\item{\vskip -1.9ex 
\membername{checkWidget}
{\tt protected void {\bf checkWidget}(  )
}%end signature
}%end item
\divideents{dispose}
\item{\vskip -1.9ex 
\membername{dispose}
{\tt public void {\bf dispose}(  )
}%end signature
}%end item
\divideents{getData}
\item{\vskip -1.9ex 
\membername{getData}
{\tt public Object {\bf getData}(  )
}%end signature
}%end item
\divideents{getData}
\item{\vskip -1.9ex 
\membername{getData}
{\tt public Object {\bf getData}( {\tt java.lang.String } {\bf arg0} )
}%end signature
}%end item
\divideents{getDisplay}
\item{\vskip -1.9ex 
\membername{getDisplay}
{\tt public Display {\bf getDisplay}(  )
}%end signature
}%end item
\divideents{getListeners}
\item{\vskip -1.9ex 
\membername{getListeners}
{\tt public Listener {\bf getListeners}( {\tt int } {\bf arg0} )
}%end signature
}%end item
\divideents{getStyle}
\item{\vskip -1.9ex 
\membername{getStyle}
{\tt public int {\bf getStyle}(  )
}%end signature
}%end item
\divideents{isDisposed}
\item{\vskip -1.9ex 
\membername{isDisposed}
{\tt public boolean {\bf isDisposed}(  )
}%end signature
}%end item
\divideents{isListening}
\item{\vskip -1.9ex 
\membername{isListening}
{\tt public boolean {\bf isListening}( {\tt int } {\bf arg0} )
}%end signature
}%end item
\divideents{notifyListeners}
\item{\vskip -1.9ex 
\membername{notifyListeners}
{\tt public void {\bf notifyListeners}( {\tt int } {\bf arg0},
{\tt org.eclipse.swt.widgets.Event } {\bf arg1} )
}%end signature
}%end item
\divideents{removeDisposeListener}
\item{\vskip -1.9ex 
\membername{removeDisposeListener}
{\tt public void {\bf removeDisposeListener}( {\tt org.eclipse.swt.events.DisposeListener } {\bf arg0} )
}%end signature
}%end item
\divideents{removeListener}
\item{\vskip -1.9ex 
\membername{removeListener}
{\tt public void {\bf removeListener}( {\tt int } {\bf arg0},
{\tt org.eclipse.swt.widgets.Listener } {\bf arg1} )
}%end signature
}%end item
\divideents{removeListener}
\item{\vskip -1.9ex 
\membername{removeListener}
{\tt protected void {\bf removeListener}( {\tt int } {\bf arg0},
{\tt org.eclipse.swt.internal.SWTEventListener } {\bf arg1} )
}%end signature
}%end item
\divideents{setData}
\item{\vskip -1.9ex 
\membername{setData}
{\tt public void {\bf setData}( {\tt java.lang.Object } {\bf arg0} )
}%end signature
}%end item
\divideents{setData}
\item{\vskip -1.9ex 
\membername{setData}
{\tt public void {\bf setData}( {\tt java.lang.String } {\bf arg0},
{\tt java.lang.Object } {\bf arg1} )
}%end signature
}%end item
\divideents{toString}
\item{\vskip -1.9ex 
\membername{toString}
{\tt public String {\bf toString}(  )
}%end signature
}%end item
\end{itemize}
}}
}
\startsection{Class}{GUINetwork}{l1610}{%
\startsubsubsection{Declaration}{
\fbox{\vbox{
\hbox{\vbox{\small public 
class 
GUINetwork}}
\noindent\hbox{\vbox{{\bf extends} org.eclipse.zest.core.widgets.GraphContainer}}
\noindent\hbox{\vbox{{\bf implements} 
NodeContainer}}
}}}
\startsubsubsection{Constructors}{
\vskip -2em
\begin{itemize}
\item{\vskip -1.9ex 
\membername{GUINetwork}
{\tt public {\bf GUINetwork}( {\tt uk.ac.ic.doc.neuralnets.graph.neural.NeuralNetwork } {\bf network},
{\tt org.eclipse.zest.core.widgets.IContainer } {\bf container},
{\tt org.eclipse.zest.core.widgets.Graph } {\bf g},
{\tt int } {\bf style} )
\label{l1697}\label{l1698}}%end signature
\begin{itemize}
\sld
\item{
\sld
{\bf Usage}
  \begin{itemize}\isep
   \item{
Creates a GUI Network which can contain more GUI Networks or GUI Nodes.
}%end item
  \end{itemize}
}
\item{
\sld
{\bf Parameters}
\sld\isep
  \begin{itemize}
\sld\isep
   \item{
\sld
{\tt network} - Network to model in GUI}
   \item{
\sld
{\tt container} - Graph to insert GUI Network into}
   \item{
\sld
{\tt g} - Contents of network in a displayable format}
   \item{
\sld
{\tt style} - Style of GUI Network}
  \end{itemize}
}%end item
\end{itemize}
}%end item
\end{itemize}
}
\startsubsubsection{Methods}{
\vskip -2em
\begin{itemize}
\item{\vskip -1.9ex 
\membername{createToolTip}
{\tt public void {\bf createToolTip}(  )
\label{l1699}\label{l1700}}%end signature
}%end item
\divideents{getNode}
\item{\vskip -1.9ex 
\membername{getNode}
{\tt public Node {\bf getNode}(  )
\label{l1701}\label{l1702}}%end signature
}%end item
\divideents{persistLocation}
\item{\vskip -1.9ex 
\membername{persistLocation}
{\tt public void {\bf persistLocation}(  )
\label{l1703}\label{l1704}}%end signature
\begin{itemize}
\sld
\item{
\sld
{\bf Usage}
  \begin{itemize}\isep
   \item{
Persists the location of this node in the GUI to the model node.
}%end item
  \end{itemize}
}
\end{itemize}
}%end item
\divideents{setNode}
\item{\vskip -1.9ex 
\membername{setNode}
{\tt public void {\bf setNode}( {\tt uk.ac.ic.doc.neuralnets.graph.Node } {\bf network} )
\label{l1705}\label{l1706}}%end signature
}%end item
\end{itemize}
}
\startsubsubsection{Methods inherited from class {\tt org.eclipse.zest.core.widgets.GraphContainer}}{
\par{\small 
\refdefined{l1707}\vskip -2em
\begin{itemize}
\item{\vskip -1.9ex 
\membername{applyLayout}
{\tt public void {\bf applyLayout}(  )
}%end signature
}%end item
\divideents{close}
\item{\vskip -1.9ex 
\membername{close}
{\tt public void {\bf close}( {\tt boolean } {\bf arg0} )
}%end signature
}%end item
\divideents{getGraph}
\item{\vskip -1.9ex 
\membername{getGraph}
{\tt public Graph {\bf getGraph}(  )
}%end signature
}%end item
\divideents{getItemType}
\item{\vskip -1.9ex 
\membername{getItemType}
{\tt public int {\bf getItemType}(  )
}%end signature
}%end item
\divideents{getNodeFigure}
\item{\vskip -1.9ex 
\membername{getNodeFigure}
{\tt public IFigure {\bf getNodeFigure}(  )
}%end signature
}%end item
\divideents{getNodes}
\item{\vskip -1.9ex 
\membername{getNodes}
{\tt public List {\bf getNodes}(  )
}%end signature
}%end item
\divideents{getScale}
\item{\vskip -1.9ex 
\membername{getScale}
{\tt public double {\bf getScale}(  )
}%end signature
}%end item
\divideents{initFigure}
\item{\vskip -1.9ex 
\membername{initFigure}
{\tt protected void {\bf initFigure}(  )
}%end signature
}%end item
\divideents{open}
\item{\vskip -1.9ex 
\membername{open}
{\tt public void {\bf open}( {\tt boolean } {\bf arg0} )
}%end signature
}%end item
\divideents{refreshLocation}
\item{\vskip -1.9ex 
\membername{refreshLocation}
{\tt protected void {\bf refreshLocation}(  )
}%end signature
}%end item
\divideents{setCustomFigure}
\item{\vskip -1.9ex 
\membername{setCustomFigure}
{\tt public void {\bf setCustomFigure}( {\tt org.eclipse.draw2d.IFigure } {\bf arg0} )
}%end signature
}%end item
\divideents{setLayoutAlgorithm}
\item{\vskip -1.9ex 
\membername{setLayoutAlgorithm}
{\tt public void {\bf setLayoutAlgorithm}( {\tt org.eclipse.zest.layouts.LayoutAlgorithm } {\bf arg0},
{\tt boolean } {\bf arg1} )
}%end signature
}%end item
\divideents{setScale}
\item{\vskip -1.9ex 
\membername{setScale}
{\tt public void {\bf setScale}( {\tt double } {\bf arg0} )
}%end signature
}%end item
\divideents{updateFigureForModel}
\item{\vskip -1.9ex 
\membername{updateFigureForModel}
{\tt protected void {\bf updateFigureForModel}( {\tt org.eclipse.draw2d.IFigure } {\bf arg0} )
}%end signature
}%end item
\end{itemize}
}}
\startsubsubsection{Methods inherited from class {\tt org.eclipse.zest.core.widgets.GraphNode}}{
\par{\small 
\refdefined{l1672}\vskip -2em
\begin{itemize}
\item{\vskip -1.9ex 
\membername{cacheLabel}
{\tt public boolean {\bf cacheLabel}(  )
}%end signature
}%end item
\divideents{createFigureForModel}
\item{\vskip -1.9ex 
\membername{createFigureForModel}
{\tt protected IFigure {\bf createFigureForModel}(  )
}%end signature
}%end item
\divideents{dispose}
\item{\vskip -1.9ex 
\membername{dispose}
{\tt public void {\bf dispose}(  )
}%end signature
}%end item
\divideents{fishEye}
\item{\vskip -1.9ex 
\membername{fishEye}
{\tt protected IFigure {\bf fishEye}( {\tt boolean } {\bf arg0},
{\tt boolean } {\bf arg1} )
}%end signature
}%end item
\divideents{getBackgroundColor}
\item{\vskip -1.9ex 
\membername{getBackgroundColor}
{\tt public Color {\bf getBackgroundColor}(  )
}%end signature
}%end item
\divideents{getBorderColor}
\item{\vskip -1.9ex 
\membername{getBorderColor}
{\tt public Color {\bf getBorderColor}(  )
}%end signature
}%end item
\divideents{getBorderHighlightColor}
\item{\vskip -1.9ex 
\membername{getBorderHighlightColor}
{\tt public Color {\bf getBorderHighlightColor}(  )
}%end signature
}%end item
\divideents{getBorderWidth}
\item{\vskip -1.9ex 
\membername{getBorderWidth}
{\tt public int {\bf getBorderWidth}(  )
}%end signature
}%end item
\divideents{getFont}
\item{\vskip -1.9ex 
\membername{getFont}
{\tt public Font {\bf getFont}(  )
}%end signature
}%end item
\divideents{getForegroundColor}
\item{\vskip -1.9ex 
\membername{getForegroundColor}
{\tt public Color {\bf getForegroundColor}(  )
}%end signature
}%end item
\divideents{getGraphModel}
\item{\vskip -1.9ex 
\membername{getGraphModel}
{\tt public Graph {\bf getGraphModel}(  )
}%end signature
}%end item
\divideents{getHighlightColor}
\item{\vskip -1.9ex 
\membername{getHighlightColor}
{\tt public Color {\bf getHighlightColor}(  )
}%end signature
}%end item
\divideents{getItemType}
\item{\vskip -1.9ex 
\membername{getItemType}
{\tt public int {\bf getItemType}(  )
}%end signature
}%end item
\divideents{getLayoutEntity}
\item{\vskip -1.9ex 
\membername{getLayoutEntity}
{\tt public LayoutEntity {\bf getLayoutEntity}(  )
}%end signature
}%end item
\divideents{getLocation}
\item{\vskip -1.9ex 
\membername{getLocation}
{\tt public Point {\bf getLocation}(  )
}%end signature
}%end item
\divideents{getNodeFigure}
\item{\vskip -1.9ex 
\membername{getNodeFigure}
{\tt public IFigure {\bf getNodeFigure}(  )
}%end signature
}%end item
\divideents{getNodeStyle}
\item{\vskip -1.9ex 
\membername{getNodeStyle}
{\tt public int {\bf getNodeStyle}(  )
}%end signature
}%end item
\divideents{getSize}
\item{\vskip -1.9ex 
\membername{getSize}
{\tt public Dimension {\bf getSize}(  )
}%end signature
}%end item
\divideents{getSourceConnections}
\item{\vskip -1.9ex 
\membername{getSourceConnections}
{\tt public List {\bf getSourceConnections}(  )
}%end signature
}%end item
\divideents{getStyle}
\item{\vskip -1.9ex 
\membername{getStyle}
{\tt public int {\bf getStyle}(  )
}%end signature
}%end item
\divideents{getTargetConnections}
\item{\vskip -1.9ex 
\membername{getTargetConnections}
{\tt public List {\bf getTargetConnections}(  )
}%end signature
}%end item
\divideents{getTooltip}
\item{\vskip -1.9ex 
\membername{getTooltip}
{\tt public IFigure {\bf getTooltip}(  )
}%end signature
}%end item
\divideents{highlight}
\item{\vskip -1.9ex 
\membername{highlight}
{\tt public void {\bf highlight}(  )
}%end signature
}%end item
\divideents{initFigure}
\item{\vskip -1.9ex 
\membername{initFigure}
{\tt protected void {\bf initFigure}(  )
}%end signature
}%end item
\divideents{initModel}
\item{\vskip -1.9ex 
\membername{initModel}
{\tt protected void {\bf initModel}( {\tt org.eclipse.zest.core.widgets.IContainer } {\bf arg0},
{\tt java.lang.String } {\bf arg1},
{\tt org.eclipse.swt.graphics.Image } {\bf arg2} )
}%end signature
}%end item
\divideents{isDisposed}
\item{\vskip -1.9ex 
\membername{isDisposed}
{\tt public boolean {\bf isDisposed}(  )
}%end signature
}%end item
\divideents{isSelected}
\item{\vskip -1.9ex 
\membername{isSelected}
{\tt public boolean {\bf isSelected}(  )
}%end signature
}%end item
\divideents{isSizeFixed}
\item{\vskip -1.9ex 
\membername{isSizeFixed}
{\tt public boolean {\bf isSizeFixed}(  )
}%end signature
}%end item
\divideents{isVisible}
\item{\vskip -1.9ex 
\membername{isVisible}
{\tt public boolean {\bf isVisible}(  )
}%end signature
}%end item
\divideents{refreshLocation}
\item{\vskip -1.9ex 
\membername{refreshLocation}
{\tt protected void {\bf refreshLocation}(  )
}%end signature
}%end item
\divideents{setBackgroundColor}
\item{\vskip -1.9ex 
\membername{setBackgroundColor}
{\tt public void {\bf setBackgroundColor}( {\tt org.eclipse.swt.graphics.Color } {\bf arg0} )
}%end signature
}%end item
\divideents{setBorderColor}
\item{\vskip -1.9ex 
\membername{setBorderColor}
{\tt public void {\bf setBorderColor}( {\tt org.eclipse.swt.graphics.Color } {\bf arg0} )
}%end signature
}%end item
\divideents{setBorderHighlightColor}
\item{\vskip -1.9ex 
\membername{setBorderHighlightColor}
{\tt public void {\bf setBorderHighlightColor}( {\tt org.eclipse.swt.graphics.Color } {\bf arg0} )
}%end signature
}%end item
\divideents{setBorderWidth}
\item{\vskip -1.9ex 
\membername{setBorderWidth}
{\tt public void {\bf setBorderWidth}( {\tt int } {\bf arg0} )
}%end signature
}%end item
\divideents{setCacheLabel}
\item{\vskip -1.9ex 
\membername{setCacheLabel}
{\tt public void {\bf setCacheLabel}( {\tt boolean } {\bf arg0} )
}%end signature
}%end item
\divideents{setFont}
\item{\vskip -1.9ex 
\membername{setFont}
{\tt public void {\bf setFont}( {\tt org.eclipse.swt.graphics.Font } {\bf arg0} )
}%end signature
}%end item
\divideents{setForegroundColor}
\item{\vskip -1.9ex 
\membername{setForegroundColor}
{\tt public void {\bf setForegroundColor}( {\tt org.eclipse.swt.graphics.Color } {\bf arg0} )
}%end signature
}%end item
\divideents{setHighlightColor}
\item{\vskip -1.9ex 
\membername{setHighlightColor}
{\tt public void {\bf setHighlightColor}( {\tt org.eclipse.swt.graphics.Color } {\bf arg0} )
}%end signature
}%end item
\divideents{setImage}
\item{\vskip -1.9ex 
\membername{setImage}
{\tt public void {\bf setImage}( {\tt org.eclipse.swt.graphics.Image } {\bf arg0} )
}%end signature
}%end item
\divideents{setLocation}
\item{\vskip -1.9ex 
\membername{setLocation}
{\tt public void {\bf setLocation}( {\tt double } {\bf arg0},
{\tt double } {\bf arg1} )
}%end signature
}%end item
\divideents{setNodeStyle}
\item{\vskip -1.9ex 
\membername{setNodeStyle}
{\tt public void {\bf setNodeStyle}( {\tt int } {\bf arg0} )
}%end signature
}%end item
\divideents{setSize}
\item{\vskip -1.9ex 
\membername{setSize}
{\tt public void {\bf setSize}( {\tt double } {\bf arg0},
{\tt double } {\bf arg1} )
}%end signature
}%end item
\divideents{setText}
\item{\vskip -1.9ex 
\membername{setText}
{\tt public void {\bf setText}( {\tt java.lang.String } {\bf arg0} )
}%end signature
}%end item
\divideents{setTooltip}
\item{\vskip -1.9ex 
\membername{setTooltip}
{\tt public void {\bf setTooltip}( {\tt org.eclipse.draw2d.IFigure } {\bf arg0} )
}%end signature
}%end item
\divideents{setVisible}
\item{\vskip -1.9ex 
\membername{setVisible}
{\tt public void {\bf setVisible}( {\tt boolean } {\bf arg0} )
}%end signature
}%end item
\divideents{toString}
\item{\vskip -1.9ex 
\membername{toString}
{\tt public String {\bf toString}(  )
}%end signature
}%end item
\divideents{unhighlight}
\item{\vskip -1.9ex 
\membername{unhighlight}
{\tt public void {\bf unhighlight}(  )
}%end signature
}%end item
\divideents{updateFigureForModel}
\item{\vskip -1.9ex 
\membername{updateFigureForModel}
{\tt protected void {\bf updateFigureForModel}( {\tt org.eclipse.draw2d.IFigure } {\bf arg0} )
}%end signature
}%end item
\end{itemize}
}}
\startsubsubsection{Methods inherited from class {\tt org.eclipse.zest.core.widgets.GraphItem}}{
\par{\small 
\refdefined{l1673}\vskip -2em
\begin{itemize}
\item{\vskip -1.9ex 
\membername{checkStyle}
{\tt protected boolean {\bf checkStyle}( {\tt int } {\bf arg0} )
}%end signature
}%end item
\divideents{dispose}
\item{\vskip -1.9ex 
\membername{dispose}
{\tt public void {\bf dispose}(  )
}%end signature
}%end item
\divideents{getGraphModel}
\item{\vskip -1.9ex 
\membername{getGraphModel}
{\tt public abstract Graph {\bf getGraphModel}(  )
}%end signature
}%end item
\divideents{getItemType}
\item{\vskip -1.9ex 
\membername{getItemType}
{\tt public abstract int {\bf getItemType}(  )
}%end signature
}%end item
\divideents{highlight}
\item{\vskip -1.9ex 
\membername{highlight}
{\tt public abstract void {\bf highlight}(  )
}%end signature
}%end item
\divideents{isVisible}
\item{\vskip -1.9ex 
\membername{isVisible}
{\tt public abstract boolean {\bf isVisible}(  )
}%end signature
}%end item
\divideents{setVisible}
\item{\vskip -1.9ex 
\membername{setVisible}
{\tt public abstract void {\bf setVisible}( {\tt boolean } {\bf arg0} )
}%end signature
}%end item
\divideents{unhighlight}
\item{\vskip -1.9ex 
\membername{unhighlight}
{\tt public abstract void {\bf unhighlight}(  )
}%end signature
}%end item
\end{itemize}
}}
\startsubsubsection{Methods inherited from class {\tt org.eclipse.swt.widgets.Item}}{
\par{\small 
\refdefined{l1674}\vskip -2em
\begin{itemize}
\item{\vskip -1.9ex 
\membername{checkSubclass}
{\tt protected void {\bf checkSubclass}(  )
}%end signature
}%end item
\divideents{getImage}
\item{\vskip -1.9ex 
\membername{getImage}
{\tt public Image {\bf getImage}(  )
}%end signature
}%end item
\divideents{getText}
\item{\vskip -1.9ex 
\membername{getText}
{\tt public String {\bf getText}(  )
}%end signature
}%end item
\divideents{setImage}
\item{\vskip -1.9ex 
\membername{setImage}
{\tt public void {\bf setImage}( {\tt org.eclipse.swt.graphics.Image } {\bf arg0} )
}%end signature
}%end item
\divideents{setText}
\item{\vskip -1.9ex 
\membername{setText}
{\tt public void {\bf setText}( {\tt java.lang.String } {\bf arg0} )
}%end signature
}%end item
\end{itemize}
}}
\startsubsubsection{Methods inherited from class {\tt org.eclipse.swt.widgets.Widget}}{
\par{\small 
\refdefined{l1675}\vskip -2em
\begin{itemize}
\item{\vskip -1.9ex 
\membername{addDisposeListener}
{\tt public void {\bf addDisposeListener}( {\tt org.eclipse.swt.events.DisposeListener } {\bf arg0} )
}%end signature
}%end item
\divideents{addListener}
\item{\vskip -1.9ex 
\membername{addListener}
{\tt public void {\bf addListener}( {\tt int } {\bf arg0},
{\tt org.eclipse.swt.widgets.Listener } {\bf arg1} )
}%end signature
}%end item
\divideents{checkSubclass}
\item{\vskip -1.9ex 
\membername{checkSubclass}
{\tt protected void {\bf checkSubclass}(  )
}%end signature
}%end item
\divideents{checkWidget}
\item{\vskip -1.9ex 
\membername{checkWidget}
{\tt protected void {\bf checkWidget}(  )
}%end signature
}%end item
\divideents{dispose}
\item{\vskip -1.9ex 
\membername{dispose}
{\tt public void {\bf dispose}(  )
}%end signature
}%end item
\divideents{getData}
\item{\vskip -1.9ex 
\membername{getData}
{\tt public Object {\bf getData}(  )
}%end signature
}%end item
\divideents{getData}
\item{\vskip -1.9ex 
\membername{getData}
{\tt public Object {\bf getData}( {\tt java.lang.String } {\bf arg0} )
}%end signature
}%end item
\divideents{getDisplay}
\item{\vskip -1.9ex 
\membername{getDisplay}
{\tt public Display {\bf getDisplay}(  )
}%end signature
}%end item
\divideents{getListeners}
\item{\vskip -1.9ex 
\membername{getListeners}
{\tt public Listener {\bf getListeners}( {\tt int } {\bf arg0} )
}%end signature
}%end item
\divideents{getStyle}
\item{\vskip -1.9ex 
\membername{getStyle}
{\tt public int {\bf getStyle}(  )
}%end signature
}%end item
\divideents{isDisposed}
\item{\vskip -1.9ex 
\membername{isDisposed}
{\tt public boolean {\bf isDisposed}(  )
}%end signature
}%end item
\divideents{isListening}
\item{\vskip -1.9ex 
\membername{isListening}
{\tt public boolean {\bf isListening}( {\tt int } {\bf arg0} )
}%end signature
}%end item
\divideents{notifyListeners}
\item{\vskip -1.9ex 
\membername{notifyListeners}
{\tt public void {\bf notifyListeners}( {\tt int } {\bf arg0},
{\tt org.eclipse.swt.widgets.Event } {\bf arg1} )
}%end signature
}%end item
\divideents{removeDisposeListener}
\item{\vskip -1.9ex 
\membername{removeDisposeListener}
{\tt public void {\bf removeDisposeListener}( {\tt org.eclipse.swt.events.DisposeListener } {\bf arg0} )
}%end signature
}%end item
\divideents{removeListener}
\item{\vskip -1.9ex 
\membername{removeListener}
{\tt public void {\bf removeListener}( {\tt int } {\bf arg0},
{\tt org.eclipse.swt.widgets.Listener } {\bf arg1} )
}%end signature
}%end item
\divideents{removeListener}
\item{\vskip -1.9ex 
\membername{removeListener}
{\tt protected void {\bf removeListener}( {\tt int } {\bf arg0},
{\tt org.eclipse.swt.internal.SWTEventListener } {\bf arg1} )
}%end signature
}%end item
\divideents{setData}
\item{\vskip -1.9ex 
\membername{setData}
{\tt public void {\bf setData}( {\tt java.lang.Object } {\bf arg0} )
}%end signature
}%end item
\divideents{setData}
\item{\vskip -1.9ex 
\membername{setData}
{\tt public void {\bf setData}( {\tt java.lang.String } {\bf arg0},
{\tt java.lang.Object } {\bf arg1} )
}%end signature
}%end item
\divideents{toString}
\item{\vskip -1.9ex 
\membername{toString}
{\tt public String {\bf toString}(  )
}%end signature
}%end item
\end{itemize}
}}
}
\startsection{Class}{GUINode}{l1611}{%
{\small Represents a Neurone in the Zest graph.}
\vskip .1in 
\startsubsubsection{Declaration}{
\fbox{\vbox{
\hbox{\vbox{\small public 
class 
GUINode}}
\noindent\hbox{\vbox{{\bf extends} org.eclipse.zest.core.widgets.GraphNode}}
\noindent\hbox{\vbox{{\bf implements} 
NodeContainer}}
}}}
\startsubsubsection{Constructors}{
\vskip -2em
\begin{itemize}
\item{\vskip -1.9ex 
\membername{GUINode}
{\tt public {\bf GUINode}( {\tt org.eclipse.zest.core.widgets.IContainer } {\bf graphModel},
{\tt int } {\bf style} )
\label{l1708}\label{l1709}}%end signature
}%end item
\divideents{GUINode}
\item{\vskip -1.9ex 
\membername{GUINode}
{\tt public {\bf GUINode}( {\tt uk.ac.ic.doc.neuralnets.graph.Node } {\bf node},
{\tt org.eclipse.zest.core.widgets.IContainer } {\bf graphModel},
{\tt int } {\bf style} )
\label{l1710}\label{l1711}}%end signature
}%end item
\end{itemize}
}
\startsubsubsection{Methods}{
\vskip -2em
\begin{itemize}
\item{\vskip -1.9ex 
\membername{createFigureForModel}
{\tt protected IFigure {\bf createFigureForModel}(  )
\label{l1712}\label{l1713}}%end signature
}%end item
\divideents{createToolTip}
\item{\vskip -1.9ex 
\membername{createToolTip}
{\tt public void {\bf createToolTip}(  )
\label{l1714}\label{l1715}}%end signature
}%end item
\divideents{getNode}
\item{\vskip -1.9ex 
\membername{getNode}
{\tt public Node {\bf getNode}(  )
\label{l1716}\label{l1717}}%end signature
}%end item
\divideents{highlight}
\item{\vskip -1.9ex 
\membername{highlight}
{\tt public void {\bf highlight}(  )
\label{l1718}\label{l1719}}%end signature
\begin{itemize}
\sld
\item{
\sld
{\bf Usage}
  \begin{itemize}\isep
   \item{
Highlights the node.
}%end item
  \end{itemize}
}
\end{itemize}
}%end item
\divideents{persistLocation}
\item{\vskip -1.9ex 
\membername{persistLocation}
{\tt public void {\bf persistLocation}(  )
\label{l1720}\label{l1721}}%end signature
\begin{itemize}
\sld
\item{
\sld
{\bf Usage}
  \begin{itemize}\isep
   \item{
Persists the location of this node in the GUI to the model node.
}%end item
  \end{itemize}
}
\end{itemize}
}%end item
\divideents{setNode}
\item{\vskip -1.9ex 
\membername{setNode}
{\tt public void {\bf setNode}( {\tt uk.ac.ic.doc.neuralnets.graph.Node } {\bf node} )
\label{l1722}\label{l1723}}%end signature
}%end item
\divideents{setOverlayColor}
\item{\vskip -1.9ex 
\membername{setOverlayColor}
{\tt public void {\bf setOverlayColor}( {\tt org.eclipse.swt.graphics.Color } {\bf c} )
\label{l1724}\label{l1725}}%end signature
\begin{itemize}
\sld
\item{
\sld
{\bf Usage}
  \begin{itemize}\isep
   \item{
Change the background color of the charge overlay to the specified color.
}%end item
  \end{itemize}
}
\item{
\sld
{\bf Parameters}
\sld\isep
  \begin{itemize}
\sld\isep
   \item{
\sld
{\tt c} - - the new overlay color.}
  \end{itemize}
}%end item
\end{itemize}
}%end item
\divideents{unhighlight}
\item{\vskip -1.9ex 
\membername{unhighlight}
{\tt public void {\bf unhighlight}(  )
\label{l1726}\label{l1727}}%end signature
\begin{itemize}
\sld
\item{
\sld
{\bf Usage}
  \begin{itemize}\isep
   \item{
Unhightlights the node.
}%end item
  \end{itemize}
}
\end{itemize}
}%end item
\divideents{updateChargeOverlay}
\item{\vskip -1.9ex 
\membername{updateChargeOverlay}
{\tt public void {\bf updateChargeOverlay}(  )
\label{l1728}\label{l1729}}%end signature
\begin{itemize}
\sld
\item{
\sld
{\bf Usage}
  \begin{itemize}\isep
   \item{
Update the size of the charge overlay. Should be called when the model
 node ticks.
}%end item
  \end{itemize}
}
\end{itemize}
}%end item
\end{itemize}
}
\startsubsubsection{Methods inherited from class {\tt org.eclipse.zest.core.widgets.GraphNode}}{
\par{\small 
\refdefined{l1672}\vskip -2em
\begin{itemize}
\item{\vskip -1.9ex 
\membername{cacheLabel}
{\tt public boolean {\bf cacheLabel}(  )
}%end signature
}%end item
\divideents{createFigureForModel}
\item{\vskip -1.9ex 
\membername{createFigureForModel}
{\tt protected IFigure {\bf createFigureForModel}(  )
}%end signature
}%end item
\divideents{dispose}
\item{\vskip -1.9ex 
\membername{dispose}
{\tt public void {\bf dispose}(  )
}%end signature
}%end item
\divideents{fishEye}
\item{\vskip -1.9ex 
\membername{fishEye}
{\tt protected IFigure {\bf fishEye}( {\tt boolean } {\bf arg0},
{\tt boolean } {\bf arg1} )
}%end signature
}%end item
\divideents{getBackgroundColor}
\item{\vskip -1.9ex 
\membername{getBackgroundColor}
{\tt public Color {\bf getBackgroundColor}(  )
}%end signature
}%end item
\divideents{getBorderColor}
\item{\vskip -1.9ex 
\membername{getBorderColor}
{\tt public Color {\bf getBorderColor}(  )
}%end signature
}%end item
\divideents{getBorderHighlightColor}
\item{\vskip -1.9ex 
\membername{getBorderHighlightColor}
{\tt public Color {\bf getBorderHighlightColor}(  )
}%end signature
}%end item
\divideents{getBorderWidth}
\item{\vskip -1.9ex 
\membername{getBorderWidth}
{\tt public int {\bf getBorderWidth}(  )
}%end signature
}%end item
\divideents{getFont}
\item{\vskip -1.9ex 
\membername{getFont}
{\tt public Font {\bf getFont}(  )
}%end signature
}%end item
\divideents{getForegroundColor}
\item{\vskip -1.9ex 
\membername{getForegroundColor}
{\tt public Color {\bf getForegroundColor}(  )
}%end signature
}%end item
\divideents{getGraphModel}
\item{\vskip -1.9ex 
\membername{getGraphModel}
{\tt public Graph {\bf getGraphModel}(  )
}%end signature
}%end item
\divideents{getHighlightColor}
\item{\vskip -1.9ex 
\membername{getHighlightColor}
{\tt public Color {\bf getHighlightColor}(  )
}%end signature
}%end item
\divideents{getItemType}
\item{\vskip -1.9ex 
\membername{getItemType}
{\tt public int {\bf getItemType}(  )
}%end signature
}%end item
\divideents{getLayoutEntity}
\item{\vskip -1.9ex 
\membername{getLayoutEntity}
{\tt public LayoutEntity {\bf getLayoutEntity}(  )
}%end signature
}%end item
\divideents{getLocation}
\item{\vskip -1.9ex 
\membername{getLocation}
{\tt public Point {\bf getLocation}(  )
}%end signature
}%end item
\divideents{getNodeFigure}
\item{\vskip -1.9ex 
\membername{getNodeFigure}
{\tt public IFigure {\bf getNodeFigure}(  )
}%end signature
}%end item
\divideents{getNodeStyle}
\item{\vskip -1.9ex 
\membername{getNodeStyle}
{\tt public int {\bf getNodeStyle}(  )
}%end signature
}%end item
\divideents{getSize}
\item{\vskip -1.9ex 
\membername{getSize}
{\tt public Dimension {\bf getSize}(  )
}%end signature
}%end item
\divideents{getSourceConnections}
\item{\vskip -1.9ex 
\membername{getSourceConnections}
{\tt public List {\bf getSourceConnections}(  )
}%end signature
}%end item
\divideents{getStyle}
\item{\vskip -1.9ex 
\membername{getStyle}
{\tt public int {\bf getStyle}(  )
}%end signature
}%end item
\divideents{getTargetConnections}
\item{\vskip -1.9ex 
\membername{getTargetConnections}
{\tt public List {\bf getTargetConnections}(  )
}%end signature
}%end item
\divideents{getTooltip}
\item{\vskip -1.9ex 
\membername{getTooltip}
{\tt public IFigure {\bf getTooltip}(  )
}%end signature
}%end item
\divideents{highlight}
\item{\vskip -1.9ex 
\membername{highlight}
{\tt public void {\bf highlight}(  )
}%end signature
}%end item
\divideents{initFigure}
\item{\vskip -1.9ex 
\membername{initFigure}
{\tt protected void {\bf initFigure}(  )
}%end signature
}%end item
\divideents{initModel}
\item{\vskip -1.9ex 
\membername{initModel}
{\tt protected void {\bf initModel}( {\tt org.eclipse.zest.core.widgets.IContainer } {\bf arg0},
{\tt java.lang.String } {\bf arg1},
{\tt org.eclipse.swt.graphics.Image } {\bf arg2} )
}%end signature
}%end item
\divideents{isDisposed}
\item{\vskip -1.9ex 
\membername{isDisposed}
{\tt public boolean {\bf isDisposed}(  )
}%end signature
}%end item
\divideents{isSelected}
\item{\vskip -1.9ex 
\membername{isSelected}
{\tt public boolean {\bf isSelected}(  )
}%end signature
}%end item
\divideents{isSizeFixed}
\item{\vskip -1.9ex 
\membername{isSizeFixed}
{\tt public boolean {\bf isSizeFixed}(  )
}%end signature
}%end item
\divideents{isVisible}
\item{\vskip -1.9ex 
\membername{isVisible}
{\tt public boolean {\bf isVisible}(  )
}%end signature
}%end item
\divideents{refreshLocation}
\item{\vskip -1.9ex 
\membername{refreshLocation}
{\tt protected void {\bf refreshLocation}(  )
}%end signature
}%end item
\divideents{setBackgroundColor}
\item{\vskip -1.9ex 
\membername{setBackgroundColor}
{\tt public void {\bf setBackgroundColor}( {\tt org.eclipse.swt.graphics.Color } {\bf arg0} )
}%end signature
}%end item
\divideents{setBorderColor}
\item{\vskip -1.9ex 
\membername{setBorderColor}
{\tt public void {\bf setBorderColor}( {\tt org.eclipse.swt.graphics.Color } {\bf arg0} )
}%end signature
}%end item
\divideents{setBorderHighlightColor}
\item{\vskip -1.9ex 
\membername{setBorderHighlightColor}
{\tt public void {\bf setBorderHighlightColor}( {\tt org.eclipse.swt.graphics.Color } {\bf arg0} )
}%end signature
}%end item
\divideents{setBorderWidth}
\item{\vskip -1.9ex 
\membername{setBorderWidth}
{\tt public void {\bf setBorderWidth}( {\tt int } {\bf arg0} )
}%end signature
}%end item
\divideents{setCacheLabel}
\item{\vskip -1.9ex 
\membername{setCacheLabel}
{\tt public void {\bf setCacheLabel}( {\tt boolean } {\bf arg0} )
}%end signature
}%end item
\divideents{setFont}
\item{\vskip -1.9ex 
\membername{setFont}
{\tt public void {\bf setFont}( {\tt org.eclipse.swt.graphics.Font } {\bf arg0} )
}%end signature
}%end item
\divideents{setForegroundColor}
\item{\vskip -1.9ex 
\membername{setForegroundColor}
{\tt public void {\bf setForegroundColor}( {\tt org.eclipse.swt.graphics.Color } {\bf arg0} )
}%end signature
}%end item
\divideents{setHighlightColor}
\item{\vskip -1.9ex 
\membername{setHighlightColor}
{\tt public void {\bf setHighlightColor}( {\tt org.eclipse.swt.graphics.Color } {\bf arg0} )
}%end signature
}%end item
\divideents{setImage}
\item{\vskip -1.9ex 
\membername{setImage}
{\tt public void {\bf setImage}( {\tt org.eclipse.swt.graphics.Image } {\bf arg0} )
}%end signature
}%end item
\divideents{setLocation}
\item{\vskip -1.9ex 
\membername{setLocation}
{\tt public void {\bf setLocation}( {\tt double } {\bf arg0},
{\tt double } {\bf arg1} )
}%end signature
}%end item
\divideents{setNodeStyle}
\item{\vskip -1.9ex 
\membername{setNodeStyle}
{\tt public void {\bf setNodeStyle}( {\tt int } {\bf arg0} )
}%end signature
}%end item
\divideents{setSize}
\item{\vskip -1.9ex 
\membername{setSize}
{\tt public void {\bf setSize}( {\tt double } {\bf arg0},
{\tt double } {\bf arg1} )
}%end signature
}%end item
\divideents{setText}
\item{\vskip -1.9ex 
\membername{setText}
{\tt public void {\bf setText}( {\tt java.lang.String } {\bf arg0} )
}%end signature
}%end item
\divideents{setTooltip}
\item{\vskip -1.9ex 
\membername{setTooltip}
{\tt public void {\bf setTooltip}( {\tt org.eclipse.draw2d.IFigure } {\bf arg0} )
}%end signature
}%end item
\divideents{setVisible}
\item{\vskip -1.9ex 
\membername{setVisible}
{\tt public void {\bf setVisible}( {\tt boolean } {\bf arg0} )
}%end signature
}%end item
\divideents{toString}
\item{\vskip -1.9ex 
\membername{toString}
{\tt public String {\bf toString}(  )
}%end signature
}%end item
\divideents{unhighlight}
\item{\vskip -1.9ex 
\membername{unhighlight}
{\tt public void {\bf unhighlight}(  )
}%end signature
}%end item
\divideents{updateFigureForModel}
\item{\vskip -1.9ex 
\membername{updateFigureForModel}
{\tt protected void {\bf updateFigureForModel}( {\tt org.eclipse.draw2d.IFigure } {\bf arg0} )
}%end signature
}%end item
\end{itemize}
}}
\startsubsubsection{Methods inherited from class {\tt org.eclipse.zest.core.widgets.GraphItem}}{
\par{\small 
\refdefined{l1673}\vskip -2em
\begin{itemize}
\item{\vskip -1.9ex 
\membername{checkStyle}
{\tt protected boolean {\bf checkStyle}( {\tt int } {\bf arg0} )
}%end signature
}%end item
\divideents{dispose}
\item{\vskip -1.9ex 
\membername{dispose}
{\tt public void {\bf dispose}(  )
}%end signature
}%end item
\divideents{getGraphModel}
\item{\vskip -1.9ex 
\membername{getGraphModel}
{\tt public abstract Graph {\bf getGraphModel}(  )
}%end signature
}%end item
\divideents{getItemType}
\item{\vskip -1.9ex 
\membername{getItemType}
{\tt public abstract int {\bf getItemType}(  )
}%end signature
}%end item
\divideents{highlight}
\item{\vskip -1.9ex 
\membername{highlight}
{\tt public abstract void {\bf highlight}(  )
}%end signature
}%end item
\divideents{isVisible}
\item{\vskip -1.9ex 
\membername{isVisible}
{\tt public abstract boolean {\bf isVisible}(  )
}%end signature
}%end item
\divideents{setVisible}
\item{\vskip -1.9ex 
\membername{setVisible}
{\tt public abstract void {\bf setVisible}( {\tt boolean } {\bf arg0} )
}%end signature
}%end item
\divideents{unhighlight}
\item{\vskip -1.9ex 
\membername{unhighlight}
{\tt public abstract void {\bf unhighlight}(  )
}%end signature
}%end item
\end{itemize}
}}
\startsubsubsection{Methods inherited from class {\tt org.eclipse.swt.widgets.Item}}{
\par{\small 
\refdefined{l1674}\vskip -2em
\begin{itemize}
\item{\vskip -1.9ex 
\membername{checkSubclass}
{\tt protected void {\bf checkSubclass}(  )
}%end signature
}%end item
\divideents{getImage}
\item{\vskip -1.9ex 
\membername{getImage}
{\tt public Image {\bf getImage}(  )
}%end signature
}%end item
\divideents{getText}
\item{\vskip -1.9ex 
\membername{getText}
{\tt public String {\bf getText}(  )
}%end signature
}%end item
\divideents{setImage}
\item{\vskip -1.9ex 
\membername{setImage}
{\tt public void {\bf setImage}( {\tt org.eclipse.swt.graphics.Image } {\bf arg0} )
}%end signature
}%end item
\divideents{setText}
\item{\vskip -1.9ex 
\membername{setText}
{\tt public void {\bf setText}( {\tt java.lang.String } {\bf arg0} )
}%end signature
}%end item
\end{itemize}
}}
\startsubsubsection{Methods inherited from class {\tt org.eclipse.swt.widgets.Widget}}{
\par{\small 
\refdefined{l1675}\vskip -2em
\begin{itemize}
\item{\vskip -1.9ex 
\membername{addDisposeListener}
{\tt public void {\bf addDisposeListener}( {\tt org.eclipse.swt.events.DisposeListener } {\bf arg0} )
}%end signature
}%end item
\divideents{addListener}
\item{\vskip -1.9ex 
\membername{addListener}
{\tt public void {\bf addListener}( {\tt int } {\bf arg0},
{\tt org.eclipse.swt.widgets.Listener } {\bf arg1} )
}%end signature
}%end item
\divideents{checkSubclass}
\item{\vskip -1.9ex 
\membername{checkSubclass}
{\tt protected void {\bf checkSubclass}(  )
}%end signature
}%end item
\divideents{checkWidget}
\item{\vskip -1.9ex 
\membername{checkWidget}
{\tt protected void {\bf checkWidget}(  )
}%end signature
}%end item
\divideents{dispose}
\item{\vskip -1.9ex 
\membername{dispose}
{\tt public void {\bf dispose}(  )
}%end signature
}%end item
\divideents{getData}
\item{\vskip -1.9ex 
\membername{getData}
{\tt public Object {\bf getData}(  )
}%end signature
}%end item
\divideents{getData}
\item{\vskip -1.9ex 
\membername{getData}
{\tt public Object {\bf getData}( {\tt java.lang.String } {\bf arg0} )
}%end signature
}%end item
\divideents{getDisplay}
\item{\vskip -1.9ex 
\membername{getDisplay}
{\tt public Display {\bf getDisplay}(  )
}%end signature
}%end item
\divideents{getListeners}
\item{\vskip -1.9ex 
\membername{getListeners}
{\tt public Listener {\bf getListeners}( {\tt int } {\bf arg0} )
}%end signature
}%end item
\divideents{getStyle}
\item{\vskip -1.9ex 
\membername{getStyle}
{\tt public int {\bf getStyle}(  )
}%end signature
}%end item
\divideents{isDisposed}
\item{\vskip -1.9ex 
\membername{isDisposed}
{\tt public boolean {\bf isDisposed}(  )
}%end signature
}%end item
\divideents{isListening}
\item{\vskip -1.9ex 
\membername{isListening}
{\tt public boolean {\bf isListening}( {\tt int } {\bf arg0} )
}%end signature
}%end item
\divideents{notifyListeners}
\item{\vskip -1.9ex 
\membername{notifyListeners}
{\tt public void {\bf notifyListeners}( {\tt int } {\bf arg0},
{\tt org.eclipse.swt.widgets.Event } {\bf arg1} )
}%end signature
}%end item
\divideents{removeDisposeListener}
\item{\vskip -1.9ex 
\membername{removeDisposeListener}
{\tt public void {\bf removeDisposeListener}( {\tt org.eclipse.swt.events.DisposeListener } {\bf arg0} )
}%end signature
}%end item
\divideents{removeListener}
\item{\vskip -1.9ex 
\membername{removeListener}
{\tt public void {\bf removeListener}( {\tt int } {\bf arg0},
{\tt org.eclipse.swt.widgets.Listener } {\bf arg1} )
}%end signature
}%end item
\divideents{removeListener}
\item{\vskip -1.9ex 
\membername{removeListener}
{\tt protected void {\bf removeListener}( {\tt int } {\bf arg0},
{\tt org.eclipse.swt.internal.SWTEventListener } {\bf arg1} )
}%end signature
}%end item
\divideents{setData}
\item{\vskip -1.9ex 
\membername{setData}
{\tt public void {\bf setData}( {\tt java.lang.Object } {\bf arg0} )
}%end signature
}%end item
\divideents{setData}
\item{\vskip -1.9ex 
\membername{setData}
{\tt public void {\bf setData}( {\tt java.lang.String } {\bf arg0},
{\tt java.lang.Object } {\bf arg1} )
}%end signature
}%end item
\divideents{toString}
\item{\vskip -1.9ex 
\membername{toString}
{\tt public String {\bf toString}(  )
}%end signature
}%end item
\end{itemize}
}}
}
}
}
\newpage
\def\packagename{uk.ac.ic.doc.neuralnets.gui.listeners}
\chapter{\bf Package uk.ac.ic.doc.neuralnets.gui.listeners}{
\vskip -.25in
\hbox to \hsize{\it Package Contents\hfil Page}
\rule{\hsize}{.7mm}
\vskip .13in
\hbox{\bf Classes}
\entityintro{ContinueQuestion}{l1730}{Prompts the user for to confirm continuing with an action}
\vskip .1in
\rule{\hsize}{.7mm}
\vskip .1in
\newpage
\section{Classes}{
\startsection{Class}{ContinueQuestion}{l1730}{%
{\small Prompts the user for to confirm continuing with an action}
\vskip .1in 
\startsubsubsection{Declaration}{
\fbox{\vbox{
\hbox{\vbox{\small public 
class 
ContinueQuestion}}
\noindent\hbox{\vbox{{\bf extends} java.lang.Object}}
}}}
\startsubsubsection{Constructors}{
\vskip -2em
\begin{itemize}
\item{\vskip -1.9ex 
\membername{ContinueQuestion}
{\tt public {\bf ContinueQuestion}(  )
\label{l1731}\label{l1732}}%end signature
}%end item
\end{itemize}
}
\startsubsubsection{Methods}{
\vskip -2em
\begin{itemize}
\item{\vskip -1.9ex 
\membername{ask}
{\tt public static boolean {\bf ask}( {\tt org.eclipse.swt.widgets.Shell } {\bf parent} )
\label{l1733}\label{l1734}}%end signature
\begin{itemize}
\sld
\item{
\sld
{\bf Usage}
  \begin{itemize}\isep
   \item{
Ask a question with the standard description: 
 "All unsaved changes will be lost!".
}%end item
  \end{itemize}
}
\item{
\sld
{\bf Parameters}
\sld\isep
  \begin{itemize}
\sld\isep
   \item{
\sld
{\tt parent} - - root shell}
  \end{itemize}
}%end item
\item{{\bf Returns} - 
true to continue, false otherwise 
}%end item
\end{itemize}
}%end item
\divideents{ask}
\item{\vskip -1.9ex 
\membername{ask}
{\tt public static boolean {\bf ask}( {\tt org.eclipse.swt.widgets.Shell } {\bf parent},
{\tt java.lang.String } {\bf desc} )
\label{l1735}\label{l1736}}%end signature
\begin{itemize}
\sld
\item{
\sld
{\bf Usage}
  \begin{itemize}\isep
   \item{
Ask a continue question of the user.
}%end item
  \end{itemize}
}
\item{
\sld
{\bf Parameters}
\sld\isep
  \begin{itemize}
\sld\isep
   \item{
\sld
{\tt parent} - - root shell}
   \item{
\sld
{\tt desc} - - question description}
  \end{itemize}
}%end item
\item{{\bf Returns} - 
true to continue, false otherwise 
}%end item
\end{itemize}
}%end item
\end{itemize}
}
}
}
}
\end{document}
